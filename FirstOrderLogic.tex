\chapter{Predicate - / first order logic}
\defin{A First order Language}{
    consists of infinetely many distinct symbols such that no symbol is a proper 
    initial segment of another symbol and the symbols are divided into 2 groups:
    \begin{enumerate}
        \item logical symbols (These elements have a fixed meaning and the equivalence symbol $=$ is optional)
        \begin{flalign*}
            (,),\lnot, \to, v_1,v_2,\dots,=&&
        \end{flalign*}
        \item parameters
        \begin{itemize}
            \item quantifier symbol: $\forall$ (the range is subject of interpretation)
            \item predicate symbols: for every $n>0$ we have a set of $n$-ary predicates $P$
            \item constant symbols: Some set of constants (could also be $\varnothing$)
            \item function symbols: for every $n>0$ we have a set of $n$-ary function symbols
        \end{itemize}
    \end{enumerate}
}
\note{}{
    \begin{itemize}
        \item We could drop constants and instead introduce $0$-ary function symbols
        \item to specify language we need to specify the parameters and say if $=$ is included
    \end{itemize}
}
\bsp{}{
    \begin{itemize}
        \item $\mathcal{L}_\text{set} = \{\in\}$, \quad $=$ included and the binary predicate symbol $\in$ "element in"
        \item $\mathcal{L}_\text{arith} = \{<,0,S,E,+,\cdot\}$
        \begin{itemize}
            \item [$=$] included
            \item [$<$] is a binary rel. symbol
            \item [$0$] is a constant
            \item [$S$] is a unary function symbol
            \item [$E$] exponentiation function symbol
            \item [$+,\cdot$] binary function symbols
        \end{itemize}
        \item $\mathcal{L}_\text{ring} = \{=,+,\cdot,-,0,1\}$
        \begin{itemize}
            \item [$=$] included
            \item [$0,1$] are constants
            \item [$-$] is a unary function symbol (additive inverse)
            \item [$+,\cdot$] binary function symbols
        \end{itemize}
    \end{itemize}
}

\section{Formulas}
\defin{Expression}{An \graybf{expression} is any finite sequence of symbols.
    There exist two kinds of expressions that makes sense "grammatically"
    \begin{itemize}[leftmargin=1.6cm]
        \item[Terms:] \begin{itemize}
            \item points to an object
            \item they are built up from variables and constants using function symbols %(by use of polish notation)
        \end{itemize}
        \item[Formulas:]\begin{itemize}
            \item They express assertions about objects,
            \item they are built up from atomic formulas 
            \item atomic formulas these are built up from terms using predicate symbols and $=$, if included
        \end{itemize}
    \end{itemize}
}
\defin{Term Building Operations}{
    For every $n>0$ and for every $n$-place function symbol $f$ let $\mathcal{F}_f$ be an $n$-place term building operation,
    that is $\mathcal{F}_f (t_1,\dots t_n)\defeq ft_1,\dots t_n$ (polish notation for $f(t_1,\dots t_n)$).
    The Set of Terms we then define as the set of expressions that are built up from variables and constants by applying the term building operations
    finitely many times.
}
\bsp{}{Let $\mathcal{L} = \mathcal{L}_{arith}$ then the set of terms will contain $0$, $v_{42}$, $S0$, $SSS0$, $Sv_1$, $+SOv_1$}
\defin{Atomic formula}{Any expression of the form 
    $$=t_1t_2 \text{ or } Pt_1,\dots t_n, \text{ where $t_1,\dots t_n$ are terms and $P$ is an $n$-ary predicate symbol}$$}
\note{}{Atomic formulas are not defined inductively.}
\bsp{}{\emph{cont.}\quad $=v_1 v_{42}$, $<S0 SS0$ are atomic formulas, but $\lnot = v_1 v_{42}$ is not.}
\defin{Formulas}{We define $\varepsilon_\lnot$, $\varepsilon_\to, Q_i$ to be the fla. building operations, defined as follows
    $\varepsilon_\lnot(\alpha) \defeq (\lnot \alpha)$, $\varepsilon_\to \defeq (\alpha \to \beta)$ and
    $Q_i(\gamma) \defeq \forall v_i \gamma$.
    The set of formulas is the set of expressions built up from atomic formulas by applying the fla. building operations finitely many times.
}
\bsp{}{\emph{cont.} $\forall v_1 (=Sv_10)$ is a formula we get by applying $Q_1$ on the atomic formula $=Sv_1 0$.}
\subsubsection*{Free variables}
\bsp{}{We introduce the $\exists$\outernote{$\exists$ quantifier} quantifier by defining $\exists y \alpha$ means $\lnot \forall y \lnot \alpha$.\\
    "Every non-zero natual number is a succsesor" $\forall x (x\neq 0\to \exists y S(y)=x)$
    is different then "if a number is not $0$, then it is a succsesor" $x\neq 0\to \exists y S(y)=x$. 
    $x$ occurs bounded\outernote{bounded variable} in the first formula, for the latter $x$ occures free in the fla.

    If you have an expression without free variables, it is either true or false, on the other hand
    if a variable occurs free in a formula, the truth value of it depends on the variable itself.
}
\defin{Free variables}{
    Let $x$ be a variable. $x$ occurs \graybf{free in $\varphi$} is defined inductively as follows:
    \begin{enumerate}
        \item If $\varphi$ is an atomic fla. then $x$ occurs \graybf{free} in $\varphi$ iff $x$ occurs in $\varphi$
        \item If $\varphi = (\lnot \alpha)$ then $x$ occurs free in $\varphi$ iff $x$ occurs free in $\alpha$
        \item If $\varphi = (\alpha \to \beta)$ then $x$ occurs free in $\varphi$ iff $x$ occurs free in $\alpha$ or $\beta$
        \item If $\varphi = \forall v_i \alpha$ then $x$ occurs free in $\varphi$ iff $x$ occurs free in $\alpha$ and $x\lnot v_i$
    \end{enumerate}
    A formula $\alpha$ is called a sentence, if no variable occurs free in $\alpha$
}
\note{}{The above definition makes sense thanks to the recursion theorem.
    def function $h$ on the set of atoms: $h(\alpha) = \text{the set of var occ in fla } \alpha$, which is the set of all variables $v_i$ that occur free in $\alpha$.
    we now want to extend $h$ to $\overline{h}$, which is the set of all formulas.
    \begin{itemize}
        \item $\overline{h}(\lnot \alpha) = \overline{h}(\alpha)$
        \item $\overline{h}( \alpha \to \beta) = \overline{h}(\alpha)\cup \overline{h}(\beta)$
        \item $\overline{h}(Q_i(\alpha)) = \overline{h}(\alpha)\backslash \{v_i\}$
    \end{itemize}
    We say $x$ occurs free in $\alpha$ iff $x\in \overline{h}(\alpha)$.
}
\note{}{We will now use $\lnot, \land,\lor,\to, \leftrightarrow, \exists v_i$ (all can be expressed in terms of $\lnot,\to,Q_i$.)
    We will sometimes drop the $(,)$ and not always be using polish notation.
}
\section{Semantics of first order logic}
The equivalent scheme to our TA in predicate logic. The meaning of formulas is given by \emph{structures}, 
which also determine the scope of the quantifier $\forall$, the meaning of all parameters.
\defin{structure}{A \graybf{structure} $\mathcal{A}$ for a first order language $\mathcal{L}$ is a non-empty set
    set $A$ called \graybf{universe} or \graybf{underlying set of $\mathcal{A}$} together with an interpretation of each parameters of $\mathcal{L}$ i.e.
    \begin{itemize}
        \item $\forall$ ranges over the universe $A$
        \item for an $n$-ary pred. symbol $P\in \mathcal{L}$ its interpretation $P^\mathcal{A}$ is a subset of $A^n$
        \item for a constant $c\in \mathcal{L}$ its interpretation $c^\mathcal{A}$ is an element of $A$
        \item for an $n$-ary function symbol $f\in \mathcal{L}$ its interpretation $f^\mathcal{A}$ is a total function $f^\mathcal{A}: A^n \to A$
    \end{itemize}
}
\note{}{$A\neq \varnothing$, and all functions $f^\mathcal{A}$ are total.}
\bsp{}{Let $\mathcal{L} = \{\in\}$ where $\in$ is a binary relation "
    An example of an $\mathcal{L}$ structure is $(\NN, \in^\NN)$ where $\in^\NN = \{(x,y)\in \NN^2 : x<y\}$}
\defin{Extention of an assigment}{Let $\varphi$ be a $\mathcal{L}$-fla. and $\mathcal{A}$ a $\mathcal{L}$-structure. 
    Let $V$ be the set of all variables in $\mathcal{L}$ and $s:V\to A$ an assignment.
    We define the extention $\overline{s}$ of $s$ to the set of all $\mathcal{L}$-terms by
    \begin{itemize}
        \item $x\in V$ then $\overline{s}(x) \defeq s(x)$
        \item $c\in \mathcal{L}$ a constant symbol, then $\overline{s}(c) \defeq c^\mathcal{A}$
        \item $t_1,\dots t_n$ $\mathcal{L}$-terms and $f\in \mathcal{L}$ an $n$-ary function symbol, then 
        $\overline{s}(ft_1\dots t_n) \defeq f^\mathcal{A}(\overline{s}(t_1),\dots \overline{s}(t_n))$
    \end{itemize}
}

\note{}{in the previous definition point 3. for $n=1$ yields a commutative diagram.

}
\prop{}{For any given assignment $s$ there exists a unique extention $\overline{s}$ as in the previous definition.}
{
    will follow from recursion theorem and unique decomposition of terms.
}
\subsection*{Definition of truth}
\defin{Satisfy}{ We define '$\mathcal{A}$ satisfies $\varphi$ with $s$' and write
    $\mathcal{A}\models \varphi [s]$ inductively over the complexity of the formula $\varphi$
    \begin{itemize}
        \item if $\varphi$ is atomic: \begin{itemize}
            \item $\mathcal{A} \models = t_1,t_2 [s] \overline{s}(t_1) = \overline{s}(t_2)$
            \item $\mathcal{A} \models P t_1,\dots t_n [s] (\overline{s}(t_1),\dots \overline{s}(t_2))\in P^\mathcal{A}$
        \end{itemize}
        \item suppose $\mathcal{A}\models \varphi [s]$ and $\mathcal{A}\models \psi [s]$ are defined, then
        \begin{itemize}
            \item $\mathcal{A}\models \lnot\varphi [s]$ iff $\mathcal{A}\nvDash \varphi [s]$
            \item $\mathcal{A}\models \varphi\to \psi [s]$ iff $\mathcal{A}\models \psi [s]$ or $\mathcal{A}\nvDash  \varphi [s]$
            \item $\mathcal{A}\models \forall x \varphi [s]$ iff for all $a\in A$ $\mathcal{A}\models \varphi [s(x| a)]$
            where 
            \begin{equation*}
                s(x|a)(v) = \begin{cases}
                s(v) \text{ if } v \neq x\\
                a \text{ if } v = x
            \end{cases}
            \end{equation*}
            
        \end{itemize}
    \end{itemize}
}
\bsp{}{
    $\mathcal{L} = \{\forall, \leq, S, 0\}$ 
    a $\mathcal{L}$-structure then could be $\mathcal{N} = (\NN,\leq^\mathcal{N},S^\mathcal{N},0^\mathcal{N})$ together with an assignment
    $s: v_n \mapsto n-1$ then:
    \begin{itemize}
        \item $s(v_1) = 0$
        \item $\overline{s}(0) = 0^\mathcal{N}$ (a constant is always mapped to its realisation, the interpretation of constant $0$ in the structure $\mathcal{N}$)
        \item $\overline{s}(Sv_1) = S^\mathcal{N}(\overline{s}(v_1)) = S^\mathcal{N}(0) = 1$\\
        \item $\mathcal{N}\models \forall v_1 (S(v_1) \neq v_1) [s]$\\
            iff for all $a\in \NN$ we have that $\mathcal{N}\models (S(v_1) \neq v_1) [s(v_1 | a)]$\\
            iff \dots \\
            iff for all $a\in \NN$ we have $S^\mathcal{A}(a) \neq a$, which is true in our structure of the natural numbers.
        \item Is it true in $\mathcal{N}$ that $\mathcal{N}\models S(0) \leq S(v_1) [s]$? Yes because
        \begin{equation*}
            \begin{split}
                &\mathcal{N}\models S(0) \leq S(v_1) [s] \\
                &iff 1\leq 1
            \end{split}
        \end{equation*}
        
    \end{itemize}
}
\note{}{To know wheter $\mathcal{A}\models \varphi [s]$ it suffices to know where $s$ maps the variables that are free in $\varphi$}

\prop{}{Suppose $s_1,s_2:V\to A$ agree on all variables that occur free in $\varphi$ then 
 $$\mathcal{A}\models \varphi [s_1] \text{iff }\mathcal{A}\models \varphi [s_2]$$
}{
    By complexity of $\varphi$
    \begin{itemize}
        \item if $\varphi$ is $P t_1,\dots t_n$ 
        note: any var that occur in $\varphi$ occur free in $\varphi$, so $s_1,s_2$ agree on all variables that occur in the terms $t_1,\dots t_n$.\\
        So we Claim: for $t$ a term, $s_1,s_2$ assignments that agree on all variables of $t$ then $\overline{s}_1(t) = \overline{s}_2(t)$
        \begin{claimproof}
            By complexity of $t$
            \begin{itemize}
                \item[$t = v_m$] then  $\overline{s}_1(t) ={s}_1(v_m) = {s}_2(v_m) =\overline{s}_2(t)$
                \item[$t = c$] then  $\overline{s}_1(t) =c^\mathcal{A} =\overline{s}_2(t)$
                \item[$t = ft_1\dots t_n$] inductively, assume $\overline{s}_1(t_i) =\overline{s}_2(t_i)$ for all $1\leq i\leq n$ then  TODO
            \end{itemize}
        \end{claimproof}
        \item $\varphi: =t_1,t_2$ is similar
        \item $\varphi: \lnot \alpha$ then $\mathcal{A}\models \lnot \alpha [s_1]$ iff $\mathcal{A}\vDash \alpha [s_1]$iff $\mathcal{A}\vDash \alpha [s_2]$ iff $\mathcal{A}\models \lnot \alpha [s_1]$
        \item $\varphi: \alpha \to \beta$ then $\mathcal{A}\models  \alpha \to \beta [s_1]$ iff .. or .. iff for s2 iff ... or .. 
        \item $\varphi: \forall x \alpha$ then the assumption is that $s_1,s_2$ .. the free variables of $\alpha$ are the free variables of $\varphi$ exept for $x$.
        but because $s_1(x|a) = s_2(x|a)$ they both agree on all free variables of $\alpha$.
        \[\begin{split}
            \mathcal{A}\models \forall x \varphi [s_1] &\iif \fora a\in A \mathcal{A}\models \varphi [s_1(x| a)]\\
            & \iif \fora a\in A \mathcal{A}\models \varphi [s_2(x| a)]\\
            & \iif \mathcal{A}\models \forall x \varphi [s_2]
        \end{split}\]
    \end{itemize}
    
}
Notation: $\mathcal{A}\models \varphi $TODO means that all free variables of $\varphi$ are among $v_1,\dots v_n$ and $\mathcal{A}\models \varphi [s]$ whenever $s(v_i) = a_i$ for all $1\leq i\leq n$.
\coroll{If $\sigma$ is a sentence then $\mathcal{A}\models \varphi [s]$ for all $s:V\to A$ or 
$\mathcal{A}\vDash \varphi [s]$ for all $s:V\to A$.

Notation: $\mathcal{A}\models \sigma$ "$\sigma$ is true in $\mathcal{A}$, $\mathcal{A}$ is a model of $\sigma$ or $\sigma$ holds in $\mathcal{A}$.
}
\note{}{If $\sigma$ is a sentence then we can not have  $\mathcal{A}\models \sigma$ and  $\mathcal{A}\vDash \sigma$ because $A\neq \varnothing$.}
\defin{Model}{$\mathcal{A}$ is a model of a set of sentences $\Sigma$ iff for every sentence $\sigma\in \Sigma$ it holds  $\mathcal{A}\models \sigma$ }
\bsp{}{$\mathcal{L} = \{0,1,+,-,\cdot\}$
A realisation could be $\mathcal{R} = (\RR, 0,1,+,-,\cdot)$ or $\mathcal{C} = (\mathbb{C}, 0,1,+,-,\cdot)$
then the sentence 
$\sigma: \quad \exists x (x\cdot x = -1)$ then $\mathcal{R}\vDash \sigma$ but $\mathcal{C}\models \sigma$

}
\note{}{$\land,\lor,\leftrightarrow,\exists$ work as expected. That is 
 $\mathcal{A}\models (\alpha \land \beta) [s]$ iff $\mathcal{A}\models \alpha [s]$ and $\mathcal{A}\models \beta [s]$
 $\mathcal{A}\models (\alpha \lor \beta) [s]$ iff $\mathcal{A}\models \alpha [s]$ or $\mathcal{A}\models \beta [s]$
 $\mathcal{A}\models \exists x \alpha [s]$ iff $\mathcal{A}\models \lnot \forall x \lnot \alpha [s]$\\
 iff $\mathcal{A}\vDash \forall x \lnot \alpha [s]$\\
 iff it is not true that forall $a \in A$ $\mathcal{A}\models \lnot \alpha [s(x|a)]$\\
 iff there is $a\in A$ such that $\mathcal{A}\models \alpha [s(x|a)]$
}


\section{Logical implication}
Let $\Gamma$ be a set of $\mathcal{L}$-formulas, $\varphi$ a $\mathcal{L}$-formula.
\defin{Logical implication}{
    $\Gamma \models \varphi$ "$\Gamma$ logically implies $\varphi$" if for every $L$-structure $\mathcal{A}$ and for every $s:V\to A$ 

    if $\mathcal{A}\models \gamma [s]$for every $\gamma \in \Gamma$  then $\mathcal{A}\models \varphi [s]$

}

\defin{Logical equivalence}{$\varphi,\psi$ are called logically equivalent if 
$\varphi \models \psi$ and $\psi \models \varphi$.
}
\defin{Valid}{$\varphi$ is called valid iff $\models \varphi$ i.e. $\varnothing\models \varphi$ i.e. for every $\mathcal{L}$-structure $\mathcal{A}$ and every $s:V\to A$ it is $\mathcal{A}\models \varphi [s]$
}
\bsp{}{
\begin{enumerate}
    \item $\forall x_1 P x_1 \models P x_2$\\
    Suppose $\mathcal{A}\models \forall x_1 P x_1 [s]$.
    then for all $a\in A$ it is $\mathcal{A}\models P x_1 [s(x_1|a)]$ in particular, $a\in P^\mathcal{A}$ for $a = s(x_2)$
    \item  $\forall P x_2 \vDash \forall x_1 P x_1$\\
    We need a counterexample to $\forall P x_2 \models \forall x_1 P x_1$. Let $A = \{a_1,a_2\}$ $s(x_2) = a_1$ and $P^\mathcal{A} = \{a_1\}$
    then $\mathcal{A}\models Px_2[s]$.
    \item Is the following valid? $\models \exists x(Px \to \forall y P y)$ yes
    \item $\Gamma,\alpha\models \varphi$ iff $\Gamma \models \alpha \to \varphi$. (on next problem set, quite impoirtant)
\end{enumerate}
}

\section{definability in a structure}
\bsp{}{
    \begin{enumerate}
        \item $x = x$ would define the entire universe.
        \item $\lnot x = x$ would define the empty set.
    \end{enumerate}
}
\defin{definability in a structure}{ 
    We say that a general $n$-ary relation $P$ on $A$ (we will just call it $P$, it does not have to be in the language) is definable in $\mathcal{A}$, if there is a $\mathcal{L}$-formula
    $\varphi$ with free variables among $\{v_1,\dots,v_n\}$ such that 
    \[P = \{(a_1,\dots a_n) : \mathcal{A} \models \varphi [a_1,\dots a_n]\}\]
    We also say that $\varphi$ defines $P$ in the structure $\mathcal{A}$.
}
\bsp{}{
\begin{enumerate}
    \item decomposition
    \item $\mathcal{R} = (\RR,0,1,+,-,\cdot)$ Q: is $[0,\infty)$ definable in $\mathcal{R}$
    Yes because $\exists y (y\cdot y = x)$
    Indeed we can even define the $\leq$ relation on $\RR^2$ by $x\leq z \defaq \exists y (x+y\cdot y = z)$ 
\end{enumerate}
}
\defin{definability of classes of structures}{
    Let $\Sigma$ be a set of sentences. $\tau$ a sentence.
    We will say that the class of models of $\Sigma$ is the class $Mod \Sigma =\{\mathcal{A} : \mathcal{A}\models \Sigma\}$.
    Let $K$ be a class of structures. We are going to call $K$ an elementary class (EC) if there is a single sentence $\tau$ such that $Mod \tau = K$
    $K$ is an elementary class in the wider sence (EC$\Delta$) if there is a set of sentences $\Sigma$ such that $Mod \Sigma = K$

}
\bsp{}{$\mathcal{L} = \{0,1,+,\cdot\}$ 
    $\tau$ is a sentence that expresses the field axioms (the unary inverse functions are not in our language but are definable.)
    $Mod \tau$ is the class of all the fields, which is EC.
    the class of all fields of characteristic $0$. Let $\sigma_p: \lnot(1 + \dots +1 = 0)$ then $\Sigma = \{\tau\} \cup \{\sigma_p : p\in \mathbb{P}\}$ 
    yields $Mod \Sigma$ is the class of fields with characteristic $0$, therefore EC$\Delta$, we will later see that it is not $EC$.

}
\bsp{}{
    Let $E$ be a binary relation, $\mathcal{L} = \{E\}$ then a graph is a realisation $\mathcal{G} = (V,E^\mathcal{G})$ 
    such that $v\neq \varnothing$, $E^\mathcal{G}$ is irreflexive and symmetric.
    By definition the universe is not empty, we still have to check irreflexive and symmetric.
    \begin{itemize}
        \item irreflexive: $\forall x (\lnot x E x)$
        \item symmetric: $\forall x \forall y (x E y \to y E x)$
    \end{itemize}
    We take $\tau$ to be $\forall x \forall y ((\lnot x E x)\land (x E y \to y E x))$
    Then $Mod \tau$ is the class of all graphs and is EC
    Note: the class of all finite graphs is neither EC nor EC$\Delta$. proof later.
}
We want to have some notion that tells us when two graphs are the same or at least similar.
\section{Homomorphisms of structures}
\defin{Homomorphism}{
    Suppose that $\mathcal{A},\mathcal{B}$ are two $\mathcal{L}$-structures. then a Homomorphism of $\mathcal{A}$ into $\mathcal{B}$ is a map
    $h: A \to B$ that satisfy the below conditions
    \begin{itemize}
        \item for every $n$-ary predicate $P\in \mathcal{L}$ it is $(a_1,\dots a_n)\in P^\mathcal{A}$ iff $(h(a_1),\dots h(a_n))\in P^\mathcal{B}$ (this def. a strong Homomorphism, other textbooks maybe only reqire $\to$ direction)
        \item for every $n$-ary function $f\in \mathcal{L}$ and for all $\underline{a} = (a_1,\dots a_n)\in A^n$ it holds $h(f^\mathcal{A}(\underline{a})) = f^\mathcal{B}(h(a_1),\dots h(a_n))$
        \item for every constant symbol $c\in \mathcal{L}$ it is $h(c^\mathcal{A}) = c^\mathcal{B}$ (could also skip this if we consider constants as $0$-ary functions)
    \end{itemize}
}
\note{}{Intuativly a Homomorphism of $\mathcal{A}$ into $\mathcal{B}$ is a map $A \to B$ that preserve all funciton and relation symbols in some sense, (imp: not the definable relations)}
\defin{Isomorphism}{
    \begin{itemize}
        \item $h:A\to B$  is called isomorphism of $\mathcal{A}$ into $\mathcal{B}$ if $h$ is a Homomorphism and injective (in other textbooks: an isomorphic embedding of $\mathcal{A}$ into $\mathcal{B}$)
        \item $h:A\to B$  is called isomorphism of $\mathcal{A}$ onto $\mathcal{B}$ if $h$ is a Homomorphism and bijective $A\to B$
        \item \outernote{isomorphic}$\mathcal{A}$ and $\mathcal{B}$ are called isomorphic if there is an isomorphism of $\mathcal{A}$ onto $\mathcal{B}$
    \end{itemize}
}
\note{}{
    %TODO: in general 
}
\bsp{}{
    $\mathcal{L}= \{+,\cdot\}$
    $\mathcal{N} = (\NN, +^\NN, \cdot^\NN)$ and $\mathcal{B} = (B,+^\mathcal{B},\cdot^\mathcal{B})$ where $B = \{0,1\}$ and 
    \begin{tabular}{c|c c}
        $+^\mathcal{B}$ & $e$ & 0\\\hline
        $e$ & $e$ & 0\\
        0 & 0 & $e$
    \end{tabular}
    \begin{tabular}{c|c c}
        $\cdot^\mathcal{B}$ & $e$ & 0\\\hline
        $e$ & $e$ & $e$\\
        0 & $e$ & 0
    \end{tabular}
    let $h:\NN \to B$ a Homomorphism?
    $h(n) = \begin{cases}
        e & \text{ if $n$ is even}\\
        0 & \text{ else}
    \end{cases}$
    need at first that $h(m+n ) = h(m) +^\mathcal{B} h(n)$
    and $h(m\cdot n) = h(m)\cdot^\mathcal{B} h(n)$.
    it is indeed a Homomorphism.
}
\defin{Substructure}{Suppose we have two $\mathcal{L}$ structures and $A\subseteq B$ hen $\mathcal{A}$ is a substructure of $\mathcal{B}$ (notation: $\mathcal{A}\subseteq \mathcal{B}$ or we might say $\mathcal{B}$ is an extention of $\mathcal{A}$ ) if
    \begin{itemize}
        \item for every $n$-ary relation $P^\mathcal{A} = P^\mathcal{B}|_A$
        \item for every $n$-ary function $f^\mathcal{A} = f^\mathcal{B}|_A$
        \item for every constant symbol $c$ in $\mathcal{L}$ it is $c^\mathcal{A} = c^\mathcal{B}$
    \end{itemize}
}
\bsp{}{
    $\mathcal{L} = \{\leq\}$ then $\mathcal{N} = (\NN, \leq)$ and $\mathcal{P} = (\NN^+, \leq ^\mathcal{P})$ where $\leq^\mathcal{P}$ is the restriction of $\leq$ to the positive natual numbers.
    $\mathcal{P} \subseteq \mathcal{N}$ and there exists a isomorphic embedding $id: \NN^+\to \NN$ from $\mathcal{P}$ into $\mathcal{N}$
    They are even isomorphic ($h:\NN \to \NN^+, h(n) = n+1$) so in fact $\mathcal{P} \cong \mathcal{N}$.
}
\bsp{}{
    $(\mathbb{Q},+)\subseteq(\mathbb{C},+)$
}
\note{}{If $\mathcal{A}\subseteq \mathcal{B}$ then in particular $\mathcal{A}$ is closed under all constant and functions in $\mathcal{B}$
    So suppose that $\mathcal{B}$ is a substructure and $A\subseteq B$ and $A\neq \varnothing$ and $A$ is closed under $f^\mathcal{B}$, $c^\mathcal{B}$ 
    Can then $A$ be made into a substructure $\mathcal{A}$ of $\mathcal{B}$.
    $f^\mathcal{A}$ would be the restriction of $f^\mathcal{B}$ to $A^n$, constants $c^\mathcal{A} = c^\mathcal{B}$ 
    and if $P \in \mathcal{L}$ is an $n$-ary predicate then $P^\mathcal{A}$ should be $P^\mathcal{B}\cap A^n$.
    If $\mathcal{L}$ has no const. or fuction symbols then any subset can be made into a substructure of a structure on $\mathcal{L}$.
}
Our next question will be: what is the relation of the above notions with truth and satisfiability
The answer will be given by the so called Homomorphism theorem.

\prop{Homomorphism theorem}{
    $h$ homomorphism of $\mathcal{A}$ into $\mathcal{B}$, $s:V\to A$ then
    \begin{enumerate}
        \item for all terms $t$ it is $h(\overline{s}(t))=\overline{(h\circ s)}(t)$
        \item $\varphi$ a fla. that is quantifier free and does not include $=$ then $\mathcal{A}\models \varphi [s]$ iff $\mathcal{B}\models \varphi [h\circ s]$
        \item if $h$ is additionally injective then we can drop the requirement " no $=$".
        \item if $h$ is homomorphism of $\mathcal{A}$ onto $\mathcal{B}$ then we can drop the requirement "q.f." in (b)
    \end{enumerate}
}{
    \begin{enumerate}
        \item problem set
        \item \begin{itemize}
            \item $\varphi: Pt$ then $\mathcal{A}\models Pt[s]$ iff $\overline{s}(t)\in P^\mathcal{A}$ iff $h(\overline{s}(t))\in P^\mathcal{B}$ iff $\overline{(h\circ s)}(t)\in P^\mathcal{B}$ iff $\mathcal{B}\models Pt [h\circ s]$
            \item $\varphi: \lnot \psi$ $\mathcal{A}\models \lnot \psi[s]$ iff $\mathcal{A}\vDash \psi[s]$ iff $\mathcal{A}\vDash \psi[s]$ iff %TODO
            \item $\varphi: \psi \to \alpha$
        \end{itemize}
        \item $\mathcal{A}\models =t_1t_2 [s]$ iff $\overline{s}(t_1)=\overline{s}(t_2)$ iff $h(\overline{s}(t_1))=h(\overline{s}(t_2))$ iff (by (a))$\overline{(h\circ s)}(t_1) = \overline{(h\circ s)}(t_2)$ iff $\mathcal{B}\models =t_1t_2[h\circ s]$
        \item $\varphi$ $ \forall s: V\to A$ $\mathcal{A}\models \varphi [s]$ iff $\mathcal{B}\models \varphi [h\circ s]$, want $\mathcal{A}\models \forall x \varphi [s]$ iff $\mathcal{B}\models \forall x \varphi [h \circ s]$
        1. $\mathcal{B}\models \forall x \varphi [(h\circ s)]$
        iff for all $s:V\to A$, $a\in A$ (req. surjectivity) it is $\mathcal{B}\models \varphi [(h\circ s)(x | h(a))]$
        iff $\mathcal{B}\models \varphi [h\circ (s(x |a))]$
        iff (inductive assumption)$\mathcal{A}\models \varphi [s(x|a)]$ because $a$ was arbitrary it is $\mathcal{A}\models \forall x \varphi [s]$
        2. Suppose $\mathcal{B}\vDash \forall x \varphi [(h\circ s)]$ then there exists a $b\in B$ such that $\mathcal{B}\models \lnot \varphi [(h\circ s)(x | b)]$
        by surjectivity we can find $a\in A$ such that $h(a)=b$ and it is $\mathcal{B}\models \lnot \varphi [(h\circ s)(x | h(a))]$
        By the inductive assumption $\mathcal{A}\models \lnot \varphi [s(x|a)]$ and $\mathcal{A}\vDash \forall x \varphi [s]$
    \end{enumerate}
}
\note{}{$\mathcal{A}\cong \mathcal{B}$ then $\mathcal{A}$ and $\mathcal{B}$ satisfy exactly the same sentences.}
\defin{elementarily equivalent}{$\mathcal{A}$ and $\mathcal{B}$ are called elementarily equivalent ($\mathcal{A}\equiv \mathcal{B}$) if $\mathcal{A}$ and $\mathcal{B}$ satisfy the same sentences.}
\note{}{If $\mathcal{A}\cong \mathcal{B}$ implies $\mathcal{A}\equiv \mathcal{B}$
    The converse is not true.
    For instance DLO (dence linear order) w/o endpoints is complete, so two structures on DLO are equivalent 
    $(\mathbb{Q},<)\equiv (\RR,<)$ but they are not isomorphic because the universes have diffrent cardinality. 
}
\bsp{}{
    $\mathcal{N} = (\NN, \leq)$ and $\mathcal{P}=(\NN^{>0},\leq )$ $h:n\mapsto n-1:\mathcal{P}\to \mathcal{N}$ isom. 
    so in part $\mathcal{N} \equiv \mathcal{P}$.
    but $id:\mathcal{P}\to \mathcal{N}$ is only isom embedding, so for example
    $\forall y(x\neq y x\leq y)$  
    $\mathcal{P}\models \alpha [1]$ but $\mathcal{N}\nvDash \alpha [1]$
    but $\mathcal{N}\models \alpha [h(1)]$
}
\defin{Automorphism}{An automorphism is an isomorphism of the form $h:A\to A$ from $\mathcal{A}$ onto $\mathcal{A}$}
\note{}{Every structure has a trivial automorphism $id:A\to A$}
\defin{Rigid}{If the only automorphism on $\mathcal{A}$ is the trivial automorphism, then $\mathcal{A}$ is called rigid.}
\bsp{}{ If every element is definable then the structure is rigid.
    For example $(\NN,0,S)$ and $(\NN,<)$ every element is definable, therefore the structures are rigid.
    %TODO examples structures with many automorphism
    }
\coroll{
    Let $h$ be atutom of $\mathcal{A}$, $R\subseteq A^n$ definable in $\mathcal{A}$ then $\forall a\in A^n a\in R \iif (h(a_1),\dots h(a_n))\in R$ 
    Suppose $\varphi$ defines $R$ in $\mathcal{A}$ we want
    $\mathcal{A}\models \varphi[a]$ iff  $\mathcal{A}\models \varphi[h(a_1),\dots h(a_n)]$ which is true by the homom. thm.
}
\note{}{Corol can be used to show that some $R\subseteq A^n$ is not definable in $\mathcal{A}$
}
\bsp{}{
    $\mathcal{R}=(\RR,<)$ then $\NN$ is not definable in $\mathcal{R}$.
    What do automorphisms of $\mathcal{R}$ look right? 
    $h:\RR\to \RR$ is a bijection and $x<y$ iff $h(x)<h(y)$ so $h$ is strictly increasing.
    for example $x\mapsto x+ \frac{1}{2}$ or $x\mapsto x^3$.  
}
\section{Unique readability for terms}
\defin{}{
    We define $K$ on symbols from which terms are built up(variables, constants, function symbols).
    $K(s) = 1-n$ where $s$ is a symbol and $n$ is the number of terms that need to follow s in order to obtain a term.
    $K(x) = 1 = K(c)$ and $K(f) = 1-n$ where f is an $n$-ary funciton symbol
    We now extend $K$ to the set of all expressions which are built up from above symbols (variables, constants, function symbols):
    $K(s_1,\dots s_n) = K(s_1) + \dots + K(s_n)$ (unique because no symbol is a finite sequence of other symbols)
}
\lemma{}{
    $t$ a term then $K(t)=1$
}{
    $K(x) = 1 = K(c)$ and $K(ft_1,\dots t_n) = 1-n + n = 1$ 
}
\defin{}{A terminal segment of string of symbols $(s_1,\dots s_n)$ is $(s_k,s_{k+1},\dots s_n)$ for some $1\leq k\leq n$.}
\lemma{}{Any terminal segment of terms is a concatenation of one or more terms.}{
    True for variables and constants. 
    $ft_1\dots t_n$ the only non trivial case is $t'_k t_{k+1}\dots t_m$ where $t_k$ is $t''_k t_k'$
}
\coroll{If $t_1$ is a proper initial segment of a term $t$ then 
 its $K(t_1)<1$. proof:
    let $t$ be $t_1 t_2$ where $t_1$ is a proper initial segment then $K(t) = 1$ and $K(t_2)\geq1$ therefore $K(t_1)\leq 0$
}
\subsubsection{Unique readability for terms}
The set of terms is freely generated from the set of variables(Var), the set of constant symbols (Const) by the term building operations $\mathcal{F}_f$ for the function symbols $f$.
\begin{proof}
    
     \begin{itemize}
        \item disjointment of ranges: Let $f$ and $g$ be two distinct funciton symbols then $\rng\mathcal{F}_f \cap \rng\mathcal{F}_g =\varnothing$
        $\rng\mathcal{F}_f \cap Var =\varnothing$
        $\rng\mathcal{F}_f \cap Const =\varnothing$
        \item $\mathcal{F}_f|_{\text{terms}}$ are 1-1:
        assume $ft_1\dots t_n = f t'_1\dots t'_n$ and assume $t_1 \neq t_1'$ then one is an initial segment of the other.
        Then its $K$-value has to be less than $1$ so it is not a term.
        $t_1 = t_1' \dots t_n=t_n'$.
     \end{itemize}
\end{proof}
\defin{}{Extend $K$ as follows: $K(() = -1$ $K()) = 1$ $K(\forall) = 1$ $K(\lnot) = 0$ $K(\to) = -1$ $K(P) = 1-n$ for an $n$-ary rel. symb. $P$. $K(=) = -1$.
    Extend $K$ to the set of all expressions by $K(s_1,\dots s_n) = K(s_1) + K(s_n)$
    The idea is that $K$ tells us the number of symbols that at least need to follow to obtain a formula.
}
\lemma{}{
    for every formula $\varphi$ it is $K(\varphi) = 1$
}{
    induction on $\varphi$
}
\lemma{}{for every proper initial segment $\alpha'$ of a fla. $\alpha$ we have $K(\alpha')<1$}{}
\coroll{No proper initial segment of a fla. is a fla.}
The set of flas. is freely generated from the set of atomic flas. by operations $\mathcal{E}_\lnot,\mathcal{E}_\to,Q_i$
\begin{proof}
    \begin{itemize}
        \item $\mathcal{E}_\lnot,Q_i$ are one to one
        \item $\mathcal{E}_\to|_{\text{Flas.}}$ then itemwise and use of prev. lemmas
        \item p.w. disjointness of ranges
    \end{itemize}
\end{proof}
\section{A parsing algorithm for first order logic}

\section{Deductions (formal proofs)}
\section{Generalization and deduction theorem}
TODO evt noch sectioons
