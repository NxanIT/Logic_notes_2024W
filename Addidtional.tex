\appendix
\chapter{Appendix}
\setcounter{section}{2}
\section{On Model theory}
\thm{Löwenheim-Skolem}{
    Let $\mathcal{L}$ be a language of cardinality $\lambda$. $\Gamma$ a set of formulas and $\Sigma$ a set of sentences.
    \begin{enumerate}[label = (\roman*)]
        \item If $\Gamma$ is satisfiable, then it is satisfiable in some structure of cardinality at most $\lambda$
        \item If $\Sigma$ has any model, then it has a model of cardinality at most $\lambda$.
    \end{enumerate}
}{
    by using the LST theorem.

}

\section{On Boolean Algebras}
\defin{lattice}{%wikipedia
    A \imp{lattice} is a set $L$ with two binary, commutative and associative operations $\lor, \land$ satisfying the absorbtion axioms.
    \[\begin{aligned}
        &\forall a\forall b \: a\land (a\lor b) = a \\
        &\forall a \forall b \:  a \lor (a\land b) = a
    \end{aligned}\]
    A lattice is called \subimp{distributive}, if the distributive axioms hold.\\
    \[\begin{aligned}
        &\forall a\forall b\forall c \: a \land (b\lor c) = (a\land b)\lor (a\land c)\\
        &\forall a \forall b \forall c \: a \lor (b\land c) = (a\lor b)\land (a\lor c)\\
    \end{aligned}\]
    A lattice is called \subimp{bounded}, if it has a least element $0$ and a greatest element $1$.\\
    \[\begin{aligned}
        &\exists 0\forall a\: a \lor 0 = a\\
        &\forall 1 \forall a\: 1 \lor a = 1\\
    \end{aligned}\]
    A lattice is called \subimp{complemented}, if it is bounded and every element $a$ 
    has a complement $b$\\
    satisfying $a\lor b = 1$ and $a\land b = 0$.
    \[\forall a \exists b \: a\lor b = 1 \text{ and } a\land b = 0\]

}
\defin{Alternative Def: Boolean Algebra}{%from last year
    A \graybf{boolean algebra} is a set $B$ with
    \begin{itemize}
        \item distinguished elements $0,1$ (called zero and unit of $B$)
        \item a unary operation $'$ on $B$ (called \graybf{complementation})
        \item two binary operations $\lor$ called \graybf{join} and $\land$ called \graybf{meet} s.t. for all $x,y,z \in B$ 
        \begin{enumerate}[label=(\roman*)]
            \item $x\lor 0 = x$ \qquad  $x\land 1 = x$
            \item $x\lor x' = 1$ \qquad   $x\land x' = 0$
            \item $x \lor y = y \lor x$ \qquad   $x\land y = y\land x$
            \item $(x\lor y)\lor z = x\lor (y\lor z)$ \qquad   $(x\land y)\land z = x\land (y\land z)$
            \item $x\lor (y\land z) = (x\lor y)\land (x\lor z)$\qquad    $x\land (y\lor z) = (x\land y)\lor (x\land z)$
        \end{enumerate}
    \end{itemize}
}
With this definition a boolean algebra is exactly a complemented distributive lattice closed under the additional complementation map. The definition is compatible with the definition given in the chapter of boolean algebras.
However we must be careful, a subalgebra of a Boolean algebra must again be closed under the restricted complementation map.\footnote{See \url{https://math.nmsu.edu/people/personal-pages/files/ESSLLI2.pdf} on slide 7 for example.}
\bsp{}{Let $X\neq \varnothing$ be a set, $B \defeq \mathcal{P}(X)$ the power set of $X$, $0\defeq \varnothing$ and $1\defeq S$, 
    $$': \mathcal{P}(S)\to \mathcal{P}(S), x' \defeq S\backslash x \qquad x\lor y \defeq x\cup y, \quad x\land y \defeq x\cap y \text{ for } x,y\in \mathcal{P}(S)$$
}

\lemma{}{ Let $(B,',\lor,\land,0,1)$ be a boolean algebra. Then it holds
    \begin{enumerate}[label=\alph*)]
        \item $0' = 1$, $1' = 0$
        \item $x\lor x = x$, $x\land x = x$
        \item $(x')'= x$
        \item $(x\lor y)' = x' \land y'$, $(x\land y)' = x' \lor y'$
        \item $x\lor y = y \text{ iff } x\land y = x$
    \end{enumerate}
}{}
\lemma{}{
    \begin{enumerate}[label=\alph*)]
        \item $x\leq y \defaq x\lor y = y$ defines a partial ordering on $B$ (inclusion) and it holds
        \item $x\lor y$ is the least upper bound of $\{x,y\}$ in $B$\\
            $x\land y$ is the greatest lower bound of  $\{x,y\}$ in $B$
        \item $0\leq x\leq 1$ for all $x\in B$
    \end{enumerate}
}{}

\defin{Opposite of boolean algebra}{Let $(B,',\lor,\land,0,1)$ be a boolean algebra. The boolean algebra $B^{\text{op}}$ is defined by
    $$B^\text{op}\defeq B,\quad 0^\text{op} \defeq 1,\quad 1^\text{op} \defeq 0,\quad' \text{ stayes the same as for} B,\quad\lor^\text{op} \defeq \land,\quad\land^\text{op} \defeq \lor$$
    Note: $(B^\text{op})^\text{op} = B$
}
\defin{Subalgebra}{A \graybf{subalgebra} of $B$ is a subset $A\subseteq B$ s.t. $0,1\in A$ and $A$ is closed under $',\land,\lor$.
    The subalgebra generated by $P\subseteq B$ is defined to be the smallest subalgebra containing $P$. Equivalently it is the 
    intersection of all Subalgebras of $B$ that contain $P$.
}
\bsp{Power set algebra}{Let $S$ be a set then $\mathcal{P}(S)$ defines a boolean algebra on $S$.
    $B \defeq \{x\in \mathcal{P}(S): x \text{is finite or cofinite}\}$ is a subalgebra of $\mathcal{P}(S)$
    w/ set of generators $\{\{s\}:s\in S\}$}
\note{}{We will prove the Tarski-Stone Theorem: every boolean algebra is isomorphic to an algebra on a set.}

\bsp{Lindenbaum Algebra of $\Sigma$}{
    Let $A$ be a set of prop. atoms, $\propM(A)$ the set of prop. generated by $A$.
    Further let $\Sigma \subseteq \propM(A)$ and $p,q,r$ range over $\propM(A)$.\\
    We say $p$ is $\Sigma$-equivalent to $q$ iff $\Sigma \models_\text{taut} p\leftrightarrow q$
    $\Sigma$-Equivalence is an equivalent relation on $\propM(A)$ and $\propM(A)/\Sigma$ is a boolean algebra with
    $$0\defeq \bot/\Sigma,\quad1\defeq \top/\Sigma,\quad(p/\Sigma)' \defeq (\lnot p)/ \Sigma,\quad(p/\Sigma \lor q/ \Sigma)\defeq (p\lor q)/ \Sigma,\quad(p/\Sigma \land q/ \Sigma)\defeq (p\land q)/ \Sigma$$
    a set of generators is $\{a/\Sigma : a\in A\}$
}
\defin{Homomorphisms of boolean algebras}{Let $B,C$ be boolean algebras. A map $\phi: B\to C$ is a (homo)morphism of boolean algebras iff
    $\forall x,y\in B$ it holds
    \begin{itemize}
        \item $\phi(0_B) = 0_C$
        \item $\phi(x') = \phi(x)'$
        \item $\phi(x\lor y) = \phi(x)\lor \phi(y)$
        \item $\phi(x\land y) = \phi(x)\land \phi(y)$
    \end{itemize}
    If $\phi:B\to C$ is bijective too , we call $\phi$ an isomorphism and $\phi^{-1}:C\to B$ is also a morphism of boolean algebras.
}
\note{}{$\phi(B)$ is subalgebra of $C$}
\bsp{}{Let $S,T$ be sets then a function $f:S\to T$ induces a morphism of boolean algebras $\mathcal{P}(T)\to \mathcal{P}(S): y\mapsto f^{-1}(y)$
If $S\subseteq T$ and $f$ the inclusion map $S\hookrightarrow T$ then we get a boolean algebra morphism $Y\to Y\cap S$.\\
    \begin{itemize*}
        \item $id_B: B\to B$ \qquad 
        \item $x\mapsto x': B\to B^{\text{op}}$ are both isomorphism
    \end{itemize*}
}
\note{}{A boolean algebra morphism $\phi: B\to C$ is injective iff $\ker f = 0_B$}
\lemma{}{\label{boolLemma}
    Let $X_1,\dots X_m\subseteq S$ and $\mathcal{A}$ a boolean algebra on $S$ generated by $\{X_1,\dots X_m\}$. Then $\mathcal{A}$ 
    is finite and isomorphic to $\mathcal{P}(\{1,2,\dots n\})$ for some $n\leq 2^m$.
}{
    TODO
}
\defin{Trivial algebras}{\begin{itemize}
    \item $B$ is trivial if $|B| = 1$ (equivalently $0=1\in B$) 
    according to \ref{boolLemma} $B$ is isomorphic to $\mathcal{P}(\varnothing)$
    \item If $|S|=1$ then $|\mathcal{P}(S)| = 2$ 
    TODO
\end{itemize}}
\defin{Ideal}{An ideal of $B$ is a subset of $I\subseteq B$ s.t.
    \begin{itemize}
        \item[(I1)] $0\in I$
        \item[(I2)] $\forall a,b \in B$ it holds \qquad 
            $a\leq b$ and $b \in I\implies a\in I$\qquad and \qquad $a,b\in I\implies a\lor b\in I$ 
    \end{itemize}
}
\bsp{}{$F_{\text{in}} = \{F\subseteq S: F \text{ finite}\}$
    is ideal in $\mathcal{P}(S)$.
}
\note{}{If $I$ is an ideal of $B$ then 
    $I\lor b \defeq \{x\in B: x = a\lor b \text{ for some } a \in I\}$ is the smallest ideal w/ respect of $\subseteq$ of $B$ that contains $I\cup \{b\}$.
}
\bsp{}{\begin{itemize}
\item For a boolean algebra morphism $\phi: B\to C$ the kernel $\ker(\phi)$ is an ideal in $B$.
\item If $I$ is an ideal in $B$ then $a =_I b \defaq a\lor x = b\lor x$ for some $x\in I$ defines an equivalent relation and
$B/_{=_I}$ is a boolean algebra w/ 
$$0\defeq 0/_{=_I}\quad 1\defeq 1/_{=_I}\quad (a/_{=_I})' \defeq a'/_{=_I}\quad a/_{=_I}\lor b/_{=_I} \defeq (a\lor b)/_{=_I}\quad a/_{=_I}\land b/_{=_I} \defeq (a\land b)/_{=_I}$$
Then $\phi: B\to B/_{=_I}: b\mapsto b/_{=_I}$ is a boolean algebra morphism w/ $\ker(\phi)=I$
\end{itemize}}
\newpage
\subsection{Notes on Stone spaces}\label{Appendix:Stone}
\defin{topological properties}{\label{Appendix:def:top}
    Let $X$ be a \imp{topological space}. $X$ is said to 
    \begin{enumerate}[label=(\roman*)]
        \item be \subimp{compact}, if 
        every open cover of $X$ has a finite subcover. A subset $K\subseteq X$ of a topological space is called compact if it is a compact subspace of $X$.
        \item be a \subimp{T0-space} or equivalently hausdorff, if
        \[\forall x \forall y \: x\neq y \to \exists U,V\in \tau \: (x\in U\land y \notin U)\lor (x\notin V \land y\in V)\]
        \item be a \subimp{T2-space}, if 
        \[\forall x \forall y \: x\neq y \to \exists U,V\in \tau \: (x\in U\land y\in V \land U\cap V = \varnothing)\]
        \item be \subimp{totally seperated}, if 
        \[\forall x \forall y \: x\neq y \to \exists U,V\in \tau \: (x\in U\land y\in V \land U\cap V = \varnothing \land U\cup V = X)\]
        \item be \subimp{zero-dimensional} with respect to the \underline{small inductive dimension}, if it has a base for the topology consisting of clopen sets.
        \[\forall U \in \tau \,\forall x\in U \, \exists W\in \tau \: (x\in W\subseteq U\land W^c\in \tau)\]
        \item\label{Appendix:Top:irred} be \subimp{irreducible}, if one of the equivalent conditions below is satisfied
        \footnote{from: \url{https://en.wikipedia.org/wiki/Hyperconnected_space}}
        \begin{enumerate}
            \item No two nonempty open sets are disjoint.
            \item $X$ cannot be written as the union of two proper closed subsets. Proper means that none of the sets are equal to $X$ or the empty set.
            \item Every nonempty open set is dense in $X$.
            \item The interior of every proper closed subset of $X$ is empty.
            \item Every subset is dense or nowhere dense in $X$.
            \item\label{Appendix:Top:irred:cond6} No two points can be separated by disjoint neighbourhoods.
        \end{enumerate}
        An irreducible set is a subset of a topological space for which the subspace topology is irreducible.
        \item have a \subimp{generic point}, if there is a singleton $\{p\}\subseteq X$, whose closure is $X$.
        A subset $Z\subseteq$ of a topological space is said to have a gerneric point, if
        \[\exists! p\in Z \: \overline{\{p\}} = Z\]%TODO check uniqueness
        \item be \subimp{sober}, if every nonempty irreducible closed subset of $X$ has a (necessarily unique) generic point.
        \[\exists p\in X \: \overline{\{p\}}=X\]%TODO check if needs to be unique
        \item be \subimp{coherent}, if all of the following conditions are satisfied.\\
        Let $K^\circ(X) = \{V \subseteq X : V\in \tau \land V \text{ is compact}\}$
        \begin{enumerate}
            \item $X$ is compact, T0 and sober
            \item $K^\circ(X)$ is a basis for the topology
            \item $K^\circ(X)$ is closed under finite intersections
        \end{enumerate}
    \end{enumerate}

}
\newpage
\lemma{}{
    The definitions in \ref{Appendix:def:top} \ref{Appendix:Top:irred} are indeed equivalent.
}{  $(c)\implies(d)$, $(d)\implies(e)$ and $(f)\implies(a)$ are left as an exercise.
    \begin{enumerate}[leftmargin=2.1cm]
        \item[$(a)\implies(b)$] Suppose $X$ can be written es the union of two proper closed sets. 
        Then $X = A \cup B$ with $A,B\neq \varnothing$ and $A,B\neq X$. Then $A^c,B^c$ are open, nonempty and disjoint ($A^c\cap B^c = (A\cup B)^c = \varnothing$).
        \item[$(b)\implies(c)$] Suppose there is a nonempty open set $U$ with $\overline{U} \neq X$.
        Take $A \defeq U^c$ and $B \defeq \overline{U}$. We have $\varnothing \neq \overline{U}^c\subseteq U^c \neq X$ and $\varnothing \neq U\subseteq \overline{U} \neq X$ so $A,B$ are proper closed sets with $X = A \cup B$.
        \item[$(e)\implies(f)$] Suppose $(f)$ fails, then there exist $x,y\in X$ with $x\neq y$ and 
        there exist neighbourhoods $U_x\in \mathcal{U}(x)$ and $U_y\in \mathcal{U}(y)$ with 
        $U_x\cap U_y = \varnothing$. By definition there exist $V_x, V_y\in \tau$ with 
        $x\in V_x\subseteq U_x$ and 
        $y\in V_y\subseteq U_y$. Hence $x\in V_x^\circ = V_x$ and 
        $y\notin \overline{V_x}\subseteq V_y^c$ and $(e)$ fails too.
    \end{enumerate}
    \vspace{-0.6cm}
}
\defin{Stone space}{
    A Stone space is a zero-dimensional, compact, hausdorff topological space $X$.
}
There are other equivalent equivalent definitions of Stone spaces in other mathematical works.
\prop{}{The following statements are equivalent:
    \begin{enumerate}[label=(\roman*)]
        \item $X$ is a Stone space
        \item $X$ is compact and totally seperated
        \item $X$ is compact, T0 and zero-dimensional
        \item $X$ is T2 and coherent
    \end{enumerate}
}{
    We show $(\romanNum{1})\Longleftrightarrow (\romanNum{2})$, $(\romanNum{3})\implies (\romanNum{1})$, $(\romanNum{1})\implies (\romanNum{4})$, $(\romanNum{4})\implies (\romanNum{3})$
    \begin{enumerate}[leftmargin=2cm]
        \item[$(\romanNum{1})\implies(\romanNum{2})$:] We just have to show that $X$ is totally seperated. Let $x,y\in X$ with $x\neq y$. Since $X$ is hausdorff there exists a open set $U$ such that $x\in U$ and $y\notin U$. For this $U$ there exists a $W\subset U$ with $x\in W$ which is clopen. Therefore $W^c$ is clopen too and $y\in U^c\subseteq W^c$, $W^c \cup W = X$.
        
        \item[$(\romanNum{1})\impliedby(\romanNum{2})$:] T2 follows from totally seperatedness, so
        let $U$ be an open set and $x\in U$. We have to show that there exists a clopen subset $W\subseteq U$ with $x\in W$. We define $V_y$ to be the clopen set $V$ we get from totally seperatedness with respect to $x$ and $y$. From $y\in V_y$ we get that
        $\{U\}\cup \{V_y : y\in U^c\}$ is an open cover of $X$, which is compact so there is a natural number $n$ and $y_1, \dots y_n\in U^c$ with $U\cup \bigcup_{i\leq n} V_{y_i} = X$,
        hence $\bigcup_{i\leq n} V_{y_i}\supseteq U^c$ and therefore $\bigcap_{i\leq n} V_{y_i}^c\subseteq U$.
        $W\defeq \bigcap_{i\leq n} V_{y_i}^c$ is clopen, since finite intersections of clopen sets are clopen. Furthermore for every $i\leq n$ we have $x\in V_{y_i}^c$, so $x\in W$.

        \item[$(\romanNum{3})\implies (\romanNum{1})$:] Let $x,y\in X$ with $x\neq y$. Since $X$ is T0,
        There exists open sets $U,V$ with \\
        $x\in U\land y \notin U$ or $x\notin V \land y\in V$
        Assuming $x\in U\land y \notin U$ does not hold, then $x\notin V \land y\in V$.
        Since $X$ is zero-dimensional with respect to the small inductive dimension, there exists 
        a clopen set $W\subseteq V$ with $y\in W$. Therefore $W^c\in \tau$, $x\in W^c$ and we have shown that $X$ is a T2-space.

        \item[$(\romanNum{1})\implies (\romanNum{4})$:] 
        \begin{enumerate}
            \item compact and T2 are clear, we have to show that $X$ is sober.
            Let $Z\subseteq X$ be a nonempty irreducible closed subset. Subspaces of T2 are T2, and 
            since $Z$ is irreducible, using condition \ref{Appendix:Top:irred:cond6}
            ``No two points can be separated by disjoint neighbourhoods.'' we get that $Z$ has to be a singleton, and for T2-spaces, singletons have a unique generic point, hence $X$ is sober.
            \item Let $\sigma$ be the basis for the topology $\tau$ consisting of clopen sets. Since $X$ is T2, the closed sets are exactly the compact sets. Therefore
            $\sigma\subseteq K^\circ(X)\subseteq \tau$ and $K^\circ(X)$ is a basis for the topology.
            \item Let $n\in \NN$, $A_1,\dots A_n\in K^\circ(X)$. Like above, all $A_i$ are closed, hence
            clopen. The finite intersection of clopen sets $\bigcap_{i\leq n}A_i$ is clopen and therefore open and compact and we have $\bigcap_{i\leq n}A_i\in K^\circ(X)$
        \end{enumerate}
        \item[$(\romanNum{4})\implies (\romanNum{3})$:]
        Compact and T0 are given by definition. $K^\circ(X)$ is a basis for the topology and since $X$ is T2, every set of $K^\circ(X)$ is closed, hence $X$ is zero-dimensional.
    \end{enumerate}
    \vspace{-0.7cm}
}
Note that in the previous lemma $(\romanNum{1})\implies (\romanNum{3})$ is for free.

\lemma{}{
    Let $\mathcal{B}$ be a boolean Algebra and $U\subseteq B$. 
    Then $U$ is a ultrafilter on $\mathcal{B}$ if and only if $h:B\to \{0,1\}$, $h(x) \defeq\raisebox{\depth}{$\chi$}_U(x)$ is a homeomorphism from $\mathcal{B}$ to the two-element boolean algebra.
}{
    Let $x,y\in B$. 
    \begin{itemize}
        \item[$\implies$:] Clearly $x+y\geq x$ and $x+y\geq y$. Suppose $h(x)+h(y)=1$ then $x\in U$ or $y\in U$, by (F3) $h(x+y) = 1$.
        Otherwise $x,y\notin U$. By (UF) we have $\overline{x},\overline{y}\in U$.
        By (F2) we have $\overline{x}\cdot \overline{y}\in U$, but $\overline{x}\cdot \overline{y} = \overline{x+y}$.
        By (F1) we have $x+y\notin U$.

        If $h(x)\cdot h(y) = 1$ then $x,y\in U$, by (F2) $x\cdot y\in U$ and $h(x\cdot y) = 1$.
        Similarly $x\cdot y \leq x,y$, so if $h(x\cdot y) = 1$ then $x\cdot y\in U$ and by (F3) $h(x)\cdot h(y) = 1$.

        $h(0) = 0$ follows from (F1).

        $h(1) = 1$ because $U\neq \varnothing$ and (F3).

        $\overline{h(x)} = h(\overline{x})$ by (UF) and (F1). 
        Suppose $1 = \overline{h(x)}$ then $x\notin U$ by (UF) we have $\overline{x}\in U$, so $h(\overline{x})=1$
        Suppose $1=h(\overline{x})$ then $\overline{x}\in U$ by (F1) $x\notin U$ and therefore $\overline{h(x)}=1$

        \item[$\impliedby$:] (F1): Suppose $0\in U$ then $h(0) = 1 \neq 0$.
        
        (F2): Let $x,y\in U$ then $1 = h(x)\cdot h(y) = h(x\cdot y)$, so $x\cdot y\in U$.

        (F3): Let $x\in U$ and $y\in B$ with $x\leq y$. Then by definition $x+y = y$, hence
        $h(y) = h(x+y) = h(x+y) = h(x)+ h(y) = 1+h(y) = 1$.
        
        (UF): Suppose there is a $x\in B$ with $x,\overline{x}\notin U$ then 
        $0 = h(x) + h(\overline{x}) = h(x+\overline{x}) = h(1) = 1$.
    \end{itemize}
}


Existence and uniqueness of a topology generated by a subset of the power set
\thm{}{\label{Appendix:Top:BasisThm}
    Let $X$ be a set, $\sigma\subseteq \mathcal{P}(X)$. Then it is equivalent:
    \begin{enumerate}
        \item $\sigma$ is a basis for a uniquely determined topology $\tau\supseteq \sigma$.
        \item $X = \bigcup_{B\in \sigma}B $ and 
        \begin{flalign*}
            \forall B_1, B_2\in \sigma \,\forall p\in B_1\cap B_2\, 
            \exists B_3\in \sigma\: (p\in B_3\subseteq B_1\cap B_2)&&
        \end{flalign*}
        
        
    \end{enumerate}
}{}


\thm{}{\label{Appendix:Thm:Stone}
    \begin{enumerate}[label=(\roman*)]
        \item If $\mathcal{B}\models BA$, then $S(\mathcal{B})$ is a Stone-space
        \item If $\mathcal{S}$ is a Stone space then the clopen subsets of $\mathcal{S}$ form a boolean algebra of sets denoted by $B(\mathcal{S})$.
        \item Every boolean algebra $\mathcal{B}$ is isomorphic to the boolean algebra $B(S(\mathcal{B}))$ with $a\mapsto \langle a\rangle$. Hence $\mathcal{B}$ is isomorphic to a subalgebra of the boolean algebra $\mathcal{P}(S(\mathcal{B}))$ of sets
        \item Every Stone space $\mathcal{S}$ is homeomorphic to the Stone space $S(B(\mathcal{S}))$
        $$x\mapsto \{a\in S(\mathcal{B}) : x\in a\}$$
    \end{enumerate}
}{ (\romannumeral 1) is proven in \ref{Bool:Thm:Stone}
    \begin{enumerate}[label=(\roman*)]
        \stepcounter{enumi} 
        \item Let $B(S) = \{V\subseteq  S : V \in \tau \land V^c\in \tau\}$
        The functions on a boolean algebra of sets are defined as in \ref{Bool:Bsp:AlgOfSets}. Hence we can, for simplicity write $\cap$ instead of $\cdot$ \dots.

        Clearly clopen sets are closed under finite operations of $\cup$, $\cap$, $\overline{\phantom{x}}$.
        Checking the axioms of boolean algebras
        \begin{equation*}
            \begin{matrix*}[l]
                \forall x,y,z \: \bigl(x+(y+z) = (x+y)+z \land x\cdot (y\cdot z) = (x \cdot y ) \cdot z\bigr) & \text{(Associativity $+,\cdot$)}\\[3pt]
                \forall x,y \: \bigl(x+y=y+x \land x\cdot y = y\cdot x\bigr) & \text{(Commutativity of $+,\cdot$) }\\[3pt]
                \forall x \: \bigl(x+x = x \land x\cdot x = x\bigr)& \text{(Idempotence) }\\[3pt]
                \forall x,y,z \: \bigl(x\cdot (y+z) = x\cdot y + x\cdot z \land x+(y\cdot z) = (x+y)\cdot (x+z)\bigr) & \text{(Distributivity) }\\[3pt]
                \forall x,y \: \bigl(x\cdot (x+ y) = x  = x+ (x\cdot y) \bigr)& \text{(Absorbtion)}\\[3pt]
                \forall x,y \: \bigl(\overline{x+y} = \overline{x}\cdot \overline{y}\land \overline{x\cdot y} = \overline{x}+ \overline{y}\bigr)& \text{(De Morgan's Laws) }\\[3pt]
                \forall x \: \bigl( x+0 = x \land x\cdot 0 = 0 \bigr)& \text{(Laws of $0$)}\\[3pt]
                \forall x \: \bigl( x+1 = 1\land x\cdot 1 = x \bigr)& \text{(Laws of $1$)}\\[3pt]
                \forall x \: \bigl(x + \overline{x} = 1 \land x\cdot \overline{x} = 0 \land \overline{\overline{x}} = x\bigr)& \text{(Laws of $\overline{\phantom{x}}$)}\\
            \end{matrix*}
        \end{equation*}
        reveals that they imediately follow from the properties of $\cup$, $\cap$ and $\overline{\phantom{x}}$.
        \item Let $\mathcal{B}$ be a boolean algebra, $C\defeq B(S(\mathcal{B}))$ like above and $h: B\to C$, $h(a) = \langle a \rangle$.
        \begin{flalign*}
            & h(0_B) = \langle 0 \rangle = \varnothing \eqdef 0_C & \\
            & h(a)^c = \langle a \rangle ^c = \langle \overline{a} \rangle = h(\overline{a}) & \\
            & h(a\cdot b) = \langle a\cdot b\rangle = \langle a\rangle \cap \langle b\rangle & \\
            & h(a+b) = h(\overline{a}\cdot \overline{b})^c = (h(\overline{a})\cap h(\overline{b}))^c = h(\overline{a})\cup h(\overline{b}) &
        \end{flalign*}
        Hence $h$ is a homomorphism of boolean algebras.
        \item Let $\tau$ be the Stone topology on $S$. 
        \[S(B(S)) = \{F\subseteq B(S) : \: F \text{ is ultrafilter on }B(S)\}\]
        Since $B(S)$ is a boolean algebra, we have $S(B(S))$ is a stone space.
        Let $\tau'$ be the Stone topology on $S(B(S))$. Let $h: S\to S(B(S))$, $h(x) = \{a\in S(\mathcal{B}) : x\in a\}$.

        \textbf{Claim: } $h$ is a bijection.
        \begin{claimproof}
            Let $x,y\in S$ with $ h(x) = h(y) $. 
            Therefore
            \[\{a\in S(\mathcal{B}) : x\in a\} = \{a\in S(\mathcal{B}) : y\in a\}\]
            
        \end{claimproof}
        \textbf{Claim: }
        \begin{claimproof}
            
        \end{claimproof}
    \end{enumerate}
}

\newpage
