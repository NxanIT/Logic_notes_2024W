\appendix
\chapter{Appendix}
\setcounter{section}{2}
\section{On Model theory}
\thm{Löwenheim-Skolem}{
    Let $\mathcal{L}$ be a language of cardinality $\lambda$. $\Gamma$ a set of formulas and $\Sigma$ a set of sentences.
    \begin{enumerate}[label = (\roman*)]
        \item If $\Gamma$ is satisfiable, then it is satisfiable in some structure of cardinality at most $\lambda$
        \item If $\Sigma$ has any model, then it has a model of cardinality at most $\lambda$.
    \end{enumerate}
}{
    by using the LST theorem.

}

\section{On Boolean Algebras}
\defin{lattice}{%wikipedia
    A \imp{lattice} is a set $L$ with two binary, commutative and associative operations $\lor, \land$ satisfying the absorbtion axioms.
    \[\begin{aligned}
        &\forall a\forall b \: a\land (a\lor b) = a \\
        &\forall a \forall b \:  a \lor (a\land b) = a
    \end{aligned}\]
    A lattice is called \subimp{distributive}, if the distributive axioms hold.\\
    \[\begin{aligned}
        &\forall a\forall b\forall c \: a \land (b\lor c) = (a\land b)\lor (a\land c)\\
        &\forall a \forall b \forall c \: a \lor (b\land c) = (a\lor b)\land (a\lor c)\\
    \end{aligned}\]
    A lattice is called \subimp{bounded}, if it has a least element $0$ and a greatest element $1$.\\
    \[\begin{aligned}
        &\exists 0\forall a\: a \lor 0 = a\\
        &\forall 1 \forall a\: 1 \lor a = 1\\
    \end{aligned}\]
    A lattice is called \subimp{complemented}, if it is bounded and every element $a$ 
    has a complement $b$\\
    satisfying $a\lor b = 1$ and $a\land b = 0$.
    \[\forall a \exists b \: a\lor b = 1 \text{ and } a\land b = 0\]

}
\defin{Alternative Def: Boolean Algebra}{%from last year
    A \graybf{boolean algebra} is a set $B$ with
    \begin{itemize}
        \item distinguished elements $0,1$ (called zero and unit of $B$)
        \item a unary operation $'$ on $B$ (called \graybf{complementation})
        \item two binary operations $\lor$ called \graybf{join} and $\land$ called \graybf{meet} s.t. for all $x,y,z \in B$ 
        \begin{enumerate}[label=(\roman*)]
            \item $x\lor 0 = x$ \qquad  $x\land 1 = x$
            \item $x\lor x' = 1$ \qquad   $x\land x' = 0$
            \item $x \lor y = y \lor x$ \qquad   $x\land y = y\land x$
            \item $(x\lor y)\lor z = x\lor (y\lor z)$ \qquad   $(x\land y)\land z = x\land (y\land z)$
            \item $x\lor (y\land z) = (x\lor y)\land (x\lor z)$\qquad    $x\land (y\lor z) = (x\land y)\lor (x\land z)$
        \end{enumerate}
    \end{itemize}
}
With this definition a boolean algebra is exactly a complemented distributive lattice closed under the additional complementation map. The definition is compatible with the definition given in the chapter of boolean algebras.
However we must be careful, a subalgebra of a Boolean algebra must again be closed under the restricted complementation map.\footnote{See \url{https://math.nmsu.edu/people/personal-pages/files/ESSLLI2.pdf} on slide 7 for example.}
\bsp{}{Let $X\neq \varnothing$ be a set, $B \defeq \mathcal{P}(X)$ the power set of $X$, $0\defeq \varnothing$ and $1\defeq S$, 
    $$': \mathcal{P}(S)\to \mathcal{P}(S), x' \defeq S\backslash x \qquad x\lor y \defeq x\cup y, \quad x\land y \defeq x\cap y \text{ for } x,y\in \mathcal{P}(S)$$
}

\lemma{}{ Let $(B,',\lor,\land,0,1)$ be a boolean algebra. Then it holds
    \begin{enumerate}[label=\alph*)]
        \item $0' = 1$, $1' = 0$
        \item $x\lor x = x$, $x\land x = x$
        \item $(x')'= x$
        \item $(x\lor y)' = x' \land y'$, $(x\land y)' = x' \lor y'$
        \item $x\lor y = y \text{ iff } x\land y = x$
    \end{enumerate}
}{}
\lemma{}{
    \begin{enumerate}[label=\alph*)]
        \item $x\leq y \defaq x\lor y = y$ defines a partial ordering on $B$ (inclusion) and it holds
        \item $x\lor y$ is the least upper bound of $\{x,y\}$ in $B$\\
            $x\land y$ is the greatest lower bound of  $\{x,y\}$ in $B$
        \item $0\leq x\leq 1$ for all $x\in B$
    \end{enumerate}
}{}
\note{}{A boolean algebra is a complemented distributive lattice.}
\defin{Opposite of boolean algebra}{Let $(B,',\lor,\land,0,1)$ be a boolean algebra. The boolean algebra $B^{\text{op}}$ is defined by
    $$B^\text{op}\defeq B,\quad 0^\text{op} \defeq 1,\quad 1^\text{op} \defeq 0,\quad' \text{ stayes the same as for} B,\quad\lor^\text{op} \defeq \land,\quad\land^\text{op} \defeq \lor$$
    Note: $(B^\text{op})^\text{op} = B$
}
\defin{Subalgebra}{A \graybf{subalgebra} of $B$ is a subset $A\subseteq B$ s.t. $0,1\in A$ and $A$ is closed under $',\land,\lor$.
    The subalgebra generated by $P\subseteq B$ is defined to be the smallest subalgebra containing $P$. Equivalently it is the 
    intersection of all Subalgebras of $B$ that contain $P$.
}
\bsp{Power set algebra}{Let $S$ be a set then $\mathcal{P}(S)$ defines a boolean algebra on $S$.
    $B \defeq \{x\in \mathcal{P}(S): x \text{is finite or cofinite}\}$ is a subalgebra of $\mathcal{P}(S)$
    w/ set of generators $\{\{s\}:s\in S\}$}
\note{}{We will prove the Tarski-Stone Theorem: every boolean algebra is isomorphic to an algebra on a set.}

\bsp{Lindenbaum Algebra of $\Sigma$}{
    Let $A$ be a set of prop. atoms, $\propM(A)$ the set of prop. generated by $A$.
    Further let $\Sigma \subseteq \propM(A)$ and $p,q,r$ range over $\propM(A)$.\\
    We say $p$ is $\Sigma$-equivalent to $q$ iff $\Sigma \models_\text{taut} p\leftrightarrow q$
    $\Sigma$-Equivalence is an equivalent relation on $\propM(A)$ and $\propM(A)/\Sigma$ is a boolean algebra with
    $$0\defeq \bot/\Sigma,\quad1\defeq \top/\Sigma,\quad(p/\Sigma)' \defeq (\lnot p)/ \Sigma,\quad(p/\Sigma \lor q/ \Sigma)\defeq (p\lor q)/ \Sigma,\quad(p/\Sigma \land q/ \Sigma)\defeq (p\land q)/ \Sigma$$
    a set of generators is $\{a/\Sigma : a\in A\}$
}
\defin{Homomorphisms of boolean algebras}{Let $B,C$ be boolean algebras. A map $\phi: B\to C$ is a (homo)morphism of boolean algebras iff
    $\forall x,y\in B$ it holds
    \begin{itemize}
        \item $\phi(0_B) = 0_C$
        \item $\phi(x') = \phi(x)'$
        \item $\phi(x\lor y) = \phi(x)\lor \phi(y)$
        \item $\phi(x\land y) = \phi(x)\land \phi(y)$
    \end{itemize}
    If $\phi:B\to C$ is bijective too , we call $\phi$ an isomorphism and $\phi^{-1}:C\to B$ is also a morphism of boolean algebras.
}
\note{}{$\phi(B)$ is subalgebra of $C$}
\bsp{}{Let $S,T$ be sets then a function $f:S\to T$ induces a morphism of boolean algebras $\mathcal{P}(T)\to \mathcal{P}(S): y\mapsto f^{-1}(y)$
If $S\subseteq T$ and $f$ the inclusion map $S\hookrightarrow T$ then we get a boolean algebra morphism $Y\to Y\cap S$.\\
    \begin{itemize*}
        \item $id_B: B\to B$ \qquad 
        \item $x\mapsto x': B\to B^{\text{op}}$ are both isomorphism
    \end{itemize*}
}
\note{}{A boolean algebra morphism $\phi: B\to C$ is injective iff $\ker f = 0_B$}
\lemma{}{\label{boolLemma}
    Let $X_1,\dots X_m\subseteq S$ and $\mathcal{A}$ a boolean algebra on $S$ generated by $\{X_1,\dots X_m\}$. Then $\mathcal{A}$ 
    is finite and isomorphic to $\mathcal{P}(\{1,2,\dots n\})$ for some $n\leq 2^m$.
}{
    TODO
}
\defin{Trivial algebras}{\begin{itemize}
    \item $B$ is trivial if $|B| = 1$ (equivalently $0=1\in B$) 
    according to \ref{boolLemma} $B$ is isomorphic to $\mathcal{P}(\varnothing)$
    \item If $|S|=1$ then $|\mathcal{P}(S)| = 2$ 
    TODO
\end{itemize}}
\defin{Ideal}{An ideal of $B$ is a subset of $I\subseteq B$ s.t.
    \begin{itemize}
        \item[(I1)] $0\in I$
        \item[(I2)] $\forall a,b \in B$ it holds \qquad 
            $a\leq b$ and $b \in I\implies a\in I$\qquad and \qquad $a,b\in I\implies a\lor b\in I$ 
    \end{itemize}
}
\bsp{}{$F_{\text{in}} = \{F\subseteq S: F \text{ finite}\}$
    is ideal in $\mathcal{P}(S)$.
}
\note{}{If $I$ is an ideal of $B$ then 
    $I\lor b \defeq \{x\in B: x = a\lor b \text{ for some } a \in I\}$ is the smallest ideal w/ respect of $\subseteq$ of $B$ that contains $I\cup \{b\}$.
}
\bsp{}{\begin{itemize}
\item For a boolean algebra morphism $\phi: B\to C$ the kernel $\ker(\phi)$ is an ideal in $B$.
\item If $I$ is an ideal in $B$ then $a =_I b \defaq a\lor x = b\lor x$ for some $x\in I$ defines an equivalent relation and
$B/_{=_I}$ is a boolean algebra w/ 
$$0\defeq 0/_{=_I}\quad 1\defeq 1/_{=_I}\quad (a/_{=_I})' \defeq a'/_{=_I}\quad a/_{=_I}\lor b/_{=_I} \defeq (a\lor b)/_{=_I}\quad a/_{=_I}\land b/_{=_I} \defeq (a\land b)/_{=_I}$$
Then $\phi: B\to B/_{=_I}: b\mapsto b/_{=_I}$ is a boolean algebra morphism w/ $\ker(\phi)=I$
\end{itemize}}


