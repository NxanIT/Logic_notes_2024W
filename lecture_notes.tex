\documentclass{report}
\usepackage{amsthm,amssymb}
%\usepackage{amsmath} %already loaded by mathtools
\usepackage{titlesec}%required for \titleformat
\usepackage{mathtools}%automatically loads amsmath and is required for \defeq macro
\usepackage{tcolorbox}%required for \definecolor
\usepackage[colorlinks=true, linkcolor=purple2,citecolor=purple]{hyperref}
\usepackage{geometry}
\usepackage[inline]{enumitem}%for "enumerate*" (i.e. horizontal lists)
\usepackage{ifthen}%for "ifthenelse"
\usepackage{fancyhdr} %note only for this document: if inconsistent behaviour - use [twoside]
\usepackage{lastpage}
\usepackage{datetime2}
\usepackage{tocloft}
\usepackage[backend=biber,style=alphabetic]{biblatex}
\usepackage{graphicx}
\usepackage{tikz}
\usetikzlibrary{calc}
\usepackage{float}%benötigt für floating point-positionierung eines bildes
\usepackage{stmaryrd}%for double left brackets
\usepackage{nameref}%for section names in list of theorems
\usepackage{multicol}%for multi colums in list of theorems
\usepackage[acronym]{glossaries}%package to work with glossaries, %2024-12-12 not used
\graphicspath{ {./images/} }%relativen pfad für bilder eingeben

\setlength\parindent{0pt}%disables indent after paragraph

\makeatletter
\newcommand*{\currentname}{\@currentlabelname}
\makeatother

\setlength{\headheight}{18.0pt}

\addbibresource{test.bib}

\newcommand{\listdefinitionsname}{\Large List of definitions}
\newlistof{definitions}{dfn}{\listdefinitionsname}

\newcommand{\listtheoremsname}{\Large List of Theorems}
\newlistof{theorems}{thn}{\listtheoremsname}

\newcommand{\listofabbreviationsname}{\vspace{-1.2em}\Large List of Abbreviations\vspace{-1.5em}}
\newlistof{abbreviations}{abf}{\listofabbreviationsname}

\newcommand{\listofouternotesname}{\Large List of Outernotes}
\newlistof{outernotes}{otf}{\listofouternotesname}

\newcommand{\addAbbrev}[2]{
    \addcontentsline{abf}{section}{#1 \hspace{1em} - \hspace{1em} #2}
}

\newcommand{\version}{0.7.0}
\newif\ifPrintTwosided
%\PrintTwosidedtrue
\ifPrintTwosided
\geometry{
    a4paper,
    twoside,
    inner=14mm,
    outer=46mm,
    bottom=15mm,
    top=20mm,
    marginparwidth = 30mm,
    marginparsep = 8mm
}
\else 
\geometry{
    a4paper,
    inner=14mm,
    outer=46mm,
    bottom=15mm,
    top=20mm,
    marginparwidth = 30mm,
    marginparsep = 8mm
}
\fi

\titleformat{\chapter}[display]{\bfseries\Large} % format
{\vspace{-1cm}\fontfamily{phv}\selectfont\scshape Chapter \thechapter} % label
{1ex} % sep
{   \fontfamily{phv}\selectfont
    \bfseries\huge\centering
} % before-code
[
    \vspace{-0.7cm}
    \addcontentsline{thn}{chapter}{\color{black}\currentname}
    \addcontentsline{dfn}{chapter}{\color{black}\currentname}
    \addcontentsline{otf}{section}{\color{black}\textbf{Chapter \thechapter}}
] % after-code

\titleformat{\section}[display]{\Large} % format
{} % label
{-0.4cm} % sep
{   
    \fontfamily{phv}\selectfont \scshape\thesection\quad
} % before-code
[
    \vspace{-0.2cm}
    %\stepcounter{propcount}
    \addcontentsline{thn}{section}{\color{black}\currentname}
] % after-code

\definecolor{purple2}{HTML}{500066}
\definecolor{rhoda}{HTML}{0000B7}
\definecolor{tanita}{HTML}{5173BD}
\definecolor{yellow}{HTML}{FCF434}
\definecolor{purple}{HTML}{9C59d1}
\definecolor{darkgray}{HTML}{2C2C2C}
\definecolor{mediumgray}{HTML}{4C4C4C}
\definecolor{lightgray}{HTML}{C2C2C2}

%Highlight
\newcommand{\purplebf}[1]{\color{purple}\textbf{#1 }\color{black}}
\newcommand{\graybf}[1]{\color{darkgray}\textbf{#1\hspace{4px}}\color{black}}
\newcommand{\bigemph}[1]{\fontfamily{qag}\selectfont\color{darkgray}\textbf{#1 }\color{black}\normalfont}
\newcommand{\imp}[1]{\textbf{#1}\outernote{#1}\hspace{-3px}}
\newcommand{\subimp}[1]{\underline{\smash{#1}}\subouternote{#1}\hspace{-3px}}%the smash is for a closer underline

\newcommand{\defeq}{\vcentcolon=} 
\newcommand{\eqdef}{=\vcentcolon}
\newcommand{\defaq}{\vcentcolon\Leftrightarrow}
\newcommand{\sledom}{\Relbar\joinrel\mathrel{|}}
\renewcommand{\AA}{\mathbb{A}}
\newcommand{\KK}{\mathbb{K}}
\newcommand{\NN}{\mathbb{N}}
\newcommand{\RR}{\mathbb{R}}
\newcommand{\CC}{\mathbb{C}}
\newcommand{\ZZ}{\mathbb{Z}}
\DeclareMathOperator{\Erf}{Erf}
\DeclareMathOperator{\nicht}{nicht}
\DeclareMathOperator{\Successor}{S}
\DeclareMathOperator{\varS}{var}
\DeclareMathOperator{\TA}{TA}
\DeclareMathOperator{\frei}{frei}
\DeclareMathOperator{\iif}{iff}
\DeclareMathOperator{\fora}{for all}
\DeclareMathOperator{\rng}{ran}%range
\DeclareMathOperator{\dom}{dom}%domain
\DeclareMathOperator{\Mod}{Mod}%class of models
\DeclareMathOperator{\Thm}{Thm}%class of theorems
%\DeclareMathOperator{command}{definition}
\DeclareMathOperator{\Th}{Th}
\DeclareMathOperator{\st}{st}
\DeclareMathOperator{\imag}{im}
\DeclareMathOperator{\On}{On}%class of all ordinals
\DeclareMathOperator{\Ord}{Ord}%class of all ordinals
\DeclareMathOperator{\cof}{cof}
\DeclareMathOperator{\rk}{rk}%rank of sets / ordinal
\DeclareMathOperator{\cl}{cl}% closure of a set


\DeclareMathOperator{\TERMSYMB}{TERMSYMB}

\DeclareMathOperator{\VARIAB}{VARIABLES}
\DeclareMathOperator{\CONST}{CONSTANTS}
\DeclareMathOperator{\FUNCT}{FUNCTSYMB}
\DeclareMathOperator{\TERMS}{TERMS}
\DeclareMathOperator{\EXPR}{EXPR}

\DeclareMathOperator{\propM}{Prop} % for example in boolean algebra
%\DeclareMathOperator{\gdw}{\quad:gdw\quad}

\newif\ifSimplifiedVersion
%\SimplifiedVersiontrue

%%\newtheorem{name}[counter]{Printed output}
\newcounter{defcount}[chapter]
\renewcommand{\thedefcount}{\theHchapter.\arabic{defcount}}

\theoremstyle{definition}
\newtheorem{defi}[defcount]{Definition}
\newcommand{\defin}[2]{
    %Input: 1-Name(or empty), 2-definition
    \ifSimplifiedVersion
        #1
    \else 
        \color{rhoda}   
        \begin{defi}
            \addcontentsline{dfn}{section}{\color{rhoda}Def. \protect{\color{tanita}{\thedefi \hspace{5px} #1}}}
            %\normalfont
            \color{tanita}\fontfamily{txtt}\selectfont\textbf{{{#1}}:}
            \normalfont\color{black}#2
        \end{defi}
        \color{black}
    \fi
}

\newcommand{\outernote}[1]{
    \marginpar[\raggedleft #1]{\raggedright #1}%\marginpar[\raggedright #1]{\raggedleft #1}%
    \addcontentsline{otf}{section}{#1}
    \hspace{-6px}
}

\newcommand{\subouternote}[1]{
    \marginpar[\raggedleft #1]{\raggedright #1}%\marginpar[\raggedright #1]{\raggedleft #1}%
    \addcontentsline{otf}{subsection}{#1}
    \hspace{-6px}
}

%\newtheorem{notes}[]{Bemerkung}
\newcommand{\note}[2]{
    \ifSimplifiedVersion
        Bem: #1 - #2

    \else 
        \color{rhoda}   
        %\begin{notes}
            \noindent Note \color{tanita}\fontfamily{txtt}\selectfont{}#1:
            \hspace{5px}\normalfont\color{black}#2

        %\end{notes}
        \color{black}
    \fi
}

\newcounter{propcount}[section]
\renewcommand{\thepropcount}{\theHchapter.\arabic{section}.\arabic{propcount}}

\theoremstyle{plain}
\newtheorem{propo}[propcount]{Proposition}
\newcommand{\prop}[3]{
    %Input: 1-Name(or empty), 2-statement, 3-proof(or empty)
    \ifSimplifiedVersion
        Prop: #1 - #2

    \else 
        \color{rhoda}   
        \begin{propo}
            \propOrThm{#1}{#2}{#3}
        \end{propo}
        \color{black}
    \fi
    \addcontentsline{thn}{subsection}{\color{rhoda}Prop. \protect{\color{tanita}{\thepropcount \hspace{5px} #1}}} 
}
%\newcounter{theoremcount}[section]
%\renewcommand{\thetheoremcount}{\thesection.\arabic{propcount}}

\newtheorem{theorem}[propcount]{Theorem}
\newcommand{\thm}[3]{
    %Input: 1-Name(or empty), 2-statement, 3-proof(or empty)
    \ifSimplifiedVersion
        Prop: #1 - #2

    \else 
        \color{rhoda}   
        \begin{theorem}
            \propOrThm{#1}{#2}{#3}
        \end{theorem}
        \color{black}
    \fi
    \addcontentsline{thn}{subsection}{\color{rhoda}Thm. \protect{\color{tanita}{\thepropcount \hspace{5px} #1}}} 
}

\newtheorem{lemmata}[propcount]{Lemma}
\newcommand{\lemma}[3]{
    %Input: 1-Name, 2-statement, 3-proof
    
    \ifSimplifiedVersion
        Lemma: #1 - #2

    \else 
        \color{rhoda}   
        \begin{lemmata}
            \propOrThm{#1}{#2}{#3}
        \end{lemmata}
        \color{black}
    \fi
    \addcontentsline{thn}{subsection}{\color{rhoda}Lemma \protect{\color{tanita}{\thepropcount \hspace{5px} #1}}} 
}
\newtheorem{hypo}[propcount]{Hypothesis}
\newcommand{\hypothesis}[2]{
    \ifSimplifiedVersion
        Hypo: #1 - #2

    \else 
        \color{rhoda}   
        \begin{hypo}
            \propOrThm{#1}{#2}{}
        \end{hypo}
        \color{black}
    \fi
    \addcontentsline{thn}{subsection}{\color{rhoda}Hyp. \protect{\color{tanita}{\thepropcount \hspace{5px} #1}}} 
}

\newcommand{\propOrThm}[3]{
    \ifx\foo#1\foo 
    \else
        \color{tanita}\fontfamily{txtt}\selectfont\textbf{\textup{#1}}:
    \fi
    \fontfamily{cmr}\selectfont\color{black}#2
    \ifx\foo#3\foo
    \else
        \begin{proof}
            #3%\vspace{5pt}
        \end{proof}
    \fi
}


\newcounter{subcorollcount}[propcount]

\renewcommand{\thesubcorollcount}{\thepropcount-\Alph{subcorollcount}}
\newtheorem{corollary}[subcorollcount]{Corollary}
\newcommand{\coroll}[1]{
    %\addtocounter{corollcount}{-1}
    %Input: 1-Statement
    \ifSimplifiedVersion
        Cor: #1

    \else 
        \color{rhoda}   
        \begin{corollary}
            \color{black}#1\vspace{-5px}
        \end{corollary}
        \color{black}
    \fi
}

\newcounter{excount}[chapter]
\renewcommand{\theexcount}{\theHchapter.\arabic{excount}}
\newtheorem{example}[excount]{Example}
\newcommand{\bsp}[2]{
    \ifSimplifiedVersion
        Bsp(#1): #2

    \else 
        
        \color{rhoda}
        \begin{example}
            \ifx\foo#1\foo 
            \else
                \color{tanita}\fontfamily{txtt}\selectfont\textbf{\textup{#1}}:
            \fi
            \normalfont\color{black}#2
        \end{example}
        \color{black}
    \fi
}

\newcommand*{\claimproofname}{proof of claim.}
\newenvironment{claimproof}[1][\claimproofname]{\vspace*{-10px}\begin{proof}[#1]\renewcommand*{\qedsymbol}{\(\boxtimes\)}}{\end{proof}}

\newcommand{\romanNum}[1]{
  \textup{{\romannumeral#1\relax}}
}
\newcommand{\RomanNum}[1]{
  \textup{\uppercase\expandafter{\romannumeral#1\relax}}
}

\fancypagestyle{two_sided}{
    \pagestyle{fancy}   %needed for changing headers/footers
    \fancyhf{}          %cleares headers and footers
    \fancyhead[RO]{\small\rightmark}
    \fancyhead[LE]{\small\leftmark}
    \fancyhead[LO,RE]{Logic Lecture Notes 2024W}
    \fancyfoot[C]{}
    \fancyfoot[LO,RE]{\fontfamily{cmtt}\small\color{gray}J.Petermann: LogicNotes [V\version-\today{ }at \DTMcurrenttime] }% old: \href[]{https://creativecommons.org/licenses/by-nc-nd/3.0/de/}{\color{gray}CC BY-NC-ND 3.0 DE}
    \fancyfoot[LE,RO]{Seite \thepage \hspace{3pt}von \pageref*{LastPage}}
}
\fancypagestyle{one_sided}{
    \pagestyle{fancy}   %needed for changing headers/footers
    \fancyhf{}          %cleares headers and footers
    \fancyhead[R]{
        \small\rightmark
        % \begin{tikzpicture}[overlay,remember picture]
        %     \fill [color=lightgray](current page.north east) rectangle
        %         ($ (current page.south east) + (-4.5cm,0cm) $);
        % \end{tikzpicture}
    }
    \fancyhead[L]{
        Logic Lecture Notes 2024W
        %\begin{tikzpicture}[overlay,remember picture]
        %     \fill [color=mediumgray] (current page.north west) rectangle
        %         ($ (current page.south west) + (1.5cm,0cm) $);
        % \end{tikzpicture}
    }
    \fancyfoot[C]{}
    \fancyfoot[L]{\fontfamily{cmtt}\small\color{gray}J.Petermann: LogicNotes [V\version-\today{ }at \DTMcurrenttime] }% old: \href[]{https://creativecommons.org/licenses/by-nc-nd/3.0/de/}{\color{gray}CC BY-NC-ND 3.0 DE}
    \fancyfoot[R]{Seite \thepage \hspace{3pt}von \pageref*{LastPage}}
}
\fancypagestyle{one_sided_outer}[one_sided]{
    \fancyhead[R]{
        \small\rightmark%(current page.north east)
        \begin{tikzpicture}[overlay,remember picture]
            \fill [color=lightgray]
            ($ (current page.north east) + (-4.2cm,-1cm) $)
            rectangle
            ($ (current page.south east) + (-4.cm,+1cm) $);
        \end{tikzpicture}
    }
    
    \fancyhead[L]{
        Logic Lecture Notes 2024W
        %\begin{tikzpicture}[overlay,remember picture]
        %    \fill [color=mediumgray] (current page.north west) rectangle
        %        ($ (current page.south west) + (1.5cm,0cm) $);
        %\end{tikzpicture}
    }
}

\begin{document}

\fancypagestyle{plain}{}
\ifPrintTwosided
\pagestyle{two_sided}
\else
\pagestyle{one_sided_outer}
\fi

\begin{center}
    \fontfamily{qag}\selectfont
    \Huge\textbf{Lecture notes\\ Einführung in die Logik 2024W}\\
    \normalfont
\end{center}
\begin{center}
    \framebox[15cm]{\parbox{\dimexpr\linewidth-2\fboxsep-2\fboxrule}{
    This is a summary of the material discussed in the lecture "Mathematische Logik". 
    It is still a work in progress and there \textbf{may be mistakes} in this work. 
    If you find any, feel free to let me know and I will correct them\\\hspace{10px}
    The content of this script relies on \cite{EndertonHerbertB2001AMIt}, \cite{van1998tame} and \cite{krivine1998théorie}
    Dieses Skript ist noch nicht vollständig und wird regelmäßig aktualisiert.}
}
\end{center}

\setlength{\cftbeforetoctitleskip}{0em}
\setlength{\cftaftertoctitleskip}{0em}
\tableofcontents



    
\chapter{Propositional logic}
\addAbbrev{prop.}{propositional}
\defin{Language of PL}{
The Language \outernote{Language} of Propositional logic is a set containing
    \begin{itemize}
        \item logical symbols: consisting of the \graybf{sentential connective} symbols $\lnot, \land, \lor, \to, \leftrightarrow$ and parenthesis $(,)$
        \item non-logical symbols: $ A_1, A_2,A_3,\dots$ (also called sentential atoms, variables)
    \end{itemize}
    from which we assume (for unique readability) that no symbol is a finite sequence of any other symbols.
}
\note{}{
    \begin{enumerate}
        \item The role of the logical symbols doesn't change, the sentential atoms we see as variables, they function as placeholders or variables.
        \item we assumed the set of non-logical symbols is countable, for most of our conclusions you could use any set of prop. atoms of any size
    \end{enumerate}
}
\addAbbrev{exp.}{expression(s)}
\defin{Expression / prop. sentence}{An \graybf{expression} is a any finite sequence of symbols
    We define \graybf{grammatically correct exp.} recursive
    \begin{enumerate}
        \item every prop. atom is a prop. sentence
        \item if $\alpha, \beta$ are prop. sentences, then also $\lnot \alpha, \alpha \land \beta, \alpha \lor \beta, \alpha \to \beta, \alpha \leftrightarrow \beta$
        \item nothing else
    \end{enumerate}
    and call them \graybf{prop. sentences.}
    Equivalently stated every prop. sentence. is built up by applying finitly many operations
    TODO
    This allows us to symbolize the \graybf{expression tree} 
}
\defin{Construction sequence}{
    Given a prop. sentence $\alpha$ a construction sequence of $\alpha$ is a finite sequence $\left\langle \alpha_1,\dots \alpha_{n-1},\alpha\right\rangle$ such that for all $i\leq n$
    the following holds
    \begin{itemize}
        \item $\alpha_i$ is a sentential atom
        \item or $\alpha_i= \varepsilon_\lnot(\alpha_j)$ for some $j< i$
        \item or $\alpha_i= \varepsilon_{\square }(\alpha_j,\alpha_k)$ for some $j,k<i$ and $\square\in\{\land,\lor,\to,\leftrightarrow\}$
    \end{itemize}
}
\defin{}{Let $S$ be a set. We say $S$ is \graybf{closed} under an $n$-ary operational symbol $f$
    iff for all $s\in S$ it holds $f(s)\in S$
}
\addAbbrev{sent.}{sentence(s)}
\addAbbrev{seq.}{sequence}
\noindent\graybf{Induction principle:} Suppose $S$ is a set of prop. sentences
containing all prop. atoms and closed under the 5 formula building operations, 
then $S$ is the set of all prop. sentences.
\begin{proof}
    let $PS = \text{set of all prop. sent.}$
    \begin{itemize}[leftmargin=2cm]
        \item[$S\subseteq PS$:] is clear
        \item[$S\supseteq PS$:] let $\alpha\in PS$ then $\alpha$ has a construction seq. $\left\langle \alpha_1,\dots \alpha_{n-1},\alpha\right\rangle$ and $\alpha_1\in S$
        lets assume that $\alpha_k$ for $k<n$ is in $S$ then $\alpha_{k+1}$ is either an atom and therefore in $S$ or its obtained by one of the formula building operations 
        and therefore $\alpha_{k+1}\in S$
    \end{itemize}
\end{proof}
\addAbbrev{TA}{truth assignment}
\section{Truth assignments}
We will answer the question when does a prop. sent. follow from other prop. sentences.
\addAbbrev{fla.}{formula}
\addAbbrev{TV}{truth value}
\defin{Truth assignment}{\label{TAConditions}
Let $\{0,1\}$ be the set of truth values. \outernote{Truth assigment} A truth assignment \outernote{TA}(TA) for a set $S$ of prop. atoms is a map $\nu:S\to\{0,1\}$}
We now want to extend $\nu$ to $\overline{\nu}: \overline{S}\to \{0,1\}$, where $\overline{S}$ is the closure of $S$ under the 5 fla. building operations such that
\begin{enumerate}
    \item $\overline{\nu}(A) = \nu(A)$
    \item $\overline{\nu}(\lnot \alpha) = 1- \nu(\alpha)$
    \item $\overline{\nu}(\alpha \land \beta) = \begin{cases}
        1 & \text{iff } \overline{\nu}(\alpha) = 1 = \overline{\nu}(\beta)\\
        0 & \text{otherwise}
    \end{cases}$
    \item $\overline{\nu}(\alpha \lor \beta) = \begin{cases}
        1 & \text{iff } \overline{\nu}(\alpha) = 1 \text{ or } \overline{\nu}(\beta) = 1\\
        0 & \text{otherwise}
    \end{cases}$
    \item $\overline{\nu}(\alpha \to \beta) = \begin{cases}
        1 & \text{iff } \overline{\nu}(\alpha) = 0 \text{ or } \overline{\nu}(\beta) = 1\\
        0 & \text{otherwise}
    \end{cases}$
    \item $\overline{\nu}(\alpha \leftrightarrow \beta) = \begin{cases}
        1 & \text{iff } \overline{\nu}(\alpha) =  \overline{\nu}(\beta)\\
        0 & \text{otherwise}
    \end{cases}$
\end{enumerate}
\prop{}{\label{extendetTruthAss}\label{ThrmUniqueExt}
    $\forall$ TA $\nu$ for a set $S$ $\exists ! \overline{\nu}:\overline{S}\to\{0,1\}$ satisfying the above properties}{}
    We will proof this later
\defin{Satisfaction}{A TA $\nu$ satisfies a prop. sent. $\alpha$ 
    iff $\overline{\nu}(\alpha)=1$ (that is, provided that everery atom of $\alpha$ is in the domain of $\nu$)}
\defin{Tautological implication}{
    Let $\Sigma$ be a set of prop. sent. and $\alpha$ a prop. sent. then we say:
    $\Sigma$ tautologically imlies $\alpha$ iff $\forall$ TA that satisfies $\Sigma$ then $\alpha$ is also satisfied and we write $\Sigma\models \alpha$\\
    If $\Sigma = \{\beta\}$, we simply write $\beta \models \alpha$ If $\Sigma = \varnothing$ then we write $\models \alpha$ for $\varnothing \models \alpha$ and $\alpha$ is called a \graybf{tautology}\\
    $\alpha, \beta$ are called \graybf{tautologically equivalent} iff $\alpha\models \beta$ and $\beta\models \alpha$ we then write $\alpha \sledom \models \beta$
}
\note{}{Suppose there is no TA that satisfies $\Sigma$, then we have $\Sigma \models \alpha$ for every prop. sent. $\alpha$}
\bsp{}{$\{\lnot A \lor B\} \sledom \models A \to B$ }
\note{}{In order to check if a prop. sent. is satisfiable we need to check $2^N$ TAs, where $N=\text{\# of atoms}$. It is unknown if this can be done by an algorithm in polynomial time. Answering this 
    would settle the debate whether $P=NP$}
TODO: Add section here?
\prop{Compactness theorem}{Let $\Sigma$ be an infinite set op prop. sent. such that 
    $$\forall \Sigma_0 \subseteq \Sigma, \Sigma_0 \text{finite} \exists \text{ TA satisfying every member of } \Sigma_0$$
    then there is a TA satsfying every member of $\Sigma$.}
{
    Let $\mathcal{A} = \{A_0, A_1,\dots\}$ be the set of all prop. atoms. We are going to identify TAs 
    with elements in $\{0,1\}^\mathcal{A}\defeq \{f: \mathcal{A}\to \{0,1\}\}$
    TODO
}
\section{A parsing algorithm}
To prove Thm. \ref{extendetTruthAss} we essentially need to show that we have enough parenthesis to make the reading of a prop. sent. unique.
TODO Bsp
\addAbbrev{w/}{with}
\lemma{}{Every prop. sent. has the same number of left and right parenthesis.}{
    Let $M = \text{set of prop. sent. w/ \# left parenthesis = \# right parenthesis}$ and \\
    $PS = \text{set of all prop. sent.}$
    We have $M\subseteq PS$. Since atoms have no parenthesis, they are in $M$. we just need to show that
    $M$ is closed under the 5 construction operations.\\
    $\varepsilon_{\lnot} = (\lnot \alpha)$ \dots
}
\lemma{}{No proper initial segment of a prop. sent. is itself a prop. sent.}{
    Let $\alpha = \alpha_1\alpha_2\dots \alpha_n$ be a prop. sent. By proper initial segment we understand $\beta = \alpha_1\dots \alpha_i$ for $1\leq i<n$.
    We will prove that every proper initial segment has an excess of left parenthesis, then we use the previous lemma.
    \begin{itemize}
        \item Atoms: since the empty sequence is no prop. sent. they have no proper initial segment.
        \item If the above is true for $\alpha, \beta$ then the proper initial segments of $(\lnot \alpha)$ are of the form
        \begin{itemize}
            \item[] $(\lnot \alpha$
            \item[] $(\lnot \alpha'$ where $\alpha'$ is a propper initial segment of $\alpha$
            \item[] $($ \qquad or
            \item[] $(\lnot$
        \end{itemize}
        Therefore $\varepsilon_\lnot$ preserves this property and 
        under $\varepsilon_\land, \varepsilon_\lor, \varepsilon_\to, \varepsilon_\leftrightarrow$ this is also the case.
    \end{itemize}
}
\subsubsection*{Parsing algorithm}
We now give a parsing algorithm procedure. For input we take some expression $\tau$ and the algorithm will determine if $\tau$ is a prop. sent.
If so, it will generate a unique construction tree (in form of a rooted tree) for $\tau$.
\begin{enumerate}
    \item[0.] create the root and label it $\tau$
    \item HALT if all leaves are labled w/ prop. atom and return: "$\tau$ is a prop. sent."
    \item select a leaf of the graph which is not labled w/ prop. atom
    \item if the first symbol of label under consideration is not a left parenthesis, then halt and return: "$\tau$ is not a prop. sent."
    \item if the second symbol of the label is "$\lnot$" then GOTO 6.
    \item scan the expression from left to right\\
    if we reach a proper initial segment of the form "$(\beta$" where $\# lp(\beta) = \#rp(\beta)$ and $\beta$ is followed by one of thesection
    $\land,\lor,\to,\leftrightarrow$ and the remainder of the expression is of the form $\beta')$, where $\# lp(\beta') = \#rp(\beta')$
    \begin{itemize}
        \item [Then:] create two child nodes (left,right) to the selected element and label them (left $\defeq \beta$, right $\defeq \beta'$) GOTO 1.
        \item [Else:] HALT and return "$\tau$ is not a prop. sent."
    \end{itemize}
    
    \item if the expression is of the form $(\lnot \beta)$ where $\# lp(\beta) = \#rp(\beta)$
    \begin{itemize}
        \item [Then:] construct one childnode and label it $\beta$ and GOTO 1.
        \item [Else:] HALT and return: "$\tau$ is not a prop. sent."
    \end{itemize}
\end{enumerate}
\bsp{
    TODO
}{}
\subsection*{Correctness of the parsing algorithm}
\begin{itemize}
    \item The algorithm always halts, because the length of a child is less than the label of a parent.
    \item If the algorithm halts with the conclusion that $\tau$ is a prop. sent. 
    then we can prove inductively (starting from the leaves) that each label is a prop. sent
    \item Unique way to make choices in the algorithm: in particular $\beta, \beta'$ in step 5.
    If there was a shorter choice for $\beta$ it would be a proper initial segment of $\beta$ but such prop. sent. can not exist.
    (This also works under the assumption that a longer choice exists).
    \item rejections are made correctly
\end{itemize}
Back to proving the existence and uniqueness of $\overline{\nu}$ in \ref{ThrmUniqueExt}.
Let $\alpha$ be a prop. sent. of $\overline{S}$. We apply the parsing algorithm to $\alpha$ to get a unique construction tree
For the leaves, use $\nu$ go get the truth values then work our way up using the conditions (1-6) in \ref{TAConditions}.
\subsection*{A more formal notation}
TODO
\section{Induction and recursion}
A simple case: let $\mathcal{U}$ be a set and $B\subseteq \mathcal{U}$ our initial set.
$\mathcal{F} = \{f,g\}$ a class of functions containing just $f$ and $g$, where $$f:\mathcal{U}\times \mathcal{U}\to \mathcal{U}, \qquad g: \mathcal{U}\to \mathcal{U}$$
We want to construct the smallest subset $\mathcal{C}\subseteq \mathcal{U}$ such that $B\subseteq \mathcal{C}$ and $\mathcal{C}$ is closed under all elements of $\mathcal{F}$.
\defin{Closedness, Inductiveness}{ We say $\mathcal{C}$ is 
    \begin{itemize}
        \item\graybf{closed} under $f$ and $g$ iff $\forall x,y\in \mathcal{C}\: (f(x,y)\in \mathcal{C} \land g(x)\in \mathcal{C})$
        \item\graybf{inductive} if $B\subseteq \mathcal{C}$ and $\mathcal{C}$ is closed under $\mathcal{F}$
    \end{itemize}
}
Big
TODO
\section{Sentential connectives}
\defin{Tautological equivalence relation}{For $\alpha,\beta$ prop. sent. we define $\alpha ~ \beta$ 
iff $\alpha \sledom\models  \beta$. This defines an equivalent relation.}
\bsp{}{$A \to B \sledom\models  \lnot A \lor B$}
\note{}{
    A $k$-place boolean function is a functon of the form $f: \{0,1\}^k\to \{0,1\}$ and we 
    define $0,1$ as the $0$-place boolean functions.\\
    If $\alpha$ is a prop. sent. then it determines a $k$-place boolean function, 
    where $k$ is the number of atoms, $\alpha$ is built up from.
    If $\alpha$ is $A_1\lor \lnot A_2$ then $B_\alpha: \{0,1\}^2\to \{0,1\}$ and asign its values corresponding a truth table.
    TODO extend / rearange function
}
\prop{}{If $\alpha,\beta$ are prop. sent. with at most $n$ prop. Atoms (combined), then
    \begin{enumerate}
        \item $\alpha \models \beta $ iff $\forall x\in \{0,1\}^n$ it holds $B_\alpha(x)\leq B_\beta(x)$
        \item $\alpha \sledom \models \beta $ iff $\forall x\in \{0,1\}^n$ it holds $B_\alpha(x) = B_\beta(x)$
        \item $\models \alpha $ iff $\forall x\in \{0,1\}^n$ it holds $B_\alpha(x)=1$
    \end{enumerate}
}{}
\prop{Realisation}{
    Let $G$ be an $n$-ary boolean function for $n\geq 1$. Then there is a prop. sent. $\alpha$ such that. $B_\alpha = G$.
    We say $\alpha$ realizes $G$.
}{
    \begin{enumerate}
        \item if $G$ is constantly equal to $0$ then set $\alpha$ to $A_1 \land \lnot A_1$.
        \item Otherwise the set of inputs $\{\vec{x}_1,\vec{x}_2,\dots \vec{x}_k\}$ for which $G(\vec{x}_i)=1$ holds is not empty.\\
        We denote $\vec{x}_i = (x_{i1},x_{i2},\dots x_{in})$ and define a matrix $(x_{ij})_{k\times n}$
        We further set $\beta_{ij} = \begin{cases}
            A_j & \text{iff } x_{ij}=1\\
            \lnot A_j & \text{iff } x_{ij}=0
        \end{cases}$\\
        \graybf{Example:} 
        \begin{equation*}
            (x_{ij})=
            \begin{pmatrix}
                0&1&0\\
                1&1&0
            \end{pmatrix}\leadsto 
            \begin{pmatrix}
                \lnot A_1 & A_2 & \lnot A_3\\
                A_1 & A_2 & \lnot A_3\\
            \end{pmatrix}=(\beta_{ij})
        \end{equation*}
        We define $\gamma_i$ as $\beta_{i1} \land \beta_{i2}\land \dots \beta_{in}$ for $1\leq i\leq k$\\
    and $\alpha$ as $\gamma_1 \lor \gamma_2\lor \dots \gamma_k = \vee_{i=1}^{k}{\gamma_i} $
    Then $B_\alpha = G$ is fulfilled.
    \end{enumerate}
}
\note{}{$\alpha$ as constructed in the proof is in the so-called Disjunctive normal form (DNF).}
\coroll{Every prop. sent. is tautologically equivalent to a sentence in DNF}
\addAbbrev{i.e.}{id est (that is)}
\coroll{$\{\lnot,\land,\lor\}$ is a complete set of logical connectives, i.e. every prop. sent. is tautologically 
    equivalent to a sentence built up from atoms and $\lnot,\land,\lor$.
}
\prop{}{Both $\{\lnot, \land\}$ and $\{\lnot, \lor\}$ are complete.}{
    Its sufficient to show that every $k$-place boolean function is realisable by a prop. sent.
    built up using only $\lnot$ and $\land$. This is, because $\alpha\land \beta \sledom \models \lnot (\lnot \alpha \lor \lnot \beta)$
    We prove this by induction over the number of disjuctions of a prop. sent. $\alpha$ in DNF.
    Suppose the statement is true for $k \leq n$. For $n+1$ and $\alpha = \bigvee_{j=1}^{n+1}{\gamma_j}$ there exists an $\alpha' \sledom \models \bigvee_{j=1}^{n}{\gamma_j}$ and 
    $$\alpha = \bigvee_{j=1}^{n+1}{\gamma_j} \sledom \models \alpha' \lor \gamma_{n+1} \sledom \models \lnot (\lnot \alpha' \land \lnot \gamma_{n+1})$$
    %$\alpha\land \beta \sledom \models \lnot (\lnot \alpha \lor \lnot \beta)$
}
\note{}{We used the observation that, if $\alpha \sledom \models \beta$ and we replace a subsequence of $\alpha$ by a so called tautological equivalence 
    then the result is also tautologically equivalent to $\beta$}
TODO S.10
\bsp{$\{\to, \land\}$ is not complete.}{Let $\alpha\in PS$ built up from only $\to,\land$ from the atoms $A_1,\dots A_n$ then we claim
    $$A_1\land A_2\land \dots \land A_n \models \alpha$$
    %Furhter we can observe that $\{\to, \lor \}$ is not complete, because if $\alpha\in PS$ is only built up from $\to,\lor$ then $\lnot \alpha$
    %can be built up from $\to, \land$. This is because of
    %$$\lnot(A\to B)\sledom \models \lnot B \to \lnot A \quad \text{and}\quad \lnot(\alpha\lor \beta) \sledom \models \lnot \alpha \land \lnot \beta$$
    We can also say $\{\to, \land\}$ is not complete bc. $\lnot A$ is not tautological equivalent to a sent. built up from $\to, \land$
    \begin{proof}
        Let $C \defeq \{\alpha \in PS \text{ built up from }\to,\land \text{ and }A_1,\dots A_n \text{ for which } \bigwedge_{i=1}^n{A_i}\models \alpha\}$
        we want to show that $C = \{\alpha \in PS \text{ built up from }\to,\land \text{ and }A_1,\dots A_n \}$
        \begin{itemize}
            \item We have $\{A_1,A_2\dots,A_n\}\subseteq C$
            \item for $\alpha,\beta\in C$ it holds
            \begin{itemize}
                \item[(1)] $A_1\land\dots\land A_n \models \alpha\to\beta$
                \item[(2)] $A_1\land\dots\land A_n \models \alpha\land \beta$
            \end{itemize}
        \end{itemize}
        Therefore $C$ is closed under the fla. building operations and we have proven our claim.
    \end{proof}
    }
\note{}{$\{\land,\lor,\to,\leftrightarrow \}$ is still not complete.}
\note{}{The number of $n$-ary boolean functions existing is $2^{2^n}$
    We define a notation for $n=0$: $\bot$ (for TV = $0$) and $\top$ (for TV = $1$)
    We can conclude that $\{\lnot,\to\}$ and $\{\to, \bot\}$ are both complete, it holds $\lnot A \sledom \models A\to \bot$
}
\defin{Satisfiability}{\\ A set of prop. sent. $\Sigma$ is called \graybf{satisfiable} iff $\exists$ TA that satisfies every member of $\Sigma$.}
\section{Compactness Theorem}
\prop{Compactness Theorem}{\label{CompThrm}
    $\Sigma$ is satisfiable iff every finite subset $\Sigma_0\subseteq \Sigma$ is satisfiable. (i.e. $\Sigma$ is finitely satisfied)}{
    Let $\Sigma$ be a finitely satisfiable set of prop. sent. Outline of the proof:
    \begin{enumerate}
        \item extend $\Sigma$ to a maximal finitely satisfiable set $\Delta$ of prop. sent.
        \item construct a thruth assigment using $\Delta$
    \end{enumerate}
    \begin{enumerate}
        \item Let $\alpha_1,\alpha_2,\dots$ be an enumeration of all prop. sent. 
        and define $\Delta_n$ inductively by $\Delta_0 \defeq \Sigma$
        $$\Delta_{n+1}\defeq \begin{cases}
            \Delta_n\cup \{\alpha_{n+1}\} & \text{if satisfiable}\\
            \Delta_n\cup \{\lnot\alpha_{n+1}\} & \text{otherwise}
        \end{cases}$$
        \textbf{Claim:} $\Delta_n$ is finitely satisfiable for each $n$
        \begin{claimproof}
            By regular induction over $n$. $\Delta_0$ is finitely satisfiable. Let us assume $\Delta_n$ is finitely satisfiable.
            If $\Delta_{n+1} = \Delta_n\cup \{\alpha_{n+1}\}$ then we are finished. 
            Otherwise let $\Delta' \subseteq \Delta_n$ be a finite set that $\Delta' \cup \{\alpha_{n+1}\}$ is not satisfiable.
            It holds $\Delta' \models \lnot \alpha_{n+1}$.
            We assume that $\Delta_n\cup \{\lnot\alpha_{n+1}\}$ is not finitely satisfiable. 
            Then there exists a finite subset $\Delta'' \subseteq \Delta_n $ such that $\Delta'' \cup \{\lnot\alpha_{n+1}\}$ is (finite and) not satisfiable.
            It therefore holds $\Delta'' \models \alpha_{n+1}$
            But $\Delta'\cup \Delta''$ is a finite subset of $\Delta_n$ and by above observations $\Delta'\cup \Delta''\models \alpha_{n+1}$ and $\Delta'\cup \Delta''\models \lnot \alpha_{n+1}$
            A contradiction to the assumption that $\Delta_n$ is finitely satisfiable.
        \end{claimproof}
        We set $\Delta \defeq \bigcup_{i\in\NN}{\Delta_i}$ and get
        \begin{enumerate}
            \item $\Sigma\subseteq \Delta$
            \item (Maximality): for every prop. sent. $\alpha$ it holds $\alpha\in \Delta$ or $\lnot \alpha\in \Delta$
            \item (Satisfiability): $\Delta$ is finitely satisfiable. For every finite subset there exists a $\Delta_n$ which is a superset.
        \end{enumerate}
        \item Let $\nu$ be a TA for the prop. atoms $A_1, A_2,\dots$ such that $\nu(A) = 1$ iff $A\in \Delta$
        
        \textbf{Claim:} For every prop. sent. $\varphi$ it holds $\overline{\nu}(\varphi) =1 $ iff $\varphi\in\Delta$.
        \begin{claimproof}
            Let $S = \{\varphi \in PS \text{ s.t. } \overline{\nu}(\varphi) = 1 \text{ iff } \varphi \in \Delta\}$. \\
            \begin{itemize}
                \item $PS\supseteq S$ is clear.
                \item $PS\subseteq S$ 
                \begin{enumerate}
                    \item $\{A_1,A_2\dots\}\subseteq S$ by definition of $\nu$
                    \item closure under $\epsilon_\lnot$: Let $\varphi\in S$ then we get by maximality and satisfiability of $\Delta$: 
                    \begin{equation*}
                        \begin{split}
                            &\overline{\nu}(\lnot\varphi) = 1\\
                            \text{iff }\quad&\overline{\nu}(\varphi) = 0\\
                            \text{iff }\quad& \varphi \notin \Delta\\
                            \text{iff }\quad& (\lnot \varphi)\in \Delta
                        \end{split}
                    \end{equation*}
                    closure under $\epsilon_\to$: Let $\varphi_1,\varphi_2\in S$ similiarly
                    \begin{equation*}
                        \begin{split}
                            &\overline{\nu}(\varphi_1\to \varphi_2) = 0\\
                            \text{iff }\quad&\overline{\nu}(\varphi_1) = 1 \text{ and } \overline{\nu}(\varphi_2) = 0\\
                            \text{iff }\quad& \varphi_1 \in \Delta \text{ and }\varphi_2 \notin \Delta\\
                            \text{iff }\quad& (\varphi_1\to \varphi_2)\notin \Delta 
                        \end{split}
                    \end{equation*}
                    The closure under the other fla. building operations are similar.\qedhere
                \end{enumerate}
            \end{itemize}
        \end{claimproof}
        By this claim $\overline{\nu}$ satisfies $\Sigma$.\qedhere
    \end{enumerate}
}
\coroll{\label{CorCompThrm}
    If $\Sigma\models \tau$ then there exists a finite subset $\Sigma' \subseteq \Sigma$ s.t. $\Sigma' \models \tau$}
\begin{proof}
    Recall: $\Sigma\models \tau$ iff $\Sigma\cup\{\lnot \tau\}$ is not satisfiable.
    Suppose $\Sigma\models \tau$ but no finite subset does. \\
    Then $\forall \Sigma'\subseteq \Sigma \text{ finite } \Sigma'\cup \{\lnot \tau\}$ is satisfiable.
    By the compactness theorem $\Sigma\cup \{\lnot \tau\}$ is satisfiable which is a contradiction to $\Sigma\models \tau$.
\end{proof}
\note{}{\ref{CompThrm} and \ref{CorCompThrm} are equivalent.}


\chapter{Predicate - / first order logic}
\defin{A First order Language}{
    consists of infinetely many distinct symbols such that no symbol is a proper 
    initial segment of another symbol and the symbols are divided into 2 groups:
    \begin{enumerate}
        \item logical symbols (These elements have a fixed meaning and the equivalence symbol $=$ is optional)
        \begin{flalign*}
            (,),\lnot, \to, v_1,v_2,\dots,=&&
        \end{flalign*}
        \item parameters
        \begin{itemize}
            \item quantifier symbol: $\forall$ (the range is subject of interpretation)
            \item predicate symbols: for every $n>0$ we have a set of $n$-ary predicates $P$
            \item constant symbols: Some set of constants (could also be $\varnothing$)
            \item function symbols: for every $n>0$ we have a set of $n$-ary function symbols
        \end{itemize}
    \end{enumerate}
}
\note{}{
    \begin{itemize}
        \item We could drop constants and instead introduce $0$-ary function symbols
        \item to specify language we need to specify the parameters and say if $=$ is included
    \end{itemize}
}
\bsp{}{
    \begin{itemize}
        \item $\mathcal{L}_\text{set} = \{\in\}$, \quad $=$ included and the binary predicate symbol $\in$ "element in"
        \item $\mathcal{L}_\text{arith} = \{<,0,S,E,+,\cdot\}$
        \begin{itemize}
            \item [$=$] included
            \item [$<$] is a binary rel. symbol
            \item [$0$] is a constant
            \item [$S$] is a unary function symbol
            \item [$E$] exponentiation function symbol
            \item [$+,\cdot$] binary function symbols
        \end{itemize}
        \item $\mathcal{L}_\text{ring} = \{=,+,\cdot,-,0,1\}$
        \begin{itemize}
            \item [$=$] included
            \item [$0,1$] are constants
            \item [$-$] is a unary function symbol (additive inverse)
            \item [$+,\cdot$] binary function symbols
        \end{itemize}
    \end{itemize}
}

\section{Formulas}
\defin{Expression}{An \graybf{expression} is any finite sequence of symbols.
    There exist two kinds of expressions that makes sense "grammatically"
    \begin{itemize}[leftmargin=1.6cm]
        \item[Terms:] \begin{itemize}
            \item points to an object
            \item they are built up from variables and constants using function symbols %(by use of polish notation)
        \end{itemize}
        \item[Formulas:]\begin{itemize}
            \item They express assertions about objects,
            \item they are built up from atomic formulas 
            \item atomic formulas these are built up from terms using predicate symbols and $=$, if included
        \end{itemize}
    \end{itemize}
}
\defin{Term Building Operations}{
    For every $n>0$ and for every $n$-place function symbol $f$ let $\mathcal{F}_f$ be an $n$-place term building operation,
    that is $\mathcal{F}_f (t_1,\dots t_n)\defeq ft_1,\dots t_n$ (polish notation for $f(t_1,\dots t_n)$).
    The Set of Terms we then define as the set of expressions that are built up from variables and constants by applying the term building operations
    finitely many times.
}
\bsp{}{Let $\mathcal{L} = \mathcal{L}_{arith}$ then the set of terms will contain $0$, $v_{42}$, $S0$, $SSS0$, $Sv_1$, $+SOv_1$}
\defin{Atomic formula}{Any expression of the form 
    $$=t_1t_2 \text{ or } Pt_1,\dots t_n, \text{ where $t_1,\dots t_n$ are terms and $P$ is an $n$-ary predicate symbol}$$}
\note{}{Atomic formulas are not defined inductively.}
\bsp{}{\emph{cont.}\quad $=v_1 v_{42}$, $<S0 SS0$ are atomic formulas, but $\lnot = v_1 v_{42}$ is not.}
\defin{Formulas}{We define $\varepsilon_\lnot$, $\varepsilon_\to, Q_i$ to be the fla. building operations, defined as follows
    $\varepsilon_\lnot(\alpha) \defeq (\lnot \alpha)$, $\varepsilon_\to \defeq (\alpha \to \beta)$ and
    $Q_i(\gamma) \defeq \forall v_i \gamma$.
    The set of formulas is the set of expressions built up from atomic formulas by applying the fla. building operations finitely many times.
}
\bsp{}{\emph{cont.} $\forall v_1 (=Sv_10)$ is a formula we get by applying $Q_1$ on the atomic formula $=Sv_1 0$.}
\subsubsection*{Free variables}
\bsp{}{We introduce the $\exists$\outernote{$\exists$ quantifier} quantifier by defining $\exists y \alpha$ means $\lnot \forall y \lnot \alpha$.\\
    "Every non-zero natual number is a succsesor" $\forall x (x\neq 0\to \exists y S(y)=x)$
    is different then "if a number is not $0$, then it is a succsesor" $x\neq 0\to \exists y S(y)=x$. 
    $x$ occurs bounded\outernote{bounded variable} in the first formula, for the latter $x$ occures free in the fla.

    If you have an expression without free variables, it is either true or false, on the other hand
    if a variable occurs free in a formula, the truth value of it depends on the variable itself.
}
\defin{Free variables}{
    Let $x$ be a variable. $x$ occurs \graybf{free in $\varphi$} is defined inductively as follows:
    \begin{enumerate}
        \item If $\varphi$ is an atomic fla. then $x$ occurs \graybf{free} in $\varphi$ iff $x$ occurs in $\varphi$
        \item If $\varphi = (\lnot \alpha)$ then $x$ occurs free in $\varphi$ iff $x$ occurs free in $\alpha$
        \item If $\varphi = (\alpha \to \beta)$ then $x$ occurs free in $\varphi$ iff $x$ occurs free in $\alpha$ or $\beta$
        \item If $\varphi = \forall v_i \alpha$ then $x$ occurs free in $\varphi$ iff $x$ occurs free in $\alpha$ and $x\lnot v_i$
    \end{enumerate}
    A formula $\alpha$ is called a sentence, if no variable occurs free in $\alpha$
}
\note{}{The above definition makes sense thanks to the recursion theorem.
    def function $h$ on the set of atoms: $h(\alpha) = \text{the set of var occ in fla } \alpha$, which is the set of all variables $v_i$ that occur free in $\alpha$.
    we now want to extend $h$ to $\overline{h}$, which is the set of all formulas.
    \begin{itemize}
        \item $\overline{h}(\lnot \alpha) = \overline{h}(\alpha)$
        \item $\overline{h}( \alpha \to \beta) = \overline{h}(\alpha)\cup \overline{h}(\beta)$
        \item $\overline{h}(Q_i(\alpha)) = \overline{h}(\alpha)\backslash \{v_i\}$
    \end{itemize}
    We say $x$ occurs free in $\alpha$ iff $x\in \overline{h}(\alpha)$.
}
\note{}{We will now use $\lnot, \land,\lor,\to, \leftrightarrow, \exists v_i$ (all can be expressed in terms of $\lnot,\to,Q_i$.)
    We will sometimes drop the $(,)$ and not always be using polish notation.
}
\section{Semantics of first order logic}
The equivalent scheme to our TA in predicate logic. The meaning of formulas is given by \emph{structures}, 
which also determine the scope of the quantifier $\forall$, the meaning of all parameters.
\defin{structure}{A \graybf{structure} $\mathcal{A}$ for a first order language $\mathcal{L}$ is a non-empty set
    set $A$ called \graybf{universe} or \graybf{underlying set of $\mathcal{A}$} together with an interpretation of each parameters of $\mathcal{L}$ i.e.
    \begin{itemize}
        \item $\forall$ ranges over the universe $A$
        \item for an $n$-ary pred. symbol $P\in \mathcal{L}$ its interpretation $P^\mathcal{A}$ is a subset of $A^n$
        \item for a constant $c\in \mathcal{L}$ its interpretation $c^\mathcal{A}$ is an element of $A$
        \item for an $n$-ary function symbol $f\in \mathcal{L}$ its interpretation $f^\mathcal{A}$ is a total function $f^\mathcal{A}: A^n \to A$
    \end{itemize}
}
\note{}{$A\neq \varnothing$, and all functions $f^\mathcal{A}$ are total.}
\bsp{}{Let $\mathcal{L} = \{\in\}$ where $\in$ is a binary relation "
    An example of an $\mathcal{L}$ structure is $(\NN, \in^\NN)$ where $\in^\NN = \{(x,y)\in \NN^2 : x<y\}$}
\defin{Extention of an assigment}{Let $\varphi$ be a $\mathcal{L}$-fla. and $\mathcal{A}$ a $\mathcal{L}$-structure. 
    Let $V$ be the set of all variables in $\mathcal{L}$ and $s:V\to A$ an assignment.
    We define the extention $\overline{s}$ of $s$ to the set of all $\mathcal{L}$-terms by
    \begin{itemize}
        \item $x\in V$ then $\overline{s}(x) \defeq s(x)$
        \item $c\in \mathcal{L}$ a constant symbol, then $\overline{s}(c) \defeq c^\mathcal{A}$
        \item $t_1,\dots t_n$ $\mathcal{L}$-terms and $f\in \mathcal{L}$ an $n$-ary function symbol, then 
        $\overline{s}(ft_1\dots t_n) \defeq f^\mathcal{A}(\overline{s}(t_1),\dots \overline{s}(t_n))$
    \end{itemize}
}

\note{}{in the previous definition point 3. for $n=1$ yields a commutative diagram.

}
\prop{}{For any given assignment $s$ there exists a unique extention $\overline{s}$ as in the previous definition.}
{
    will follow from recursion theorem and unique decomposition of terms.
}
\subsection*{Definition of truth}
\defin{Satisfy}{ We define '$\mathcal{A}$ satisfies $\varphi$ with $s$' and write
    $\mathcal{A}\models \varphi [s]$ inductively over the complexity of the formula $\varphi$
    \begin{itemize}
        \item if $\varphi$ is atomic: \begin{itemize}
            \item $\mathcal{A} \models = t_1,t_2 [s] \overline{s}(t_1) = \overline{s}(t_2)$
            \item $\mathcal{A} \models P t_1,\dots t_n [s] (\overline{s}(t_1),\dots \overline{s}(t_2))\in P^\mathcal{A}$
        \end{itemize}
        \item suppose $\mathcal{A}\models \varphi [s]$ and $\mathcal{A}\models \psi [s]$ are defined, then
        \begin{itemize}
            \item $\mathcal{A}\models \lnot\varphi [s]$ iff $\mathcal{A}\nvDash \varphi [s]$
            \item $\mathcal{A}\models \varphi\to \psi [s]$ iff $\mathcal{A}\models \psi [s]$ or $\mathcal{A}\nvDash  \varphi [s]$
            \item $\mathcal{A}\models \forall x \varphi [s]$ iff for all $a\in A$ $\mathcal{A}\models \varphi [s(x| a)]$
            where 
            \begin{equation*}
                s(x|a)(v) = \begin{cases}
                s(v) \text{ if } v \neq x\\
                a \text{ if } v = x
            \end{cases}
            \end{equation*}
            
        \end{itemize}
    \end{itemize}
}
\bsp{}{
    $\mathcal{L} = \{\forall, \leq, S, 0\}$ 
    a $\mathcal{L}$-structure then could be $\mathcal{N} = (\NN,\leq^\mathcal{N},S^\mathcal{N},0^\mathcal{N})$ together with an assignment
    $s: v_n \mapsto n-1$ then:
    \begin{itemize}
        \item $s(v_1) = 0$
        \item $\overline{s}(0) = 0^\mathcal{N}$ (a constant is always mapped to its realisation, the interpretation of constant $0$ in the structure $\mathcal{N}$)
        \item $\overline{s}(Sv_1) = S^\mathcal{N}(\overline{s}(v_1)) = S^\mathcal{N}(0) = 1$\\
        \item $\mathcal{N}\models \forall v_1 (S(v_1) \neq v_1) [s]$\\
            iff for all $a\in \NN$ we have that $\mathcal{N}\models (S(v_1) \neq v_1) [s(v_1 | a)]$\\
            iff \dots \\
            iff for all $a\in \NN$ we have $S^\mathcal{A}(a) \neq a$, which is true in our structure of the natural numbers.
        \item Is it true in $\mathcal{N}$ that $\mathcal{N}\models S(0) \leq S(v_1) [s]$? Yes because
        \begin{equation*}
            \begin{split}
                &\mathcal{N}\models S(0) \leq S(v_1) [s] \\
                &iff 1\leq 1
            \end{split}
        \end{equation*}
        
    \end{itemize}
}
\note{}{To know wheter $\mathcal{A}\models \varphi [s]$ it suffices to know where $s$ maps the variables that are free in $\varphi$}

\prop{}{Suppose $s_1,s_2:V\to A$ agree on all variables that occur free in $\varphi$ then 
 $$\mathcal{A}\models \varphi [s_1] \text{iff }\mathcal{A}\models \varphi [s_2]$$
}{
    By complexity of $\varphi$
    \begin{itemize}
        \item if $\varphi$ is $P t_1,\dots t_n$ 
        note: any var that occur in $\varphi$ occur free in $\varphi$, so $s_1,s_2$ agree on all variables that occur in the terms $t_1,\dots t_n$.\\
        So we Claim: for $t$ a term, $s_1,s_2$ assignments that agree on all variables of $t$ then $\overline{s}_1(t) = \overline{s}_2(t)$
        \begin{claimproof}
            By complexity of $t$
            \begin{itemize}
                \item[$t = v_m$] then  $\overline{s}_1(t) ={s}_1(v_m) = {s}_2(v_m) =\overline{s}_2(t)$
                \item[$t = c$] then  $\overline{s}_1(t) =c^\mathcal{A} =\overline{s}_2(t)$
                \item[$t = ft_1\dots t_n$] inductively, assume $\overline{s}_1(t_i) =\overline{s}_2(t_i)$ for all $1\leq i\leq n$ then  TODO
            \end{itemize}
        \end{claimproof}
        \item $\varphi: =t_1,t_2$ is similar
        \item $\varphi: \lnot \alpha$ then $\mathcal{A}\models \lnot \alpha [s_1]$ iff $\mathcal{A}\vDash \alpha [s_1]$iff $\mathcal{A}\vDash \alpha [s_2]$ iff $\mathcal{A}\models \lnot \alpha [s_1]$
        \item $\varphi: \alpha \to \beta$ then $\mathcal{A}\models  \alpha \to \beta [s_1]$ iff .. or .. iff for s2 iff ... or .. 
        \item $\varphi: \forall x \alpha$ then the assumption is that $s_1,s_2$ .. the free variables of $\alpha$ are the free variables of $\varphi$ exept for $x$.
        but because $s_1(x|a) = s_2(x|a)$ they both agree on all free variables of $\alpha$.
        \[\begin{split}
            \mathcal{A}\models \forall x \varphi [s_1] &\iif \fora a\in A \mathcal{A}\models \varphi [s_1(x| a)]\\
            & \iif \fora a\in A \mathcal{A}\models \varphi [s_2(x| a)]\\
            & \iif \mathcal{A}\models \forall x \varphi [s_2]
        \end{split}\]
    \end{itemize}
    
}
Notation: $\mathcal{A}\models \varphi $TODO means that all free variables of $\varphi$ are among $v_1,\dots v_n$ and $\mathcal{A}\models \varphi [s]$ whenever $s(v_i) = a_i$ for all $1\leq i\leq n$.
\coroll{If $\sigma$ is a sentence then $\mathcal{A}\models \varphi [s]$ for all $s:V\to A$ or 
$\mathcal{A}\vDash \varphi [s]$ for all $s:V\to A$.

Notation: $\mathcal{A}\models \sigma$ "$\sigma$ is true in $\mathcal{A}$, $\mathcal{A}$ is a model of $\sigma$ or $\sigma$ holds in $\mathcal{A}$.
}
\note{}{If $\sigma$ is a sentence then we can not have  $\mathcal{A}\models \sigma$ and  $\mathcal{A}\vDash \sigma$ because $A\neq \varnothing$.}
\defin{Model}{$\mathcal{A}$ is a model of a set of sentences $\Sigma$ iff for every sentence $\sigma\in \Sigma$ it holds  $\mathcal{A}\models \sigma$ }
\bsp{}{$\mathcal{L} = \{0,1,+,-,\cdot\}$
A realisation could be $\mathcal{R} = (\RR, 0,1,+,-,\cdot)$ or $\mathcal{C} = (\mathbb{C}, 0,1,+,-,\cdot)$
then the sentence 
$\sigma: \quad \exists x (x\cdot x = -1)$ then $\mathcal{R}\vDash \sigma$ but $\mathcal{C}\models \sigma$

}
\note{}{$\land,\lor,\leftrightarrow,\exists$ work as expected. That is 
 $\mathcal{A}\models (\alpha \land \beta) [s]$ iff $\mathcal{A}\models \alpha [s]$ and $\mathcal{A}\models \beta [s]$
 $\mathcal{A}\models (\alpha \lor \beta) [s]$ iff $\mathcal{A}\models \alpha [s]$ or $\mathcal{A}\models \beta [s]$
 $\mathcal{A}\models \exists x \alpha [s]$ iff $\mathcal{A}\models \lnot \forall x \lnot \alpha [s]$\\
 iff $\mathcal{A}\vDash \forall x \lnot \alpha [s]$\\
 iff it is not true that forall $a \in A$ $\mathcal{A}\models \lnot \alpha [s(x|a)]$\\
 iff there is $a\in A$ such that $\mathcal{A}\models \alpha [s(x|a)]$
}


\section{Logical implication}
Let $\Gamma$ be a set of $\mathcal{L}$-formulas, $\varphi$ a $\mathcal{L}$-formula.
\defin{Logical implication}{
    $\Gamma \models \varphi$ "$\Gamma$ logically implies $\varphi$" if for every $L$-structure $\mathcal{A}$ and for every $s:V\to A$ 

    if $\mathcal{A}\models \gamma [s]$for every $\gamma \in \Gamma$  then $\mathcal{A}\models \varphi [s]$

}

\defin{Logical equivalence}{$\varphi,\psi$ are called logically equivalent if 
$\varphi \models \psi$ and $\psi \models \varphi$.
}
\defin{Valid}{$\varphi$ is called valid iff $\models \varphi$ i.e. $\varnothing\models \varphi$ i.e. for every $\mathcal{L}$-structure $\mathcal{A}$ and every $s:V\to A$ it is $\mathcal{A}\models \varphi [s]$
}
\bsp{}{
\begin{enumerate}
    \item $\forall x_1 P x_1 \models P x_2$\\
    Suppose $\mathcal{A}\models \forall x_1 P x_1 [s]$.
    then for all $a\in A$ it is $\mathcal{A}\models P x_1 [s(x_1|a)]$ in particular, $a\in P^\mathcal{A}$ for $a = s(x_2)$
    \item  $\forall P x_2 \vDash \forall x_1 P x_1$\\
    We need a counterexample to $\forall P x_2 \models \forall x_1 P x_1$. Let $A = \{a_1,a_2\}$ $s(x_2) = a_1$ and $P^\mathcal{A} = \{a_1\}$
    then $\mathcal{A}\models Px_2[s]$.
    \item Is the following valid? $\models \exists x(Px \to \forall y P y)$ yes
    \item $\Gamma,\alpha\models \varphi$ iff $\Gamma \models \alpha \to \varphi$. (on next problem set, quite impoirtant)
\end{enumerate}
}

\section{definability in a structure}
\bsp{}{
    \begin{enumerate}
        \item $x = x$ would define the entire universe.
        \item $\lnot x = x$ would define the empty set.
    \end{enumerate}
}
\defin{definability in a structure}{ 
    We say that a general $n$-ary relation $P$ on $A$ (we will just call it $P$, it does not have to be in the language) is definable in $\mathcal{A}$, if there is a $\mathcal{L}$-formula
    $\varphi$ with free variables among $\{v_1,\dots,v_n\}$ such that 
    \[P = \{(a_1,\dots a_n) : \mathcal{A} \models \varphi [a_1,\dots a_n]\}\]
    We also say that $\varphi$ defines $P$ in the structure $\mathcal{A}$.
}
\bsp{}{
\begin{enumerate}
    \item decomposition
    \item $\mathcal{R} = (\RR,0,1,+,-,\cdot)$ Q: is $[0,\infty)$ definable in $\mathcal{R}$
    Yes because $\exists y (y\cdot y = x)$
    Indeed we can even define the $\leq$ relation on $\RR^2$ by $x\leq z \defaq \exists y (x+y\cdot y = z)$ 
\end{enumerate}
}
\defin{definability of classes of structures}{
    Let $\Sigma$ be a set of sentences. $\tau$ a sentence.
    We will say that the class of models of $\Sigma$ is the class $Mod \Sigma =\{\mathcal{A} : \mathcal{A}\models \Sigma\}$.
    Let $K$ be a class of structures. We are going to call $K$ an elementary class (EC) if there is a single sentence $\tau$ such that $Mod \tau = K$
    $K$ is an elementary class in the wider sence (EC$\Delta$) if there is a set of sentences $\Sigma$ such that $Mod \Sigma = K$

}
\bsp{}{$\mathcal{L} = \{0,1,+,\cdot\}$ 
    $\tau$ is a sentence that expresses the field axioms (the unary inverse functions are not in our language but are definable.)
    $Mod \tau$ is the class of all the fields, which is EC.
    the class of all fields of characteristic $0$. Let $\sigma_p: \lnot(1 + \dots +1 = 0)$ then $\Sigma = \{\tau\} \cup \{\sigma_p : p\in \mathbb{P}\}$ 
    yields $Mod \Sigma$ is the class of fields with characteristic $0$, therefore EC$\Delta$, we will later see that it is not $EC$.

}
\bsp{}{
    Let $E$ be a binary relation, $\mathcal{L} = \{E\}$ then a graph is a realisation $\mathcal{G} = (V,E^\mathcal{G})$ 
    such that $v\neq \varnothing$, $E^\mathcal{G}$ is irreflexive and symmetric.
    By definition the universe is not empty, we still have to check irreflexive and symmetric.
    \begin{itemize}
        \item irreflexive: $\forall x (\lnot x E x)$
        \item symmetric: $\forall x \forall y (x E y \to y E x)$
    \end{itemize}
    We take $\tau$ to be $\forall x \forall y ((\lnot x E x)\land (x E y \to y E x))$
    Then $Mod \tau$ is the class of all graphs and is EC
    Note: the class of all finite graphs is neither EC nor EC$\Delta$. proof later.
}
We want to have some notion that tells us when two graphs are the same or at least similar.
\section{Homomorphisms of structures}
\defin{Homomorphism}{
    Suppose that $\mathcal{A},\mathcal{B}$ are two $\mathcal{L}$-structures. then a Homomorphism of $\mathcal{A}$ into $\mathcal{B}$ is a map
    $h: A \to B$ that satisfy the below conditions
    \begin{itemize}
        \item for every $n$-ary predicate $P\in \mathcal{L}$ it is $(a_1,\dots a_n)\in P^\mathcal{A}$ iff $(h(a_1),\dots h(a_n))\in P^\mathcal{B}$ (this def. a strong Homomorphism, other textbooks maybe only reqire $\to$ direction)
        \item for every $n$-ary function $f\in \mathcal{L}$ and for all $\underline{a} = (a_1,\dots a_n)\in A^n$ it holds $h(f^\mathcal{A}(\underline{a})) = f^\mathcal{B}(h(a_1),\dots h(a_n))$
        \item for every constant symbol $c\in \mathcal{L}$ it is $h(c^\mathcal{A}) = c^\mathcal{B}$ (could also skip this if we consider constants as $0$-ary functions)
    \end{itemize}
}
\note{}{Intuativly a Homomorphism of $\mathcal{A}$ into $\mathcal{B}$ is a map $A \to B$ that preserve all funciton and relation symbols in some sense, (imp: not the definable relations)}
\defin{Isomorphism}{
    \begin{itemize}
        \item $h:A\to B$  is called isomorphism of $\mathcal{A}$ into $\mathcal{B}$ if $h$ is a Homomorphism and injective (in other textbooks: an isomorphic embedding of $\mathcal{A}$ into $\mathcal{B}$)
        \item $h:A\to B$  is called isomorphism of $\mathcal{A}$ onto $\mathcal{B}$ if $h$ is a Homomorphism and bijective $A\to B$
        \item \outernote{isomorphic}$\mathcal{A}$ and $\mathcal{B}$ are called isomorphic if there is an isomorphism of $\mathcal{A}$ onto $\mathcal{B}$
    \end{itemize}
}
\note{}{
    %TODO: in general 
}
\bsp{}{
    $\mathcal{L}= \{+,\cdot\}$
    $\mathcal{N} = (\NN, +^\NN, \cdot^\NN)$ and $\mathcal{B} = (B,+^\mathcal{B},\cdot^\mathcal{B})$ where $B = \{0,1\}$ and 
    \begin{tabular}{c|c c}
        $+^\mathcal{B}$ & $e$ & 0\\\hline
        $e$ & $e$ & 0\\
        0 & 0 & $e$
    \end{tabular}
    \begin{tabular}{c|c c}
        $\cdot^\mathcal{B}$ & $e$ & 0\\\hline
        $e$ & $e$ & $e$\\
        0 & $e$ & 0
    \end{tabular}
    let $h:\NN \to B$ a Homomorphism?
    $h(n) = \begin{cases}
        e & \text{ if $n$ is even}\\
        0 & \text{ else}
    \end{cases}$
    need at first that $h(m+n ) = h(m) +^\mathcal{B} h(n)$
    and $h(m\cdot n) = h(m)\cdot^\mathcal{B} h(n)$.
    it is indeed a Homomorphism.
}
\defin{Substructure}{Suppose we have two $\mathcal{L}$ structures and $A\subseteq B$ hen $\mathcal{A}$ is a substructure of $\mathcal{B}$ (notation: $\mathcal{A}\subseteq \mathcal{B}$ or we might say $\mathcal{B}$ is an extention of $\mathcal{A}$ ) if
    \begin{itemize}
        \item for every $n$-ary relation $P^\mathcal{A} = P^\mathcal{B}|_A$
        \item for every $n$-ary function $f^\mathcal{A} = f^\mathcal{B}|_A$
        \item for every constant symbol $c$ in $\mathcal{L}$ it is $c^\mathcal{A} = c^\mathcal{B}$
    \end{itemize}
}
\bsp{}{
    $\mathcal{L} = \{\leq\}$ then $\mathcal{N} = (\NN, \leq)$ and $\mathcal{P} = (\NN^+, \leq ^\mathcal{P})$ where $\leq^\mathcal{P}$ is the restriction of $\leq$ to the positive natual numbers.
    $\mathcal{P} \subseteq \mathcal{N}$ and there exists a isomorphic embedding $id: \NN^+\to \NN$ from $\mathcal{P}$ into $\mathcal{N}$
    They are even isomorphic ($h:\NN \to \NN^+, h(n) = n+1$) so in fact $\mathcal{P} \cong \mathcal{N}$.
}
\bsp{}{
    $(\mathbb{Q},+)\subseteq(\mathbb{C},+)$
}
\note{}{If $\mathcal{A}\subseteq \mathcal{B}$ then in particular $\mathcal{A}$ is closed under all constant and functions in $\mathcal{B}$
    So suppose that $\mathcal{B}$ is a substructure and $A\subseteq B$ and $A\neq \varnothing$ and $A$ is closed under $f^\mathcal{B}$, $c^\mathcal{B}$ 
    Can then $A$ be made into a substructure $\mathcal{A}$ of $\mathcal{B}$.
    $f^\mathcal{A}$ would be the restriction of $f^\mathcal{B}$ to $A^n$, constants $c^\mathcal{A} = c^\mathcal{B}$ 
    and if $P \in \mathcal{L}$ is an $n$-ary predicate then $P^\mathcal{A}$ should be $P^\mathcal{B}\cap A^n$.
    If $\mathcal{L}$ has no const. or fuction symbols then any subset can be made into a substructure of a structure on $\mathcal{L}$.
}
Our next question will be: what is the relation of the above notions with truth and satisfiability
The answer will be given by the so called Homomorphism theorem.

\prop{Homomorphism theorem}{
    $h$ homomorphism of $\mathcal{A}$ into $\mathcal{B}$, $s:V\to A$ then
    \begin{enumerate}
        \item for all terms $t$ it is $h(\overline{s}(t))=\overline{(h\circ s)}(t)$
        \item $\varphi$ a fla. that is quantifier free and does not include $=$ then $\mathcal{A}\models \varphi [s]$ iff $\mathcal{B}\models \varphi [h\circ s]$
        \item if $h$ is additionally injective then we can drop the requirement " no $=$".
        \item if $h$ is homomorphism of $\mathcal{A}$ onto $\mathcal{B}$ then we can drop the requirement "q.f." in (b)
    \end{enumerate}
}{
    \begin{enumerate}
        \item problem set
        \item \begin{itemize}
            \item $\varphi: Pt$ then $\mathcal{A}\models Pt[s]$ iff $\overline{s}(t)\in P^\mathcal{A}$ iff $h(\overline{s}(t))\in P^\mathcal{B}$ iff $\overline{(h\circ s)}(t)\in P^\mathcal{B}$ iff $\mathcal{B}\models Pt [h\circ s]$
            \item $\varphi: \lnot \psi$ $\mathcal{A}\models \lnot \psi[s]$ iff $\mathcal{A}\vDash \psi[s]$ iff $\mathcal{A}\vDash \psi[s]$ iff %TODO
            \item $\varphi: \psi \to \alpha$
        \end{itemize}
        \item $\mathcal{A}\models =t_1t_2 [s]$ iff $\overline{s}(t_1)=\overline{s}(t_2)$ iff $h(\overline{s}(t_1))=h(\overline{s}(t_2))$ iff (by (a))$\overline{(h\circ s)}(t_1) = \overline{(h\circ s)}(t_2)$ iff $\mathcal{B}\models =t_1t_2[h\circ s]$
        \item $\varphi$ $ \forall s: V\to A$ $\mathcal{A}\models \varphi [s]$ iff $\mathcal{B}\models \varphi [h\circ s]$, want $\mathcal{A}\models \forall x \varphi [s]$ iff $\mathcal{B}\models \forall x \varphi [h \circ s]$
        1. $\mathcal{B}\models \forall x \varphi [(h\circ s)]$
        iff for all $s:V\to A$, $a\in A$ (req. surjectivity) it is $\mathcal{B}\models \varphi [(h\circ s)(x | h(a))]$
        iff $\mathcal{B}\models \varphi [h\circ (s(x |a))]$
        iff (inductive assumption)$\mathcal{A}\models \varphi [s(x|a)]$ because $a$ was arbitrary it is $\mathcal{A}\models \forall x \varphi [s]$
        2. Suppose $\mathcal{B}\vDash \forall x \varphi [(h\circ s)]$ then there exists a $b\in B$ such that $\mathcal{B}\models \lnot \varphi [(h\circ s)(x | b)]$
        by surjectivity we can find $a\in A$ such that $h(a)=b$ and it is $\mathcal{B}\models \lnot \varphi [(h\circ s)(x | h(a))]$
        By the inductive assumption $\mathcal{A}\models \lnot \varphi [s(x|a)]$ and $\mathcal{A}\vDash \forall x \varphi [s]$
    \end{enumerate}
}
\note{}{$\mathcal{A}\cong \mathcal{B}$ then $\mathcal{A}$ and $\mathcal{B}$ satisfy exactly the same sentences.}
\defin{elementarily equivalent}{$\mathcal{A}$ and $\mathcal{B}$ are called elementarily equivalent ($\mathcal{A}\equiv \mathcal{B}$) if $\mathcal{A}$ and $\mathcal{B}$ satisfy the same sentences.}
\note{}{If $\mathcal{A}\cong \mathcal{B}$ implies $\mathcal{A}\equiv \mathcal{B}$
    The converse is not true.
    For instance DLO (dence linear order) w/o endpoints is complete, so two structures on DLO are equivalent 
    $(\mathbb{Q},<)\equiv (\RR,<)$ but they are not isomorphic because the universes have diffrent cardinality. 
}
\bsp{}{
    $\mathcal{N} = (\NN, \leq)$ and $\mathcal{P}=(\NN^{>0},\leq )$ $h:n\mapsto n-1:\mathcal{P}\to \mathcal{N}$ isom. 
    so in part $\mathcal{N} \equiv \mathcal{P}$.
    but $id:\mathcal{P}\to \mathcal{N}$ is only isom embedding, so for example
    $\forall y(x\neq y x\leq y)$  
    $\mathcal{P}\models \alpha [1]$ but $\mathcal{N}\nvDash \alpha [1]$
    but $\mathcal{N}\models \alpha [h(1)]$
}
\defin{Automorphism}{An automorphism is an isomorphism of the form $h:A\to A$ from $\mathcal{A}$ onto $\mathcal{A}$}
\note{}{Every structure has a trivial automorphism $id:A\to A$}
\defin{Rigid}{If the only automorphism on $\mathcal{A}$ is the trivial automorphism, then $\mathcal{A}$ is called rigid.}
\bsp{}{ If every element is definable then the structure is rigid.
    For example $(\NN,0,S)$ and $(\NN,<)$ every element is definable, therefore the structures are rigid.
    %TODO examples structures with many automorphism
    }
\coroll{
    Let $h$ be atutom of $\mathcal{A}$, $R\subseteq A^n$ definable in $\mathcal{A}$ then $\forall a\in A^n a\in R \iif (h(a_1),\dots h(a_n))\in R$ 
    Suppose $\varphi$ defines $R$ in $\mathcal{A}$ we want
    $\mathcal{A}\models \varphi[a]$ iff  $\mathcal{A}\models \varphi[h(a_1),\dots h(a_n)]$ which is true by the homom. thm.
}
\note{}{Corol can be used to show that some $R\subseteq A^n$ is not definable in $\mathcal{A}$
}
\bsp{}{
    $\mathcal{R}=(\RR,<)$ then $\NN$ is not definable in $\mathcal{R}$.
    What do automorphisms of $\mathcal{R}$ look right? 
    $h:\RR\to \RR$ is a bijection and $x<y$ iff $h(x)<h(y)$ so $h$ is strictly increasing.
    for example $x\mapsto x+ \frac{1}{2}$ or $x\mapsto x^3$.  
}
\section{Unique readability for terms}
\defin{}{
    We define $K$ on symbols from which terms are built up(variables, constants, function symbols).
    $K(s) = 1-n$ where $s$ is a symbol and $n$ is the number of terms that need to follow s in order to obtain a term.
    $K(x) = 1 = K(c)$ and $K(f) = 1-n$ where f is an $n$-ary funciton symbol
    We now extend $K$ to the set of all expressions which are built up from above symbols (variables, constants, function symbols):
    $K(s_1,\dots s_n) = K(s_1) + \dots + K(s_n)$ (unique because no symbol is a finite sequence of other symbols)
}
\lemma{}{
    $t$ a term then $K(t)=1$
}{
    $K(x) = 1 = K(c)$ and $K(ft_1,\dots t_n) = 1-n + n = 1$ 
}
\defin{}{A terminal segment of string of symbols $(s_1,\dots s_n)$ is $(s_k,s_{k+1},\dots s_n)$ for some $1\leq k\leq n$.}
\lemma{}{Any terminal segment of terms is a concatenation of one or more terms.}{
    True for variables and constants. 
    $ft_1\dots t_n$ the only non trivial case is $t'_k t_{k+1}\dots t_m$ where $t_k$ is $t''_k t_k'$
}
\coroll{If $t_1$ is a proper initial segment of a term $t$ then 
 its $K(t_1)<1$. proof:
    let $t$ be $t_1 t_2$ where $t_1$ is a proper initial segment then $K(t) = 1$ and $K(t_2)\geq1$ therefore $K(t_1)\leq 0$
}
\subsubsection{Unique readability for terms}
The set of terms is freely generated from the set of variables(Var), the set of constant symbols (Const) by the term building operations $\mathcal{F}_f$ for the function symbols $f$.
\begin{proof}
    
     \begin{itemize}
        \item disjointment of ranges: Let $f$ and $g$ be two distinct funciton symbols then $\rng\mathcal{F}_f \cap \rng\mathcal{F}_g =\varnothing$
        $\rng\mathcal{F}_f \cap Var =\varnothing$
        $\rng\mathcal{F}_f \cap Const =\varnothing$
        \item $\mathcal{F}_f|_{\text{terms}}$ are 1-1:
        assume $ft_1\dots t_n = f t'_1\dots t'_n$ and assume $t_1 \neq t_1'$ then one is an initial segment of the other.
        Then its $K$-value has to be less than $1$ so it is not a term.
        $t_1 = t_1' \dots t_n=t_n'$.
     \end{itemize}
\end{proof}
\defin{}{Extend $K$ as follows: $K(() = -1$ $K()) = 1$ $K(\forall) = 1$ $K(\lnot) = 0$ $K(\to) = -1$ $K(P) = 1-n$ for an $n$-ary rel. symb. $P$. $K(=) = -1$.
    Extend $K$ to the set of all expressions by $K(s_1,\dots s_n) = K(s_1) + K(s_n)$
    The idea is that $K$ tells us the number of symbols that at least need to follow to obtain a formula.
}
\lemma{}{
    for every formula $\varphi$ it is $K(\varphi) = 1$
}{
    induction on $\varphi$
}
\lemma{}{for every proper initial segment $\alpha'$ of a fla. $\alpha$ we have $K(\alpha')<1$}{}
\coroll{No proper initial segment of a fla. is a fla.}
The set of flas. is freely generated from the set of atomic flas. by operations $\mathcal{E}_\lnot,\mathcal{E}_\to,Q_i$
\begin{proof}
    \begin{itemize}
        \item $\mathcal{E}_\lnot,Q_i$ are one to one
        \item $\mathcal{E}_\to|_{\text{Flas.}}$ then itemwise and use of prev. lemmas
        \item p.w. disjointness of ranges
    \end{itemize}
\end{proof}
\section{A parsing algorithm for first order logic}

\section{Deductions (formal proofs)}
\section{Generalization and deduction theorem}
TODO evt noch sectioons

\chapter{Model Theory}
The sections \ref{mt:sec:LST} to \ref{MT:sec:NSA} are sourced from \cite[chapter~2]{EndertonHerbertB2001AMIt} and and the theory of o-minimality (from \ref{MT:sec:o-min} onwards) can be found in \cite{van1998tame}.
\section{LST-Theorem}\label{mt:sec:LST}
LST stands for Löwenheim-Skolem-Tarski and is the combination of the ``upward Löwenheim-Skolem theorem'' with the ``downward Löwenheim-Skolem theorem''.
\thm{LST-Theorem}{Let $\Gamma$ be a set of $\mathcal{L}$-formulas. $|\mathcal{L}| = \lambda$ and lets assume 
    $\Gamma$ is satisfiable in some infinite structure.\\
    Then for every cardinal $\kappa\geq \lambda$, $\Gamma$ is satisfiable in a structure of cardinality $\kappa$.
}{ %2024-12-13: TODO check proof
    add $\kappa$ many new constants to the language $\mathcal{L}$.

    $\mathcal{L}' = \mathcal{L}\cup \{c_\alpha : \alpha < \kappa\}$

    $\Sigma = \{c_\alpha \neq c_\beta : \alpha\leq \beta, \ \alpha,\beta\leq \kappa\}$

    Then $\Gamma\cup \Sigma$ is finitly satisfiable in $\mathcal{L}'$.
    This is because $\Gamma$ is satisfiable in some infinite structure. By compactness $\Gamma\cup \Sigma$ is satisfiable.
    We have $\mathcal{A}\models \Gamma \cup \Sigma$ then $|\mathcal{A}|\geq \kappa$.

    By the proof of completeness theorem, $\Gamma\cup \Sigma$ has a model of size $\leq \kappa$.
    Hence it is exactly of size $\kappa$. Take the reduct of $\mathcal{A}$ to the language $\mathcal{L}$.%TODO: define reduct
}
\bsp{}{The language of ZFC $\mathcal{L} = \{\in\}$ is countable, so Löwenheim-Skolem guaranties that ZFC has a countable model.
But ZFC knows that there are uncountable sets (see Cantors Theorem \ref{5:Thm:Cantor}).
This is called skolems paradox.
explanation: some bijections are missing
}
\bsp{}{\label{ComplNotImplyCath}\begin{enumerate}
    \item $\overline{\RR}$ real field. $\Thm(\overline{\RR})$ has a countable model. $\RR_\text{alg}$
    \item $\mathcal{N} = (\NN,0,S,+,\cdot)$\\
    Claim: there exists a countable structure $\mathcal{M}$ such that $\mathcal{N}\equiv \mathcal{M}$ but $\mathcal{N}\ncong \mathcal{M}$
    One way is to add new constant c to language
    $\Sigma = \{0<c, S0<c,\dots\}$ is fin satisfiable. So $\Sigma\cup Th (\mathcal{N})$ is fin satisfiable by compactness it is satisfiable

    Take the reduct to original language. $\mathcal{M}$. and $\mathcal{M}$ not isomorphic to $\mathcal{N}$, bc
    A bijection of $M\to \NN$ would have to map $c$ somewhere but for every $S^k0<c$ for every $k$ wont be preserved by any map.
\end{enumerate}
}
\section{Theories and completeness}
\defin{Theory}{A theory $T$ is a set of sentences that is closed under logical implication.
\[T\models \sigma \implies \sigma \in T\]
}
\note{}{If $\mathcal{L}$ is a language. Then 
\begin{itemize}
    \item there is a smallest $\mathcal{L}$-theory. The set of all valid $\mathcal{L}$-sentences.
    \item and also a largest $\mathcal{L}$-theory. The set of all $\mathcal{L}$-sentences.
\end{itemize}
}


\defin{Theory of structures}{Let $\mathcal{K}$ some class of $\mathcal{L}$- structures. Then 
    $$\Th(\mathcal{K}) = \{\sigma : \sigma \text{ $\mathcal{L}$-sentence and for every $K \in \mathcal{K}$ } \sigma\in \Th(K)\}$$
}
\note{}{$\Th(\mathcal{K})$ is a theory. 
\[\text{if }\Th(\mathcal{K})\models \sigma \text{ then }  \sigma\in \Th(\mathcal{K}) \]
}

\bsp{}{

\begin{itemize}
    \item $\mathcal{L} = \{0,1,+,\cdot,-\}$ $\mathcal{F}$ the class of fields then $\Th(\mathcal{F})$ is the set of sentences truein every field.
\end{itemize}
}
%2024-12-13: TODO edit below
Recall that $\Mod(\Sigma)$ is the class of all models of $\Sigma$.
$\Th(\Mod \Sigma)$ might not be the set $\Sigma$ but it is the set of all 
sentences true in all models of $\Sigma$.
Which is the set of all sentences that are logically implied by $\Sigma$ 

Or in other words: The set of all consequences of $\Sigma$
\defin{$C_n$}{$ C_n(\Sigma)\defeq \Th(\Mod \Sigma)$
}
\note{}{$\Sigma$ is a theory iff $C_n(\Sigma) = \Sigma$}

\defin{}{We say that a theory $T$ is complete, if for every sentence 
$\sigma$ either $\sigma\in T$ or $\lnot \sigma\in T$.}

\bsp{}{
    $\mathcal{A}$ a $\mathcal{L}$-structure, then $\Th(\mathcal{A})$ is complete.
}

\note{}{$\Th(\mathcal{K})$ is complete, iff any $K_1,K_2\in \mathcal{K}$ are elementarily equivalent.

A theory $T$ is complete iff any to models are elementarily equivalent.
}

\bsp{}{
    \begin{itemize}
        \item The theory of fields is not complete.
        \item The theory of algebraically closed fields of characteristic $0$ is complete 
        (That is non-trivial)
    \end{itemize}
}
\defin{axiomatizability}{
    \begin{itemize}
        \item A theory $T$ is finitely axiomatizable if there is a sentence 
        $\sigma$ such that $C_n(\sigma) = T$.
        \item A theory $T$ is axiomatizable, if there is a decidable set 
        $\Sigma$ such that $C_n(\Sigma) = T$.
    \end{itemize}
}
\bsp{}{
    \begin{itemize}
        \item The theory of fields (common theory of all fields) is finitely axiomatizable.
        \item Theo theory of fields of characteristic $0$ is axiomatizable. $\Psi\cup \{1+1\neq 0, 1+1+1\neq 0,\dots\}$ 
        It is however not finitely axiomatizable. If 
        $\Psi_0\subseteq \Psi\cup \{1+1\neq 0, 1+1+1\neq 0,\dots\}$ 
        finite, then $\Psi_0$ has a model of characteristic $p$ for some sufficiently large $p$.
    \end{itemize}
}
\thm{}{If $C_n(\Sigma)$ is finitlely axiomatizable then there exists a finite subset           
    $\Sigma_0\subseteq \Sigma$ such that $C_n(\Sigma_0) = C_n(\Sigma)$}{
    Suppose $C_n(\Sigma)$ is finitely axiomatizable. So $C_n(\sigma) = C_n(\Sigma)$. 
    Then there is a finite subset $\Sigma_0\subseteq \Sigma$ such that $\Sigma_0 \models \sigma$.
    And we get $C_n(\Sigma_0) = C_n(\Sigma)$
}
\defin{}{A theory $T$ is $\aleph_0$-categorical, if any two infinite countable models of 
    $T$ are isomorphic.
    Futhermore for some infinite cardinal $\kappa$ a theory $T$ is called $\kappa$-categorical, if every two models of cardinality $\kappa$ are isomorphic.
}
\thm{Los-Vaught test}{For a theory $T$ in a countable language with only infinite models it holds\\
    If $T$ is $\kappa$-categorical for some infinite cardinality $\kappa$ then $T$ is complete.
}{
    Let $T$ be $\kappa$-categorical.
    Want: If $\mathcal{A},\mathcal{B}\models T$ then $\mathcal{A}\equiv \mathcal{B}$.
    Note: both $\mathcal{A}$ and $\mathcal{B}$ are infinite.
    By LST there exists structures $\mathcal{A}'$ and $\mathcal{B}'$ with $\mathcal{A}\equiv \mathcal{A}'$ and $\mathcal{B} \equiv \mathcal{B}'$ and $|\mathcal{A}'|, |\mathcal{B}'| = \kappa$.
    By $\kappa$-categorical we have $\mathcal{A}' \cong \mathcal{B}'$ so $\mathcal{A}\equiv \mathcal{B}$
}
\note{}{completness does not imply categorical.
\begin{itemize}
    \item The theory of natural numbers is not $\aleph_0$-categorical. See Example \ref{ComplNotImplyCath}
    \item RCF not $\kappa$-categorical for all infinite cardinalities $\kappa$
    Not $\aleph_0$ categorical real clo of $\mathbb{Q}(\pi)$, real closure of $\mathbb{Q}$
    not uncountable categorical $\overline{\mathbb{R}}$, $\overline{\mathbb{R}(\varepsilon)}, 0<\varepsilon <\frac{1}{n}$ for every $n\in \NN$.
\end{itemize}
}
\section{Theory of algebraic closed fields}
\thm{}{The theory of algebraic closed fields of characteristic $p$ ACF$_p$, where $p$ is either $0$ or prime is complete.\cite[Theorem 26J, p.158]{EndertonHerbertB2001AMIt}}{Note that we have a
\begin{itemize}
    \item  countable language
    \item with no finite models
\end{itemize}
Let $\mathcal{K}_1,\mathcal{K}_2\models \text{ACF}_p$ such that $|K_1| = |K_2|=\kappa$ uncountable.
$F_1$ prime field of $\mathcal{K}_1$, $F_2$ prime field of $\mathcal{K}_2$.

Note $F_1, F_2$ are determined by $p$ if $p=0$ then $F_1 = F_2 = \mathbb{Q}$ and if $p$ prime then $F_1 = F_2 = \mathbb{F}_p$

Define $F \defeq F_1 = F_2$.
$B_1$ trancendence base of $\mathcal{K}_1$ over $F$ 
$B_2$ trancendence base of $\mathcal{K}_2$ over $F$ 
\begin{itemize}
    \item $B$ is trancendence base of $K$ over $F$ if $B$ is a $\subseteq$-maximal subset of $K$ which is algebraically closed then 
    \item $B\subseteq K$ is algebraically TODO
\end{itemize}
$F(B_1)$, $F(B_2)$ subfields of $\mathcal{K}_1,\mathcal{K}_2$
\begin{itemize}
    \item alg cl $F(B_1) = \mathcal{K}_1$
    \item alg cl $F(B_2) = \mathcal{K}_2$
\end{itemize}
Fact: Let $F$ subfield of $K$. if $F$ is countable and $K$ uncountable, then any transe basis $B$ of $K$ oer $F$ is of cardinality $|K|$, hence uncountable.

Steinitz: Two ACF are isomorphic iff they have the same characteristic and there trancendence spaces have the same cardinality.
}
\subsubsection{Lefschetz Principle}%2024-12-13: TODO source needed
\prop{Lefschetz Principle}{Let $\mathcal{C} = (\mathbb{C},0,1,+,\cdot,-)$ For a sentence in the language of $\mathcal{C}$
    Then the following are equivalent:
    \begin{itemize}
        \item $\mathcal{C}\models \sigma$
        \item $\mathcal{A}\models \sigma $ for every $\mathcal{A}\models $ ACF$_0$
        \item ACF$_0$ $\models \sigma$
        \item for all sufficiently large primes $p$ ACF$_p$ $\models \sigma$
        \item For infinitely many primes $p$ ACF$_p$ $models \sigma$
    \end{itemize}}{ 
    Sketch:
    \begin{itemize}
        \item[(a), (b), (c)] are equivalent by completeness of ACF$_0$
        \item[(c)$\implies$ (d)] ACF$_0$ $\models \sigma$ 
        so there is $T_0\subseteq \text{ACF}_0$ such that $T_0\models \sigma$ therefore there exists a sufficiently large prime $p$ such that $\text{ACF}_p\models \sigma$. 
        \item[(d) $\implies$ (e)]TODO
        \item[(e) $\implies$ (c)] If $\text{ACF}_0\vDash \sigma$ than $\text{ACF}_0\models \sigma$
    \end{itemize}
}
Example of the Lefschetz Principle:
\prop{Ax–Grothendieck}{\footnote{Alexander Grothendieck}
    Let $f:\mathbb{C}^n \to \mathbb{C}^n$ be a polynomial map. If $f$ is injective, then $f$ is surjective.
}{
    Our language is $\mathcal{L} = \{0,1,+,-,\cdot\}$. Note that there is an $\mathcal{L}$-sentence $\Phi_d$ such that a Field $F$ 
    
    $F\models \Phi_d$ iff for every polynomial map $f:F^n\to F^n$ whose TODO coord. function is of degree at most $d$, if $f$ is injective then $f$ is surjective.

    By Lefschetz principle it is enough to show for sufficiently large 
    primes $p$, $\text{ACF}_p\models \Phi_d$ for all $d\in \NN$.
    Since $\text{ACF}_p$ is complete, it is enough to show that every injective polynomial map $f:K^n\to K^n$ is surjective, where $K = TODO$
    Let $f:K^n\to K^n$ be a polynomial map.

    Then there is a finite subfield $K_0$ of $K$ such that all coefficients of $f$ come from $K_0$.
    Let $y\in K^n$. Then there is a finite subfield $K_1$ of $K$ such that $y\in K_1$ and $K_0\subseteq K_1 \subseteq K$. Since $f:K^n_1\to K^n_1$ is injective and $K_1$ finite, $f|_{K_1}$ is surjective onto $K_1$. So there is $x\in K_1^n$ such that $f(x)=y$.
}
\note{}{Later, a purely geometric proof was found by Borel.}

Another use of \L oś-Vaught

\prop{}{\[(\mathbb{Q},<_\mathbb{Q}) \equiv (\RR, <_\RR)\]}{%2024-12-13: TODO from here on \prop might actually be theorems, check
    $\mathcal{L} = \{<\}$ and note that both $(\mathbb{Q},<_\mathbb{Q}), (\RR, <_\RR)$ are DLO without endpoints, i.e. they satisfy the following axioms
    \begin{enumerate}
        \item $\forall x \forall y (x<y\lor x=y\lor y<x)$
        \item $\forall x \forall y (x<y\to \lnot(y<x))$
        \item $\forall x \forall y \forall z ((x<y\land y<z)\to x<z)$
        \item $\forall x \forall y (x<y\to \exists z(x<z\land z<y))$
        \item $\forall x \exists y \exists z (y<x\land x<z)$
    \end{enumerate}
    TODO
}

\section{Nonstandard Analysis}\label{MT:sec:NSA}
%TODO historical background
\begin{enumerate}
    \item Language $\mathcal{L}$: $=$, $\forall$ ranging over $\RR$,
    \begin{itemize}
        \item $P_R$ TODO
    \end{itemize}
    \item standard structure for $\mathcal{L}$: $\mathcal{R}$ with universe $\RR$, $c_r^\mathcal{R} = r$, $P_R^\mathcal{R} = R$, $f^\mathcal{R}_F = F$.
    \item Nonstandard structure for $\mathcal{L}$: $\mathcal{R}^*$, which is constructed using the compactness theorem 
    \[\Gamma \defeq \Th(\mathcal{R}) \cup \{c_r P_< v_1 : r\in \RR\}\]
    Compactness theorem $\implies$ there exists a $\mathcal{L}$-structure $\mathcal{R}^*$ with $\mathcal{R}^*\models \Gamma [(v_1 | a)]$ for some $a\in R^*$. We have $\mathcal{R}\equiv \mathcal{R}^*$.
    Moreover, $h: \RR \to R^*$ defined by $r\mapsto c_r^*$ is an isomorphism of $\mathcal{R}$ into $\mathcal{R}^*$
    \begin{itemize}
        \item $h$ is injective:
        \item TODO
    \end{itemize}
\end{enumerate}
\note{}{WMA $\mathcal{R}$ substructure of $\mathcal{R}^*$ (se PS)}
Notation: We will write $\!^*B$ instead of $P_B^{\mathcal{R}^*}$.
\bsp{}{what is $\!^*\RR$?
We have that $\mathcal{R}\models \forall x P_\RR$, hence $\mathcal{R}^*\models \forall x \!^*\RR$, so $\!^*\RR = R^* = \text{universe of } \mathcal{R}^*$.
\note{}{
    Let $F$ be an $n$-ary operator on $\RR$. Then $F$ is the restriction of $\!^*F$ to $\RR$. $\!^*c_r = r$.
}
Idea: If we want to show that $\!^*R$ or $\!^*F$ has certain property, then we show
\begin{itemize}
    \item $R$ or $F$ have that property.
    \item property can be expressed in $\mathcal{L}$
\end{itemize}
}
TODO

%26.11.2024
$\mathcal{R}^*\supseteq \mathcal{R}$ such that $\mathcal{R}^*\equiv \mathcal{R}$.
$\mathcal{F} = \{x\in \mathcal{R}^* : \exists r\in \RR \: ^*|x|^* \leq r\}$
$\mathcal{I} = \{x\in \mathcal{R}^* : \forall r\in \RR \: ^*|x|^* < r\}$
\prop{}{
    \begin{enumerate}
        \item $\mathcal{F}$ is a subring of $\mathcal{R}^*$
        \item $\mathcal{I}$ is an ideal in $\mathcal{R}*$
    \end{enumerate}
}{
    \begin{enumerate}
        \item Let $x,y\in \mathcal{F}$ then there exists $a,b\in \RR^{>0}$ such that $^*|x|^* \leq a$ and $^*|y|^* \leq b$. then \[
        \begin{aligned}
            ^*|x \pm^* y|^* \leq ^*|x|^* + ^*|y|^* \leq a+b\in \RR^{>0}\\
            ^*|x \cdot^* y|^* = ^*|x|^*\ ^*\cdot ^*|y|^* \leq a\cdot b\in \RR^{>0}
        \end{aligned}
        \]
        \item $x,y\in \mathcal{I}$ then $\forall a\in \RR^{>0}$ we have $|x|<\frac{a}{2}$
        Then $$|x\pm y| \leq \frac{a}{2} + \frac{a}{2} = a$$
        Let $z$ be finite then $|z|<b\in \RR^{>0}$
        Let $a\in \RR^{>0}$ then $|x|< \frac{a}{b}$ so 
        \[|xz| < \frac{a}{b}b = a \] 
    \end{enumerate}
}
\defin{infinitely close}{$x,y$ are called to be infinitely close ($x\simeq y$),\outernote{$\simeq$} if $y-x\in \mathcal{I}$}
\prop{}{
    \begin{enumerate}
        \item $\simeq$ is an equivalence relation
        \item $\simeq$ is congruent with $^*+,^*\cdot,^*-$
    \end{enumerate}
}{}
\lemma{}{Suppose $\lnot x\simeq y$ and at least one of $x,y$ is finite then there exists $q\in \RR$ 
such that $q$ is betweeen $x$ and $y$}{
    $y-x\notin \mathcal{I}$, wlog. $x<y$ then there exists $b\in \RR$ such that $0<b<y-x$ 
    and by the archimedian property there is $m\in \NN^{>0}$ such that $x<mb$.
    Let $m$ be the smallest such. i.e. $(m-1)b\leq x<mb$.
    And $mb<y$.
}
\prop{}{
    For every $x\in \mathcal{F}$ therre is exactly one $r\in \RR$ such that $x\sim r$
}{
    Let $S \defeq \{r\in \RR : r<x\}$. $S$ is bounded in $\RR$ because $|x|<r_0$ for 
    some $r_0\in \RR^{>0}$.
    Then $r\defeq \sup S$. Claim: $r\simeq x$. Lets assume by contradiction that this is not the case. 
    By the previous lemma, there is $q\in \RR$ such that 
    $r<q<x$ or $x<q<r$. but neither of this things can happen.
    \begin{itemize}
        \item $r<q<x$ is contradiction to $r$ not being an upper bound.
        \item $x<q<r$ is contradiction to $r$ is not the least upper bound.
    \end{itemize}
}
A concequence of that:
\coroll{for each $x\in \mathcal{F}$ there is a unique way of writing of $x$ in the form $r+i$ where $r\in \RR$ and $i\in \mathcal{I}$}
\note{}{If $x= r+i$ then we also write $\st(x) = r$.

}
\prop{}{
    \begin{itemize}
        \item $\st: \mathcal{F}\twoheadrightarrow \RR$ 
        \item $\st (x) = 0$ iff $x\in \mathcal{I}$
        \item $\st ( x\ ^* + y) = \st(x) + \st(y)$
        \item $\st ( x\ ^* \cdot y) = \st(x) \cdot \st(y)$
    \end{itemize}
}{}
\note{}{this says that $\st$ is a homomorphism of $\mathcal{F}$ onto field $\overline{\RR}$ with 
$\ker(\st) = \mathcal{I}$ and $\mathcal{F}/_\mathcal{I}\cong \overline{\RR}$

}
\defin{Convergence (non-standard definition)}{
    $F$ converges at $a$ to $b$ if whenever $x\simeq a$ and $x\neq a$ 
then $^*F(x)\simeq b$.}
\note{}{This definition is equivalent to $\varepsilon-\delta$ definition of convergence in Analysis.
\begin{itemize}
    \item Suppose $F$ converges to $b$ at $a$ in $\varepsilon-\delta$-sense
    \[\mathcal{R}\models \forall \varepsilon>0 \exists \delta>0 \forall z 
    ( |z-a|<\delta \implies |F(z)-b| < \varepsilon)\]
    \[\mathcal{R}^*\models \forall \varepsilon>0 \exists \delta>0 \forall z 
    ( |z-a|<\delta \implies |F(z)-b| < \varepsilon)\]
    Let $\varepsilon>0$ and $\delta>0$ corresponding to $\varepsilon$.
    Let $x\simeq a$ then $|x-a|<r$ for all positive $r\in \RR^{>0}$ so in particular $|x-a|<\delta$, 
    therefore $|F(x)-b|<\varepsilon$ but $\varepsilon$ was arbitrarily, so $st(F(x))=b$.
    \item Suppose $F$ convergences to $b$ at $a$ in the non-standard-sense.
    Then $\forall \varepsilon \in \RR^{>0}$ 
    $$\mathcal{R}^*\models \exists \delta >0 \forall x (|x-a|<\delta \to |F(x)-b|<\varepsilon)$$
    Because $\delta\in \mathcal{I}$ works.
    But then 
    $$\mathcal{R}\models \exists \delta >0 \forall x (|x-a|<\delta \to |F(x)-b|<\varepsilon)$$
\end{itemize}
}
\note{}{If $F$ converges to $b$ at $a$ then $b$ is unique such that for every $i\in \mathcal{I}$ 
the standard part $\st (F(a+i)) = b$. And we use the general notation $\lim_{x\to a}{F(x)}=b$}

\coroll{$F$ continuous at $a$ then $x\simeq a\implies \: ^*F(x)\simeq ^*F(a)$}
Derivatives
From Analysis: If $F:\RR \to \RR$ then $F'(a) = \lim_{h\to 0}\frac{F(a+h)-F(a)}{h}$ 
\defin{}{We will say that $F'(a) = b$ iff $\forall dx\in \mathcal{I}$, $dx\neq 0$ then 
\[\st\bigl(\frac{F(a+dx)-F(a)}{dx}\bigr) = b\]
}
$dF \defeq ^*F(a+dx) - F(a)$ then
$F'(a) = b$ iff $\forall dx\in \mathcal{I}, dx\neq 0$  we have $\st(\frac{dF}{dx}) = b$
$\frac{dF}{dx}$ is an actual division.

\bsp{}{
    $F(x) = x^2$
    \[\frac{dF}{dx} = \frac{(a+dx)^2 - a^2}{dx} = \frac{2dxa + (dx)^2}{dx} = 2a + dx\]
    and $\st\frac{dF}{dx} = 2a$
}
\prop{}{(standard) If $F'(a)$ exists, then $F$ is continuous at $a$.}{
    Assume $F'(a)$ (in the standard sense) exist, then $F'(a)$ is a finite number and 
    $F'(a) \simeq \frac{F(a+dx)-F(a)}{dx}$.
    Therefore $F(a+dx)-F(a)$ has to be infinitesimal ($\in \mathcal{I}$). 
    Which means $F(a+dx) \simeq F(a)$.
}
\prop{Chain rule}{Suppose $G'(a)$ and $F'(G(a))$ exist then $(F\circ G)'(a) = F'(G(a))\cdot G'(a)$
}{
    Note: $^*(F\circ G) = ^*F \circ ^* G$ because 
    $\mathcal{R} \models \forall x F_{f\circ g}(x) = (F_{f}\circ F_{g})(x)$
    \[dG \defeq ^*G(a+dx) - ^*G(a)\]
    \[\begin{aligned}
        dF \defeq& ^*(F\circ G)(a+dx) - ^*(F\circ G)(a)\\
        =& ^*F(^*G(a+dx)) - ^*F(^*G(a))\\
        =& ^*F(^*G(a) + dG)) - ^*F(^*G(a))\\
    \end{aligned}
    \]
    We know that $G(a)$ exists so $G$ is continuous at $a$ and therefore $dG\simeq 0$
    \begin{itemize}
        \item case $dG\neq 0$ then $\frac{dF}{dG} \simeq F'(G(a))$.
        We can re-write $$\frac{dF}{dx} = \frac{dF}{dG}\frac{dG}{dx} = F'(G(a)) \cdot G'(a)$$
        \item case $dG = 0$ then $dF = 0$ and $G'(a) = \frac{dG}{dx} = 0$ and therefore ($dx\neq 0$)
        \[\frac{dF}{dx} =0 = F'(G(a))\frac{dG}{dx}\]
    \end{itemize}
}
\section{o-minimality}\label{MT:sec:o-min}
\bsp{}{
    $\overline{\RR} = (\RR,+,-,\cdot, 0,1,\leq)$ $\overline{\RR} = (\RR,\leq)$ TODO or at least 
    in a very similar language, 
    then by quantifier elimination (QE, Tarski) all the definable sets of $\RR$ are finite unions 
    of points and intervals.
}
\defin{o-minimality}{Let $\mathcal{L} = \{\leq,\dots\}$, $\mathcal{M}$ is an $\mathcal{L}$-structure such that 
$\mathcal{M}\models \text{DLO}$ and the only definable subsets of $M$ are finite union of points and 
intervalls. Then $\mathcal{M}$ or equivalent $\Th(\mathcal{M})$ is called o-minimal.}
o-minimal is not a first order property so to say that a theory is o-minimal is non trivial.
\note{}{Cell decomposition means 
    Suppose $X$ is definable in an o-minimal structure $\mathcal{M}$, $X\subseteq M^n$ then 
    $X$ is a finite union of cells (in dimension 1 these are points or intervalls)
    in $M^2$ it is either the graph of a continuous function or everything inbetween two graphs of continuous 
    functions. (its an inductive definition)
}
\note{}{Have Dedekind complete for definable subets of $M$: 
For $X\subseteq M$ definable then $\inf X, \sup X$ exist in $M_{\pm \infty}$. }

\note{}{If $M$ contains infinitely small elements, for example $M = \!^*\RR$ then $(0,1)\subseteq M$ is not connected.
$O_1 = \{x: \forall n\in \NN^* 0<x<\frac{1}{n}\}$ is open and so is its complement in $(0,1)$. We have 

Note that $O_1$ is however not definable in $M$. If $O_1$ would be definable it would be a finite union 
of points and intervals. It is convex, and not a point. But it is also not an intervall, because then it 
would have by Dedekind completness that $\sup O_1$ exists in $M_{\pm \infty}$, a contradiction. TODO: 
}
\defin{definably connectedness}{$X\subseteq M^m$ is said to be definably connected, if $X$ is definable and $X$ is not the 
disjoint union of two definable, non-empty open sets.}
\lemma{}{
    \begin{enumerate}
        \item The definably connected subsets of $M$ are the intervalls (including singletons) and $\varnothing$.
        \item The image of a definable connected subset $X\subseteq M^n$ under a definable continuous map $f:X\to M^n$ is definably connected. ($f$ is called to be definable, if its graph $\Gamma f \subseteq M^{mn}$ is).
        \item (IVP) If $f:[a,b]\to M$ definable and continuous, then $f$ assumes all values between $f(a)$ and $f(b)$.
    \end{enumerate}
}{Exercise}
\section{o-minimal ordered groups and rings}
\defin{ordered group}{A ordered group is a group with a linear order such that 
\[\forall x \forall y \forall z x<y \to (zx<zy \land xz < yz)\]
}
\bsp{}{
    \begin{itemize}
        \item $(\RR,<,+)$ 
        \item $(\RR^{>0},<,\cdot)$
        \item non-example: $(\RR^*, \cdot,<)$
    \end{itemize}
}
Recall:
\begin{itemize}
    \item $(G,\cdot)$ is divisible, if $\forall n \forall g \exists x g = x^n$, equivalent to $\forall n G^n = G$.
    \item $(G,\cdot)$ is torsion-free, if no element has finite order except for $1$.
\end{itemize}
\prop{}{$(M,<,\cdot, \dots)$ o-minimal such that $(M,<,\cdot)$ ordered group, then
    $(M,<,\cdot)$ abelian, divisible and torsion-free.
}{}
\lemma{}{If $G$ is a definable subgroup of $M$ then $G$ is convex.
}{
    Suppose $G$ is not convex, then there exists $1<a<g$ for some $g\in G$ and $a\in M\backslash G$. 
    Then 
    \[1<a<g<ag<g^2<ag^2<g^3<\dots \]
    but elements alternate being in $G$ and outside of $G$ so $G$ is not definable (finite union).
}
\lemma{}{The only definable subsets of $M$ that are subgroups are $\{1\}$ and $M$.
}{
    Suppose $G\neq \{1\}$ wts. $G = M$.
    From the previous lemma we know that $G$ is convex.
    The idea is $s\defeq \sup G$ then $1<s$ and $(1,s)\subseteq G$. If $s=+\infty$ then $G = M$.
    Suppose $s\neq +\infty$ then 
    Take $1<g<s$ then $g^{-1}s\in (1,s)$
    So $s = g g^{-1}s\in G$ 
    and $s<sg$ thats a contradiction with $s = \sup G$. 
}
\begin{proof}
    of Proposition.
    \begin{itemize}
        \item $(M,\cdot)$ abelian:
        For any $a\in M$ we can look at $C_a = \{x\in M : xa = ax\}$ it is a definable subgroup and contains $a$ it therefore is non-trivial and we have $C_a = M$ for every $a\in M$, so abelian.
        \item For any $n\in \NN^{>0}$ look at $\{x^n: x\in M\}$ non trivial definable subgroup of $M$, hence $=M$.
        \item Every ordered group is torsion-free.
    \end{itemize}
\end{proof}
\defin{Ordered ring}{A ring (assumed to always be associative, with $1$)
 equipped with a linear order such that
\begin{enumerate}
    \item $0<1$
    \item $<$ translation invariant
    \item $<$ invariant under multiplication by positive elements
\end{enumerate}
}

\note{}{
    \begin{itemize}
        \item The additive group $(R,<,+)$ of an ordered ring is a ordered group. 
        \item Ordered rings have no zero-divisors $\forall x \forall y xy = 0 \to (x=0 \lor y = 0)$
        \item $x^2\geq 0$ 
        \item $k\mapsto k \cdot 1 : \ZZ \to \text{ring}$ is a strictly increasing embedding with resprect to the usual ordering on $\ZZ$ that means our characteristic is $0$.
    \end{itemize}
}
\note{}{
    \begin{itemize}
        \item A division ring is a field without commutativity of multiplication, so 
        \[\forall x x\neq 0 \to \exists y xy = 1\]
        \item Suppose ordered ring is also a division ring. Then such $y$ are unique and $yx=1$. Further, $x>0 \to y>0$.
        
        Also the additive group is divisible, the underlying set is DLO w/o endpoints and 
        $(x,y)\to x \cdot y$
        $x\to x^{-1}$ are continuous with resprect to itervall topology.
    \end{itemize}
}
\defin{ordered field}{An ordered field is an ordered division ring with commutative multiplication.}
\defin{real closed field}{ordered field $R$ \outernote{RCF} such that if $f(X)\in R[X]$ and $a<b$ are such that 
$f(a)<0<f(b)$ then there exists a $c\in (a,b)$ such that $f(c)=0$}
\bsp{}{
    \begin{itemize}
        \item $(\RR,+,\cdot,<)$ is RCF
        \item $(\mathbb{Q},+,\cdot,<)$ is not a RCF
    \end{itemize}
}
\prop{}{$(M,<,+,\cdot,\dots)$ o-minimal such that $(M,<,+,\cdot)$ is ordered ring then $(M,<,+,\cdot)$ is RCF.}{
    \begin{itemize}
        \item wts. $(M;<,+,\cdot)$ is ordered division ring.
            For all $a\in M$ $aM$ is additive subgroup of $(M,+)$ hence $aM = M$ if $a\neq 0$.

        \item wts. commutativity of $\cdot$. $\text{Pos}(M) \defeq \{a\in M: a>0\}$ is a subgroup of the multiplicative group of $M$. Let $a\in M$ then bc $M = aM$ we have $b\in M$ such that $1=a \cdot b$ and by note $0<a$ then $0<a^{-1}$.
        so multiplication is commutative on $M$
        \item IVP property for polynomials:
        The ring operations are continuous, see note and use lemma about IVP (c).
    \end{itemize}
}
\subsubsection{Cell decomposition}
Base step: 
\prop{Monotonicity Theorem}{
    Suppose $f:(a,b)\to M$ definable, then there are $a<c_1< \dots < c_k<b$ such that for $(a,c_1)$, $(c_i,c_{i+1})$, $(c_k,b)$ subsets of $(a,b)$ we have: $f$ is either constant or strictly monotinic and continuous.
}{}
\lemma{}{$\exists$ subinterval on which $f$ is const or injective.}{}
\lemma{}{If $f$ injective, then strictly monotone on a subinterval.}{}
\lemma{}{If $f$ strictly monotone, then $f$ continuous on a subinterval.}{}

\begin{proof}
    Proof of Monotonicity theorem:
    Consider 
    $$X \defeq \biggl\{x\in (a,b) : \begin{aligned}
        &\text{on some subinterval containing $x$,}\\
        &\text{$f$ is either constant or strictly monotone and continuous}
    \end{aligned} \biggr\}$$
    remark: $X$ is a definable set.
    Look at $(a,b)-X$ is finite. If not, it would contain subinterval use lemma to get contradiction.
    WMA:\addAbbrev{WMA}{We may assume} $X = (a,b)$ in particular we may assume $f$ continuous.
    By subdividin $(a,b)$ further WMA that we are in one of the following cases
    \begin{enumerate}
        \item[Case 1:] $\forall x \in (a,b)$ $f$ constant on some neighborhood of $x$
        \item[Case 2:] $\forall x \in (a,b)$ $f$ is strictly monotone increasing on some neighborhood 
        of $x$
        \item[Case 3:] $\forall x \in (a,b)$ $f$ is strictly monotone decreasing on some neighborhood 
        of $x$
    \end{enumerate}
    \begin{enumerate}
        \item[Case 1:] $x_0 \in (a,b)$ then 
        $s \defeq \{ x: x_0<x<b \land f \text{ cont. on $[x_0,x)$}\}$
        wts $s=b$ suppose $s<b$ then $f$ constant on neighborhood of $s$ contradiction with 
        definition of $s$ so $f$ continuous on $[x_0,b)$. $f$ constant on $(a,x_0]$ similar.
        \item[Case 2:] $x_0\in (a,b)$ then $s\defeq \{x: x_0<x<b \land f \text{strictly incr. on } 
        [x_0,x) \}$. wts: $s=b$ assume $s<b$ then $f$ is strictly increasing on some neighborhood of 
        $s$ so $f$ strictly increasing on $[x_0,s+\delta)$ for some $\delta>0$, a contradiction to 
        definition of $s$. 
        \item[Case 3:] similar to Case 2. 
    \end{enumerate}
\end{proof}
%03.12.2024
Proof of Lemma 1:
\begin{proof}
    Statement: ``There exists a subinterval on which $f$ is constant or injective.''

    \begin{itemize}
        \item If $y\in R$ so that $f^{-1}$ is infinite (it has to be a finite union of points and intervals), 
        then $f^{-1}(y)$ contains an interval and $f(x)=y$ on that interval.
        \item Suppose $f^{-1}(y)$ is finite for every $y\in R$.\\
        $f(I)$ is infinite and is definable because $f$ is definable, so it contains an interval $J$.
        We can define an inverse to $f$ on $J$ $g:J\to I$, $g(y)$ is the first $x\in I$ such that $f(x)=y<$ (this is definable). $g$ is necessarily injective. $g(J)$ infinite, so contains a subinterval on which $f$ is injective.
    \end{itemize}
\end{proof}

Proof of Lemma 2:
\begin{proof}
    Statement: ``If $f$ injective, then strictly monotone on a subinterval.''

    Suppose $f$ is injective. $f:I=(a,b)\to R$
    pick $x\in (a,b)$ then $(a,x) = \{y\in (a,x): f(y)<f(x)\} \uplus \{y\in (a,x): f(x)<f(y)\}$ is definable disjoint union of definable sets, so one of the subsets has to contain an interval $(c,x)$ with $a \leq c$, similarly for $(x,d)$.
    So for all $x\in I$ we have $x$ satisfies one of the following:
    \begin{itemize}
        \item $\Phi_{++}(x)$ iff $\exists c_1,c_2 (c_1<x<c_2 \land \forall c \in (c_1,x) f(c)>f(x) \land \forall c\in (x,c_2) f(c)>f(x)$
        \item $\Phi_{--}(x)$
        \item $\Phi_{+-}(x)$
        \item $\Phi_{-+}(x)$
    \end{itemize}
    The set of all $x$ that satisfy each $\Phi$ is definable, it therefore is a finite union of points and intervalls.
    After passing to subinterval $(a,b)$ of $I$ WMA that each $x\in I$ 
    satisfies the same $\Phi_{\pm\pm}$.
    \begin{itemize}
        \item $\Phi_{-+}(x)$, on the left everybody is smaller, on the right everybody is bigger.
        \[\forall x\in I s(x)\defeq \sup\{s\in(x,b): f(x)<f(s)\}\]
        If $s(x)<b$ then $\Phi_{-+}(s(x))$, and therefore there is an element $s'>s(x)$ such that $f(x)
        \leq f(s(x))<f(s')$ so $s(x)\geq s'$ which is a contradiction to definition to $s(x)$, therefore 
        $s(x) = b$ for every $x\in (a,b)$. Then $f$ has to be strictly increasing on $(a,b)$.
        \item $\Phi_{+-}(x)$ similar  (monotinic decreasing)
        \item $\Phi_{++}(x)$ $\forall x \in I$.
        $$B \defeq \{x\in I: \forall y\in I (x<y \to f(y)>f(x))\}$$
        $B$ is definable, if $B$ is infinite, it has to contain a subinterval on which $f$ is strictly 
        increasing. 

        WMA $B$ is finite.
        We restrict ourselves to subinterval and may assume $B=\varnothing$. 
        So by injectivity: 
        $$\circledast \quad\forall x \in I \exists y\in I x<y\land f(x)>f(y)$$
        Let $c\in I$. Claim: for every large enough $y\in I$ we have $f(y)<f(c)$.
        \begin{claimproof}
            By contradiction. suppose we can not find a neighborhood of $b$ such that for all elements in this neighborhood $f(y)<f(c)$
            otherwise $f(y)>f(c)$ for all large enough $y$.
            Let $d<b$ be minimal such that 
            \[\forall y\in (d,b) f(y)>f(c)\]
            \begin{itemize}
                \item case $f(d)>f(c)$: $\Phi_{++}(d)$, contradiction with minimality of $d$.
                \item case $f(d)<f(c)$: By $\circledast$ there has to be an $e$ with $d<e<b$ and 
                $f(e)<f(d)$. So $f(e)<f(c)$ which is a contradiction to $\Phi_{++}(d)$
            \end{itemize}
        \end{claimproof} 
        Define $y(c)$ to be the least element of $[c,b)$ for which 
        $$\forall y y(c)<y<b \: f(c)>f(y)$$ 
        $c$ satisfies $\Phi_{++}$, therefore $c<y(c)$ and $f(y(c))<f(c)$ if $y(c)<y<b$.
        The minimality of $y(c)$ implies that $y(c)$ satisfies $\Psi_{+-}$, where 
        \[\Psi_{+-}(v) \iif \exists v_1,v_2 \in I \bigl(v_1<v<v_2 \land \forall z_1,z_2 (v_1<z_1<v\land v<z_2<v_2)\to f(z_1)>f(z_2)\bigr)\]
        But $c$ was arbitrarily so $\forall x\in I \exists v\in I (x<v\land \Psi_{+-}(v))$
        On subinterval $\Psi_{+-}$ 
        we have a contradiction with $\Phi_{++}$, similarly on subinterval for $\Psi_{-+}$.
        \item $\Phi_{--}$ similar to above
    \end{itemize}
\end{proof}
Proof of Lemma 3:
\begin{proof}
    Statement: ``If $f$ strictly monotone, then $f$ continuous on a subinterval.''

    WMA: $f:(a,b)\to R$ strictly monotone increasing.
    $f(I)$ infinite and definable, so $f(I)$ contains an interval $J$.
    Let $r,s\in J$ $r<s$ and $d,e\in I$ with $d<e$ and $f(d)=r$ and $f(e) = s$.
    restrict $f$ to $(d,e)$ and we get an increasing bijection $(d,e)\to (r,s)$.
    Our topology is the order topology, so $f$ is continuous on $(d,e)$
\end{proof}
So we have proved the monotonicity Theorem.
\note{}{If $f:(a,b)\to R$ is definable, then $\lim_{x\to c^-}f(x)$ exists in $R_{\infty}$ 
    for $c\in (a,b]$. 
    And further $\lim_{x\to c^+}f(x)$ exists in $R_\infty$ for $c\in [a,b)$ 
    If furthermore $f:[a,b]\to R$ is continuous and definable, then $f$ assumes a 
    minimum and maximum on $[a,b]$ 
}
On of the important tools in o-minimality theory is the cell cecomposition theorem:
\defin{Cell}{
    Let $(i_1,\dots i_n)$ a sequence in $\{0,1\}$. An $(i_1,\dots i_n)$-cell is defined inductively:
    \begin{itemize}
        \item $(0)$-cell: $\{r\}\subseteq R$,
        \item $(1)$-cell: $(a,b)\subseteq R$, $a<b$, $a,b\in R$.
        \item $(i_1,\dots i_k, 0)$-cell: $\Gamma f\subseteq R^{k+1}$, where $f$ is definable and continuous function $f:X\to R$, where $X$ is a $(i_1,\dots, i_k)$-cell
        \item $(i_1,\dots i_k, 1)$-cell: is a the set 
        $$(f,g) = \{(\underline{x},x_{k+1})\in R^{k+1} : f(\underline{x}) < x_{k+1}<g(\underline{x})\}$$ 
        $f:X\to R, g: X\to R$, $f<g$ $f,g$ are definable and continuous on $X$, which is a $(i_1,\dots i_k)$-cell. $f$ may be constantly $-\infty$ and $g$ may be constantly $\infty$.
    \end{itemize}
}
\note{Cells have nice topological properties}{
    \begin{itemize}
        \item every $(1,\dots 1)$-cell are precisely the cells that are open in their ambient space. continuity of the functions is important.
        \item The union of finitlely many non-open cells has empty interior.
        \item Each cells is locally closed i.e. open in its closure.
        \item Each cell is homeomorphic to an open cell under a coordinate projection Example $(1,0,0,1)$-cell or $(1,0)$-cell with $(x_1,x_2)\mapsto x_2$
        \item If $A\subseteq R^{n+1}$, then $\pi A\subseteq R^n$ cell $\pi(x_1,\dots x_{n+1})\mapsto (x_1,\dots x_n)$
        \item Every cell is definalby connected. You can proof this by induction on the cell. $\{r\}$ and open intervals are definably connected. If the projection of a cell is definably connected, then the fibre above it is either an open interval or a single point. It is even definable path connected. If there would exist an open disjoint cover there exists an open disjoint cover of the fibre, which is not possible.
    \end{itemize}
}
\defin{decomposition}{A decomposition of $R^m$ is a finite partition of $R^m$ into cells defined inductively:
\begin{itemize}
    \item decomposition of $R^1 = R$: 
    \[\bigl\{(-\infty,a_1), \{a_1\},(a_1,a_2)\dots (a_k,\infty)\bigr\}\]
    \item A decomposition of $R^{n+1}$ is a finite partition of $R^{n+1}$ into cells $C$ such that the collection of $\pi C$ is a decomposition of $R^n$.
\end{itemize}
}
\thm{Cell decomposition}{
    \begin{itemize}
        \item[$(\RomanNum{1} _m)$] Let  $A_1,\dots A_k\subseteq R^m$ definable sets. Then there is a decomposition of $R^m$ partitioning each $A_i$.
        \item[$(\RomanNum{2} _m)$] Given a definable function $f: A\to R$, $A\subseteq R^m$ there is a decomposition $\mathcal{D}$ of $R^m$ partitioning $A$ such that for every $B\in \mathcal{D}$ $f|_B:B\to R$ is continuous. 
    \end{itemize}
}{
    By induction on $m$.
    Base step: \begin{itemize}
        \item $(\RomanNum{1} _1)$ o-minimality
        \item $(\RomanNum{2} _1)$ monotonicity theorem.
    \end{itemize}
    Proof idea: Suppose we have $$
    \begin{cases}
        (\RomanNum{1} _1)\dots (\RomanNum{1} _m)\\
        (\RomanNum{2} _1)\dots (\RomanNum{2} _m)
    \end{cases}\bigr\} \implies (\RomanNum{1} _{m+1}), (\RomanNum{2} _{m+1})
    $$ 
}
\defin{}{A definably connected component of a non-empty definable Subset $X\subseteq R^m$ is a definably-connected subset of $X$ which is maximal wrt being definably connected}
\bsp{}{$X\subseteq R^m$ definable. Then it is definably connected iff $X$ definably path connected 
i.e. 
$$\forall x , y\in X\exists f:[0,1]\to X \text{ definable and continuous with } f(0)=x \land f(1)=y$$
}

\prop{}{Suppose $X\subseteq R^m$ is definable and non-empty, then $X$ has only finitlely many 
    definably connected components. The components are both open and closed in $X$ and they form a 
    finite partition of $X$.}
{
    Let $\{C_1, \dots C_k\}$ be a partition of $X$ into cells. $I\subseteq \{1,\dots k\}$ then
    $C_I \defeq \bigcup_{i\in I}C_i$. 
    Let $C'$ be the maximal among the $C_I$ that is definable connected.
    Claim: For $Y\subseteq X$ definable connected such that $Y\cap C'\neq \varnothing$ then $Y\subseteq C'$.
    \begin{claimproof}
        $C_Y \defeq \bigcup \{C_i : C_i\cap Y \neq \varnothing\}$
        Then $Y\subseteq C_Y$. and $C_Y$ is definably (finite union) connected union of definably connected set $Y$ and finitly meny cells that have non-empty intersection with $Y$.
        Then $C_Y\cap C'$ contains $Y\cap C'\neq \varnothing$. So if we take $C_Y \cup C'$ has to be again definably connected. By maximality $C_Y \cup C'= C'$ and $Y\subseteq C_Y \subseteq C'$.
    \end{claimproof}
    Hence 
    \begin{itemize}
        \item $C'$ definably connected component of $X$
        \item The sets $C'$ form a finite partition of $X$
        \item $C'$ are the only definable connected components of $X$
    \end{itemize}
    The closure in $X$ of a definably connected subset of $X$ is definable connected. (see topology)
    So the $C'$ are closed in $X$. 
    They are also open because the complement in $X$ is a finite union of closed subsets.
}
\note{}{The above Proposition is not true if we drop the requirement ``definable''}

\defin{Definable families}{
    Let $S\subseteq R^{m+n}$ definable. For $a\in R^m$ we put
    $$S_a = \{\underline{x}\in R^n : (a,\underline{x})\in S\}\subseteq R^n$$
    $S$ describes the family of sets $(S_a)_{a\in R^m}$.
    And the sets $S_a$ are called the fibers of $S$. 
}
\bsp{}{$\mathcal{R}=(\RR,<,+,\cdot)$ 
\[ax^2 + bxy + c y^2 + dx + ey + f = 0\]
defines $S\subseteq R^6\times R^2$.
The fibers are: 
\begin{itemize}
    \item points, circles, ellipse, hyperbola, parabola
    \item and the limiting cases: $\varnothing$, 2 lines intersecting each other, 2 parallel lines, one line, $RR^2$
\end{itemize}
}
\note{}{In o-minimal structures there are only finitly many homomorphism types in a definably family. (If there are infinitely many fibres, then only finitlely many are not homeomorphic to each other).}
\prop{}{
    \begin{enumerate}
        \item $C$ cell in $R^{m+n}$, $a\in \pi_m^{m+n}C$ (where 
        $\pi^{m+n}_m (x_1,\dots x_{m+n}) = (x_1,\dots x_m)$) Then $C_a$ is a cell in $R^n$
     \item $\mathcal{D}$ decomposition of $R^{m+n}$, and $a\in R^m$ then 
     $$\mathcal{D}_a = \{C_a : C\in \mathcal{D} \land a \in \pi^{m+n}_m(C)\}$$ is a decomposition  of $R^m$.
    \end{enumerate}
}{
    \begin{enumerate}
        \item induction on $n$. If $n=1$, $a\in \pi^{m+1}_m C$ 
        Then $C_a$ is one of the below
        \begin{itemize}
            \item If $C$ is a $(i_1,\dots i_m,0)$-cell then $C = \Gamma f$, $f:\pi^{m+1}_m C \to R$ definalby continuous. 
            $a\in \pi^{m+1}_mC$ then $C_a = \{f(a)\}\subseteq R$
            \item If $C$ is a $(i_1,\dots i_m,1)$-cell then $C = (f,g)$, $C_a = (f(a),g(a))$
        \end{itemize}
        Suppose the statement holds for some $n$ then let
        $C\subseteq R^{m+n+1}$ be a cell. Consider the two projections
        $\pi^{m+n+1}_{m+n}, \pi^{m+n}_m$ and
        $$\pi^{m+n}_m\circ \pi^{m+n+1}_{m+n} : R^{m+n+1}\to R^m$$ 
        Two options: 
        Either $C = \Gamma f$, then 
        $$C_a = \Gamma f_a \text{ where } f_a: ( \pi^{m+n+1}_{m+n} C)_a\to R$$ and
        $f_a(x) = f(a,x)$\\
        Or $C = (f,g)_D$ i.e. $f,g:D\to R$, $D\subseteq R^{m+n}$ cell, $D = \pi^{m+n+1}_{m+n}$
        Then $C_a = (f_a, g_a)_E$, $E = D_a$. in both cases, $C_a$ is a cell.
        \item Exercise.
    \end{enumerate}
}

\coroll{Let $S\subseteq R^m \times R^n$ a definable family then there exists $M_S\in \NN$ such that 
    for all $a\in R^m$ $S_a\subseteq R^m$ has a partition into $M_S$ many cells.
}
\begin{proof}
    $S\subseteq R^m\times R^n$ $\mathcal{D}$ decomposition of $R^m\times R^n$ that partions $S$. Then $S$ is a finite union of cells from $\mathcal{D}$, each fiber $S_a$ is a finite union of $C_a, C\in \mathcal{D}$ but $C_a$ is a cell by Proposition. A bound: $|\mathcal{D}|$.
\end{proof}

\note{}{There is a uniform bound on $\#$ of definable connected somponents of sets in definable family.}


\thm{}{$\mathcal{R}= (R;<,\dots)$ o-minimal $\mathcal{L}$-structure, $R'=(R';<,\dots)$ $\mathcal{L}$-structure. 
If $R\equiv R'$ then $R'$ is o-minimal.
}{
    $S\subseteq R$ definable, $S = \{r\in R: \mathcal{R}\to \varphi(x) [r]\}$ might use parameters from 
    $R$. If $\varphi$ is a $\mathcal{L}$-fla. over $\varnothing$.
    Then 
    $$\begin{aligned}
        \mathcal{R}\models \exists x_1,x_2,x_3 &(x_1 \neq x_2\land x_1\neq x_3 \land x_2\neq x_3 \\
        &\land \forall c ((x_1<c<x_2 \to \varphi(c))\land (c=x_3 \to \varphi(c))\\
        &\qquad\land \lnot (x_1<c<x_2\lor c = x_3)\to \lnot \varphi(c)))
    \end{aligned}$$ 
    But if $\varphi$ uses parameters, we TODO
    For all $S\subseteq R^{m+1}$ need formula $\forall \underline{a}\in R^m$ ``$S_a$ is finite union of points and intervals''
    Idea: Subset of $R$ definableby formula w/ param is just a fiber in a definable family that is parameter-free definable.
    $S_a\subseteq R$ definable. By the note, there is some number $M_S$ that only depends on the TODO
    Such that for each $\underline{a}\in R^m$ $S_a$ is a finite union of at most $M_S$ cells.

    Then \[\begin{aligned}
        \mathcal{R}\models \forall \underline{z}\exists x_1 \dots \exists x_{M_S+1}&\bigl(\forall y (y<x_1 \to \varphi_{\underline{z}}(y))\lor \forall y (y<x_1 \to \lnot \varphi_{\underline{z}}(y))\bigr)\\
        &\land \bigl(\forall y (x_1<y<x_2\to \varphi_{\underline{z}}(y))\lor\forall y (x_1<y<x_2\to \lnot\varphi_{\underline{z}}(y))\bigr)\\
        &\dots%TODO
    \end{aligned}\]
    By elementarily equivalence:
    $\mathcal{R}'\models \dots$.
}
%\note{}{0-min is strong o-minimality.} TODO


\chapter{Boolean Algebra}
\stepcounter{section}
\defin{Boolean Algebra}{
    A \graybf{boolean algebra} is a set $B$ with
    \begin{itemize}
        \item distinguished elements $0,1$ (called zero and unit of $B$)
        \item a unary operation $'$ on $B$ (called \graybf{complementation})
        \item two binary operations $\lor$ called \graybf{join} and $\land$ called \graybf{meet} s.t. for all $x,y,z \in B$ 
        \begin{enumerate}
            \item $x\lor 0 = x$ \qquad  $x\land 1 = x$
            \item $x\lor x' = 1$ \qquad   $x\land x' = 0$
            \item $x \lor y = y \lor x$ \qquad   $x\land y = y\land x$
            \item $(x\lor y)\lor z = x\lor (y\lor z)$ \qquad   $(x\land y)\land z = x\land (y\land z)$
            \item $x\lor (y\land z) = (x\lor y)\land (x\lor z)$\qquad    $x\land (y\lor z) = (x\land y)\lor (x\land z)$
        \end{enumerate}
    \end{itemize}
}
\bsp{}{Let $S$ be a set, $B \defeq \mathcal{P}(S)$ the power set of $S$, $0\defeq \varnothing$ and $1\defeq S$, 
    $$': \mathcal{P}(S)\to \mathcal{P}(S), x' \defeq S\backslash x \qquad x\lor y \defeq x\cup y, \quad x\land y \defeq x\cap y \text{ for } x,y\in \mathcal{P}(S)$$
}
\lemma{}{ Let $(B,',\lor,\land,0,1)$ be a boolean algebra. Then it holds
    \begin{enumerate}[label=\alph*)]
        \item $0' = 1$, $1' = 0$
        \item $x\lor x = x$, $x\land x = x$
        \item $(x')'= x$
        \item $(x\lor y)' = x' \land y'$, $(x\land y)' = x' \lor y'$
        \item $x\lor y = y \text{ iff } x\land y = x$
    \end{enumerate}
}{}
\lemma{}{
    \begin{enumerate}[label=\alph*)]
        \item $x\leq y \defaq x\lor y = y$ defines a partial ordering on $B$ (inclusion) and it holds
        \item $x\lor y$ is the least upper bound of $\{x,y\}$ in $B$\\
            $x\land y$ is the greatest lower bound of  $\{x,y\}$ in $B$
        \item $0\leq x\leq 1$ for all $x\in B$
    \end{enumerate}
}{}
\note{}{A boolean algebra is a complemented distributive lattice.}
\defin{Opposite of boolean algebra}{Let $(B,',\lor,\land,0,1)$ be a boolean algebra. The boolean algebra $B^{\text{op}}$ is defined by
    $$B^\text{op}\defeq B,\quad 0^\text{op} \defeq 1,\quad 1^\text{op} \defeq 0,\quad' \text{ stayes the same as for} B,\quad\lor^\text{op} \defeq \land,\quad\land^\text{op} \defeq \lor$$
    Note: $(B^\text{op})^\text{op} = B$
}
\defin{Subalgebra}{A \graybf{subalgebra} of $B$ is a subset $A\subseteq B$ s.t. $0,1\in A$ and $A$ is closed under $',\land,\lor$.
    The subalgebra generated by $P\subseteq B$ is defined to be the smallest subalgebra containing $P$. Equivalently it is the 
    intersection of all Subalgebras of $B$ that contain $P$.
}
\bsp{Power set algebra}{Let $S$ be a set then $\mathcal{P}(S)$ defines a boolean algebra on $S$.
    $B \defeq \{x\in \mathcal{P}(S): x \text{is finite or cofinite}\}$ is a subalgebra of $\mathcal{P}(S)$
    w/ set of generators $\{\{s\}:s\in S\}$}
\note{}{We will prove the Tarski-Stone Theorem: every boolean algebra is isomorphic to an algebra on a set.}

\bsp{Lindenbaum Algebra of $\Sigma$}{
    Let $A$ be a set of prop. atoms, $\propM(A)$ the set of prop. generated by $A$.
    Further let $\Sigma \subseteq \propM(A)$ and $p,q,r$ range over $\propM(A)$.\\
    We say $p$ is $\Sigma$-equivalent to $q$ iff $\Sigma \models_\text{taut} p\leftrightarrow q$
    $\Sigma$-Equivalence is an equivalent relation on $\propM(A)$ and $\propM(A)/\Sigma$ is a boolean algebra with
    $$0\defeq \bot/\Sigma,\quad1\defeq \top/\Sigma,\quad(p/\Sigma)' \defeq (\lnot p)/ \Sigma,\quad(p/\Sigma \lor q/ \Sigma)\defeq (p\lor q)/ \Sigma,\quad(p/\Sigma \land q/ \Sigma)\defeq (p\land q)/ \Sigma$$
    a set of generators is $\{a/\Sigma : a\in A\}$
}
\defin{Homomorphisms of boolean algebras}{Let $B,C$ be boolean algebras. A map $\phi: B\to C$ is a (homo)morphism of boolean algebras iff
    $\forall x,y\in B$ it holds
    \begin{itemize}
        \item $\phi(0_B) = 0_C$
        \item $\phi(x') = \phi(x)'$
        \item $\phi(x\lor y) = \phi(x)\lor \phi(y)$
        \item $\phi(x\land y) = \phi(x)\land \phi(y)$
    \end{itemize}
    If $\phi:B\to C$ is bijective too , we call $\phi$ an isomorphism and $\phi^{-1}:C\to B$ is also a morphism of boolean algebras.
}
\note{}{$\phi(B)$ is subalgebra of $C$}
\bsp{}{Let $S,T$ be sets then a function $f:S\to T$ induces a morphism of boolean algebras $\mathcal{P}(T)\to \mathcal{P}(S): y\mapsto f^{-1}(y)$
If $S\subseteq T$ and $f$ the inclusion map $S\hookrightarrow T$ then we get a boolean algebra morphism $Y\to Y\cap S$.\\
    \begin{itemize*}
        \item $id_B: B\to B$ \qquad 
        \item $x\mapsto x': B\to B^{\text{op}}$ are both isomorphism
    \end{itemize*}
}
\note{}{A boolean algebra morphism $\phi: B\to C$ is injective iff $\ker f = 0_B$}
\lemma{}{\label{boolLemma}
    Let $X_1,\dots X_m\subseteq S$ and $\mathcal{A}$ a boolean algebra on $S$ generated by $\{X_1,\dots X_m\}$. Then $\mathcal{A}$ 
    is finite and isomorphic to $\mathcal{P}(\{1,2,\dots n\})$ for some $n\leq 2^m$.
}{
    TODO
}
\defin{Trivial algebras}{\begin{itemize}
    \item $B$ is trivial if $|B| = 1$ (equivalently $0=1\in B$) 
    according to \ref{boolLemma} $B$ is isomorphic to $\mathcal{P}(\varnothing)$
    \item If $|S|=1$ then $|\mathcal{P}(S)| = 2$ 
    TODO
\end{itemize}}
\defin{Ideal}{An ideal of $B$ is a subset of $I\subseteq B$ s.t.
    \begin{itemize}
        \item[(I1)] $0\in I$
        \item[(I2)] $\forall a,b \in B$ it holds \qquad 
            $a\leq b$ and $b \in I\implies a\in I$\qquad and \qquad $a,b\in I\implies a\lor b\in I$ 
    \end{itemize}
}
\bsp{}{$F_{\text{in}} = \{F\subseteq S: F \text{ finite}\}$
    is ideal in $\mathcal{P}(S)$.
}
\note{}{If $I$ is an ideal of $B$ then 
    $I\lor b \defeq \{x\in B: x = a\lor b \text{ for some } a \in I\}$ is the smallest ideal w/ respect of $\subseteq$ of $B$ that contains $I\cup \{b\}$.
}
\bsp{}{\begin{itemize}
\item For a boolean algebra morphism $\phi: B\to C$ the kernel $\ker(\phi)$ is an ideal in $B$.
\item If $I$ is an ideal in $B$ then $a =_I b \defaq a\lor x = b\lor x$ for some $x\in I$ defines an equivalent relation and
$B/_{=_I}$ is a boolean algebra w/ 
$$0\defeq 0/_{=_I}\quad 1\defeq 1/_{=_I}\quad (a/_{=_I})' \defeq a'/_{=_I}\quad a/_{=_I}\lor b/_{=_I} \defeq (a\lor b)/_{=_I}\quad a/_{=_I}\land b/_{=_I} \defeq (a\land b)/_{=_I}$$
Then $\phi: B\to B/_{=_I}: b\mapsto b/_{=_I}$ is a boolean algebra morphism w/ $\ker(\phi)=I$
\end{itemize}}

























\chapter{Set Theory of ZFC}
The contents on this chapter are at least partially sourced on \cite{krivine1998théorie}.
\bsp{Russel's paradox}{Let $A = \{a : a\notin a\}$. If any collection of elements is a set, then $A$ would be a set.
Question: is $A\in A$? if yes, then $A\notin A$, if not then $A\in A$}
\noindent Trying to resolve this, we will introduce the ZFC (Zermelo-Frankel axioms w/ choice) System.
Let $\mathcal{L}=\{\in\}$ be a Language of first order, where $\in$ \dots binary relation "beeing element of"
For $(\mathcal{U},\in)$ 
If $\mathcal{A} = (\mathcal{U},\in^\mathcal{A})\models \text{ZFC}$, then the elements of the universe $\mathcal{U}$ are called sets.
We will show roughly that some definably sets are not sets (in the sense of ZFC), others are not. The latter will be called classes.
\section{First axioms of ZFC}
\axiom{Axiom of extensionality}{\label{Set:Ax1}
    $$\forall x \forall y (x=y \leftrightarrow \forall u (u\in x \leftrightarrow u\in y))$$}
In other words, two sets are the same if they have the same elements. This will give us later uniqueness in construction of other sets.
\axiom{Pairing Axiom}{\label{Set:Ax2}
    for any two sets $a,b$ one can form a set whose elements are precicely $a,b$
$$\forall x\forall y \exists z \forall u \: u\in z \leftrightarrow (u = x \lor u = y)$$
Our notation will be $z=\{x,y\}$
}
In words: For any two sets there exists a set whose members are those two sets.

\note{}{
    \begin{itemize}
        \item $\{x,y\}$ is unique by \ref{Set:Ax1}
        \item $\{x\}$ is a set. from \ref{Set:Ax2}, take $x= y$
    \end{itemize}
}
% \lemma{}{Let $x,y$ be sets. We define the ordered pair $(x,y) \defeq \{\{x\},\{x,y\} \}$. 
%     Then it holds $(x,y) = (a,b)$ iff $x = a$ and $y = b$
% }{By cases
%     \begin{itemize}
%         \item if $x=y$, then $(x,y) = \{\{x\} \}$ therefore $a=b$ and by \ref{Set:Ax1} it holds $x=a$.
%         \item if $x\neq y$, then $\{\{x\},\{x,y\} \} = \{\{a\},\{a,b\} \}$ iff $\{x\} = \{a\}$ and $\{x,y\} = \{a,b\}$. That is, iff $x=a$ and $y=b$.
%     \end{itemize} }

Let $x_1,x_2,\dots x_n$ be sets We define the \imp{n-tuple} $(x_1,\dots x_n)$ inductively:
\begin{itemize}
    \item  $(x_1,x_2) \defeq \{\{x\},\{x,y\} \}$. 
    \item $(x_1,\dots x_n) \defeq (x_1 (x_2,\dots x_n))$
\end{itemize}
\note{}{The set $(x,y)$ exists, because its obtained by repeatedly using \ref{Set:Ax2}}

\lemma{}{Let $x,y,a,b$ be sets. Then $(x,y) = (a,b)$ iff $x = a$ and $y = b$}{
    If $x = a$ and $y = b$ then by \ref{Set:Ax1} $(x,y) = (a,b)$. The other direction by cases:
    \begin{itemize}
        \item case $x=y$, then $(x,y) = \{\{x\}\}$ is a singleton then $(a,b)$ is a singleton, 
        wlog $\{a\} = \{a,b\}$ then $a = b = x$.
        \item case $x\neq y$ and $\{\{x\},\{x,y\}\} = \{\{a\},\{a,b\}\}$ then $\{x\} = \{a\}$ and $\{x,y\} = \{a,b\}$ because by \ref{Set:Ax1} a singleton can not be equal to a set of size $2$.
    \end{itemize}
    
}
\lemma{}{For all $n,m>1$ and for all sets $x_1,\dots, x_n,y_1,\dots, y_m$:\\
    $(x_1,\dots, x_n) = (y_1,\dots, y_m)$ iff $n=m$ and $\forall i\leq n x_i = y_i$
}{By induction, left as an exercise}

\axiom{Union Axiom}{For every set $x$ there is a set $z$ consisting of all elements of the elements of $x$.
    $$\forall x \exists z \forall y (y\in z \leftrightarrow \exists u (u\in x \land y\in u))$$
    We call $z$ the union of $x$, notation: $\cup_x\defeq z$\\
    The union of two sets is often abbreviated with $x\cup y \defeq \cup_{\{x,y\}}$
}
\bsp{}{
    \begin{enumerate}
        \item $\bigcup_{(x,y)} = \{x,y\}$.
        \item $(x_1,x_2,\dots x_n) = \bigcup_{\{x_1\}, {x_2,\dots x_n}}$ 
    \end{enumerate}
}   
\note{}{
    \begin{itemize}
        \item For all sets $ x_1, \dots x_n$ there is exactly one set with elements $x_1,\dots x_n$
        \item The union is asociative $x\cup (y\cup z) = (x\cup y)\cup z$
    \end{itemize}
}
\axiom{Power set Axiom}{Let $x\subseteq y$ be the abbreviation for $\forall z(z \in x \to z\in y)$.
For every set $x$
there exists a set $z$ consisting of all subsetes $y\subseteq x$ that are themselve sets.
    $$\forall x\exists z \forall y (y\in z \leftrightarrow y\subseteq x)$$
    Notation: $\mathcal{P}(x)\defeq z$.
}
Or in words: ``For every set $x$ there is a set $z$ consisting of all subcollections of $x$ that are themselve sets.''
\subsection{Classes and functions}
\defin{Classes}{All the unary $\mathcal{L}$-definable relations (w/ parameters) are called classes.}
\bsp{}{
    \begin{itemize}
        \item $\varphi (x) \equiv x = x$ defines the universe $\mathcal{U}$, a class that is not a set
        \item $\varphi(x) \equiv \exists u (u\in x \land \forall v \: (v\in u \to v\in x))$
    \end{itemize}
}
\defin{Class functions}{
    Suppose we have a formula $\phi(x_1,\dots x_n, y)$. Then we say $\phi$ defines a class function $R_\phi$ iff 
    \[\forall x_1\dots \forall x_n \forall y \forall y' ((\phi(\underline{x},y)\land \phi(\underline{x},y'))\to y = y')\]
    We can then define the domain and image of the class function.
    \[\dom R_\phi \defeq\{(x_1,\dots x_n) : \:\exists y \phi( \underline{x},y)\}\]
    \[\imag R_\phi \defeq\{y : \:\exists \underline{x} \phi(\underline{x},y)\}\]
    Note that $R_\phi(\underline{x})=y$ iff $\phi(\underline{x},y)$
}
\axiom{Axiom of replacement / substitution}{\label{Ax5}
    Let $\varphi(x,y,\underbar{a})$ a $\mathcal{L}$-fla., w/ free variables among $x,y$ and set-parameters $\underbar{a}$.
    Suppose $\varphi$ defines a class function on $\mathcal{U}$, than the followoing is an axiom:
    $$\forall u \exists z \forall y\: (y\in z \leftrightarrow \exists x (x\in u \land \varphi(x,y,\underbar{a})))$$
    i.e. the image of a set under a class function is a set.
}
\axiom{Axiom scheme of comprehension}{\label{Ax6}
    Let $\psi(x,\underline{a})$ be an $\mathcal{L}$-formula. Then the followoing is an axiom:
    \[\forall u \exists z \forall v \:(v\in z \leftrightarrow(v\in u\land \forall \psi(v,\underline{a})))\]
    i.e. all elements of a set that satisfy a given $\mathcal{L}$-formula form a set.
}
\note{}{\ref{Ax6} follows from \ref{Ax5}}
\axiom{Set existence}{\label{Ax7}
    \[\exists x x = x\]
    i.e. $U\neq \varnothing$. - this is clear when we view it as a universe of a structure.
}
\note{on the existence of the empty set}{
    Let $u$ be any set (there exists one by \ref{Ax7}), $\psi(x) \equiv x\neq x$ then by \ref{Ax6} $\varnothing \defeq \{x\in u : \psi(x)\}$
    is a set.
}
\datenote{17.12.2024}
\note{}{We can derive pairing from replacement \ref{Ax5}, extensionality, powerset and set existence.
From set existence: $\varnothing$ is a set
By powerset, replacement(comprehension): $\{\varnothing\}$ is a set.
    \[\mathcal{P}(\{\varnothing\}) = \{\varnothing ,\{\varnothing\}\}\]
    is a set.
    Then by defining a class function $R_\phi (x) = y$, 
    $$\phi (x,y)\equiv (x=\varnothing\land y = a)\lor (x = \{\varnothing\}\land y = b)$$
}

\note{}{If the domain of a class function happens to be a set then the graph of the class function is a set. %check if really is graph
    $R_\phi$ defined by $\phi(x,y,\underline{a})$ then the domain $u\defeq \dom R_\phi \in \mathcal{U}$
    Image $v\defeq \imag R_\phi \in \mathcal{U}$ by replacement 
    \[\{(x,y) : x\in u \land y \in v \land \phi(x,y,\underline{a})\}\]
    is the graph of $R_\phi$
    The above would be a set if $u\times v$ which can be shown by using comprehenson (exercise).
}
\defin{Function}{A function $f:a\to b$ where $a,b$ are sets is a subset of $a\times b$ that satisfies the followoing
    \begin{itemize}
        \item $\forall x \: (x\in a \to \exists y \in b \:(x,y)\in f)$
        \item $\forall x \forall y \forall y' \:(((x,y)\in f \land (x,y')\in f)\to y = y')$
    \end{itemize}
}
\subsubsection*{Families of sets and cartesian products}
Suppose we have a function $a: I \to X$. Let $a_i$ be the unique $x\in X$ s.th. $(i,x)\in a$.
We define the following sets:
    \[\bigcup_{i\in I}a_i \defeq \{z\in \bigcup X : \exists i\in I z\in a_i\}\]
    \[\bigcap_{i\in I}a_i \defeq \{z\in \bigcup X : \forall i\in I z\in a_i\}\]
    \[\prod_{i\in I}a_i \defeq \{f:I\to \bigcup X : \forall i\in I z\in a_i\}\]

\note{}{
    If $I=\varnothing$ then $\bigcap_{i\in I}a_i = \bigcup X$
}
\subsubsection*{Class relations and well ordering}
Types of well ordered sets
\defin{Strict (linear) order}{
    Let $R$ be a class relation, $C$ be a class.\\
    Then $R$ defines a strict ordering on $C$, if
    \begin{enumerate}[label = (\roman*)]
        \item  $ \forall x \forall y \forall z \: R(x,y)\to (C(x)\land C(y)) $ 
        \item  $ \forall x \forall y  \: \lnot(R(x,y)\land R(y,x)) $ 
        \item  $ \forall x \forall y \forall z \: (R(x,y)\land R(y,z))\to R(x,z) $ 
    \end{enumerate}
    The ordering is linear, if additionally
    \begin{enumerate}[label=(\roman*)]
        \setcounter{enumi}{3}
        \item $\forall x \forall y \: (C(x)\land C(y))\to (x=y\lor R(x,y)\lor R(y,x))$
    \end{enumerate}
}
\defin{Well ordering}{Let $R$ be a strict ordering on $C$ and 
    $x$ be a set such that $\forall y \in x \: C(y)$
    Then $x$ is called well-ordered by $R$, if
    \[\forall \varnothing \neq y \subseteq x \text{ $y$ has a smallest element}\]
    i.e.
    \[\forall y \bigl((\varnothing \neq y \land y\subseteq x) \to \exists y' \bigl(y'\in y \land \forall z (z\in y \to( R(z,y')\lor y'=z))\bigr)\bigr)\]
}
\defin{Initial segment}{
    Let $x$ be a set, well ordered by $R$.
    Then $y\subseteq x$ is called an initial segment of $x$, if
    \[\forall s\forall t \: (s\in x \land t \in x) \to  \bigl((t\in y \land R(s,t))\to s\in y\bigr)\]
    Let $x$ be well-ordered by $<$ and $y\in x$.
    Then $\delta^\leq_y(x) \defeq \{z\in x : z<y\}$. 
    If there is no ambiguity among the well ordering, we abbreviate $\delta_z(x) \defeq \delta^\leq_y(x)$.
    With
    $<$ above strict it holds $y\notin \delta_y(x)$
}

\note{}{If $x$ is well-ordered by $<$ and $y\subseteq x$ then
    \begin{center}
        $y$ is an initial segment of $x$ iff $y = x $ or $y = \delta_z(x)$ for some $z\in x$
    \end{center}
    \begin{proof}
        $\delta_z(x)$ is well ordered for $z\in x$.\\
        Let $y\subseteq x$ an initial segment. Suppose $x\neq y$ that means $x\smallsetminus y \neq \varnothing$
        Let $z$ be the smallest element of $x\smallsetminus y$ (exists by well-ordering of $x$).
        Suppose $y \neq \delta_z(x)$. Then there is $a\in y$ $z<a$ and $y$ is not an initial segment.
    \end{proof}
    
}
\defin{Propper class}{A class $C$ is called a proper class if it is not a set.\\
    i.e. if $C$ is given by $\phi(x,\underline{a})$ then there is no $z\in \mathcal{U}$ such that
    $\forall x x\in z \iif \phi (x,\underline{a})$
}
\bsp{}{$\mathcal{U}$ is a proper class: 
    If $\mathcal{U}$ was a set then $\{x : x\notin x\}$ would be a set.

    $\Ord$, the class of all ordinals is a proper class.
}

\defin{well-ordering (class)}{A class relation $R$ defining a strict ordering on a class $C$
    is called a well-ordering, if \\
    for every $x\in C$ the class initial segment $\delta^R_x(C) = \{y : R(y,x)\}$ is a set that is well ordered by $R$.
    \[\forall x C(x)\to \exists z \: z = \delta^R_x(C) \land \text{ $z$ has a smallest element}\]
}
\section{Ordinals}
\defin{Tranistivity of sets}{A set $x$ is called transitive, if 
$\forall y\: (y\in x \to y\subseteq x)$
}
\note{}{
    It corresponds to Tranistivity of the belonging relation ``$\in$''. $z \in y\in x \to z\in x$
}
\defin{Ordinal}{An ordinal is a transitive set which is well ordered by $\in$.}
\note{}{The collection of all ordinals form a class relation, notation: 
$\Ord, \On$\outernote{$\Ord$}\outernote{$\On$}\\
    Proof: Write down formula 
}
\bsp{}{
    \begin{itemize}
        \item $\varnothing$, $\{\varnothing\}$, $\{\varnothing , \{\varnothing\}\}$ are ordinals
    \end{itemize}
}
\lemma{Characterization of ordinals}{\label{Set:Lemma:CharOfOrdinals}
    Let $\alpha$ be a set. $\alpha$ is an ordinal, iff
    \begin{itemize}
        \item the initil segments of $\alpha$ are $\alpha$ itself and the elements of $\alpha$
        \item if $\beta \in \alpha$ then $\beta $ is an ordinal.
        \item $\alpha\notin \alpha$
    \end{itemize}
}{Problem set}
\lemma{}{\label{Set:Lemma:InisLinearOrder}
    Let $\alpha,\beta\in \Ord $ then either $\alpha = \beta$, or $\alpha\in \beta$ or $\beta\in \alpha$.
}{
    Let $\gamma \defeq \alpha\cap \beta$\\
    \textbf{Claim:} $\gamma$ is initial segment of both $\alpha $ and $\beta$
    \begin{claimproof}
        $x\in y\in \gamma$ then $x\in y \in \alpha$ and $x\in y \in \beta$. but $\alpha, \beta$ are ordinals, so
        $x\in \alpha$ and $x \in \beta$ and $x \in \gamma$
    \end{claimproof}
    Then by previous lemma, either 
    \begin{itemize}
        \item $\gamma = \alpha$ and $\gamma = \beta$ and we are done 
        \item $\gamma = \alpha$ and $\gamma \in \beta$, so $\alpha\in \beta$
        \item $\gamma \in \alpha$ and $\gamma = \beta$, so we have $\beta\in \alpha$.
        \item $\gamma \in \alpha$ and $\gamma \in \beta$ we have $\gamma \in \alpha\cap \beta = \gamma$ which is impossible
    \end{itemize}
    Hence the statement follows.
}
\prop{}{$\Ord$ is well-ordered by $\in$}{
    We need to show that if $\alpha\in\Ord $ then $\delta_{\alpha}(\Ord)$ is a set which is well ordered by $\in$.

    $\delta_{\alpha}(\Ord) = \{\beta \in\Ord  : \beta\in \alpha\} = \alpha$
    And $\alpha$ is a well-ordered set. By \ref{Set:Lemma:InisLinearOrder} $\Ord $ is even linearly ordered by $\in$.
}
\lemma{}{
    $\Ord$, the class of all ordinals is a proper class.
}{
    Suppose $\Ord $ would be a set $z$.\\
    $\Ord $ is well ordered by $\in$
    $\Ord $ is transitive: $y\in x \in\Ord $ then $y\in\Ord $ by \ref{Set:Lemma:CharOfOrdinals}
    so $\Ord $ would be an ordinal itself and we would have $\Ord \in\Ord $ which is not possible by \ref{Set:Lemma:CharOfOrdinals}.
}

\note{}{
    \begin{itemize}
        \item If $\alpha\in\Ord $ then the initial segments of $\alpha$ are $\alpha$ and the elements of $\alpha$.
        \item If $\alpha\in\Ord $ and $\beta\in \alpha$ then $\beta \in \Ord $
        \item $\alpha,\beta\in\Ord $ then $\alpha\subseteq \beta$ iff $\alpha\in \beta$ or $\alpha = \beta$
        \item $\alpha\subseteq \beta$ iff $\alpha = \beta$ or $\alpha\in \beta$.
    \end{itemize}
}

\datenote{07.01.2025}

\lemma{ }{
    If $\alpha\in \Ord $ then $\alpha\cup \{\alpha\} \in \Ord $ and $\alpha\cup \{\alpha\}$ is the successor of $\alpha$ in the ordering $\in$
}{
    \begin{itemize}
        \item $\alpha\cup \{\alpha\}$ transitive:\\
        $x\in y \in \alpha\cup \{\alpha\}$ if $y\in \alpha$ then $x\in \alpha$ hence $x\in \alpha\cup \{\alpha\}$.
        else $y = \alpha$ then $x\in \alpha\cup \{\alpha\}$ 
        \item $\alpha\cup \{\alpha\}$ well-ordered:\\
        $\varnothing\neq x\subseteq  \alpha\cup \{\alpha\}$ if $x\cap \alpha\neq \varnothing$ then there is $x_0\in x\cap \alpha$ smallest.
        $x_0 \in \alpha$ and is smallest element in  $\alpha\cup \{\alpha\}$
        otherwise $\varnothing \neq x \subseteq \{\alpha\}$ then $\alpha$ is the smallest element.
        \item  $\alpha\cup \{\alpha\}$ is successor of $\alpha$\\
        $\alpha\in \alpha\cup \{\alpha\}$
        assume  $\alpha\in \beta \in \alpha\cup \{\alpha\}$
        if $\beta\in \alpha$ then $\alpha\in\beta\in\alpha$, so by transitivity, $\alpha\in \alpha$ which is not possible.\\
        else $\beta = \alpha$
    \end{itemize}
}

\note{}{If $\alpha,\beta$ are ordinals then we will use $\alpha\in \beta$, $\alpha<\beta$, $\alpha\subsetneq \beta$ interchangable.

$\gamma\in \Ord $ then $\gamma = \{\alpha\in \Ord : \alpha\in \gamma\}$
}
\lemma{}{
    $X$ a set of ordinals, then $\sup X = \bigcup X$ is an ordinal and $\forall \alpha\in X \alpha\subseteq \bigcup X$ and $\bigcup X$ is smallest with this property.
}{
    \begin{itemize}
        \item $\bigcup X$ transitive: $x\in y\in \bigcup X$. then $\exists \alpha\in X$ such that $y\in \alpha$. then $x\in \alpha$ hence $x\in \bigcup X$
        \item $\bigcup X$ well-ordered by $\in$: $\bigcup X$ contained in $\Ord $ and is a set, but $\Ord $ is well-ordered, so $\bigcup X$ is well-ordered.
        \item $\alpha\in X$ then $\alpha\subseteq \bigcup X$.
        \item $\alpha\in X$ then $\alpha\subseteq \bigcup X$.
        Let $\beta\in \bigcup X$ then there exists $\alpha\in X$ such that $\beta\in \alpha$ so $\beta$ is not an upper bound for $X$.
    \end{itemize}
}

\lemma{}{
    Suppose that $\alpha,\beta\in \Ord $ and $f:\alpha\to \beta$ that is strictly increasing i.e. 
    $\forall \gamma,\delta\in \alpha \gamma<\delta \to f(\gamma)<f(\delta)$
    Then $\alpha\subseteq \beta$ and $\forall \gamma\: \gamma\leq f(\gamma)$
}{
    By contradiction, Let $\gamma\in \alpha$ be the smallest element with $f(\gamma)<\gamma$ then by minimality of $\gamma$,
    $f(\gamma)\leq f(f(\gamma))$. 
    Because $f$ is strictly increasing $f(f(\gamma))<f(\gamma)$. So we get $f(\gamma)\leq f(f(\gamma))<f(\gamma)$ but 
    $f(\gamma)\notin f(\gamma)$ because $f(\gamma)\in \beta$.

    Suppose $\beta\in \alpha$ then $f(\beta)<f(\alpha)<\beta$ so $f(\beta)<\beta$, a contradiction.
}

\thm{}{
    $f:\alpha\to \beta$ isomorphism between $(\alpha,\in)$, $(\beta,\in)$ and $\alpha,\beta\in \Ord $ then $\alpha = \beta$ and $f$ is unique such isomorphism, hence $f = id_\alpha$.
}{
    $\alpha = \beta$:\\
    Apply previous lemma to $f,f^{-1}$ hence $\alpha\subseteq \beta$ and $\beta\subseteq \alpha$
    uniqueness:\\
    $\gamma\in \alpha$ then $\gamma\leq f(\gamma)$  and $\gamma\leq f^{-1}(\gamma)$ by prev lemma
    we get 
    $\gamma\leq f(\gamma)\leq \gamma$ so $f(\gamma) = \gamma$
}

\note{}{$y\mapsto \beta_y$ for $y\in Y$ is function defined on $Y$ and maps to
    $Z = \{\beta(x) : x\in Y\}$ (is a set by replacement)
}

\thm{}{
    Let $(X,<_X)$ be well-ordered, then there is a unique isomorphism onto an ordinal $(\alpha, \in)$
}{
    uniqueness:\\
    Suppose we have $f:(X,<_X)\to(\alpha,\in)$, $g:(X,<_X)\to(\beta,\in)$ isomorphisms
    then $f\circ g^{-1}$ and by prev thm: $\alpha = \beta$ and $f\circ g^{-1}= id_\alpha$ so $f=g$.
    
    Existence:\\
    define $y = \{x\in X : \: (\delta_x,<_X) \text{  is isomorphic to an ordinal}\}$
    where $\delta_x \defeq \delta_x(X)$.

    For each $y\in Y$ there is a unique ordinal $\beta_y\in \Ord $ such that $(\delta_y, <_X)$ 
    and $(\beta(y), \in)$ are isomorphic.

    \textbf{Claim:} $Y$ is initial segment of $X$
    \begin{claimproof}
        If $x<_Xy\in Y$, $f:\delta_y\to \beta(y)$ isomorphism, then $f$ maps $\delta_x\subseteq \delta_y$ to initial segment of $\beta$, hence to an ordinal. 
    \end{claimproof}
    $y\mapsto \beta_y$ for $y\in Y$ is function defined on $Y$ and maps to
    $Z = \{\beta(x) : x\in Y\}$ (is a set by replacement)\\
    \textbf{Claim:} $Z = \{\beta(x) : x\in Y\}$ is an initial segment in $\Ord $
    \begin{claimproof}
        if $\gamma\in \beta(x)$, $x\in Y$ have isomorphism between $(\delta_x,<_X)$ and $(\beta_x,\in)$ so its preimage $y$ and $(\delta_y, <_X)$ is mapped to the initial segment determined by gamma, so to $(\gamma, \in)$ hence $(\delta_y, <_X)$ isom to $\gamma$
    \end{claimproof}
    So $Z$ is initial segment of $\Ord $ and $Z$ is a set. So $\alpha\defeq Z$ is itself an ordinal
    and $y\mapsto \beta_y$ isomorphism between $Y$ and $\alpha$.

    Assuming $Y\subsetneq X$, then there is a minimal $x_0\in X\backslash Y$ 
    $\delta_{x_0} = Y$ ($Y$ is initial segment) $Y\cong \alpha$ hence $x_0\in Y$, a contradiction.
}

\section{Transfinite induction/recursion} % or inductive definitions

Suppose $\phi(x)$ (possibly with parameters)
to prove 
\begin{equation}\label{star}
    \forall \alpha\in \Ord  \phi(\alpha) \iif \forall \alpha \forall \beta ((\beta<\alpha\to \phi(\beta))\to \phi(\alpha))
\end{equation}
\ref{star} $\implies \forall \alpha\in \Ord  \phi(\alpha)$

Suppose ther is $\alpha\in \Ord $ such that $\lnot \phi(\alpha)$ then let $\alpha$ be smallest with the property.

proof by induction on ordinals (proof by transfinite induction) is a proof of $\forall \alpha\in \Ord  \phi(\alpha)$ by proving \ref{star}

Let $F$ be a class function in one variable and $a$ a set contained in $\dom (F)$ 
Then $F|_{a} = \{(x,y)\in (a,b) : F(x) = y\}$ where $b = \{F(x) : x\in a\}$ (which is a set by replacement).

Let $H$ be any class function in one variable.

\defin{H-inductive}{A function $f$ is called H-inductive, if 
    \begin{enumerate*}
        \item $\alpha \defeq \dom(f) \in \Ord $ and 
        \item $\forall \beta \in \alpha f|_\beta\in \dom (H)$ and 
        \item $f(\beta) = H(f|_\beta)$
    \end{enumerate*}
}   

$f:\alpha\to X$ then $H$ gives you a way to extend the $f$.
$H$ extends $f$ to a function on $\alpha\cup\{\alpha\}$
$$f(\alpha) = H(f)$$

\lemma{}{
    For every class function $H$ and ordinal $\alpha$ there is at most one H-inductive function on $\alpha$ with domain $\alpha$.
}{
    Suppose not. $f,g:\alpha\to X$ different H-inductive functions.
    Let $x_0$ the smallest element of $\alpha$ such that $f(x_0)\neq g(x_0)$. By $x_0$ smallest, 
    $f|_{x_0}= g|_{x_0}$
    By H-inductiveness
    $$f(x_0) = H(f|_{x_0}= H(g|_{x_0}) = g(x_0)$$
    A contradiction.
}
\lemma{}{
    Let $H$ be a class function, $\alpha\in \Ord $ such that any function $f:\beta\to X$ where $\beta\in \Ord $
    belongs to $\dom (H)$ then there is an H-inductive function $f:\alpha\to X$.
}{
    $\tau = \{\beta<\alpha : \text{ there is H-inductive }f_\beta : \alpha\to X\}$
    $\tau$ is a set and initial segment of $\alpha$ hence $\tau \in \Ord $ and $\tau\subseteq \alpha$

    $\beta\mapsto f_\beta$ for $\beta\in \tau$ is well-defined function by uniqueness of $f_\beta$.

    Moreover for $\gamma<\beta<\tau$ we have $f|_\beta|_\gamma  = f|_\gamma$ (H-ind, uniqueness)

    $f\defeq \bigcup_{\beta<\tau}f_\beta$ is H-inductive function (graphs agree on intersection, each of the $f_\beta$ are H-ind). The domain 
    $$\dom(f) = \sup_{\beta<\tau}(\beta) = \bigcup_{\beta<\tau}(\beta) = \sigma\in \Ord $$

    If $\sigma =\alpha$ %sigma or tau???
    we are finished, otherwise we can define 

    $\tilde{f}$ such that $\tilde{f}|_\sigma = f$ and $\tilde{f}(\sigma) = H(f)$
    $\tilde{f}$ is now H-inductive, and $\dom (\tilde{f}) = \sigma\cup\{\sigma\}$ 
    a contradiction $\sigma$, the domain of $f$.
}

\datenote{09.01.2025}
%H-induction is like transfinite recursion

\thm{Transfinite Recursion}{
    Let $A$ be a class, $M$ be a class of all functions $f:\alpha\to X$ for $\alpha\in\Ord$ arb.
    $X$ a subset of $A$ and $H$ a class function in one variable defined on all of $M$ with values in $A$.
    Then there exists a unique class function $F$ defined on $\Ord$ such that $\forall \alpha F(\alpha) = H(F|_\alpha)$
}{
    define $F$ by 
    $$F(\alpha) = y \iif \text{there is an H-inductive function $f:\alpha\to X$, $X$ subset of $A$ and $y = H(f)$}$$
    It is well defined by the prev two lemmas TODO
}

\section{Axiom of Choice and Zermelo's Theorem}
\axiom{Axiom of Choice (AC)}{
    For every set $X$ and $A\subseteq \mathcal{P}(X)$ that consists of pairwise disjoint, non-empty subsets of $X$ there is a set $T\subseteq X$ such that $\forall a\in A \:\# a\cap T = 1$. \\
    $T$ as above is called a transversal.
}
\defin{(AC')}{
    (Existence of choice function.) 
    For every set $X$ there exists a function $\pi : \mathcal{P}\setminus \{\varnothing\}\to X$ such that for every non-empty subset $a\subseteq X$ $\pi(a)\in a$
}
\defin{(AC'')}{
    Let $(X_i)_{i\in I}$ be an indexed family of non-empty sets then \\
    $\prod_{i\in I}{X_i}\neq \varnothing$.
    ($(X_i)_{i\in I}$  could also be thought of as a function $i\mapsto X_i:I\to X$)
}
In the next Problemset: $AC \leftrightarrow AC'\leftrightarrow AC''$

\thm{Zermelo}{
    ``Well-ordering theorem'': Every set can be well-ordered.\\
}{
    By contradiction. Suppose $X$ is a set which can not be well-ordered. 
    Choose a choice function $\pi:\mathcal{P}(X)\backslash \varnothing \to X$ 
    define a class function $H$ by $H(f) = y$ iff $f$ is a function with
    \begin{enumerate*}[label = (\roman*)]
        \item $\dom f = \alpha \in \Ord$
        \item $\imag f\subsetneq X$
        \item $y = \pi (X\backslash \imag f)$
    \end{enumerate*}

    note: \begin{itemize}
        \item $H$ is defined on class of all $H$-inductive functions, 
    whose image is a proper subset of $X$. $\imag f\subsetneq X$.
        \item each $H$-inductive function is injective
    \end{itemize}
    If $f$ is a $H$-inductive function that is also surjective, then we are done because $f$ induces well-ordering on $X$.

    By our assumption, every $H$-inductive function has to be not surjective.
    So $H$ is defined on all $H$-inductive functions, and we can use Transfinite recursion theorem
    and get an $H$-inductive class function $F:\Ord\to X$ 
    which is injective

    Suppose $\alpha<\beta<\gamma$ and $F(\alpha) = F(\beta)$. 
    $F(\alpha) = \pi(X\backslash\imag F|_\alpha)$
    $F(\beta) = \pi(X\backslash\imag F|_\beta)$
    
    Then $F$ is an injection from a proper class into $\imag F$, a set, which is impossible.
}

\thm{Zorn's Lemma}{
    Let $(X,\leq)$ be a partially ordered set (poset) such that all linearly ordered subsets (called chains) have an upperbound. Then $(X,\leq)$ has a maximal element. i.e. $\exists y\in X \forall x\in X y\not < x$
}{
    Let $A \defeq \{Y\subseteq X : \exists x\in X \forall y\in Y y<x\}$
    Take $\pi:\mathcal{P}(X)\backslash \{\varnothing\}\to X$ a choice function.

    Define $$p:A\to X, \quad p(Y) \defeq \pi (\{x\in X : \forall y\in Y y<x\})$$
    Define a class function $H$ by $H(f) = y$ iff 
    \begin{enumerate*}[label = (\roman*)]
        \item $f$ is a function with $\dom f \in \Ord$
        \item $\imag f\in A$
        \item $y = \pi(\imag f)$
    \end{enumerate*}

    We get 
    \begin{itemize}
        \item  any $H$-inductive $f:\alpha\to X$ is strictly increasing. ($\star$)
        $f:\alpha\to X, f(\alpha\cup \{\alpha\}) = H(f)>\imag f$
        \item The image of any $H$-inductive function $f:\alpha \to X$ is linearly ordered, so by assumption has an upperbound.
        
        i.e. $\exists x_f\in X \: \forall \beta < \alpha f(\beta) < x_f$
    \end{itemize}
    The idea now is similar to above therorem.
    Suppose $f:\alpha \to X$ is $H$-inductive but $H$ is not defined on $f$, then the image of $f$ has no strict majorant, so there is $\beta<\alpha$ such that $x_f = f(\beta)$. 
    Then $x_f$ has to be maximal for $X$.

    If $H$ is actually defined on all $H$-inductive functions, then there is an $H$-inductive class function $F:\Ord \to X$.
    $F$ is strictly increasing by $\star$.
    So have injective of proper class into set. a contradiction.
}

\section{Ordinal arithmetic and the size of a set}

Or how to think of the natural numbers to be contained in $\Ord$

\defin{successor / limit ordinals}{
    \begin{itemize}
        \item $0\defeq \varnothing$ is the smallest ordinal
        \item $\beta\in \Ord$ then its successor ordinal is defined by $\beta+1 \defeq \beta\cup\{\beta\}$
        \item $\beta$ is called a \imp{successor ordinal} if there is $\alpha\in \Ord$ such that 
        $\beta = \alpha + 1$
        \item $\beta\in \Ord$ is called a \imp{limit ordinal} if $\beta \neq 0$ and $\beta$ is not a successor ordinal
        \item $\beta\in \Ord$ is called a natual number / finite ordinal, if $\beta = 0$ or for every $\alpha\leq \beta$ we have ``$\alpha$ is a successor ordinal or $\alpha = 0$''
    \end{itemize}
}
\bsp{}{
    $0 = \varnothing$, $1 = \varnothing \cup \{\varnothing\} = \{\varnothing\}$,\dots 
}
\note{}{
    If $(X,<_X), (Y,<_Y)$ are well-ordered sets then we can well-order both their cartesian product 
    $X\times Y$
    and \imp{disjoint union} $X \sqcup Y\defeq (X\times \{0\})\cup (Y\times \{1\})$ by the reverse lexiographical order:
    $$(x_0,y_0)\sphericalangle (x_1,y_1)\iif y_0<_Y y_1 \lor (y_0 = y_1\land x_0<_X x_1)$$
    For $(x_0,y_0), (x_1,y_1)$ in $X\times Y$ or $X\sqcup Y$, in the latter case $y_0 <_Y y_1$, if $y_0 = 0$ and $y_1 = 1$.
    Note that in this case the the ordering of $X \sqcup Y$ corresponds to
    $$(a,i)\sphericalangle (b,j) \iif \begin{cases}
        i=j=0 \text{ and } a<_X b\text{, or}\\
         i=j=1 \text{ and } a<_Y b\text{, or}\\
         i<j
    \end{cases}$$
}
\newpage
\defin{Sum and product of ordinals}{
    Let $\alpha,\beta\in\Ord$, then 
    \begin{enumerate}[label=(\roman*)]
        \item $\alpha+\beta$ is the unique $\gamma\in \Ord$ such that $\gamma$ is order-isomorphic to the 
        sum / disjoint union $\alpha\sqcup\beta$
        \item $\alpha\cdot\beta$ is the unique $\gamma\in\Ord$ such that $\gamma$ is order-isomorphic to the 
        product of $\alpha$ and $\beta$
    \end{enumerate}
}
Properties of sum and product.
\lemma{}{
    \begin{enumerate}
        \item $+$ is associative, $0$ is a $2$-sided add. identity
        \item $\cdot$ is associative
        \item $\alpha\cdot 0 = 0$, $\alpha \cdot 1 = \alpha = 1\cdot \alpha$
        \item $\alpha\cdot(\beta+ \gamma) = \alpha\cdot\beta + \alpha\cdot\gamma$
        \item $\lambda$ limit ordinal then $\alpha\cdot \lambda = \sup_{\beta<\lambda} \alpha \cdot \beta$
    \end{enumerate}
}{Problem set}
\note{}{
    Right now it would be consistent to assume $\cdot$ is commutative, but no longer after the next axiom
}

\axiom{Axiom of infinity}{
    There exists an infinite ordinal.

    i.e. there exists an ordinal that is not a natural number.
}
\note{}{
    \begin{itemize}
        \item The natural numbers form an initial segment in $\Ord$.
        \item Let $\omega$ be the smallest infinite ordinal, i.p. $\omega$ is a limit ordinal
    \end{itemize}
}
\bsp{}{
    \begin{enumerate}
        \item $\omega\cdot 2 \stackrel{?}{=} 2 \cdot \omega$, observations:
            \begin{itemize}
                \item $\omega\cdot 2 = \omega + \omega$
                \item $2\cdot \omega = \omega$
                \item $\omega + \omega \neq \omega$
            \end{itemize}
        \item $\omega + 2 \neq 2 + \omega$, observations:
        \begin{itemize}
            \item $2+\omega = \omega$
            \item $\omega + 2$ has maximal element
        \end{itemize}
    \end{enumerate}
}


\datenote{14.01.2025}


\defin{Exponentiation}{
    $\alpha^\beta$ is defined recursively on $\beta$:
    \begin{enumerate}
        \item $\alpha^0 \defeq 1$
        \item $\alpha ^{\beta+1} \defeq \alpha^\beta \cdot \alpha$
        \item $\alpha^\lambda \defeq \sup_{\delta < \lambda}{\alpha^\delta}$ for a limit ordinal $\lambda$
    \end{enumerate}
}
\note{}{
    Alternatively, we can define exponentiation as given by the class function $EXP : \Ord \times \Ord \to \Ord$ defined by
    \[Exp(\alpha,\beta) = \gamma \quad\iif\quad \text{There exists a function } f:\beta+1\to \Ord 
    \text{ such that for all } \xi<\beta\text{ it holds}\]
     
    \begin{itemize}
        \item $f(\beta) = \gamma$
        \item if $\xi=0$ then $f(\xi) = 1$
        \item if $\xi = \delta+1$ then $f(\xi) = f(\delta)\cdot\alpha$
        \item if $\xi$ is a limit ordinal then $f(\xi) = \sup_{\delta < \xi}{f(\delta)}$
    \end{itemize}
}
TODO
uniqueness follows from recursion theorem.

\defin{Cardinality}{\label{Set:Def:Cardinality}Given a set $X$, the cardinality of $X$ (denoted by $|X|$, $card(X)$) is the smallest ordinal for which there is a bijection with $X$.}
\note{}{Every set has a cardinality. 
This is bc every set can be well-ordered (by Zermelo's Theorem, which is equivalent to AC) and then we get an order-preserving bijection.
In fact the statement ``$card(X)$ is defined for each set $X$'' is equivalent to AC.
}
\defin{Equinumerous sets}{We call two sets $X,Y$ equinumerous, if 
there is a bijection between them.
}
By the previous \ref{Set:Def:Cardinality} we have: $X,Y$ equinumerous iff $|X| = |Y|$
\thm{}{Let $X,Y$ be non-empty sets Then the following are equivalent.
\begin{enumerate}[label = (\roman*)]
    \item there is an injection of $X$ into $Y$
    \item there is a surjection of $Y$ into $X$
    \item $|X| \leq |Y|$
\end{enumerate}
}{
    ``$\romannumeral 1 \implies \romannumeral 2$'' don't need AC
    ``$\romannumeral 2 \implies \romannumeral 1$'' need AC
    
}
\thm{Cantor-Schröder-Bernstein}{
    $|X| = |Y|$ iff there is an injection $X\to Y$ and an injection $Y\to X$
}{
    % ``$ \implies$'' clear with AC
    % ``$ \impliedby$'' clear with AC
    
}
Proof is clear with AC, but can proof it without AC
need to do some kind of back and fourth argument
\thm{Cantor}{\label{5:Thm:Cantor}
    For every set $X$ we have $|\mathcal{P}(X)|>|X|$
}{
    Suppose its not, then there exists a set $X$ with $|X|\geq |\mathcal{P}(X)$.\\
    We can find a surjection $\pi:X\to \mathcal{P}(X)$ 
    $Y = \{x\in X : x\notin\pi(x)\}$
    Let $y\in X$ be such that $\pi(y) = Y$
    Either $y\in Y$ or $y\notin Y$
    If $y\in Y$ then by definition of $Y$, $y\notin Y$ 
    If $y\notin Y$ then by definition $y\in Y$
}
Note this says: there is no largest set.

\defin{Cardinal}{$\kappa\in\Ord$ is called a cardinal, if $\kappa = |\kappa|$}
\note{}{\textbf{Claim: }The class $\Card$ of all cardinals is a proper class.
\begin{claimproof}
    Suppose $\Card$ is a set. then $\sup \Card = \gamma\in \Ord$
    Then $|\mathcal{P}(\gamma)|>|\gamma|\geq |\lambda|$ for every cardinals $\lambda\in \Card$
    that is a contradiction %TODO
\end{claimproof}
}
\defin{finite sets}{A set $X$ is called finite, if 
    $|X|$ is a finite ordinal, otherwise $X$ is called infinite.
}
\note{}{In particular a set is infinite iff $\omega$ injects into it.}% note: this is to keep in mind, good kriterium
\prop{Galileo}{
    A set $a$ is infinite iff it properly injects into itself.
}{
    $\implies$
    Suppose $a$ is infinite, then by note use $\omega$ injects into it 
    show $\omega$ injects properly in it self

    $\impliedby$ want: set finite then it does not inject properly into itself
    show: every finite ordinal is a cardinal inductively, then $\kappa = |\kappa|$

}
\subsection*{The $\aleph$-function}
\note{}{
    $\Card$ is cofinial in $\Ord$. That is for every $\alpha\in \Ord$ there is $\gamma\in \Card$ with $\alpha<\gamma$

    And $\Card$ is a proper subclass of $\Ord$, well ordered by the same ordering $\in$ as in $\Ord$

    There is a unique function (class function) from the class of all ordinals to the infinite cardinals 
    that preserves $\in$ \dots this function is called $\aleph$
    
}
\defin{}{Instead of $\aleph(0)$ we will write subscript $\aleph_0$
}
\begin{itemize}
    \item $\aleph_0 = |\omega|$
    \item $\aleph_{\alpha+1}$ is the smallest cardinal larger than $\aleph_\alpha$
\end{itemize}
\note{}{For every cardinal $\kappa$ there is a smallest cardinal $\kappa^+$ such that $\kappa<\kappa^+$ this is not the successor in the sense of the ordinals. It is
\begin{itemize}
    \item $n^+ = n+1$
    \item $\aleph_\alpha^+ = \aleph_{\alpha+1}$
\end{itemize}
}
\note{}{For every ordinal $\gamma$ we have
    $|\gamma|\leq \gamma<|\gamma^+|$
}
Similarly as with ordinals we will call cardinals of the form $\kappa^+$ successor cardinals.\outernote{successor cardinals} and non-zero, non-successor cardinals will be called limit cardinals.\outernote{limit cardinals}
\prop{}{\label{Aleph:cont}
    The function $\aleph$ is continuous with respect to the interval topology induced by $\in$. i.e.
    If $\lambda$ is a limit ordinal then $\aleph_\lambda = \sup_{\delta<\lambda}{\aleph_\delta}$
}{
    let $\gamma \defeq \sup_{\delta<\lambda}\aleph_{\delta}$
    have $|\gamma|\leq \gamma <|\gamma|^+$
    Assume $\gamma<\aleph_\lambda$
    then there exist a $\xi_0<\lambda$ such that $|\gamma| = \alpha_{\xi_0}$
    
    Then we would have $\gamma<|\gamma|^+=\aleph_{\xi_0+1}\leq \sup_{\delta<\lambda} \aleph_{\delta}$
    which is a contradiction.
    
    we have shown that $\aleph_\lambda\leq \sup_{\delta<\lambda}\aleph_{\delta}$ (TODO: other direction?)
}
\defin{countable sets}{
    A set $X$ is called to be countable iff $|X|\leq \aleph_0$ 
    and we say $X$ is uncountable otherwise.}

\subsubsection*{Continuum hypothesis (CH)}
\hypothesis{(CH)}{
    \[|\mathcal{P}(\omega)| = \aleph_1\]
    i.e. there is no cardinality between $|\RR|$ and $|\NN|$
    or $|\RR|$ is the first uncountable cardinality.
}

It can be shown that CH is independent of ZFC. (method to show this is called forcing, very popular method)

\section{Cardinal arithmetic}
\defin{}{Let 
    $\kappa,\lambda$ be cardinals then 
    $\kappa\otimes \lambda\defeq |\kappa\times\lambda|$
    $\kappa\oplus \lambda\defeq |(\kappa\times\{0\})\cup(\lambda\times\{1\})|$
    
}
\note{}{$\oplus,\otimes$ are commutative and associative.}
\thm{}{
    If $\kappa$ is infinite, then $\kappa\otimes\kappa = \kappa$
}{
    By induction on $\kappa$. What is ment by that is induction on $\alpha$ where $\aleph_\alpha =\kappa$.

    Base case: We should check it for $\omega$ 
    $\aleph_0\times \aleph_0 = \aleph_0$ is like finding bijection of $\omega^2 $ onto $\omega$.(diagonal)

    Now suppose: 
    $\forall \beta<\kappa |\beta|\otimes |\beta||\beta\times\beta| = |\beta|$

    On The cartesian product define order such that we get order isomorphism, then this is also a bijection.

    Define ordering on $\kappa\times \kappa$
    \[(\alpha,\beta)\prec (\alpha',\beta')\iff \begin{cases}
        \max\{\alpha,\beta\}<\max\{\alpha',\beta'\}\text{, or}\\
        \max\{\alpha,\beta\} = \max\{\alpha',\beta'\}\text{ and } \alpha<\alpha'\text{, or}\\
        \max\{\alpha,\beta\} = \max\{\alpha',\beta'\}\text{ and } \alpha =\alpha'\text{ and }\beta<\beta'\\
    \end{cases}\]
    \textbf{Claim:}$\prec$ is a well-ordering (Exercise)

    \textbf{Claim:} $(\kappa\times \kappa,\prec)$ is order-isomorphic to $(\kappa,\in)$
    From the claim we imediately get $|\kappa\times \kappa| = |\kappa|$
    \begin{claimproof}
        $\kappa\times \kappa = \bigcup_{\alpha<\kappa}\alpha\times \alpha$
        increasing union.
        then $\{\xi\times \xi\}$ is an initial segment of $\kappa\times\kappa$ (with respect to $\prec$)
        Consider $(\xi\times\xi,\prec)$ then by our inductive assumption this will have to be isomorphic to some ordinal $\gamma\in\Ord$.
        We know that the cardinality of that ordinal $\gamma$
        $|\gamma| = |\xi\times \xi| \stackrel{\text{ind. Hyp.}}{=}|\xi|<|\kappa|$
        So also $|\gamma|<|\kappa|$. so then $(\kappa\times\kappa,\prec)$ is order-isomorphic to $(\kappa,\in)$
    \end{claimproof}
}
\coroll{
    For every infinite cardinal $\kappa$ we have $\kappa \oplus \kappa = \kappa$
}
\begin{proof}
    Proof of corollary:
    $\kappa\oplus \kappa = |\kappa\times 2| \leq |\kappa\times\kappa| =|\kappa|$
\end{proof}

\defin{}{Define for $\kappa,\lambda$ cardinals
    $\kappa^\lambda \defeq |\{f: f \text{ is a function from $\lambda$ to $kappa$}\}|$
}
Note $2^\kappa = |\mathcal{P}(X)|$
\lemma{}{
    If $\lambda\geq \omega$ and $2\leq \kappa\leq \lambda$ then $\kappa^\lambda = 2^\lambda$.
}{
    \[2 ^{\lambda} = 2 ^{\lambda\otimes\lambda} = 2 ^{\lambda^\lambda}\geq\lambda^\lambda\geq \kappa^\lambda\]
    $ 2 ^{\lambda^\lambda}\geq\lambda^\lambda$ think of function that constantly maps to $0$ is part TODO image 5
    The other implication is obvious.
}
\datenote{16.01.2025}
\defin{}{
    A function $f:\alpha\to\beta$, where $\alpha,\beta$ are ordinals is said to be cofinal,
    if $\imag f$ is unbounded in $\beta$
    i.e. 
    \[\forall \gamma\in \beta\exists \xi \in \alpha \gamma\leq f(\xi)\]
}
The cofiniality of $\beta\in\Ord$ (denoted by $cof(\beta)$)
    is the smallest ordinal $\alpha$ such that there exists a function $f:\alpha\to\beta$ that is cofinal. Note: $cof(\beta)\leq \beta$, 

\bsp{}{
    $cof 1 = cof (\{\varnothing\}) = 1$ and in more generality
    $cof(\alpha + 1) = cof(\alpha \cup \{\alpha\}) = 1$
}
\note{}{ $cof (\beta)$ is always a cardinal.
    Let $\beta\in \Ord$ and let $f:cof(\beta)\to \beta$ be cofinal.
    Then $|cof(\beta)|\leq cof (\beta)|$ and there exists a bijection $h: |cof(\beta)|\to cof\beta$,
    so $f\circ h$ yields cofinal map and by minimality of $cof(\beta)$ have $|cof(\beta)| = cof (\beta)|$ 

    If $\beta$ limit ordinal then there is a strict increasing cofinal map $h:cof(\beta)\to \beta$ 
    Let $f:cof(\beta)\to \beta$ be cofinal define $h:cof(\beta)\to \beta$ by
    $$h(\xi) \defeq \max\{f(\xi), \sup_{\gamma<\xi}(h(\gamma)+1)\}$$
}
\prop{}{\label{limit:samecof}
    Suppose $\alpha,\beta$ are limit ordinals, $f:\alpha\to \beta$ strictly increasing and cofinal.
    Then $cof(\alpha) = cof(\beta)$
}{ One side is obvious
    \begin{itemize}
        \item $cof(\alpha)\geq cof(\beta)$ is clear
        \item $cof(\alpha)\leq cof(\beta)$
        Let $g:cof(\beta)\to \beta$ be cofinal
        define $h:cof(\beta)\to\alpha$ by 
        $$h(\xi) = \min\{\gamma<\alpha : f(\gamma)>g(\xi)\}$$
        cofinal in $\alpha$
    \end{itemize}
}
\coroll{For a limit ordinal $\alpha$, $cof(\aleph_\alpha) = cof(\alpha)$}
\begin{proof}
    Use \ref{Aleph:cont} and \ref{limit:samecof} on $f = \aleph$
\end{proof}
\coroll{For every ordinal $\beta$, $cof(\beta) = cof(cof(\beta))$}
\begin{proof}
    By cases
    \begin{itemize}
        \item $\beta$, $cof(\beta)$ are limit ordinals: there exists a strictly increasing map $cof(\beta) \to \beta$ result follows from \ref{limit:samecof}
        \item $\beta, cof(\beta)$ are not limits
        then $\beta = \alpha\cup \{\alpha\}$, $cof(\beta) = 1$ and $cof(cof(\beta)) = cof(1) = 1$
    \end{itemize}
\end{proof}
\defin{regular ordinals}{An ordinal $\beta$ is called regular, if $cof(\beta) = \beta$ (it is fixed point of cofininality map)}
\note{}{regular ordinals are cardinals and the first regular, infinite cardinal is $\omega$}
\lemma{}{$\kappa^+$ is reglular for $\kappa\geq \omega$}{
    If $f:\alpha\to \kappa^+$ is cofinal, then $\kappa^+ = \bigcup_{\gamma<\alpha} f(\gamma) = \sup_{\gamma<\alpha}f(\gamma)$.
    Each $f(\gamma)$ is an ordinal of cardinality less then $\kappa^+$ hence is less or equal than $\kappa$
    $$\kappa^+ = |\bigcup_{\gamma<\alpha}f(\gamma)| \leq |\alpha\times \kappa| \leq \max\{|\alpha|,|\kappa|\}$$
    because $\kappa<\kappa^+$ we have $\kappa^+ \leq|\alpha| = \alpha$ TODO check
}
\note{}{$\alpha$ limit ordinal then $cof(\aleph_\alpha) = \alpha$\\
If $\aleph_\alpha$ is regular then $cof \aleph_\alpha = cof \alpha \leq \alpha\leq \aleph_\alpha$
So $\alpha = \aleph_\alpha$
}
\defin{}{Let $\kappa$ be a cardinal.
    \begin{itemize}
        \item $\kappa$ is called weakly inaccessible, if
    $\alpha$ is a regular limit cardinal strictly greater than $\omega$
    \item $\kappa$ is called (strongly) inaccessible, if $\kappa>\omega$, $\kappa$ regular and
    for every $\lambda<\omega$ we have $2^\lambda<\kappa$
    \end{itemize}
}
We will see that 
$\kappa$ inaccessible implies that $\kappa$ is not a union of fewer that $\kappa$ sets each of cardinality less that $\kappa$.
\note{}{Existence of inaccessible cardinals does not follow from ZFC.}
The question is, why care then?

Axiom of existence of inaccessible cardinals:
\begin{equation}
    (IC) \quad\text{There exists some inaccessible cardinal.}
\end{equation}

(AC) has less desirable consequences e.g. Banach-Tarski-Paradox

If we drop AC we could not prove that the lebesgue meassure is countably additive,
so we may want to replace AC by something weaker (e.g. DC, dependent choice).
Various nice results that depent on AC still hold. 

LM the axiom that states: every set of reals is lebesgue meassurable
\thm{}{Con\dots consistency. 
    Con(ZF+DC+LM) is equal to Con(ZF + IC)
}{}
Note: We do now that ZF can not prove IC, but we dont know yet if ZF can prove the negation of IC.

\lemma{}{
    If $\kappa$ is an infinite cardinal and $\lambda\geq cof(\kappa)$ then $\kappa^\lambda>\kappa$
}{
    Fix a cofinal map $f:\lambda\to \kappa$.
    Consider any function $G:\kappa\to\kappa^\lambda$
    It suffices to show that $G$ can not be surjective.

    $\kappa^\lambda$ is technically the set of functions $\lambda\to\kappa$
    define $h:\lambda\to\kappa$ by 

    \[h(\xi) = \min\{\kappa \setminus(G(\alpha)(\xi)) : \alpha \leq f(\xi)\}\]
    If $h\in \imag G$ then there is $\alpha\in \kappa$ s.t. $G(\alpha) = \kappa$
    pick $\xi <\lambda$ s.t. $f(\xi)\geq \alpha$
    Then $h(\xi) = G(\alpha)(\xi)$ but by construction of $h$, $G(\alpha)(\xi) \neq h(\xi)$ so contradiction.
}
\coroll{If $\lambda\geq \omega$ then $cof(2^\lambda)>\lambda$
}
\begin{proof}
    We know that $(2^\lambda)^\lambda = 2^{(\lambda\otimes \lambda)} = 2^\lambda$
    So if $cof(2^\lambda)\leq \lambda$ then by lemma $(2^\lambda)^\lambda>2^\lambda$, a contradiction.
\end{proof}
\thm{(König)}{
    Let $I$ be a set, $(A_i)_{i\in I}$ and $(B_i)_{i\in I}$ indexed families of sets.

    If $\forall i\in I |A_i|<|B_i|$ then 
    $|\bigsqcup_{i\in I}A_i|<|\prod_{i\in I}B_i|$
}{Exercise with hints.}
Note: above thm is equivalent to AC

One can use König theorem to show that:
``$\kappa$ inaccessible implies that $\kappa$ is not a union of fewer that $\kappa$ sets each of cardinality less that $\kappa$''TODO: check if actually true

Next time : ZF can not proof ZFC
\section{Consistency of a theory}
\datenote{21.01.2025}
%Omega is not cofinal in aleph_1, also why cofinality is different from cardinality
Today we are going to state our last axiom, the axiom of foundation. We are also going to show a relative consistency theorem.
We are going to finish prob next course thursday

\axiom{Axiom of foundation (AF)}{
    Beeing an element of, does not admit an infinite decreasing chain.
    $$\forall x x\neq \varnothing \to \exists y(y\in x\land \forall z\in y z \notin x)$$
}
\note{}{
    Suppose we have a sequence of sets $(u_n)_{n\in\omega}$ s.t. $\forall n u_{n+1}\in u_n$ then $\{u_n : n\in \NN\}$ would contradict AF.
    and have $\forall x (x\notin x)$ too.
}
\defin{}{
    Class function
    $V:\Ord \to \mathcal{U}$ by transfinite induction on ordinals 
    and set $V_\beta \defeq \bigcup_{\alpha<\beta}{\mathcal{P}(V_\alpha)}$
    $V_0 = \varnothing$
    If $\alpha\leq \beta$ then $V_\alpha\subseteq V_\beta$ 
    (an example of increasing sequence, in contrast to note on axiom AF)
    $V_{\beta+1} = \mathcal{P}(V_\beta)$

    If $\lambda$ is a limit ordinal then $V_\lambda = \bigcup_{\alpha<\lambda}V_\alpha$

    $V$ is also a class given by $V(x)\equiv \exists \alpha\in\Ord\: x\in V_\alpha$
}
$V$ also gives us a way to associate a rank to each set.
\defin{}{
    For every $x$ such that $V(x)$ we define
    $\rk(x)\defeq \min\{\alpha : \: x\in V_\alpha\}$
}
\note{}{
    The rank $\rk(x)$ is always a successor ordinal.
}
\lemma{}{
    $V(x)$ iff $\forall y \: y\in x\to V(y)$
    And also if $V(x)$ then $\forall y (y\in x \to \rk(y)<\rk(x))$
}{
    ``$\implies$'' direction: 
    $V(x)$ and $\rk(x) = \beta+1$ then 
    $x\in V_{\beta+1}$ so $x\subseteq V_\beta$ hence $\forall y\in x \: y\in V_\beta$
    Also $\rk(y)\leq \beta <\beta+1 = \rk(x)$

    ``$\impliedby$''-direction:
    Suppose $\forall y\in x V(y)$
    Note that $\rk : V\to \Ord$ is bounded on $x$.
    Else, $\{\rk y : y \in x\}$ is unbounded in $\Ord$ and it is a set (image of function $\rk$ of a set)
    Take $\bigcup\{\rk y : y\in x\}\in \Ord$ and it would be an ordinal bigger than any other set, TODO

    Suppose $\{\rk y : y\in x\}$ is bounded by $\beta$ then $\forall y \in x y\in V_\beta$
    so $x\in V_{\beta+1}$.
}
\lemma{}{
    For every ordinal $\alpha\in \Ord$ we have $V(\alpha)$ and $\rk (\alpha) = \alpha+1$.
}{Exercise}
\defin{inductive closure}{
    For any set $x$ we define the function with domain $\omega$ 
    $f(0) \defeq x$
    $f(n+1) \defeq \bigcup_{y\in f(n)}y = \bigcup f(n)$
    and we define the closure of $x$ to be the union 
    $\cl x \defeq \bigcup_{n<\omega}f(n)$
}
\note{}{
    \begin{itemize}
        \item $x\subseteq \cl x$
        \item $\cl$ is transitive
        \item If $z$ is transitive set that contains $x$ then then $\cl x\subseteq z$ 
        ($\cl x$ is the unique transitive closure of $x$)
    \end{itemize}
}
\thm{}{
    (AF) holds iff $\forall x V(x)$
}{
    ``$\impliedby$''-direction: Suppose $\forall x V(x)$.
    We need to show that any set contains an element 
    Let $a\neq \varnothing$. Let $y\in a$ be of minimal rank.
    Then for every $c\in y$ we know $\rk c<\rk y$ so $c\notin a$ by minimality of $\rk y$.
    So $y\in a$ s.t. $y\cap a = \varnothing$
    
    ``$\implies$'' direction: Lets assume the axiom of foundation and by contradiction that $x$ is a set
    for which $\lnot V(x)$
    Have $x\subseteq \cl x$ 
    \textbf{Claim:} $Y = \{y\in \cl x : \: \lnot V(y)\}\neq \varnothing$
    \begin{claimproof}
        If $Y = \varnothing$ then $\forall y \in Y V(y)$ and $\rk$ bounded on $Y$. %TODO
    \end{claimproof}
    Let $y\in Y$. Then $\lnot V(y)$ so i.p. $y\not \subseteq V$, 
    so for some $z\in y$ have $\lnot V(z)$
    but because $\cl x $ is transitive, $z\in \cl x$ hence $z\in Y$
    hence $\forall y\in Y y\cap Y \neq \varnothing$, a contradiction with (AF).
}

By Gödels 2-nd incompletness theorem ZFC can not prove its own consistency.
There is a way to express consistency in the formal level, by coding and using peano arithmetic, and the above statement says that ZFC can not prove this sentence.
All we can hope for are relative consistency results, and therefore relate two theories with each other.
For example some theories are less debated about 
and it shows a way to prove independence of certain axioms from others.

\subsection{relative consistency}

\begin{itemize}
    \item ZFC The axioms \begin{enumerate*}
        \item extensionality
        \item union
        \item power-set
        \item ax scheme of replacement
        \item set ax
        \item axiom of infinity
        \item AC
        \item AF
    \end{enumerate*}
    (recall pairing and comprehension follow from the other)

    \item ZFC$^-$ is  ZFC without (AF)
    \item ZF is ZFC without (AC)
    \item ZF$^-$ is ZF without (AF)
\end{itemize}

We will use a tool that is in set theory called Relativization

Let $C$ be a class, $\phi(\underline{x},\underline{a})$, $\underline{a}\in C$

$\phi^C(\underline{x},\underline{a})$ defined by induction on compl of $\phi$
\begin{itemize}
    \item If $\phi$ is atomic then $\phi^C = \phi$
    \item $(\lnot \phi)^C = \lnot (\phi^C)$,  $( \phi\lor \psi)^C =  (\phi^C)\lor(\psi^C)$  
    \item $(\exists y \phi)^C = \exists y (C(y)\land \phi^C)$
    \item $(\forall y \phi)^C = \forall y (C(y)\to \phi^C)$
\end{itemize}


\thm{}{
    Suppose $(\mathcal{U},\in)\models ZF^-$ then $V$ constructed in $\mathcal{U}$ is such that
    $(V,\in)\models ZF$ 
    i.e. assume $ZF^-$ is consistent (has a model) and we get a model of ZF 
}{
    ZF$^-$
    Take 
    \begin{enumerate*}
        \item extensionality
        \item union
        \item power-set
        \item ax scheme of replacement
        \item set ax
        \item axiom of infinity
    \end{enumerate*}
    (recall pairing and comprehension follow from the other)
    Need to check if  $(\mathcal{U},\in)\models ZF^-$ and $V$ class defined by 
    $V = \bigcup_{\alpha\in \Ord}V_\alpha$, $V_\beta \defeq \bigcup_{\alpha<\beta}{\mathcal{P}(V_\alpha)}$
    then $(V,\in )\models ZF^-$ (AF will follow from prev. result)

    \begin{enumerate}
        \item Let $x,y\in V$ wts $(\forall z\in V z\in x\leftrightarrow z\in y)\leftrightarrow x=y$ 
        $x,y\in V$ so $x,y\subseteq V$
        and can use Axiom of extensionality in $\mathcal{U}$
        $(\forall z\in \mathcal{U} z\in x\leftrightarrow z\in y)\leftrightarrow x=y$
        
        \item Let $x\in V$ $U_x = \{z : \exists y \in x z\in y\}\subseteq V$
        so $\bigcup_x\in V$.
        \item Let $x\in V$ then $x\subseteq V$. so every subset of $x$ is a subset of $V$ hence is an 
        element of $V$. Powerset  $\mathcal{P} (x)\subseteq V$ hence  $\mathcal{P} (x)\in V$
        %TODO (range funciton is bounded on the ???)
        \item $\varphi(x,y)$ with parameters from $V$ and assume that $\varphi(x,y)$ defines a class function in $V$ i.e.
        $$\biggl(\forall x \exists ^{\leq 1}y \varphi (x,y)\biggr)^V$$
        i.e.
        $$\forall x (V(x)\to (\exists ^{\leq 1}y \varphi (x,y)) )$$
        Then $\psi (x,y)\defaq V(x)\land V(y)\land \varphi^V(x,y)$
        defines a class function in $\mathcal{U}$, use replacement in $\mathcal{U}$
        That yields 

        $$\forall a \exists b y\in b \iif (\exists x \in A \varphi(x,y))\iif \exists x\in a \varphi^V(x,y)\land V(y)$$
        $b\subseteq V$ hence $b\in V$ and $b$ is the image of $a$ under the class function given by $\varphi$.
        \item Since we proved that every ordinal is in $V$, $\varnothing\in V$.
        \item enough to show that $\omega\in V$. in fact we know that any ordinal is in $V$.
    \end{enumerate}
}

An ordinal $\alpha$ is called regular, if it is equal to its own cardinality $\cof \alpha = \alpha$
A cardinal $\kappa$ is called inaccessible if $\kappa > \omega$ and $\kappa $ is regular.

It suffices to say $\forall \lambda<\omega 2^\lambda <\kappa$
Suppose now that $(\mathcal{U}, \in)\models ZFC$ 
\lemma{}{\label{lem:inacc1}
    If $\kappa$ is an inaccessible cardinal, then $|V_\kappa| = \kappa$. Moreover 
    for every $a\subseteq V_\kappa$ we have $a\in V_\kappa$ iff $|a|<\kappa$
}{
    Recall, if $\alpha\in \Ord$ then $\alpha\subseteq V_\alpha$. i.p.
    $\kappa\subseteq V_\kappa$ so $|\kappa|\leq |V_\kappa|$
    
    The other direction: 
    For this we are going to show inductively that for all $\xi<\kappa$ that $|V_\xi|<\kappa$
    Then $\kappa\geq |V_\kappa|$ follows.
    If $|V_\xi|<\kappa$ then $|V_{\xi+1}| = |\mathcal{P}(V_\xi)|\leq |2^{|V_\xi|}|<\kappa$
    by $\kappa$ inaccessible.
    Suppose  $|V_\xi|<\kappa$ for all $\xi<\lambda<\kappa$ where $\lambda$ is a limit ordinal.
    \[|V_\lambda|  = |\bigcup_{\xi<\lambda}V_\xi| \leq  \sup{\xi<\lambda}|V_\xi |<\kappa\]
    here we use regularity of $\kappa$.

    So have $|V_\kappa| = \kappa$.
    Left to show: for every $a\subseteq V_\kappa$ have $a\in V_\kappa$ iff $|a|<\kappa$
    \begin{itemize}
        \item ``$\impliedby$''-direction: Assume $a\subseteq V_\kappa$ and $|a|<\kappa$ 
        $\rk: a \to \Ord$ is not cofinal in $\kappa$ because $cof \kappa = \kappa > |a|$
        So for some ordinal $\beta$ have $a\subseteq V_\beta$ and then $a\in V_{\beta+1}\subseteq V_\kappa$
        \item ``$\implies$''-direction: it suffices $a\in V_\kappa$ then $|a|<\kappa$
        Exercise.
    \end{itemize}
}

\datenote{23.01.2025}


\lemma{}{Let $(\mathcal{U},\in)\models ZFC$. If $\kappa$ is inaccessible, then $V_\kappa\models ZFC$.}{
    Will check (AC) and replacement, the remaining axioms are an exercise.
    \begin{itemize}
        \item[(AC)]: Suppose $a\in V_\kappa$ and $a$ is a family of pairwise disjoint, non-empty sets.
            By (AC) in $\mathcal{U}$ there is a transversal $T$ in $\mathcal{U}$ for the set $a$. What is 
            left to show is $T\in V_\kappa$.
            Have $a\subseteq V_\kappa$ then every subset of $a$ is a subset of $V_\kappa$ so in particular 
            $T\subseteq V_\kappa$.
            We do know that $a\in V_\kappa$ so $|a|<\kappa$ and $T\subseteq a$ we have $|T|\leq |a|$ so by 
            the previous lemma $T\in V_\kappa$.
        \item[(RE)]: $\varphi(x,y)$ a formula with parameters in $V_\kappa$ that defines a class function in 
            $V_\kappa$. i.e. $$\forall x\in V_\kappa \exists^{\leq 1} y\in 
            V_\kappa \varphi^{V_\kappa}(x,y)$$ 
            let $a\in V_\kappa$ then $\psi(x,y) \equiv x\in V_\kappa \land y\in V_\kappa \land \varphi(x,y)$
            does define a class function on $\mathcal{U}$ with domain contained in $V_\kappa$ 
            %TODO check class and on cal(U)
            So $f[a]\subseteq V_\kappa$ and $ |f[a]|<\kappa$ so $f[a]\in V_\kappa$.
    \end{itemize}
}
Now we are ready to state our meta-theorem, it is not a statement in first order language of set theory.
\thm{}{
    If ZFC is consistent then ZFC + ``There are no strongly inaccessible cardinals'' is also consistent
}{
    Assume we have a model of ZFC, $(\mathcal{U},\in)\models ZFC$.
    \begin{itemize}
        \item If $\mathcal{U}$ does not contain any inaccessible cardinals then we are done.
        \item Assume that $\mathcal{U}$ does contain inaccessible cardinals. Let $\kappa$ be the smallest inaccessible cardinal. 
    \end{itemize}
    What we want to show is that there are no inaccessible cardinals in $V_\kappa$.
    An ordinal is by definition a transitive set, well-ordered by $\in$.
    By (AF) $\alpha$ is transitive and $\alpha$ is linearly ordered by $\in$.
    i.e. $\alpha$ ordinal iff 
    \begin{equation}\label{equation:ordinal}
        \forall x,y \in \alpha (x\in y \lor y\in x \lor x=y)
    \land \forall x(x\in \alpha \to x\subseteq \alpha)
    \end{equation}
    \textbf{Claim 1: }The ordinals in $V_\kappa$ are the ordinals below 
    $\kappa$ i.e. $Ord^{V_\kappa}=\kappa$.\\
    \begin{claimproof}
        If $\alpha<\kappa$ then $\alpha\subseteq V_\kappa$ ($\rk \alpha = \alpha+1$ and $V_{\alpha+1}
        \subseteq V_\kappa$ so $\alpha\in V_\kappa$ ($\alpha+1<\kappa$)) $\alpha$ is an ordinal so \ref
        {equation:ordinal} holds. and $\alpha\in \Ord^{V_\kappa}$
        If $\alpha\in \Ord^{V_\kappa}$ then $\alpha\in V_\kappa$ and $\alpha$ is transitive and totally 
        ordered by $\in$.
        % TODO check: All elements of $\alpha$ are elements of $V_\kappa$. 
        Hence $\alpha$ is an ordinal. Left to show $\alpha$ is below $\kappa$.
        Have $|\alpha|<\kappa$,
        by the \ref{lem:inacc1} $\alpha<\kappa$.
    \end{claimproof}
    \textbf{Claim 2: }The cardinals in $V_\kappa$ are the cardinals that are below $\kappa$.\\
    \begin{claimproof}
        Suppose $\lambda$ is a cardinal in $V_\kappa$. in particular $\lambda$ is an ordinal in $V_\kappa$ 
        and therefore an ordinal in $\mathcal{U}$ and by the \ref{lem:inacc1} $\lambda = |\lambda| 
        <\kappa$.
        left to show: $\lambda$ is an actual cardinal in $\mathcal{U}$
        Suppose there is a bijection $f:\lambda\to \alpha$ to some smaller ordinal $\alpha$. Then by 
        Claim 1, $\alpha\in \Ord^{V_\kappa}$ and $f\subseteq V_\kappa$ (the graph $f\subseteq \lambda\times 
        \alpha\subseteq \lambda \times \lambda$ so $f\subseteq \mathcal{P}\mathcal{P}\mathcal{P}(V_{\lambda
        +1})\eqdef V_\beta$, $\rk \lambda = \lambda+1$ and have $\beta<\kappa$ by inaccessability).
        $f\subseteq V_\kappa$ and $|f|<\kappa$ so by \ref{lem:inacc1} $f\in V_\kappa$.
        $f$ is a bijection in $V_\kappa$ between $\lambda$ and a smaller ordinal, a contradiction.

        Suppose $\lambda<\kappa$ and $\lambda$ cardinal, then by \ref{lem:inacc1} $\lambda\in V_\kappa$ and $\lambda\in \Ord^{V_\kappa}$.
        If there would be a bijection in $V_\kappa$ between $\lambda$ and a strictly smaller ordinal 
        $\alpha$ then $\alpha$ would be an actual ordinal. By an argument as before would get a bijection in 
        $\mathcal{U}$ between $\lambda$ and smaller ordinal.
    \end{claimproof}
    If $\lambda\in V_\kappa$ is a cardinal in $V_\kappa$ then $\lambda<\kappa$ and by claim 2 it is a cardinal in $\mathcal{U}$, $\lambda$ can not be inaccessible. by choice of $\kappa$.

    Note that only $\mathcal{U}$ knows that $\lambda$ is not inaccessible. We therefore need to check that also $V_\kappa$ knows that $\lambda$ is inaccessible.
    Reasons for $\lambda$ to not be inaccessible:
    \begin{itemize}
        \item $\lambda\leq \omega$ (in $\mathcal{U}$ ) hence $(\lambda\leq \omega)^{V_\kappa}$ bc. $\omega\in V_\kappa$
        \item $\lambda\leq 2^\xi$ for some $\xi <\lambda<\kappa$ hence
        $\xi,2^xi\in V_\kappa$ and have $(\xi<\lambda<2^\xi)^{V_\kappa}$
        \item $\lambda$ is not regular, if we have a function $f:\alpha\to \lambda$ cofinal and $\alpha$ an ordinal $\alpha<\lambda$, then $\alpha\in V_\kappa$ hence $f\in V_\kappa$ then $\lambda$ is not regular in $V_\kappa$.
    \end{itemize}
}

Note: $Ord^{V_\kappa}=\kappa$. but $Ord^{V_\kappa}$ is no set in $V_\kappa$.

\appendix
\chapter{Appendix}
\setcounter{section}{2}
\section{On Model theory}
\thm{Löwenheim-Skolem}{
    Let $\mathcal{L}$ be a language of cardinality $\lambda$. $\Gamma$ a set of formulas and $\Sigma$ a set of sentences.
    \begin{enumerate}[label = (\roman*)]
        \item If $\Gamma$ is satisfiable, then it is satisfiable in some structure of cardinality at most $\lambda$
        \item If $\Sigma$ has any model, then it has a model of cardinality at most $\lambda$.
    \end{enumerate}
}{
    by using the LST theorem.

}

\section{On Boolean Algebras}
\defin{lattice}{%wikipedia
    A \imp{lattice} is a set $L$ with two binary, commutative and associative operations $\lor, \land$ satisfying the absorbtion axioms.
    \[\begin{aligned}
        &\forall a\forall b \: a\land (a\lor b) = a \\
        &\forall a \forall b \:  a \lor (a\land b) = a
    \end{aligned}\]
    A lattice is called \subimp{distributive}, if the distributive axioms hold.\\
    \[\begin{aligned}
        &\forall a\forall b\forall c \: a \land (b\lor c) = (a\land b)\lor (a\land c)\\
        &\forall a \forall b \forall c \: a \lor (b\land c) = (a\lor b)\land (a\lor c)\\
    \end{aligned}\]
    A lattice is called \subimp{bounded}, if it has a least element $0$ and a greatest element $1$.\\
    \[\begin{aligned}
        &\exists 0\forall a\: a \lor 0 = a\\
        &\forall 1 \forall a\: 1 \lor a = 1\\
    \end{aligned}\]
    A lattice is called \subimp{complemented}, if it is bounded and every element $a$ 
    has a complement $b$\\
    satisfying $a\lor b = 1$ and $a\land b = 0$.
    \[\forall a \exists b \: a\lor b = 1 \text{ and } a\land b = 0\]

}
\defin{Alternative Def: Boolean Algebra}{%from last year
    A \graybf{boolean algebra} is a set $B$ with
    \begin{itemize}
        \item distinguished elements $0,1$ (called zero and unit of $B$)
        \item a unary operation $'$ on $B$ (called \graybf{complementation})
        \item two binary operations $\lor$ called \graybf{join} and $\land$ called \graybf{meet} s.t. for all $x,y,z \in B$ 
        \begin{enumerate}[label=(\roman*)]
            \item $x\lor 0 = x$ \qquad  $x\land 1 = x$
            \item $x\lor x' = 1$ \qquad   $x\land x' = 0$
            \item $x \lor y = y \lor x$ \qquad   $x\land y = y\land x$
            \item $(x\lor y)\lor z = x\lor (y\lor z)$ \qquad   $(x\land y)\land z = x\land (y\land z)$
            \item $x\lor (y\land z) = (x\lor y)\land (x\lor z)$\qquad    $x\land (y\lor z) = (x\land y)\lor (x\land z)$
        \end{enumerate}
    \end{itemize}
}
With this definition a boolean algebra is exactly a complemented distributive lattice closed under the additional complementation map. The definition is compatible with the definition given in the chapter of boolean algebras.
However we must be careful, a subalgebra of a Boolean algebra must again be closed under the restricted complementation map.\footnote{See \url{https://math.nmsu.edu/people/personal-pages/files/ESSLLI2.pdf} on slide 7 for example.}
\bsp{}{Let $X\neq \varnothing$ be a set, $B \defeq \mathcal{P}(X)$ the power set of $X$, $0\defeq \varnothing$ and $1\defeq S$, 
    $$': \mathcal{P}(S)\to \mathcal{P}(S), x' \defeq S\backslash x \qquad x\lor y \defeq x\cup y, \quad x\land y \defeq x\cap y \text{ for } x,y\in \mathcal{P}(S)$$
}

\lemma{}{ Let $(B,',\lor,\land,0,1)$ be a boolean algebra. Then it holds
    \begin{enumerate}[label=\alph*)]
        \item $0' = 1$, $1' = 0$
        \item $x\lor x = x$, $x\land x = x$
        \item $(x')'= x$
        \item $(x\lor y)' = x' \land y'$, $(x\land y)' = x' \lor y'$
        \item $x\lor y = y \text{ iff } x\land y = x$
    \end{enumerate}
}{}
\lemma{}{
    \begin{enumerate}[label=\alph*)]
        \item $x\leq y \defaq x\lor y = y$ defines a partial ordering on $B$ (inclusion) and it holds
        \item $x\lor y$ is the least upper bound of $\{x,y\}$ in $B$\\
            $x\land y$ is the greatest lower bound of  $\{x,y\}$ in $B$
        \item $0\leq x\leq 1$ for all $x\in B$
    \end{enumerate}
}{}

\defin{Opposite of boolean algebra}{Let $(B,',\lor,\land,0,1)$ be a boolean algebra. The boolean algebra $B^{\text{op}}$ is defined by
    $$B^\text{op}\defeq B,\quad 0^\text{op} \defeq 1,\quad 1^\text{op} \defeq 0,\quad' \text{ stayes the same as for} B,\quad\lor^\text{op} \defeq \land,\quad\land^\text{op} \defeq \lor$$
    Note: $(B^\text{op})^\text{op} = B$
}
\defin{Subalgebra}{A \graybf{subalgebra} of $B$ is a subset $A\subseteq B$ s.t. $0,1\in A$ and $A$ is closed under $',\land,\lor$.
    The subalgebra generated by $P\subseteq B$ is defined to be the smallest subalgebra containing $P$. Equivalently it is the 
    intersection of all Subalgebras of $B$ that contain $P$.
}
\bsp{Power set algebra}{Let $S$ be a set then $\mathcal{P}(S)$ defines a boolean algebra on $S$.
    $B \defeq \{x\in \mathcal{P}(S): x \text{is finite or cofinite}\}$ is a subalgebra of $\mathcal{P}(S)$
    w/ set of generators $\{\{s\}:s\in S\}$}
\note{}{We will prove the Tarski-Stone Theorem: every boolean algebra is isomorphic to an algebra on a set.}

\bsp{Lindenbaum Algebra of $\Sigma$}{
    Let $A$ be a set of prop. atoms, $\propM(A)$ the set of prop. generated by $A$.
    Further let $\Sigma \subseteq \propM(A)$ and $p,q,r$ range over $\propM(A)$.\\
    We say $p$ is $\Sigma$-equivalent to $q$ iff $\Sigma \models_\text{taut} p\leftrightarrow q$
    $\Sigma$-Equivalence is an equivalent relation on $\propM(A)$ and $\propM(A)/\Sigma$ is a boolean algebra with
    $$0\defeq \bot/\Sigma,\quad1\defeq \top/\Sigma,\quad(p/\Sigma)' \defeq (\lnot p)/ \Sigma,\quad(p/\Sigma \lor q/ \Sigma)\defeq (p\lor q)/ \Sigma,\quad(p/\Sigma \land q/ \Sigma)\defeq (p\land q)/ \Sigma$$
    a set of generators is $\{a/\Sigma : a\in A\}$
}
\defin{Homomorphisms of boolean algebras}{Let $B,C$ be boolean algebras. A map $\phi: B\to C$ is a (homo)morphism of boolean algebras iff
    $\forall x,y\in B$ it holds
    \begin{itemize}
        \item $\phi(0_B) = 0_C$
        \item $\phi(x') = \phi(x)'$
        \item $\phi(x\lor y) = \phi(x)\lor \phi(y)$
        \item $\phi(x\land y) = \phi(x)\land \phi(y)$
    \end{itemize}
    If $\phi:B\to C$ is bijective too , we call $\phi$ an isomorphism and $\phi^{-1}:C\to B$ is also a morphism of boolean algebras.
}
\note{}{$\phi(B)$ is subalgebra of $C$}
\bsp{}{Let $S,T$ be sets then a function $f:S\to T$ induces a morphism of boolean algebras $\mathcal{P}(T)\to \mathcal{P}(S): y\mapsto f^{-1}(y)$
If $S\subseteq T$ and $f$ the inclusion map $S\hookrightarrow T$ then we get a boolean algebra morphism $Y\to Y\cap S$.\\
    \begin{itemize*}
        \item $id_B: B\to B$ \qquad 
        \item $x\mapsto x': B\to B^{\text{op}}$ are both isomorphism
    \end{itemize*}
}
\note{}{A boolean algebra morphism $\phi: B\to C$ is injective iff $\ker f = 0_B$}
\lemma{}{\label{boolLemma}
    Let $X_1,\dots X_m\subseteq S$ and $\mathcal{A}$ a boolean algebra on $S$ generated by $\{X_1,\dots X_m\}$. Then $\mathcal{A}$ 
    is finite and isomorphic to $\mathcal{P}(\{1,2,\dots n\})$ for some $n\leq 2^m$.
}{
    TODO
}
\defin{Trivial algebras}{\begin{itemize}
    \item $B$ is trivial if $|B| = 1$ (equivalently $0=1\in B$) 
    according to \ref{boolLemma} $B$ is isomorphic to $\mathcal{P}(\varnothing)$
    \item If $|S|=1$ then $|\mathcal{P}(S)| = 2$ 
    TODO
\end{itemize}}
\defin{Ideal}{An ideal of $B$ is a subset of $I\subseteq B$ s.t.
    \begin{itemize}
        \item[(I1)] $0\in I$
        \item[(I2)] $\forall a,b \in B$ it holds \qquad 
            $a\leq b$ and $b \in I\implies a\in I$\qquad and \qquad $a,b\in I\implies a\lor b\in I$ 
    \end{itemize}
}
\bsp{}{$F_{\text{in}} = \{F\subseteq S: F \text{ finite}\}$
    is ideal in $\mathcal{P}(S)$.
}
\note{}{If $I$ is an ideal of $B$ then 
    $I\lor b \defeq \{x\in B: x = a\lor b \text{ for some } a \in I\}$ is the smallest ideal w/ respect of $\subseteq$ of $B$ that contains $I\cup \{b\}$.
}
\bsp{}{\begin{itemize}
\item For a boolean algebra morphism $\phi: B\to C$ the kernel $\ker(\phi)$ is an ideal in $B$.
\item If $I$ is an ideal in $B$ then $a =_I b \defaq a\lor x = b\lor x$ for some $x\in I$ defines an equivalent relation and
$B/_{=_I}$ is a boolean algebra w/ 
$$0\defeq 0/_{=_I}\quad 1\defeq 1/_{=_I}\quad (a/_{=_I})' \defeq a'/_{=_I}\quad a/_{=_I}\lor b/_{=_I} \defeq (a\lor b)/_{=_I}\quad a/_{=_I}\land b/_{=_I} \defeq (a\land b)/_{=_I}$$
Then $\phi: B\to B/_{=_I}: b\mapsto b/_{=_I}$ is a boolean algebra morphism w/ $\ker(\phi)=I$
\end{itemize}}
\newpage
\subsection{Notes on Stone spaces}\label{Appendix:Stone}
\defin{topological properties}{\label{Appendix:def:top}
    Let $X$ be a \imp{topological space}. $X$ is said to 
    \begin{enumerate}[label=(\roman*)]
        \item be \subimp{compact}, if 
        every open cover of $X$ has a finite subcover. A subset $K\subseteq X$ of a topological space is called compact if it is a compact subspace of $X$.
        \item be a \subimp{T0-space} or equivalently hausdorff, if
        \[\forall x \forall y \: x\neq y \to \exists U,V\in \tau \: (x\in U\land y \notin U)\lor (x\notin V \land y\in V)\]
        \item be a \subimp{T2-space}, if 
        \[\forall x \forall y \: x\neq y \to \exists U,V\in \tau \: (x\in U\land y\in V \land U\cap V = \varnothing)\]
        \item be \subimp{totally seperated}, if 
        \[\forall x \forall y \: x\neq y \to \exists U,V\in \tau \: (x\in U\land y\in V \land U\cap V = \varnothing \land U\cup V = X)\]
        \item be \subimp{zero-dimensional} with respect to the \underline{small inductive dimension}, if it has a base for the topology consisting of clopen sets.
        \[\forall U \in \tau \,\forall x\in U \, \exists W\in \tau \: (x\in W\subseteq U\land W^c\in \tau)\]
        \item\label{Appendix:Top:irred} be \subimp{irreducible}, if one of the equivalent conditions below is satisfied
        \footnote{from: \url{https://en.wikipedia.org/wiki/Hyperconnected_space}}
        \begin{enumerate}
            \item No two nonempty open sets are disjoint.
            \item $X$ cannot be written as the union of two proper closed subsets. Proper means that none of the sets are equal to $X$ or the empty set.
            \item Every nonempty open set is dense in $X$.
            \item The interior of every proper closed subset of $X$ is empty.
            \item Every subset is dense or nowhere dense in $X$.
            \item\label{Appendix:Top:irred:cond6} No two points can be separated by disjoint neighbourhoods.
        \end{enumerate}
        An irreducible set is a subset of a topological space for which the subspace topology is irreducible.
        \item have a \subimp{generic point}, if there is a singleton $\{p\}\subseteq X$, whose closure is $X$.
        A subset $Z\subseteq$ of a topological space is said to have a gerneric point, if
        \[\exists! p\in Z \: \overline{\{p\}} = Z\]%TODO check uniqueness
        \item be \subimp{sober}, if every nonempty irreducible closed subset of $X$ has a (necessarily unique) generic point.
        \[\exists p\in X \: \overline{\{p\}}=X\]%TODO check if needs to be unique
        \item be \subimp{coherent}, if all of the following conditions are satisfied.\\
        Let $K^\circ(X) = \{V \subseteq X : V\in \tau \land V \text{ is compact}\}$
        \begin{enumerate}
            \item $X$ is compact, T0 and sober
            \item $K^\circ(X)$ is a basis for the topology
            \item $K^\circ(X)$ is closed under finite intersections
        \end{enumerate}
    \end{enumerate}

}
\newpage
\lemma{}{
    The definitions in \ref{Appendix:def:top} \ref{Appendix:Top:irred} are indeed equivalent.
}{  $(c)\implies(d)$, $(d)\implies(e)$ and $(f)\implies(a)$ are left as an exercise.
    \begin{enumerate}[leftmargin=2.1cm]
        \item[$(a)\implies(b)$] Suppose $X$ can be written es the union of two proper closed sets. 
        Then $X = A \cup B$ with $A,B\neq \varnothing$ and $A,B\neq X$. Then $A^c,B^c$ are open, nonempty and disjoint ($A^c\cap B^c = (A\cup B)^c = \varnothing$).
        \item[$(b)\implies(c)$] Suppose there is a nonempty open set $U$ with $\overline{U} \neq X$.
        Take $A \defeq U^c$ and $B \defeq \overline{U}$. We have $\varnothing \neq \overline{U}^c\subseteq U^c \neq X$ and $\varnothing \neq U\subseteq \overline{U} \neq X$ so $A,B$ are proper closed sets with $X = A \cup B$.
        \item[$(e)\implies(f)$] Suppose $(f)$ fails, then there exist $x,y\in X$ with $x\neq y$ and 
        there exist neighbourhoods $U_x\in \mathcal{U}(x)$ and $U_y\in \mathcal{U}(y)$ with 
        $U_x\cap U_y = \varnothing$. By definition there exist $V_x, V_y\in \tau$ with 
        $x\in V_x\subseteq U_x$ and 
        $y\in V_y\subseteq U_y$. Hence $x\in V_x^\circ = V_x$ and 
        $y\notin \overline{V_x}\subseteq V_y^c$ and $(e)$ fails too.
    \end{enumerate}
    \vspace{-0.6cm}
}
\defin{Stone space}{
    A Stone space is a zero-dimensional, compact, hausdorff topological space $X$.
}
There are other equivalent equivalent definitions of Stone spaces in other mathematical works.
\prop{}{The following statements are equivalent:
    \begin{enumerate}[label=(\roman*)]
        \item $X$ is a Stone space
        \item $X$ is compact and totally seperated
        \item $X$ is compact, T0 and zero-dimensional
        \item $X$ is T2 and coherent
    \end{enumerate}
}{
    We show $(\romanNum{1})\Longleftrightarrow (\romanNum{2})$, $(\romanNum{3})\implies (\romanNum{1})$, $(\romanNum{1})\implies (\romanNum{4})$, $(\romanNum{4})\implies (\romanNum{3})$
    \begin{enumerate}[leftmargin=2cm]
        \item[$(\romanNum{1})\implies(\romanNum{2})$:] We just have to show that $X$ is totally seperated. Let $x,y\in X$ with $x\neq y$. Since $X$ is hausdorff there exists a open set $U$ such that $x\in U$ and $y\notin U$. For this $U$ there exists a $W\subset U$ with $x\in W$ which is clopen. Therefore $W^c$ is clopen too and $y\in U^c\subseteq W^c$, $W^c \cup W = X$.
        
        \item[$(\romanNum{1})\impliedby(\romanNum{2})$:] T2 follows from totally seperatedness, so
        let $U$ be an open set and $x\in U$. We have to show that there exists a clopen subset $W\subseteq U$ with $x\in W$. We define $V_y$ to be the clopen set $V$ we get from totally seperatedness with respect to $x$ and $y$. From $y\in V_y$ we get that
        $\{U\}\cup \{V_y : y\in U^c\}$ is an open cover of $X$, which is compact so there is a natural number $n$ and $y_1, \dots y_n\in U^c$ with $U\cup \bigcup_{i\leq n} V_{y_i} = X$,
        hence $\bigcup_{i\leq n} V_{y_i}\supseteq U^c$ and therefore $\bigcap_{i\leq n} V_{y_i}^c\subseteq U$.
        $W\defeq \bigcap_{i\leq n} V_{y_i}^c$ is clopen, since finite intersections of clopen sets are clopen. Furthermore for every $i\leq n$ we have $x\in V_{y_i}^c$, so $x\in W$.

        \item[$(\romanNum{3})\implies (\romanNum{1})$:] Let $x,y\in X$ with $x\neq y$. Since $X$ is T0,
        There exists open sets $U,V$ with \\
        $x\in U\land y \notin U$ or $x\notin V \land y\in V$
        Assuming $x\in U\land y \notin U$ does not hold, then $x\notin V \land y\in V$.
        Since $X$ is zero-dimensional with respect to the small inductive dimension, there exists 
        a clopen set $W\subseteq V$ with $y\in W$. Therefore $W^c\in \tau$, $x\in W^c$ and we have shown that $X$ is a T2-space.

        \item[$(\romanNum{1})\implies (\romanNum{4})$:] 
        \begin{enumerate}
            \item compact and T2 are clear, we have to show that $X$ is sober.
            Let $Z\subseteq X$ be a nonempty irreducible closed subset. Subspaces of T2 are T2, and 
            since $Z$ is irreducible, using condition \ref{Appendix:Top:irred:cond6}
            ``No two points can be separated by disjoint neighbourhoods.'' we get that $Z$ has to be a singleton, and for T2-spaces, singletons have a unique generic point, hence $X$ is sober.
            \item Let $\sigma$ be the basis for the topology $\tau$ consisting of clopen sets. Since $X$ is T2, the closed sets are exactly the compact sets. Therefore
            $\sigma\subseteq K^\circ(X)\subseteq \tau$ and $K^\circ(X)$ is a basis for the topology.
            \item Let $n\in \NN$, $A_1,\dots A_n\in K^\circ(X)$. Like above, all $A_i$ are closed, hence
            clopen. The finite intersection of clopen sets $\bigcap_{i\leq n}A_i$ is clopen and therefore open and compact and we have $\bigcap_{i\leq n}A_i\in K^\circ(X)$
        \end{enumerate}
        \item[$(\romanNum{4})\implies (\romanNum{3})$:]
        Compact and T0 are given by definition. $K^\circ(X)$ is a basis for the topology and since $X$ is T2, every set of $K^\circ(X)$ is closed, hence $X$ is zero-dimensional.
    \end{enumerate}
    \vspace{-0.7cm}
}
Note that in the previous lemma $(\romanNum{1})\implies (\romanNum{3})$ is for free.

\lemma{}{
    Let $\mathcal{B}$ be a boolean Algebra and $U\subseteq B$. 
    Then $U$ is a ultrafilter on $\mathcal{B}$ if and only if $h:B\to \{0,1\}$, $h(x) \defeq\raisebox{\depth}{$\chi$}_U(x)$ is a homeomorphism from $\mathcal{B}$ to the two-element boolean algebra.
}{
    Let $x,y\in B$. 
    \begin{itemize}
        \item[$\implies$:] Clearly $x+y\geq x$ and $x+y\geq y$. Suppose $h(x)+h(y)=1$ then $x\in U$ or $y\in U$, by (F3) $h(x+y) = 1$.
        Otherwise $x,y\notin U$. By (UF) we have $\overline{x},\overline{y}\in U$.
        By (F2) we have $\overline{x}\cdot \overline{y}\in U$, but $\overline{x}\cdot \overline{y} = \overline{x+y}$.
        By (F1) we have $x+y\notin U$.

        If $h(x)\cdot h(y) = 1$ then $x,y\in U$, by (F2) $x\cdot y\in U$ and $h(x\cdot y) = 1$.
        Similarly $x\cdot y \leq x,y$, so if $h(x\cdot y) = 1$ then $x\cdot y\in U$ and by (F3) $h(x)\cdot h(y) = 1$.

        $h(0) = 0$ follows from (F1).

        $h(1) = 1$ because $U\neq \varnothing$ and (F3).

        $\overline{h(x)} = h(\overline{x})$ by (UF) and (F1). 
        Suppose $1 = \overline{h(x)}$ then $x\notin U$ by (UF) we have $\overline{x}\in U$, so $h(\overline{x})=1$
        Suppose $1=h(\overline{x})$ then $\overline{x}\in U$ by (F1) $x\notin U$ and therefore $\overline{h(x)}=1$

        \item[$\impliedby$:] (F1): Suppose $0\in U$ then $h(0) = 1 \neq 0$.
        
        (F2): Let $x,y\in U$ then $1 = h(x)\cdot h(y) = h(x\cdot y)$, so $x\cdot y\in U$.

        (F3): Let $x\in U$ and $y\in B$ with $x\leq y$. Then by definition $x+y = y$, hence
        $h(y) = h(x+y) = h(x+y) = h(x)+ h(y) = 1+h(y) = 1$.
        
        (UF): Suppose there is a $x\in B$ with $x,\overline{x}\notin U$ then 
        $0 = h(x) + h(\overline{x}) = h(x+\overline{x}) = h(1) = 1$.
    \end{itemize}
}


Existence and uniqueness of a topology generated by a subset of the power set
\thm{}{\label{Appendix:Top:BasisThm}
    Let $X$ be a set, $\sigma\subseteq \mathcal{P}(X)$. Then it is equivalent:
    \begin{enumerate}
        \item $\sigma$ is a basis for a uniquely determined topology $\tau\supseteq \sigma$.
        \item $X = \bigcup_{B\in \sigma}B $ and 
        \begin{flalign*}
            \forall B_1, B_2\in \sigma \,\forall p\in B_1\cap B_2\, 
            \exists B_3\in \sigma\: (p\in B_3\subseteq B_1\cap B_2)&&
        \end{flalign*}
        
        
    \end{enumerate}
}{}


\thm{}{\label{Appendix:Thm:Stone}
    \begin{enumerate}[label=(\roman*)]
        \item If $\mathcal{B}\models BA$, then $S(\mathcal{B})$ is a Stone-space
        \item If $\mathcal{S}$ is a Stone space then the clopen subsets of $\mathcal{S}$ form a boolean algebra of sets denoted by $B(\mathcal{S})$.
        \item Every boolean algebra $\mathcal{B}$ is isomorphic to the boolean algebra $B(S(\mathcal{B}))$ with $a\mapsto \langle a\rangle$. Hence $\mathcal{B}$ is isomorphic to a subalgebra of the boolean algebra $\mathcal{P}(S(\mathcal{B}))$ of sets
        \item Every Stone space $\mathcal{S}$ is homeomorphic to the Stone space $S(B(\mathcal{S}))$
        $$x\mapsto \{a\in S(\mathcal{B}) : x\in a\}$$
    \end{enumerate}
}{ (\romannumeral 1) is proven in \ref{Bool:Thm:Stone}
    \begin{enumerate}[label=(\roman*)]
        \stepcounter{enumi} 
        \item Let $B(S) = \{V\subseteq  S : V \in \tau \land V^c\in \tau\}$
        The functions on a boolean algebra of sets are defined as in \ref{Bool:Bsp:AlgOfSets}. Hence we can, for simplicity write $\cap$ instead of $\cdot$ \dots.

        Clearly clopen sets are closed under finite operations of $\cup$, $\cap$, $\overline{\phantom{x}}$.
        Checking the axioms of boolean algebras
        \begin{equation*}
            \begin{matrix*}[l]
                \forall x,y,z \: \bigl(x+(y+z) = (x+y)+z \land x\cdot (y\cdot z) = (x \cdot y ) \cdot z\bigr) & \text{(Associativity $+,\cdot$)}\\[3pt]
                \forall x,y \: \bigl(x+y=y+x \land x\cdot y = y\cdot x\bigr) & \text{(Commutativity of $+,\cdot$) }\\[3pt]
                \forall x \: \bigl(x+x = x \land x\cdot x = x\bigr)& \text{(Idempotence) }\\[3pt]
                \forall x,y,z \: \bigl(x\cdot (y+z) = x\cdot y + x\cdot z \land x+(y\cdot z) = (x+y)\cdot (x+z)\bigr) & \text{(Distributivity) }\\[3pt]
                \forall x,y \: \bigl(x\cdot (x+ y) = x  = x+ (x\cdot y) \bigr)& \text{(Absorbtion)}\\[3pt]
                \forall x,y \: \bigl(\overline{x+y} = \overline{x}\cdot \overline{y}\land \overline{x\cdot y} = \overline{x}+ \overline{y}\bigr)& \text{(De Morgan's Laws) }\\[3pt]
                \forall x \: \bigl( x+0 = x \land x\cdot 0 = 0 \bigr)& \text{(Laws of $0$)}\\[3pt]
                \forall x \: \bigl( x+1 = 1\land x\cdot 1 = x \bigr)& \text{(Laws of $1$)}\\[3pt]
                \forall x \: \bigl(x + \overline{x} = 1 \land x\cdot \overline{x} = 0 \land \overline{\overline{x}} = x\bigr)& \text{(Laws of $\overline{\phantom{x}}$)}\\
            \end{matrix*}
        \end{equation*}
        reveals that they imediately follow from the properties of $\cup$, $\cap$ and $\overline{\phantom{x}}$.
        \item Let $\mathcal{B}$ be a boolean algebra, $C\defeq B(S(\mathcal{B}))$ like above and $h: B\to C$, $h(a) = \langle a \rangle$.
        \begin{flalign*}
            & h(0_B) = \langle 0 \rangle = \varnothing \eqdef 0_C & \\
            & h(a)^c = \langle a \rangle ^c = \langle \overline{a} \rangle = h(\overline{a}) & \\
            & h(a\cdot b) = \langle a\cdot b\rangle = \langle a\rangle \cap \langle b\rangle & \\
            & h(a+b) = h(\overline{a}\cdot \overline{b})^c = (h(\overline{a})\cap h(\overline{b}))^c = h(\overline{a})\cup h(\overline{b}) &
        \end{flalign*}
        Hence $h$ is a homomorphism of boolean algebras.
        \item Let $\tau$ be the Stone topology on $S$. 
        \[S(B(S)) = \{F\subseteq B(S) : \: F \text{ is ultrafilter on }B(S)\}\]
        Since $B(S)$ is a boolean algebra, we have $S(B(S))$ is a stone space.
        Let $\tau'$ be the Stone topology on $S(B(S))$. Let $h: S\to S(B(S))$, $h(x) = \{a\in S(\mathcal{B}) : x\in a\}$.

        \textbf{Claim: } $h$ is a bijection.
        \begin{claimproof}
            Let $x,y\in S$ with $ h(x) = h(y) $. 
            Therefore
            \[\{a\in S(\mathcal{B}) : x\in a\} = \{a\in S(\mathcal{B}) : y\in a\}\]
            
        \end{claimproof}
        \textbf{Claim: }
        \begin{claimproof}
            
        \end{claimproof}
    \end{enumerate}
}

\newpage

\newpage
\newgeometry{
    outer=15mm,
    inner=14mm,
    bottom=15mm,
    %top=20mm
}
\ifPrintTwosided
\pagestyle{two_sided}
\else
\pagestyle{one_sided}
\fi

\chapter{Outernotes, Abbreviations, Overview}
\begin{multicols}{3}
    \listofouternotes
\end{multicols}
\begin{multicols}{2}
    \listofabbreviations%\printglossary[type=\acronymtype]
\end{multicols}
\newpage

\begin{multicols}{2}
    \listofdefinitions
    \listoftheorems
\end{multicols}

\restoregeometry
\printbibliography

\end{document}


%You can see this if the commit actually happend
%ello i am not a robot

% List of abbreviations: 
% append element with
% \addAbbrev{abb}{text}