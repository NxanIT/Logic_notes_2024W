\chapter{Set Theory of ZFC}
The contents on this chapter are at least partially sourced on \cite{krivine1998théorie}.
\bsp{Russel's paradox}{Let $A = \{a : a\notin a\}$. If any collection of elements is a set, then $A$ would be a set.
Question: is $A\in A$? if yes, then $A\notin A$, if not then $A\in A$}
\noindent Trying to resolve this, we will introduce the ZFC (Zermelo-Frankel axioms w/ choice) System.
Let $\mathcal{L}=\{\in\}$ be a Language of first order, where $\in$ \dots binary relation "beeing element of"
For $(\mathcal{U},\in)$ 
If $\mathcal{A} = (\mathcal{U},\in^\mathcal{A})\models \text{ZFC}$, then the elements of the universe $\mathcal{U}$ are called sets.
We will show roughly that some definably sets are not sets (in the sense of ZFC), others are not. The latter will be called classes.
\section{First axioms of ZFC}
\axiom{Axiom of extensionality}{\label{Set:Ax1}
    $$\forall x \forall y (x=y \leftrightarrow \forall u (u\in x \leftrightarrow u\in y))$$}
In other words, two sets are the same if they have the same elements. This will give us later uniqueness in construction of other sets.
\axiom{Pairing Axiom}{\label{Set:Ax2}
    for any two sets $a,b$ one can form a set whose elements are precicely $a,b$
$$\forall x\forall y \exists z \forall u \: u\in z \leftrightarrow (u = x \lor u = y)$$
Our notation will be $z=\{x,y\}$
}
In words: For any two sets there exists a set whose members are those two sets.

\note{}{
    \begin{itemize}
        \item $\{x,y\}$ is unique by \ref{Set:Ax1}
        \item $\{x\}$ is a set. from \ref{Set:Ax2}, take $x= y$
    \end{itemize}
}
% \lemma{}{Let $x,y$ be sets. We define the ordered pair $(x,y) \defeq \{\{x\},\{x,y\} \}$. 
%     Then it holds $(x,y) = (a,b)$ iff $x = a$ and $y = b$
% }{By cases
%     \begin{itemize}
%         \item if $x=y$, then $(x,y) = \{\{x\} \}$ therefore $a=b$ and by \ref{Set:Ax1} it holds $x=a$.
%         \item if $x\neq y$, then $\{\{x\},\{x,y\} \} = \{\{a\},\{a,b\} \}$ iff $\{x\} = \{a\}$ and $\{x,y\} = \{a,b\}$. That is, iff $x=a$ and $y=b$.
%     \end{itemize} }

Let $x_1,x_2,\dots x_n$ be sets We define the \imp{n-tuple} $(x_1,\dots x_n)$ inductively:
\begin{itemize}
    \item  $(x_1,x_2) \defeq \{\{x\},\{x,y\} \}$. 
    \item $(x_1,\dots x_n) \defeq (x_1 (x_2,\dots x_n))$
\end{itemize}
\note{}{The set $(x,y)$ exists, because its obtained by repeatedly using \ref{Set:Ax2}}

\lemma{}{Let $x,y,a,b$ be sets. Then $(x,y) = (a,b)$ iff $x = a$ and $y = b$}{
    If $x = a$ and $y = b$ then by \ref{Set:Ax1} $(x,y) = (a,b)$. The other direction by cases:
    \begin{itemize}
        \item case $x=y$, then $(x,y) = \{\{x\}\}$ is a singleton then $(a,b)$ is a singleton, 
        wlog $\{a\} = \{a,b\}$ then $a = b = x$.
        \item case $x\neq y$ and $\{\{x\},\{x,y\}\} = \{\{a\},\{a,b\}\}$ then $\{x\} = \{a\}$ and $\{x,y\} = \{a,b\}$ because by \ref{Set:Ax1} a singleton can not be equal to a set of size $2$.
    \end{itemize}
    
}
\lemma{}{For all $n,m>1$ and for all sets $x_1,\dots, x_n,y_1,\dots, y_m$:\\
    $(x_1,\dots, x_n) = (y_1,\dots, y_m)$ iff $n=m$ and $\forall i\leq n x_i = y_i$
}{By induction, left as an exercise}

\axiom{Union Axiom}{For every set $x$ there is a set $z$ consisting of all elements of the elements of $x$.
    $$\forall x \exists z \forall y (y\in z \leftrightarrow \exists u (u\in x \land y\in u))$$
    We call $z$ the union of $x$, notation: $\cup_x\defeq z$\\
    The union of two sets is often abbreviated with $x\cup y \defeq \cup_{\{x,y\}}$
}
\bsp{}{
    \begin{enumerate}
        \item $\bigcup_{(x,y)} = \{x,y\}$.
        \item $(x_1,x_2,\dots x_n) = \bigcup_{\{x_1\}, {x_2,\dots x_n}}$ 
    \end{enumerate}
}   
\note{}{
    \begin{itemize}
        \item For all sets $ x_1, \dots x_n$ there is exactly one set with elements $x_1,\dots x_n$
        \item The union is asociative $x\cup (y\cup z) = (x\cup y)\cup z$
    \end{itemize}
}
\axiom{Power set Axiom}{Let $x\subseteq y$ be the abbreviation for $\forall z(z \in x \to z\in y)$.
For every set $x$
there exists a set $z$ consisting of all subsetes $y\subseteq x$ that are themselve sets.
    $$\forall x\exists z \forall y (y\in z \leftrightarrow y\subseteq x)$$
    Notation: $\mathcal{P}(x)\defeq z$.
}
Or in words: ``For every set $x$ there is a set $z$ consisting of all subcollections of $x$ that are themselve sets.''
\subsection{Classes and functions}
\defin{Classes}{All the unary $\mathcal{L}$-definable relations (w/ parameters) are called classes.}
\bsp{}{
    \begin{itemize}
        \item $\varphi (x) \equiv x = x$ defines the universe $\mathcal{U}$, a class that is not a set
        \item $\varphi(x) \equiv \exists u (u\in x \land \forall v \: (v\in u \to v\in x))$
    \end{itemize}
}
\defin{Class functions}{
    Suppose we have a formula $\phi(x_1,\dots x_n, y)$. Then we say $\phi$ defines a class function $R_\phi$ iff 
    \[\forall x_1\dots \forall x_n \forall y \forall y' ((\phi(\underline{x},y)\land \phi(\underline{x},y'))\to y = y')\]
    We can then define the domain and image of the class function.
    \[\dom R_\phi \defeq\{(x_1,\dots x_n) : \:\exists y \phi( \underline{x},y)\}\]
    \[\imag R_\phi \defeq\{y : \:\exists \underline{x} \phi(\underline{x},y)\}\]
    Note that $R_\phi(\underline{x})=y$ iff $\phi(\underline{x},y)$
}
\axiom{Axiom of replacement / substitution}{\label{Ax5}
    Let $\varphi(x,y,\underbar{a})$ a $\mathcal{L}$-fla., w/ free variables among $x,y$ and set-parameters $\underbar{a}$.
    Suppose $\varphi$ defines a class function on $\mathcal{U}$, than the followoing is an axiom:
    $$\forall u \exists z \forall y\: (y\in z \leftrightarrow \exists x (x\in u \land \varphi(x,y,\underbar{a})))$$
    i.e. the image of a set under a class function is a set.
}
\axiom{Axiom scheme of comprehension}{\label{Ax6}
    Let $\psi(x,\underline{a})$ be an $\mathcal{L}$-formula. Then the followoing is an axiom:
    \[\forall u \exists z \forall v \:(v\in z \leftrightarrow(v\in u\land \forall \psi(v,\underline{a})))\]
    i.e. all elements of a set that satisfy a given $\mathcal{L}$-formula form a set.
}
\note{}{\ref{Ax6} follows from \ref{Ax5}}
\axiom{Set existence}{\label{Ax7}
    \[\exists x x = x\]
    i.e. $U\neq \varnothing$. - this is clear when we view it as a universe of a structure.
}
\note{on the existence of the empty set}{
    Let $u$ be any set (there exists one by \ref{Ax7}), $\psi(x) \equiv x\neq x$ then by \ref{Ax6} $\varnothing \defeq \{x\in u : \psi(x)\}$
    is a set.
}
\datenote{17.12.2024}
\note{}{We can derive pairing from replacement \ref{Ax5}, extensionality, powerset and set existence.
From set existence: $\varnothing$ is a set
By powerset, replacement(comprehension): $\{\varnothing\}$ is a set.
    \[\mathcal{P}(\{\varnothing\}) = \{\varnothing ,\{\varnothing\}\}\]
    is a set.
    Then by defining a class function $R_\phi (x) = y$, 
    $$\phi (x,y)\equiv (x=\varnothing\land y = a)\lor (x = \{\varnothing\}\land y = b)$$
}

\note{}{If the domain of a class function happens to be a set then the graph of the class function is a set. %check if really is graph
    $R_\phi$ defined by $\phi(x,y,\underline{a})$ then the domain $u\defeq \dom R_\phi \in \mathcal{U}$
    Image $v\defeq \imag R_\phi \in \mathcal{U}$ by replacement 
    \[\{(x,y) : x\in u \land y \in v \land \phi(x,y,\underline{a})\}\]
    is the graph of $R_\phi$
    The above would be a set if $u\times v$ which can be shown by using comprehenson (exercise).
}
\defin{Function}{A function $f:a\to b$ where $a,b$ are sets is a subset of $a\times b$ that satisfies the followoing
    \begin{itemize}
        \item $\forall x \: (x\in a \to \exists y \in b \:(x,y)\in f)$
        \item $\forall x \forall y \forall y' \:(((x,y)\in f \land (x,y')\in f)\to y = y')$
    \end{itemize}
}
\subsubsection*{Families of sets and cartesian products}
Suppose we have a function $a: I \to X$. Let $a_i$ be the unique $x\in X$ s.th. $(i,x)\in a$.
We define the following sets:
    \[\bigcup_{i\in I}a_i \defeq \{z\in \bigcup X : \exists i\in I z\in a_i\}\]
    \[\bigcap_{i\in I}a_i \defeq \{z\in \bigcup X : \forall i\in I z\in a_i\}\]
    \[\prod_{i\in I}a_i \defeq \{f:I\to \bigcup X : \forall i\in I z\in a_i\}\]

\note{}{
    If $I=\varnothing$ then $\bigcap_{i\in I}a_i = \bigcup X$
}
\subsubsection*{Class relations and well ordering}
Types of well ordered sets
\defin{Strict (linear) order}{
    Let $R$ be a class relation, $C$ be a class.\\
    Then $R$ defines a strict ordering on $C$, if
    \begin{enumerate}[label = (\roman*)]
        \item  $ \forall x \forall y \forall z \: R(x,y)\to (C(x)\land C(y)) $ 
        \item  $ \forall x \forall y  \: \lnot(R(x,y)\land R(y,x)) $ 
        \item  $ \forall x \forall y \forall z \: (R(x,y)\land R(y,z))\to R(x,z) $ 
    \end{enumerate}
    The ordering is linear, if additionally
    \begin{enumerate}[label=(\roman*)]
        \setcounter{enumi}{3}
        \item $\forall x \forall y \: (C(x)\land C(y))\to (x=y\lor R(x,y)\lor R(y,x))$
    \end{enumerate}
}
\defin{Well ordering}{Let $R$ be a strict ordering on $C$ and 
    $x$ be a set such that $\forall y \in x \: C(y)$
    Then $x$ is called well-ordered by $R$, if
    \[\forall \varnothing \neq y \subseteq x \text{ $y$ has a smallest element}\]
    i.e.
    \[\forall y \bigl((\varnothing \neq y \land y\subseteq x) \to \exists y' \bigl(y'\in y \land \forall z (z\in y \to( R(z,y')\lor y'=z))\bigr)\bigr)\]
}
\defin{Initial segment}{
    Let $x$ be a set, well ordered by $R$.
    Then $y\subseteq x$ is called an initial segment of $x$, if
    \[\forall s\forall t \: (s\in x \land t \in x) \to  \bigl((t\in y \land R(s,t))\to s\in y\bigr)\]
    Let $x$ be well-ordered by $<$ and $y\in x$.
    Then $\delta^\leq_y(x) \defeq \{z\in x : z<y\}$. 
    If there is no ambiguity among the well ordering, we abbreviate $\delta_z(x) \defeq \delta^\leq_y(x)$.
    With
    $<$ above strict it holds $y\notin \delta_y(x)$
}

\note{}{If $x$ is well-ordered by $<$ and $y\subseteq x$ then
    \begin{center}
        $y$ is an initial segment of $x$ iff $y = x $ or $y = \delta_z(x)$ for some $z\in x$
    \end{center}
    \begin{proof}
        $\delta_z(x)$ is well ordered for $z\in x$.\\
        Let $y\subseteq x$ an initial segment. Suppose $x\neq y$ that means $x\smallsetminus y \neq \varnothing$
        Let $z$ be the smallest element of $x\smallsetminus y$ (exists by well-ordering of $x$).
        Suppose $y \neq \delta_z(x)$. Then there is $a\in y$ $z<a$ and $y$ is not an initial segment.
    \end{proof}
    
}
\defin{Propper class}{A class $C$ is called a proper class if it is not a set.\\
    i.e. if $C$ is given by $\phi(x,\underline{a})$ then there is no $z\in \mathcal{U}$ such that
    $\forall x x\in z \iif \phi (x,\underline{a})$
}
\bsp{}{$\mathcal{U}$ is a proper class: 
    If $\mathcal{U}$ was a set then $\{x : x\notin x\}$ would be a set.

    $\Ord$, the class of all ordinals is a proper class.
}

\defin{well-ordering (class)}{A class relation $R$ defining a strict ordering on a class $C$
    is called a well-ordering, if \\
    for every $x\in C$ the class initial segment $\delta^R_x(C) = \{y : R(y,x)\}$ is a set that is well ordered by $R$.
    \[\forall x C(x)\to \exists z \: z = \delta^R_x(C) \land \text{ $z$ has a smallest element}\]
}
\section{Ordinals}
\defin{Tranistivity of sets}{A set $x$ is called transitive, if 
$\forall y\: (y\in x \to y\subseteq x)$
}
\note{}{
    It corresponds to Tranistivity of the belonging relation ``$\in$''. $z \in y\in x \to z\in x$
}
\defin{Ordinal}{An ordinal is a transitive set which is well ordered by $\in$.}
\note{}{The collection of all ordinals form a class relation, notation: 
$\Ord, \On$\outernote{$\Ord$}\outernote{$\On$}\\
    Proof: Write down formula 
}
\bsp{}{
    \begin{itemize}
        \item $\varnothing$, $\{\varnothing\}$, $\{\varnothing , \{\varnothing\}\}$ are ordinals
    \end{itemize}
}
\lemma{Characterization of ordinals}{\label{Set:Lemma:CharOfOrdinals}
    Let $\alpha$ be a set. $\alpha$ is an ordinal, iff
    \begin{itemize}
        \item the initil segments of $\alpha$ are $\alpha$ itself and the elements of $\alpha$
        \item if $\beta \in \alpha$ then $\beta $ is an ordinal.
        \item $\alpha\notin \alpha$
    \end{itemize}
}{Problem set}
\lemma{}{\label{Set:Lemma:InisLinearOrder}
    Let $\alpha,\beta\in \Ord $ then either $\alpha = \beta$, or $\alpha\in \beta$ or $\beta\in \alpha$.
}{
    Let $\gamma \defeq \alpha\cap \beta$\\
    \textbf{Claim:} $\gamma$ is initial segment of both $\alpha $ and $\beta$
    \begin{claimproof}
        $x\in y\in \gamma$ then $x\in y \in \alpha$ and $x\in y \in \beta$. but $\alpha, \beta$ are ordinals, so
        $x\in \alpha$ and $x \in \beta$ and $x \in \gamma$
    \end{claimproof}
    Then by previous lemma, either 
    \begin{itemize}
        \item $\gamma = \alpha$ and $\gamma = \beta$ and we are done 
        \item $\gamma = \alpha$ and $\gamma \in \beta$, so $\alpha\in \beta$
        \item $\gamma \in \alpha$ and $\gamma = \beta$, so we have $\beta\in \alpha$.
        \item $\gamma \in \alpha$ and $\gamma \in \beta$ we have $\gamma \in \alpha\cap \beta = \gamma$ which is impossible
    \end{itemize}
    Hence the statement follows.
}
\prop{}{$\Ord$ is well-ordered by $\in$}{
    We need to show that if $\alpha\in\Ord $ then $\delta_{\alpha}(\Ord)$ is a set which is well ordered by $\in$.

    $\delta_{\alpha}(\Ord) = \{\beta \in\Ord  : \beta\in \alpha\} = \alpha$
    And $\alpha$ is a well-ordered set. By \ref{Set:Lemma:InisLinearOrder} $\Ord $ is even linearly ordered by $\in$.
}
\lemma{}{
    $\Ord$, the class of all ordinals is a proper class.
}{
    Suppose $\Ord $ would be a set $z$.\\
    $\Ord $ is well ordered by $\in$
    $\Ord $ is transitive: $y\in x \in\Ord $ then $y\in\Ord $ by \ref{Set:Lemma:CharOfOrdinals}
    so $\Ord $ would be an ordinal itself and we would have $\Ord \in\Ord $ which is not possible by \ref{Set:Lemma:CharOfOrdinals}.
}

\note{}{
    \begin{itemize}
        \item If $\alpha\in\Ord $ then the initial segments of $\alpha$ are $\alpha$ and the elements of $\alpha$.
        \item If $\alpha\in\Ord $ and $\beta\in \alpha$ then $\beta \in \Ord $
        \item $\alpha,\beta\in\Ord $ then $\alpha\subseteq \beta$ iff $\alpha\in \beta$ or $\alpha = \beta$
        \item $\alpha\subseteq \beta$ iff $\alpha = \beta$ or $\alpha\in \beta$.
    \end{itemize}
}

\datenote{07.01.2025}

\lemma{ }{
    If $\alpha\in \Ord $ then $\alpha\cup \{\alpha\} \in \Ord $ and $\alpha\cup \{\alpha\}$ is the successor of $\alpha$ in the ordering $\in$
}{
    \begin{itemize}
        \item $\alpha\cup \{\alpha\}$ transitive:\\
        $x\in y \in \alpha\cup \{\alpha\}$ if $y\in \alpha$ then $x\in \alpha$ hence $x\in \alpha\cup \{\alpha\}$.
        else $y = \alpha$ then $x\in \alpha\cup \{\alpha\}$ 
        \item $\alpha\cup \{\alpha\}$ well-ordered:\\
        $\varnothing\neq x\subseteq  \alpha\cup \{\alpha\}$ if $x\cap \alpha\neq \varnothing$ then there is $x_0\in x\cap \alpha$ smallest.
        $x_0 \in \alpha$ and is smallest element in  $\alpha\cup \{\alpha\}$
        otherwise $\varnothing \neq x \subseteq \{\alpha\}$ then $\alpha$ is the smallest element.
        \item  $\alpha\cup \{\alpha\}$ is successor of $\alpha$\\
        $\alpha\in \alpha\cup \{\alpha\}$
        assume  $\alpha\in \beta \in \alpha\cup \{\alpha\}$
        if $\beta\in \alpha$ then $\alpha\in\beta\in\alpha$, so by transitivity, $\alpha\in \alpha$ which is not possible.\\
        else $\beta = \alpha$
    \end{itemize}
}

\note{}{If $\alpha,\beta$ are ordinals then we will use $\alpha\in \beta$, $\alpha<\beta$, $\alpha\subsetneq \beta$ interchangable.

$\gamma\in \Ord $ then $\gamma = \{\alpha\in \Ord : \alpha\in \gamma\}$
}
\lemma{}{
    $X$ a set of ordinals, then $\sup X = \bigcup X$ is an ordinal and $\forall \alpha\in X \alpha\subseteq \bigcup X$ and $\bigcup X$ is smallest with this property.
}{
    \begin{itemize}
        \item $\bigcup X$ transitive: $x\in y\in \bigcup X$. then $\exists \alpha\in X$ such that $y\in \alpha$. then $x\in \alpha$ hence $x\in \bigcup X$
        \item $\bigcup X$ well-ordered by $\in$: $\bigcup X$ contained in $\Ord $ and is a set, but $\Ord $ is well-ordered, so $\bigcup X$ is well-ordered.
        \item $\alpha\in X$ then $\alpha\subseteq \bigcup X$.
        \item $\alpha\in X$ then $\alpha\subseteq \bigcup X$.
        Let $\beta\in \bigcup X$ then there exists $\alpha\in X$ such that $\beta\in \alpha$ so $\beta$ is not an upper bound for $X$.
    \end{itemize}
}

\lemma{}{
    Suppose that $\alpha,\beta\in \Ord $ and $f:\alpha\to \beta$ that is strictly increasing i.e. 
    $\forall \gamma,\delta\in \alpha \gamma<\delta \to f(\gamma)<f(\delta)$
    Then $\alpha\subseteq \beta$ and $\forall \gamma\: \gamma\leq f(\gamma)$
}{
    By contradiction, Let $\gamma\in \alpha$ be the smallest element with $f(\gamma)<\gamma$ then by minimality of $\gamma$,
    $f(\gamma)\leq f(f(\gamma))$. 
    Because $f$ is strictly increasing $f(f(\gamma))<f(\gamma)$. So we get $f(\gamma)\leq f(f(\gamma))<f(\gamma)$ but 
    $f(\gamma)\notin f(\gamma)$ because $f(\gamma)\in \beta$.

    Suppose $\beta\in \alpha$ then $f(\beta)<f(\alpha)<\beta$ so $f(\beta)<\beta$, a contradiction.
}

\thm{}{
    $f:\alpha\to \beta$ isomorphism between $(\alpha,\in)$, $(\beta,\in)$ and $\alpha,\beta\in \Ord $ then $\alpha = \beta$ and $f$ is unique such isomorphism, hence $f = id_\alpha$.
}{
    $\alpha = \beta$:\\
    Apply previous lemma to $f,f^{-1}$ hence $\alpha\subseteq \beta$ and $\beta\subseteq \alpha$
    uniqueness:\\
    $\gamma\in \alpha$ then $\gamma\leq f(\gamma)$  and $\gamma\leq f^{-1}(\gamma)$ by prev lemma
    we get 
    $\gamma\leq f(\gamma)\leq \gamma$ so $f(\gamma) = \gamma$
}

\note{}{$y\mapsto \beta_y$ for $y\in Y$ is function defined on $Y$ and maps to
    $Z = \{\beta(x) : x\in Y\}$ (is a set by replacement)
}

\thm{}{
    Let $(X,<_X)$ be well-ordered, then there is a unique isomorphism onto an ordinal $(\alpha, \in)$
}{
    uniqueness:\\
    Suppose we have $f:(X,<_X)\to(\alpha,\in)$, $g:(X,<_X)\to(\beta,\in)$ isomorphisms
    then $f\circ g^{-1}$ and by prev thm: $\alpha = \beta$ and $f\circ g^{-1}= id_\alpha$ so $f=g$.
    
    Existence:\\
    define $y = \{x\in X : \: (\delta_x,<_X) \text{  is isomorphic to an ordinal}\}$
    where $\delta_x \defeq \delta_x(X)$.

    For each $y\in Y$ there is a unique ordinal $\beta_y\in \Ord $ such that $(\delta_y, <_X)$ 
    and $(\beta(y), \in)$ are isomorphic.

    \textbf{Claim:} $Y$ is initial segment of $X$
    \begin{claimproof}
        If $x<_Xy\in Y$, $f:\delta_y\to \beta(y)$ isomorphism, then $f$ maps $\delta_x\subseteq \delta_y$ to initial segment of $\beta$, hence to an ordinal. 
    \end{claimproof}
    $y\mapsto \beta_y$ for $y\in Y$ is function defined on $Y$ and maps to
    $Z = \{\beta(x) : x\in Y\}$ (is a set by replacement)\\
    \textbf{Claim:} $Z = \{\beta(x) : x\in Y\}$ is an initial segment in $\Ord $
    \begin{claimproof}
        if $\gamma\in \beta(x)$, $x\in Y$ have isomorphism between $(\delta_x,<_X)$ and $(\beta_x,\in)$ so its preimage $y$ and $(\delta_y, <_X)$ is mapped to the initial segment determined by gamma, so to $(\gamma, \in)$ hence $(\delta_y, <_X)$ isom to $\gamma$
    \end{claimproof}
    So $Z$ is initial segment of $\Ord $ and $Z$ is a set. So $\alpha\defeq Z$ is itself an ordinal
    and $y\mapsto \beta_y$ isomorphism between $Y$ and $\alpha$.

    Assuming $Y\subsetneq X$, then there is a minimal $x_0\in X\backslash Y$ 
    $\delta_{x_0} = Y$ ($Y$ is initial segment) $Y\cong \alpha$ hence $x_0\in Y$, a contradiction.
}

\section{Transfinite induction/recursion} % or inductive definitions

Suppose $\phi(x)$ (possibly with parameters)
to prove 
\begin{equation}\label{star}
    \forall \alpha\in \Ord  \phi(\alpha) \iif \forall \alpha \forall \beta ((\beta<\alpha\to \phi(\beta))\to \phi(\alpha))
\end{equation}
\ref{star} $\implies \forall \alpha\in \Ord  \phi(\alpha)$

Suppose ther is $\alpha\in \Ord $ such that $\lnot \phi(\alpha)$ then let $\alpha$ be smallest with the property.

proof by induction on ordinals (proof by transfinite induction) is a proof of $\forall \alpha\in \Ord  \phi(\alpha)$ by proving \ref{star}

Let $F$ be a class function in one variable and $a$ a set contained in $\dom (F)$ 
Then $F|_{a} = \{(x,y)\in (a,b) : F(x) = y\}$ where $b = \{F(x) : x\in a\}$ (which is a set by replacement).

Let $H$ be any class function in one variable.

\defin{H-inductive}{A function $f$ is called H-inductive, if 
    \begin{enumerate*}
        \item $\alpha \defeq \dom(f) \in \Ord $ and 
        \item $\forall \beta \in \alpha f|_\beta\in \dom (H)$ and 
        \item $f(\beta) = H(f|_\beta)$
    \end{enumerate*}
}   

$f:\alpha\to X$ then $H$ gives you a way to extend the $f$.
$H$ extends $f$ to a function on $\alpha\cup\{\alpha\}$
$$f(\alpha) = H(f)$$

\lemma{}{
    For every class function $H$ and ordinal $\alpha$ there is at most one H-inductive function on $\alpha$ with domain $\alpha$.
}{
    Suppose not. $f,g:\alpha\to X$ different H-inductive functions.
    Let $x_0$ the smallest element of $\alpha$ such that $f(x_0)\neq g(x_0)$. By $x_0$ smallest, 
    $f|_{x_0}= g|_{x_0}$
    By H-inductiveness
    $$f(x_0) = H(f|_{x_0}= H(g|_{x_0}) = g(x_0)$$
    A contradiction.
}
\lemma{}{
    Let $H$ be a class function, $\alpha\in \Ord $ such that any function $f:\beta\to X$ where $\beta\in \Ord $
    belongs to $\dom (H)$ then there is an H-inductive function $f:\alpha\to X$.
}{
    $\tau = \{\beta<\alpha : \text{ there is H-inductive }f_\beta : \alpha\to X\}$
    $\tau$ is a set and initial segment of $\alpha$ hence $\tau \in \Ord $ and $\tau\subseteq \alpha$

    $\beta\mapsto f_\beta$ for $\beta\in \tau$ is well-defined function by uniqueness of $f_\beta$.

    Moreover for $\gamma<\beta<\tau$ we have $f|_\beta|_\gamma  = f|_\gamma$ (H-ind, uniqueness)

    $f\defeq \bigcup_{\beta<\tau}f_\beta$ is H-inductive function (graphs agree on intersection, each of the $f_\beta$ are H-ind). The domain 
    $$\dom(f) = \sup_{\beta<\tau}(\beta) = \bigcup_{\beta<\tau}(\beta) = \sigma\in \Ord $$

    If $\sigma =\alpha$ %sigma or tau???
    we are finished, otherwise we can define 

    $\tilde{f}$ such that $\tilde{f}|_\sigma = f$ and $\tilde{f}(\sigma) = H(f)$
    $\tilde{f}$ is now H-inductive, and $\dom (\tilde{f}) = \sigma\cup\{\sigma\}$ 
    a contradiction $\sigma$, the domain of $f$.
}

\datenote{09.01.2025}
%H-induction is like transfinite recursion

\thm{Transfinite Recursion}{
    Let $A$ be a class, $M$ be a class of all functions $f:\alpha\to X$ for $\alpha\in\Ord$ arb.
    $X$ a subset of $A$ and $H$ a class function in one variable defined on all of $M$ with values in $A$.
    Then there exists a unique class function $F$ defined on $\Ord$ such that $\forall \alpha F(\alpha) = H(F|_\alpha)$
}{
    define $F$ by 
    $$F(\alpha) = y \iif \text{there is an H-inductive function $f:\alpha\to X$, $X$ subset of $A$ and $y = H(f)$}$$
    It is well defined by the prev two lemmas TODO
}

\section{Axiom of Choice and Zermelo's Theorem}
\axiom{Axiom of Choice (AC)}{
    For every set $X$ and $A\subseteq \mathcal{P}(X)$ that consists of pairwise disjoint, non-empty subsets of $X$ there is a set $T\subseteq X$ such that $\forall a\in A \:\# a\cap T = 1$. \\
    $T$ as above is called a transversal.
}
\defin{(AC')}{
    (Existence of choice function.) 
    For every set $X$ there exists a function $\pi : \mathcal{P}\setminus \{\varnothing\}\to X$ such that for every non-empty subset $a\subseteq X$ $\pi(a)\in a$
}
\defin{(AC'')}{
    Let $(X_i)_{i\in I}$ be an indexed family of non-empty sets then \\
    $\prod_{i\in I}{X_i}\neq \varnothing$.
    ($(X_i)_{i\in I}$  could also be thought of as a function $i\mapsto X_i:I\to X$)
}
In the next Problemset: $AC \leftrightarrow AC'\leftrightarrow AC''$

\thm{Zermelo}{
    ``Well-ordering theorem'': Every set can be well-ordered.\\
}{
    By contradiction. Suppose $X$ is a set which can not be well-ordered. 
    Choose a choice function $\pi:\mathcal{P}(X)\backslash \varnothing \to X$ 
    define a class function $H$ by $H(f) = y$ iff $f$ is a function with
    \begin{enumerate*}[label = (\roman*)]
        \item $\dom f = \alpha \in \Ord$
        \item $\imag f\subsetneq X$
        \item $y = \pi (X\backslash \imag f)$
    \end{enumerate*}

    note: \begin{itemize}
        \item $H$ is defined on class of all $H$-inductive functions, 
    whose image is a proper subset of $X$. $\imag f\subsetneq X$.
        \item each $H$-inductive function is injective
    \end{itemize}
    If $f$ is a $H$-inductive function that is also surjective, then we are done because $f$ induces well-ordering on $X$.

    By our assumption, every $H$-inductive function has to be not surjective.
    So $H$ is defined on all $H$-inductive functions, and we can use Transfinite recursion theorem
    and get an $H$-inductive class function $F:\Ord\to X$ 
    which is injective

    Suppose $\alpha<\beta<\gamma$ and $F(\alpha) = F(\beta)$. 
    $F(\alpha) = \pi(X\backslash\imag F|_\alpha)$
    $F(\beta) = \pi(X\backslash\imag F|_\beta)$
    
    Then $F$ is an injection from a proper class into $\imag F$, a set, which is impossible.
}

\thm{Zorn's Lemma}{
    Let $(X,\leq)$ be a partially ordered set (poset) such that all linearly ordered subsets (called chains) have an upperbound. Then $(X,\leq)$ has a maximal element. i.e. $\exists y\in X \forall x\in X y\not < x$
}{
    Let $A \defeq \{Y\subseteq X : \exists x\in X \forall y\in Y y<x\}$
    Take $\pi:\mathcal{P}(X)\backslash \{\varnothing\}\to X$ a choice function.

    Define $$p:A\to X, \quad p(Y) \defeq \pi (\{x\in X : \forall y\in Y y<x\})$$
    Define a class function $H$ by $H(f) = y$ iff 
    \begin{enumerate*}[label = (\roman*)]
        \item $f$ is a function with $\dom f \in \Ord$
        \item $\imag f\in A$
        \item $y = \pi(\imag f)$
    \end{enumerate*}

    We get 
    \begin{itemize}
        \item  any $H$-inductive $f:\alpha\to X$ is strictly increasing. ($\star$)
        $f:\alpha\to X, f(\alpha\cup \{\alpha\}) = H(f)>\imag f$
        \item The image of any $H$-inductive function $f:\alpha \to X$ is linearly ordered, so by assumption has an upperbound.
        
        i.e. $\exists x_f\in X \: \forall \beta < \alpha f(\beta) < x_f$
    \end{itemize}
    The idea now is similar to above therorem.
    Suppose $f:\alpha \to X$ is $H$-inductive but $H$ is not defined on $f$, then the image of $f$ has no strict majorant, so there is $\beta<\alpha$ such that $x_f = f(\beta)$. 
    Then $x_f$ has to be maximal for $X$.

    If $H$ is actually defined on all $H$-inductive functions, then there is an $H$-inductive class function $F:\Ord \to X$.
    $F$ is strictly increasing by $\star$.
    So have injective of proper class into set. a contradiction.
}

\section{Ordinal arithmetic and the size of a set}

Or how to think of the natural numbers to be contained in $\Ord$

\defin{successor / limit ordinals}{
    \begin{itemize}
        \item $0\defeq \varnothing$ is the smallest ordinal
        \item $\beta\in \Ord$ then its successor ordinal is defined by $\beta+1 \defeq \beta\cup\{\beta\}$
        \item $\beta$ is called a \imp{successor ordinal} if there is $\alpha\in \Ord$ such that 
        $\beta = \alpha + 1$
        \item $\beta\in \Ord$ is called a \imp{limit ordinal} if $\beta \neq 0$ and $\beta$ is not a successor ordinal
        \item $\beta\in \Ord$ is called a natual number / finite ordinal, if $\beta = 0$ or for every $\alpha\leq \beta$ we have ``$\alpha$ is a successor ordinal or $\alpha = 0$''
    \end{itemize}
}
\bsp{}{
    $0 = \varnothing$, $1 = \varnothing \cup \{\varnothing\} = \{\varnothing\}$,\dots 
}
\note{}{
    If $(X,<_X), (Y,<_Y)$ are well-ordered sets then we can well-order both their cartesian product 
    $X\times Y$
    and \imp{disjoint union} $X \sqcup Y\defeq (X\times \{0\})\cup (Y\times \{1\})$ by the reverse lexiographical order:
    $$(x_0,y_0)\sphericalangle (x_1,y_1)\iif y_0<_Y y_1 \lor (y_0 = y_1\land x_0<_X x_1)$$
    For $(x_0,y_0), (x_1,y_1)$ in $X\times Y$ or $X\sqcup Y$, in the latter case $y_0 <_Y y_1$, if $y_0 = 0$ and $y_1 = 1$.
    Note that in this case the the ordering of $X \sqcup Y$ corresponds to
    $$(a,i)\sphericalangle (b,j) \iif \begin{cases}
        i=j=0 \text{ and } a<_X b\text{, or}\\
         i=j=1 \text{ and } a<_Y b\text{, or}\\
         i<j
    \end{cases}$$
}
\newpage
\defin{Sum and product of ordinals}{
    Let $\alpha,\beta\in\Ord$, then 
    \begin{enumerate}[label=(\roman*)]
        \item $\alpha+\beta$ is the unique $\gamma\in \Ord$ such that $\gamma$ is order-isomorphic to the 
        sum / disjoint union $\alpha\sqcup\beta$
        \item $\alpha\cdot\beta$ is the unique $\gamma\in\Ord$ such that $\gamma$ is order-isomorphic to the 
        product of $\alpha$ and $\beta$
    \end{enumerate}
}
Properties of sum and product.
\lemma{}{
    \begin{enumerate}
        \item $+$ is associative, $0$ is a $2$-sided add. identity
        \item $\cdot$ is associative
        \item $\alpha\cdot 0 = 0$, $\alpha \cdot 1 = \alpha = 1\cdot \alpha$
        \item $\alpha\cdot(\beta+ \gamma) = \alpha\cdot\beta + \alpha\cdot\gamma$
        \item $\lambda$ limit ordinal then $\alpha\cdot \lambda = \sup_{\beta<\lambda} \alpha \cdot \beta$
    \end{enumerate}
}{Problem set}
\note{}{
    Right now it would be consistent to assume $\cdot$ is commutative, but no longer after the next axiom
}

\axiom{Axiom of infinity}{
    There exists an infinite ordinal.

    i.e. there exists an ordinal that is not a natural number.
}
\note{}{
    \begin{itemize}
        \item The natural numbers form an initial segment in $\Ord$.
        \item Let $\omega$ be the smallest infinite ordinal, i.p. $\omega$ is a limit ordinal
    \end{itemize}
}
\bsp{}{
    \begin{enumerate}
        \item $\omega\cdot 2 \stackrel{?}{=} 2 \cdot \omega$, observations:
            \begin{itemize}
                \item $\omega\cdot 2 = \omega + \omega$
                \item $2\cdot \omega = \omega$
                \item $\omega + \omega \neq \omega$
            \end{itemize}
        \item $\omega + 2 \neq 2 + \omega$, observations:
        \begin{itemize}
            \item $2+\omega = \omega$
            \item $\omega + 2$ has maximal element
        \end{itemize}
    \end{enumerate}
}


\datenote{14.01.2025}


\defin{Exponentiation}{
    $\alpha^\beta$ is defined recursively on $\beta$:
    \begin{enumerate}
        \item $\alpha^0 \defeq 1$
        \item $\alpha ^{\beta+1} \defeq \alpha^\beta \cdot \alpha$
        \item $\alpha^\lambda \defeq \sup_{\delta < \lambda}{\alpha^\delta}$ for a limit ordinal $\lambda$
    \end{enumerate}
}
\note{}{
    Alternatively, we can define exponentiation as given by the class function $EXP : \Ord \times \Ord \to \Ord$ defined by
    \[Exp(\alpha,\beta) = \gamma \quad\iif\quad \text{There exists a function } f:\beta+1\to \Ord 
    \text{ such that for all } \xi<\beta\text{ it holds}\]
     
    \begin{itemize}
        \item $f(\beta) = \gamma$
        \item if $\xi=0$ then $f(\xi) = 1$
        \item if $\xi = \delta+1$ then $f(\xi) = f(\delta)\cdot\alpha$
        \item if $\xi$ is a limit ordinal then $f(\xi) = \sup_{\delta < \xi}{f(\delta)}$
    \end{itemize}
}
TODO
uniqueness follows from recursion theorem.

\defin{Cardinality}{\label{Set:Def:Cardinality}Given a set $X$, the cardinality of $X$ (denoted by $|X|$, $card(X)$) is the smallest ordinal for which there is a bijection with $X$.}
\note{}{Every set has a cardinality. 
This is bc every set can be well-ordered (by Zermelo's Theorem, which is equivalent to AC) and then we get an order-preserving bijection.
In fact the statement ``$card(X)$ is defined for each set $X$'' is equivalent to AC.
}
\defin{Equinumerous sets}{We call two sets $X,Y$ equinumerous, if 
there is a bijection between them.
}
By the previous \ref{Set:Def:Cardinality} we have: $X,Y$ equinumerous iff $|X| = |Y|$
\thm{}{Let $X,Y$ be non-empty sets Then the following are equivalent.
\begin{enumerate}[label = (\roman*)]
    \item there is an injection of $X$ into $Y$
    \item there is a surjection of $Y$ into $X$
    \item $|X| \leq |Y|$
\end{enumerate}
}{
    ``$\romannumeral 1 \implies \romannumeral 2$'' don't need AC
    ``$\romannumeral 2 \implies \romannumeral 1$'' need AC
    
}
\thm{Cantor-Schröder-Bernstein}{
    $|X| = |Y|$ iff there is an injection $X\to Y$ and an injection $Y\to X$
}{
    % ``$ \implies$'' clear with AC
    % ``$ \impliedby$'' clear with AC
    
}
Proof is clear with AC, but can proof it without AC
need to do some kind of back and fourth argument
\thm{Cantor}{\label{5:Thm:Cantor}
    For every set $X$ we have $|\mathcal{P}(X)|>|X|$
}{
    Suppose its not, then there exists a set $X$ with $|X|\geq |\mathcal{P}(X)$.\\
    We can find a surjection $\pi:X\to \mathcal{P}(X)$ 
    $Y = \{x\in X : x\notin\pi(x)\}$
    Let $y\in X$ be such that $\pi(y) = Y$
    Either $y\in Y$ or $y\notin Y$
    If $y\in Y$ then by definition of $Y$, $y\notin Y$ 
    If $y\notin Y$ then by definition $y\in Y$
}
Note this says: there is no largest set.

\defin{Cardinal}{$\kappa\in\Ord$ is called a cardinal, if $\kappa = |\kappa|$}
\note{}{\textbf{Claim: }The class $\Card$ of all cardinals is a proper class.
\begin{claimproof}
    Suppose $\Card$ is a set. then $\sup \Card = \gamma\in \Ord$
    Then $|\mathcal{P}(\gamma)|>|\gamma|\geq |\lambda|$ for every cardinals $\lambda\in \Card$
    that is a contradiction %TODO
\end{claimproof}
}
\defin{finite sets}{A set $X$ is called finite, if 
    $|X|$ is a finite ordinal, otherwise $X$ is called infinite.
}
\note{}{In particular a set is infinite iff $\omega$ injects into it.}% note: this is to keep in mind, good kriterium
\prop{Galileo}{
    A set $a$ is infinite iff it properly injects into itself.
}{
    $\implies$
    Suppose $a$ is infinite, then by note use $\omega$ injects into it 
    show $\omega$ injects properly in it self

    $\impliedby$ want: set finite then it does not inject properly into itself
    show: every finite ordinal is a cardinal inductively, then $\kappa = |\kappa|$

}
\subsection*{The $\aleph$-function}
\note{}{
    $\Card$ is cofinial in $\Ord$. That is for every $\alpha\in \Ord$ there is $\gamma\in \Card$ with $\alpha<\gamma$

    And $\Card$ is a proper subclass of $\Ord$, well ordered by the same ordering $\in$ as in $\Ord$

    There is a unique function (class function) from the class of all ordinals to the infinite cardinals 
    that preserves $\in$ \dots this function is called $\aleph$
    
}
\defin{}{Instead of $\aleph(0)$ we will write subscript $\aleph_0$
}
\begin{itemize}
    \item $\aleph_0 = |\omega|$
    \item $\aleph_{\alpha+1}$ is the smallest cardinal larger than $\aleph_\alpha$
\end{itemize}
\note{}{For every cardinal $\kappa$ there is a smallest cardinal $\kappa^+$ such that $\kappa<\kappa^+$ this is not the successor in the sense of the ordinals. It is
\begin{itemize}
    \item $n^+ = n+1$
    \item $\aleph_\alpha^+ = \aleph_{\alpha+1}$
\end{itemize}
}
\note{}{For every ordinal $\gamma$ we have
    $|\gamma|\leq \gamma<|\gamma^+|$
}
Similarly as with ordinals we will call cardinals of the form $\kappa^+$ successor cardinals.\outernote{successor cardinals} and non-zero, non-successor cardinals will be called limit cardinals.\outernote{limit cardinals}
\prop{}{\label{Aleph:cont}
    The function $\aleph$ is continuous with respect to the interval topology induced by $\in$. i.e.
    If $\lambda$ is a limit ordinal then $\aleph_\lambda = \sup_{\delta<\lambda}{\aleph_\delta}$
}{
    let $\gamma \defeq \sup_{\delta<\lambda}\aleph_{\delta}$
    have $|\gamma|\leq \gamma <|\gamma|^+$
    Assume $\gamma<\aleph_\lambda$
    then there exist a $\xi_0<\lambda$ such that $|\gamma| = \alpha_{\xi_0}$
    
    Then we would have $\gamma<|\gamma|^+=\aleph_{\xi_0+1}\leq \sup_{\delta<\lambda} \aleph_{\delta}$
    which is a contradiction.
    
    we have shown that $\aleph_\lambda\leq \sup_{\delta<\lambda}\aleph_{\delta}$ (TODO: other direction?)
}
\defin{countable sets}{
    A set $X$ is called to be countable iff $|X|\leq \aleph_0$ 
    and we say $X$ is uncountable otherwise.}

\subsubsection*{Continuum hypothesis (CH)}
\hypothesis{(CH)}{
    \[|\mathcal{P}(\omega)| = \aleph_1\]
    i.e. there is no cardinality between $|\RR|$ and $|\NN|$
    or $|\RR|$ is the first uncountable cardinality.
}

It can be shown that CH is independent of ZFC. (method to show this is called forcing, very popular method)

\section{Cardinal arithmetic}
\defin{}{Let 
    $\kappa,\lambda$ be cardinals then 
    $\kappa\otimes \lambda\defeq |\kappa\times\lambda|$
    $\kappa\oplus \lambda\defeq |(\kappa\times\{0\})\cup(\lambda\times\{1\})|$
    
}
\note{}{$\oplus,\otimes$ are commutative and associative.}
\thm{}{
    If $\kappa$ is infinite, then $\kappa\otimes\kappa = \kappa$
}{
    By induction on $\kappa$. What is ment by that is induction on $\alpha$ where $\aleph_\alpha =\kappa$.

    Base case: We should check it for $\omega$ 
    $\aleph_0\times \aleph_0 = \aleph_0$ is like finding bijection of $\omega^2 $ onto $\omega$.(diagonal)

    Now suppose: 
    $\forall \beta<\kappa |\beta|\otimes |\beta||\beta\times\beta| = |\beta|$

    On The cartesian product define order such that we get order isomorphism, then this is also a bijection.

    Define ordering on $\kappa\times \kappa$
    \[(\alpha,\beta)\prec (\alpha',\beta')\iff \begin{cases}
        \max\{\alpha,\beta\}<\max\{\alpha',\beta'\}\text{, or}\\
        \max\{\alpha,\beta\} = \max\{\alpha',\beta'\}\text{ and } \alpha<\alpha'\text{, or}\\
        \max\{\alpha,\beta\} = \max\{\alpha',\beta'\}\text{ and } \alpha =\alpha'\text{ and }\beta<\beta'\\
    \end{cases}\]
    \textbf{Claim:}$\prec$ is a well-ordering (Exercise)

    \textbf{Claim:} $(\kappa\times \kappa,\prec)$ is order-isomorphic to $(\kappa,\in)$
    From the claim we imediately get $|\kappa\times \kappa| = |\kappa|$
    \begin{claimproof}
        $\kappa\times \kappa = \bigcup_{\alpha<\kappa}\alpha\times \alpha$
        increasing union.
        then $\{\xi\times \xi\}$ is an initial segment of $\kappa\times\kappa$ (with respect to $\prec$)
        Consider $(\xi\times\xi,\prec)$ then by our inductive assumption this will have to be isomorphic to some ordinal $\gamma\in\Ord$.
        We know that the cardinality of that ordinal $\gamma$
        $|\gamma| = |\xi\times \xi| \stackrel{\text{ind. Hyp.}}{=}|\xi|<|\kappa|$
        So also $|\gamma|<|\kappa|$. so then $(\kappa\times\kappa,\prec)$ is order-isomorphic to $(\kappa,\in)$
    \end{claimproof}
}
\coroll{
    For every infinite cardinal $\kappa$ we have $\kappa \oplus \kappa = \kappa$
}
\begin{proof}
    Proof of corollary:
    $\kappa\oplus \kappa = |\kappa\times 2| \leq |\kappa\times\kappa| =|\kappa|$
\end{proof}

\defin{}{Define for $\kappa,\lambda$ cardinals
    $\kappa^\lambda \defeq |\{f: f \text{ is a function from $\lambda$ to $kappa$}\}|$
}
Note $2^\kappa = |\mathcal{P}(X)|$
\lemma{}{
    If $\lambda\geq \omega$ and $2\leq \kappa\leq \lambda$ then $\kappa^\lambda = 2^\lambda$.
}{
    \[2 ^{\lambda} = 2 ^{\lambda\otimes\lambda} = 2 ^{\lambda^\lambda}\geq\lambda^\lambda\geq \kappa^\lambda\]
    $ 2 ^{\lambda^\lambda}\geq\lambda^\lambda$ think of function that constantly maps to $0$ is part TODO image 5
    The other implication is obvious.
}
\datenote{16.01.2025}
\defin{}{
    A function $f:\alpha\to\beta$, where $\alpha,\beta$ are ordinals is said to be cofinal,
    if $\imag f$ is unbounded in $\beta$
    i.e. 
    \[\forall \gamma\in \beta\exists \xi \in \alpha \gamma\leq f(\xi)\]
}
The cofiniality of $\beta\in\Ord$ (denoted by $cof(\beta)$)
    is the smallest ordinal $\alpha$ such that there exists a function $f:\alpha\to\beta$ that is cofinal. Note: $cof(\beta)\leq \beta$, 

\bsp{}{
    $cof 1 = cof (\{\varnothing\}) = 1$ and in more generality
    $cof(\alpha + 1) = cof(\alpha \cup \{\alpha\}) = 1$
}
\note{}{ $cof (\beta)$ is always a cardinal.
    Let $\beta\in \Ord$ and let $f:cof(\beta)\to \beta$ be cofinal.
    Then $|cof(\beta)|\leq cof (\beta)|$ and there exists a bijection $h: |cof(\beta)|\to cof\beta$,
    so $f\circ h$ yields cofinal map and by minimality of $cof(\beta)$ have $|cof(\beta)| = cof (\beta)|$ 

    If $\beta$ limit ordinal then there is a strict increasing cofinal map $h:cof(\beta)\to \beta$ 
    Let $f:cof(\beta)\to \beta$ be cofinal define $h:cof(\beta)\to \beta$ by
    $$h(\xi) \defeq \max\{f(\xi), \sup_{\gamma<\xi}(h(\gamma)+1)\}$$
}
\prop{}{\label{limit:samecof}
    Suppose $\alpha,\beta$ are limit ordinals, $f:\alpha\to \beta$ strictly increasing and cofinal.
    Then $cof(\alpha) = cof(\beta)$
}{ One side is obvious
    \begin{itemize}
        \item $cof(\alpha)\geq cof(\beta)$ is clear
        \item $cof(\alpha)\leq cof(\beta)$
        Let $g:cof(\beta)\to \beta$ be cofinal
        define $h:cof(\beta)\to\alpha$ by 
        $$h(\xi) = \min\{\gamma<\alpha : f(\gamma)>g(\xi)\}$$
        cofinal in $\alpha$
    \end{itemize}
}
\coroll{For a limit ordinal $\alpha$, $cof(\aleph_\alpha) = cof(\alpha)$}
\begin{proof}
    Use \ref{Aleph:cont} and \ref{limit:samecof} on $f = \aleph$
\end{proof}
\coroll{For every ordinal $\beta$, $cof(\beta) = cof(cof(\beta))$}
\begin{proof}
    By cases
    \begin{itemize}
        \item $\beta$, $cof(\beta)$ are limit ordinals: there exists a strictly increasing map $cof(\beta) \to \beta$ result follows from \ref{limit:samecof}
        \item $\beta, cof(\beta)$ are not limits
        then $\beta = \alpha\cup \{\alpha\}$, $cof(\beta) = 1$ and $cof(cof(\beta)) = cof(1) = 1$
    \end{itemize}
\end{proof}
\defin{regular ordinals}{An ordinal $\beta$ is called regular, if $cof(\beta) = \beta$ (it is fixed point of cofininality map)}
\note{}{regular ordinals are cardinals and the first regular, infinite cardinal is $\omega$}
\lemma{}{$\kappa^+$ is reglular for $\kappa\geq \omega$}{
    If $f:\alpha\to \kappa^+$ is cofinal, then $\kappa^+ = \bigcup_{\gamma<\alpha} f(\gamma) = \sup_{\gamma<\alpha}f(\gamma)$.
    Each $f(\gamma)$ is an ordinal of cardinality less then $\kappa^+$ hence is less or equal than $\kappa$
    $$\kappa^+ = |\bigcup_{\gamma<\alpha}f(\gamma)| \leq |\alpha\times \kappa| \leq \max\{|\alpha|,|\kappa|\}$$
    because $\kappa<\kappa^+$ we have $\kappa^+ \leq|\alpha| = \alpha$ TODO check
}
\note{}{$\alpha$ limit ordinal then $cof(\aleph_\alpha) = \alpha$\\
If $\aleph_\alpha$ is regular then $cof \aleph_\alpha = cof \alpha \leq \alpha\leq \aleph_\alpha$
So $\alpha = \aleph_\alpha$
}
\defin{}{Let $\kappa$ be a cardinal.
    \begin{itemize}
        \item $\kappa$ is called weakly inaccessible, if
    $\alpha$ is a regular limit cardinal strictly greater than $\omega$
    \item $\kappa$ is called (strongly) inaccessible, if $\kappa>\omega$, $\kappa$ regular and
    for every $\lambda<\omega$ we have $2^\lambda<\kappa$
    \end{itemize}
}
We will see that 
$\kappa$ inaccessible implies that $\kappa$ is not a union of fewer that $\kappa$ sets each of cardinality less that $\kappa$.
\note{}{Existence of inaccessible cardinals does not follow from ZFC.}
The question is, why care then?

Axiom of existence of inaccessible cardinals:
\begin{equation}
    (IC) \quad\text{There exists some inaccessible cardinal.}
\end{equation}

(AC) has less desirable consequences e.g. Banach-Tarski-Paradox

If we drop AC we could not prove that the lebesgue meassure is countably additive,
so we may want to replace AC by something weaker (e.g. DC, dependent choice).
Various nice results that depent on AC still hold. 

LM the axiom that states: every set of reals is lebesgue meassurable
\thm{}{Con\dots consistency. 
    Con(ZF+DC+LM) is equal to Con(ZF + IC)
}{}
Note: We do now that ZF can not prove IC, but we dont know yet if ZF can prove the negation of IC.

\lemma{}{
    If $\kappa$ is an infinite cardinal and $\lambda\geq cof(\kappa)$ then $\kappa^\lambda>\kappa$
}{
    Fix a cofinal map $f:\lambda\to \kappa$.
    Consider any function $G:\kappa\to\kappa^\lambda$
    It suffices to show that $G$ can not be surjective.

    $\kappa^\lambda$ is technically the set of functions $\lambda\to\kappa$
    define $h:\lambda\to\kappa$ by 

    \[h(\xi) = \min\{\kappa \setminus(G(\alpha)(\xi)) : \alpha \leq f(\xi)\}\]
    If $h\in \imag G$ then there is $\alpha\in \kappa$ s.t. $G(\alpha) = \kappa$
    pick $\xi <\lambda$ s.t. $f(\xi)\geq \alpha$
    Then $h(\xi) = G(\alpha)(\xi)$ but by construction of $h$, $G(\alpha)(\xi) \neq h(\xi)$ so contradiction.
}
\coroll{If $\lambda\geq \omega$ then $cof(2^\lambda)>\lambda$
}
\begin{proof}
    We know that $(2^\lambda)^\lambda = 2^{(\lambda\otimes \lambda)} = 2^\lambda$
    So if $cof(2^\lambda)\leq \lambda$ then by lemma $(2^\lambda)^\lambda>2^\lambda$, a contradiction.
\end{proof}
\thm{(König)}{
    Let $I$ be a set, $(A_i)_{i\in I}$ and $(B_i)_{i\in I}$ indexed families of sets.

    If $\forall i\in I |A_i|<|B_i|$ then 
    $|\bigsqcup_{i\in I}A_i|<|\prod_{i\in I}B_i|$
}{Exercise with hints.}
Note: above thm is equivalent to AC

One can use König theorem to show that:
``$\kappa$ inaccessible implies that $\kappa$ is not a union of fewer that $\kappa$ sets each of cardinality less that $\kappa$''TODO: check if actually true

Next time : ZF can not proof ZFC
\section{Consistency of a theory}
\datenote{21.01.2025}
%Omega is not cofinal in aleph_1, also why cofinality is different from cardinality
Today we are going to state our last axiom, the axiom of foundation. We are also going to show a relative consistency theorem.
We are going to finish prob next course thursday

\axiom{Axiom of foundation (AF)}{
    Beeing an element of, does not admit an infinite decreasing chain.
    $$\forall x x\neq \varnothing \to \exists y(y\in x\land \forall z\in y z \notin x)$$
}
\note{}{
    Suppose we have a sequence of sets $(u_n)_{n\in\omega}$ s.t. $\forall n u_{n+1}\in u_n$ then $\{u_n : n\in \NN\}$ would contradict AF.
    and have $\forall x (x\notin x)$ too.
}
\defin{}{
    Class function
    $V:\Ord \to \mathcal{U}$ by transfinite induction on ordinals 
    and set $V_\beta \defeq \bigcup_{\alpha<\beta}{\mathcal{P}(V_\alpha)}$
    $V_0 = \varnothing$
    If $\alpha\leq \beta$ then $V_\alpha\subseteq V_\beta$ 
    (an example of increasing sequence, in contrast to note on axiom AF)
    $V_{\beta+1} = \mathcal{P}(V_\beta)$

    If $\lambda$ is a limit ordinal then $V_\lambda = \bigcup_{\alpha<\lambda}V_\alpha$

    $V$ is also a class given by $V(x)\equiv \exists \alpha\in\Ord\: x\in V_\alpha$
}
$V$ also gives us a way to associate a rank to each set.
\defin{}{
    For every $x$ such that $V(x)$ we define
    $\rk(x)\defeq \min\{\alpha : \: x\in V_\alpha\}$
}
\note{}{
    The rank $\rk(x)$ is always a successor ordinal.
}
\lemma{}{
    $V(x)$ iff $\forall y \: y\in x\to V(y)$
    And also if $V(x)$ then $\forall y (y\in x \to \rk(y)<\rk(x))$
}{
    ``$\implies$'' direction: 
    $V(x)$ and $\rk(x) = \beta+1$ then 
    $x\in V_{\beta+1}$ so $x\subseteq V_\beta$ hence $\forall y\in x \: y\in V_\beta$
    Also $\rk(y)\leq \beta <\beta+1 = \rk(x)$

    ``$\impliedby$''-direction:
    Suppose $\forall y\in x V(y)$
    Note that $\rk : V\to \Ord$ is bounded on $x$.
    Else, $\{\rk y : y \in x\}$ is unbounded in $\Ord$ and it is a set (image of function $\rk$ of a set)
    Take $\bigcup\{\rk y : y\in x\}\in \Ord$ and it would be an ordinal bigger than any other set, TODO

    Suppose $\{\rk y : y\in x\}$ is bounded by $\beta$ then $\forall y \in x y\in V_\beta$
    so $x\in V_{\beta+1}$.
}
\lemma{}{
    For every ordinal $\alpha\in \Ord$ we have $V(\alpha)$ and $\rk (\alpha) = \alpha+1$.
}{Exercise}
\defin{inductive closure}{
    For any set $x$ we define the function with domain $\omega$ 
    $f(0) \defeq x$
    $f(n+1) \defeq \bigcup_{y\in f(n)}y = \bigcup f(n)$
    and we define the closure of $x$ to be the union 
    $\cl x \defeq \bigcup_{n<\omega}f(n)$
}
\note{}{
    \begin{itemize}
        \item $x\subseteq \cl x$
        \item $\cl$ is transitive
        \item If $z$ is transitive set that contains $x$ then then $\cl x\subseteq z$ 
        ($\cl x$ is the unique transitive closure of $x$)
    \end{itemize}
}
\thm{}{
    (AF) holds iff $\forall x V(x)$
}{
    ``$\impliedby$''-direction: Suppose $\forall x V(x)$.
    We need to show that any set contains an element 
    Let $a\neq \varnothing$. Let $y\in a$ be of minimal rank.
    Then for every $c\in y$ we know $\rk c<\rk y$ so $c\notin a$ by minimality of $\rk y$.
    So $y\in a$ s.t. $y\cap a = \varnothing$
    
    ``$\implies$'' direction: Lets assume the axiom of foundation and by contradiction that $x$ is a set
    for which $\lnot V(x)$
    Have $x\subseteq \cl x$ 
    \textbf{Claim:} $Y = \{y\in \cl x : \: \lnot V(y)\}\neq \varnothing$
    \begin{claimproof}
        If $Y = \varnothing$ then $\forall y \in Y V(y)$ and $\rk$ bounded on $Y$. %TODO
    \end{claimproof}
    Let $y\in Y$. Then $\lnot V(y)$ so i.p. $y\not \subseteq V$, 
    so for some $z\in y$ have $\lnot V(z)$
    but because $\cl x $ is transitive, $z\in \cl x$ hence $z\in Y$
    hence $\forall y\in Y y\cap Y \neq \varnothing$, a contradiction with (AF).
}

By Gödels 2-nd incompletness theorem ZFC can not prove its own consistency.
There is a way to express consistency in the formal level, by coding and using peano arithmetic, and the above statement says that ZFC can not prove this sentence.
All we can hope for are relative consistency results, and therefore relate two theories with each other.
For example some theories are less debated about 
and it shows a way to prove independence of certain axioms from others.

\subsection{relative consistency}

\begin{itemize}
    \item ZFC The axioms \begin{enumerate*}
        \item extensionality
        \item union
        \item power-set
        \item ax scheme of replacement
        \item set ax
        \item axiom of infinity
        \item AC
        \item AF
    \end{enumerate*}
    (recall pairing and comprehension follow from the other)

    \item ZFC$^-$ is  ZFC without (AF)
    \item ZF is ZFC without (AC)
    \item ZF$^-$ is ZF without (AF)
\end{itemize}

We will use a tool that is in set theory called Relativization

Let $C$ be a class, $\phi(\underline{x},\underline{a})$, $\underline{a}\in C$

$\phi^C(\underline{x},\underline{a})$ defined by induction on compl of $\phi$
\begin{itemize}
    \item If $\phi$ is atomic then $\phi^C = \phi$
    \item $(\lnot \phi)^C = \lnot (\phi^C)$,  $( \phi\lor \psi)^C =  (\phi^C)\lor(\psi^C)$  
    \item $(\exists y \phi)^C = \exists y (C(y)\land \phi^C)$
    \item $(\forall y \phi)^C = \forall y (C(y)\to \phi^C)$
\end{itemize}


\thm{}{
    Suppose $(\mathcal{U},\in)\models ZF^-$ then $V$ constructed in $\mathcal{U}$ is such that
    $(V,\in)\models ZF$ 
    i.e. assume $ZF^-$ is consistent (has a model) and we get a model of ZF 
}{
    ZF$^-$
    Take 
    \begin{enumerate*}
        \item extensionality
        \item union
        \item power-set
        \item ax scheme of replacement
        \item set ax
        \item axiom of infinity
    \end{enumerate*}
    (recall pairing and comprehension follow from the other)
    Need to check if  $(\mathcal{U},\in)\models ZF^-$ and $V$ class defined by 
    $V = \bigcup_{\alpha\in \Ord}V_\alpha$, $V_\beta \defeq \bigcup_{\alpha<\beta}{\mathcal{P}(V_\alpha)}$
    then $(V,\in )\models ZF^-$ (AF will follow from prev. result)

    \begin{enumerate}
        \item Let $x,y\in V$ wts $(\forall z\in V z\in x\leftrightarrow z\in y)\leftrightarrow x=y$ 
        $x,y\in V$ so $x,y\subseteq V$
        and can use Axiom of extensionality in $\mathcal{U}$
        $(\forall z\in \mathcal{U} z\in x\leftrightarrow z\in y)\leftrightarrow x=y$
        
        \item Let $x\in V$ $U_x = \{z : \exists y \in x z\in y\}\subseteq V$
        so $\bigcup_x\in V$.
        \item Let $x\in V$ then $x\subseteq V$. so every subset of $x$ is a subset of $V$ hence is an 
        element of $V$. Powerset  $\mathcal{P} (x)\subseteq V$ hence  $\mathcal{P} (x)\in V$
        %TODO (range funciton is bounded on the ???)
        \item $\varphi(x,y)$ with parameters from $V$ and assume that $\varphi(x,y)$ defines a class function in $V$ i.e.
        $$\biggl(\forall x \exists ^{\leq 1}y \varphi (x,y)\biggr)^V$$
        i.e.
        $$\forall x (V(x)\to (\exists ^{\leq 1}y \varphi (x,y)) )$$
        Then $\psi (x,y)\defaq V(x)\land V(y)\land \varphi^V(x,y)$
        defines a class function in $\mathcal{U}$, use replacement in $\mathcal{U}$
        That yields 

        $$\forall a \exists b y\in b \iif (\exists x \in A \varphi(x,y))\iif \exists x\in a \varphi^V(x,y)\land V(y)$$
        $b\subseteq V$ hence $b\in V$ and $b$ is the image of $a$ under the class function given by $\varphi$.
        \item Since we proved that every ordinal is in $V$, $\varnothing\in V$.
        \item enough to show that $\omega\in V$. in fact we know that any ordinal is in $V$.
    \end{enumerate}
}

An ordinal $\alpha$ is called regular, if it is equal to its own cardinality $\cof \alpha = \alpha$
A cardinal $\kappa$ is called inaccessible if $\kappa > \omega$ and $\kappa $ is regular.

It suffices to say $\forall \lambda<\omega 2^\lambda <\kappa$
Suppose now that $(\mathcal{U}, \in)\models ZFC$ 
\lemma{}{\label{lem:inacc1}
    If $\kappa$ is an inaccessible cardinal, then $|V_\kappa| = \kappa$. Moreover 
    for every $a\subseteq V_\kappa$ we have $a\in V_\kappa$ iff $|a|<\kappa$
}{
    Recall, if $\alpha\in \Ord$ then $\alpha\subseteq V_\alpha$. i.p.
    $\kappa\subseteq V_\kappa$ so $|\kappa|\leq |V_\kappa|$
    
    The other direction: 
    For this we are going to show inductively that for all $\xi<\kappa$ that $|V_\xi|<\kappa$
    Then $\kappa\geq |V_\kappa|$ follows.
    If $|V_\xi|<\kappa$ then $|V_{\xi+1}| = |\mathcal{P}(V_\xi)|\leq |2^{|V_\xi|}|<\kappa$
    by $\kappa$ inaccessible.
    Suppose  $|V_\xi|<\kappa$ for all $\xi<\lambda<\kappa$ where $\lambda$ is a limit ordinal.
    \[|V_\lambda|  = |\bigcup_{\xi<\lambda}V_\xi| \leq  \sup{\xi<\lambda}|V_\xi |<\kappa\]
    here we use regularity of $\kappa$.

    So have $|V_\kappa| = \kappa$.
    Left to show: for every $a\subseteq V_\kappa$ have $a\in V_\kappa$ iff $|a|<\kappa$
    \begin{itemize}
        \item ``$\impliedby$''-direction: Assume $a\subseteq V_\kappa$ and $|a|<\kappa$ 
        $\rk: a \to \Ord$ is not cofinal in $\kappa$ because $cof \kappa = \kappa > |a|$
        So for some ordinal $\beta$ have $a\subseteq V_\beta$ and then $a\in V_{\beta+1}\subseteq V_\kappa$
        \item ``$\implies$''-direction: it suffices $a\in V_\kappa$ then $|a|<\kappa$
        Exercise.
    \end{itemize}
}

\datenote{23.01.2025}


\lemma{}{Let $(\mathcal{U},\in)\models ZFC$. If $\kappa$ is inaccessible, then $V_\kappa\models ZFC$.}{
    Will check (AC) and replacement, the remaining axioms are an exercise.
    \begin{itemize}
        \item[(AC)]: Suppose $a\in V_\kappa$ and $a$ is a family of pairwise disjoint, non-empty sets.
            By (AC) in $\mathcal{U}$ there is a transversal $T$ in $\mathcal{U}$ for the set $a$. What is 
            left to show is $T\in V_\kappa$.
            Have $a\subseteq V_\kappa$ then every subset of $a$ is a subset of $V_\kappa$ so in particular 
            $T\subseteq V_\kappa$.
            We do know that $a\in V_\kappa$ so $|a|<\kappa$ and $T\subseteq a$ we have $|T|\leq |a|$ so by 
            the previous lemma $T\in V_\kappa$.
        \item[(RE)]: $\varphi(x,y)$ a formula with parameters in $V_\kappa$ that defines a class function in 
            $V_\kappa$. i.e. $$\forall x\in V_\kappa \exists^{\leq 1} y\in 
            V_\kappa \varphi^{V_\kappa}(x,y)$$ 
            let $a\in V_\kappa$ then $\psi(x,y) \equiv x\in V_\kappa \land y\in V_\kappa \land \varphi(x,y)$
            does define a class function on $\mathcal{U}$ with domain contained in $V_\kappa$ 
            %TODO check class and on cal(U)
            So $f[a]\subseteq V_\kappa$ and $ |f[a]|<\kappa$ so $f[a]\in V_\kappa$.
    \end{itemize}
}
Now we are ready to state our meta-theorem, it is not a statement in first order language of set theory.
\thm{}{
    If ZFC is consistent then ZFC + ``There are no strongly inaccessible cardinals'' is also consistent
}{
    Assume we have a model of ZFC, $(\mathcal{U},\in)\models ZFC$.
    \begin{itemize}
        \item If $\mathcal{U}$ does not contain any inaccessible cardinals then we are done.
        \item Assume that $\mathcal{U}$ does contain inaccessible cardinals. Let $\kappa$ be the smallest inaccessible cardinal. 
    \end{itemize}
    What we want to show is that there are no inaccessible cardinals in $V_\kappa$.
    An ordinal is by definition a transitive set, well-ordered by $\in$.
    By (AF) $\alpha$ is transitive and $\alpha$ is linearly ordered by $\in$.
    i.e. $\alpha$ ordinal iff 
    \begin{equation}\label{equation:ordinal}
        \forall x,y \in \alpha (x\in y \lor y\in x \lor x=y)
    \land \forall x(x\in \alpha \to x\subseteq \alpha)
    \end{equation}
    \textbf{Claim 1: }The ordinals in $V_\kappa$ are the ordinals below 
    $\kappa$ i.e. $Ord^{V_\kappa}=\kappa$.\\
    \begin{claimproof}
        If $\alpha<\kappa$ then $\alpha\subseteq V_\kappa$ ($\rk \alpha = \alpha+1$ and $V_{\alpha+1}
        \subseteq V_\kappa$ so $\alpha\in V_\kappa$ ($\alpha+1<\kappa$)) $\alpha$ is an ordinal so \ref
        {equation:ordinal} holds. and $\alpha\in \Ord^{V_\kappa}$
        If $\alpha\in \Ord^{V_\kappa}$ then $\alpha\in V_\kappa$ and $\alpha$ is transitive and totally 
        ordered by $\in$.
        % TODO check: All elements of $\alpha$ are elements of $V_\kappa$. 
        Hence $\alpha$ is an ordinal. Left to show $\alpha$ is below $\kappa$.
        Have $|\alpha|<\kappa$,
        by the \ref{lem:inacc1} $\alpha<\kappa$.
    \end{claimproof}
    \textbf{Claim 2: }The cardinals in $V_\kappa$ are the cardinals that are below $\kappa$.\\
    \begin{claimproof}
        Suppose $\lambda$ is a cardinal in $V_\kappa$. in particular $\lambda$ is an ordinal in $V_\kappa$ 
        and therefore an ordinal in $\mathcal{U}$ and by the \ref{lem:inacc1} $\lambda = |\lambda| 
        <\kappa$.
        left to show: $\lambda$ is an actual cardinal in $\mathcal{U}$
        Suppose there is a bijection $f:\lambda\to \alpha$ to some smaller ordinal $\alpha$. Then by 
        Claim 1, $\alpha\in \Ord^{V_\kappa}$ and $f\subseteq V_\kappa$ (the graph $f\subseteq \lambda\times 
        \alpha\subseteq \lambda \times \lambda$ so $f\subseteq \mathcal{P}\mathcal{P}\mathcal{P}(V_{\lambda
        +1})\eqdef V_\beta$, $\rk \lambda = \lambda+1$ and have $\beta<\kappa$ by inaccessability).
        $f\subseteq V_\kappa$ and $|f|<\kappa$ so by \ref{lem:inacc1} $f\in V_\kappa$.
        $f$ is a bijection in $V_\kappa$ between $\lambda$ and a smaller ordinal, a contradiction.

        Suppose $\lambda<\kappa$ and $\lambda$ cardinal, then by \ref{lem:inacc1} $\lambda\in V_\kappa$ and $\lambda\in \Ord^{V_\kappa}$.
        If there would be a bijection in $V_\kappa$ between $\lambda$ and a strictly smaller ordinal 
        $\alpha$ then $\alpha$ would be an actual ordinal. By an argument as before would get a bijection in 
        $\mathcal{U}$ between $\lambda$ and smaller ordinal.
    \end{claimproof}
    If $\lambda\in V_\kappa$ is a cardinal in $V_\kappa$ then $\lambda<\kappa$ and by claim 2 it is a cardinal in $\mathcal{U}$, $\lambda$ can not be inaccessible. by choice of $\kappa$.

    Note that only $\mathcal{U}$ knows that $\lambda$ is not inaccessible. We therefore need to check that also $V_\kappa$ knows that $\lambda$ is inaccessible.
    Reasons for $\lambda$ to not be inaccessible:
    \begin{itemize}
        \item $\lambda\leq \omega$ (in $\mathcal{U}$ ) hence $(\lambda\leq \omega)^{V_\kappa}$ bc. $\omega\in V_\kappa$
        \item $\lambda\leq 2^\xi$ for some $\xi <\lambda<\kappa$ hence
        $\xi,2^xi\in V_\kappa$ and have $(\xi<\lambda<2^\xi)^{V_\kappa}$
        \item $\lambda$ is not regular, if we have a function $f:\alpha\to \lambda$ cofinal and $\alpha$ an ordinal $\alpha<\lambda$, then $\alpha\in V_\kappa$ hence $f\in V_\kappa$ then $\lambda$ is not regular in $V_\kappa$.
    \end{itemize}
}

Note: $Ord^{V_\kappa}=\kappa$. but $Ord^{V_\kappa}$ is no set in $V_\kappa$.
