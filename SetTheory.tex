\chapter{Set Theory}
The contents on this chapter are at least partially sourced on \cite{krivine1998théorie}.
\bsp{Russel's paradox}{Let $A = \{a : a\notin a\}$. If any collection of elements is a set, then $A$ would be a set.
Question: is $A\in A$? if yes, then $A\notin A$, if not then $A\in A$}
\noindent Trying to resolve this, we will introduce the ZFC (Zermelo-Frankel axioms w/ choice) System.
Let $\mathcal{L}=\{\in\}$ be a Language of first order, where $\in$ \dots binary relation "beeing element of"
For $(\mathcal{U},\in)$ 
If $\mathcal{A} = (\mathcal{U},\in^\mathcal{A})\models \text{ZFC}$, then the elements of the universe $\mathcal{U}$ are called sets.
We will show roughly that some definably sets are not sets (in the sense of ZFC), others are not. The latter will be called classes.
\section{Axioms of ZFC}
\defin{Axiom of extensionality}{\label{Ax1}
    $$\forall x \forall y (x=y \leftrightarrow \forall u (u\in x \leftrightarrow u\in y))$$}
In other words, two sets are the same if they have the same elements. This will give us later uniqueness in construction of other sets.
\defin{Pairing Axiom}{\label{Ax2}
    for any two sets $a,b$ one can form a set whose elements are precicely $a,b$
$$\forall x\forall y \exists z (u\in z \leftrightarrow (u = x \lor u = y))$$
Our notation will be $z=\{x,y\}$
}
In words: For any two sets there exists a set whose members are those two sets.

\note{}{
    \begin{itemize}
        \item $\{x,y\}$ is unique by \ref{Ax1}
        \item $\{x\}$ is a set. from \ref{Ax2}, take $x= y$
    \end{itemize}
}
\lemma{}{Let $x,y$ be sets. We define the ordered pair $(x,y) \defeq \{\{x\},\{x,y\} \}$. 
    Then it holds $(x,y) = (a,b)$ iff $x = a$ and $y = b$
}{By cases
    \begin{itemize}
        \item if $x=y$, then $(x,y) = \{\{x\} \}$ therefore $a=b$ and by \ref{Ax1} it holds $x=a$.
        \item if $x\neq y$, then $\{\{x\},\{x,y\} \} = \{\{a\},\{a,b\} \}$ iff $\{x\} = \{a\}$ and $\{x,y\} = \{a,b\}$. That is, iff $x=a$ and $y=b$.
    \end{itemize} }
\note{}{The set $(x,y)$ exists, because its obtained by repeatedly using \ref{Ax2}}

\lemma{}{Let $x,y,a,b$ be sets. Then $(x,y) = (a,b)$ iff $x = a$ and $y = b$}{
    \begin{itemize}
        \item case $x=y$, then $(x,y) = \{\{x\}\}$ is a singleton then $(a,b)$ is a singleton, 
        wlog $\{a\} = \{a,b\}$ then $a = b = x$.
        \item case $x\neq y$ and $\{\{x\},\{x,y\}\} = \{\{a\},\{a,b\}\}$ then $\{x\} = \{a\}$ and $\{x,y\} = \{a,b\}$ because by \ref{Ax1} a singleton can not be equal to a set of size $2$.
    \end{itemize}
}
\defin{n-tuples}{
    Define $(x_1,\dots x_n)$ inductively:
    \begin{itemize}
        \item $(x_1,x_2)$ already defined
        \item $(x_1,\dots x_n) \defeq (x_1 (x_2,\dots x_n))$
    \end{itemize}
}
\lemma{}{For all $n>1$
    $(x_1,\dots, x_n) = (y_1,\dots, y_m)$ iff $n=m$ and $\forall i\leq n x_i = y_i$
}{Exercise}

\defin{Union Axiom}{For every set $x$ there is a set $z$ consisting of all elements of the elements of $x$.
    $$\forall x \exists z \forall y (y\in z \leftrightarrow \exists u (u\in x \land y\in u))$$
    We call $z$ the union of $x$, notation: $\cup_x\defeq z$}

\defin{}{\[x\cup y \defeq \bigcup_{\{x,y\}}\]

}
\bsp{}{
    \begin{enumerate}
        \item $\bigcup_{(x,y)} = \{x,y\}$.
        \item $(x_1,x_2,\dots x_n) = \bigcup_{\{x_1\}, {x_2,\dots x_n}}$ 
    \end{enumerate}
}   
\note{}{
    \begin{itemize}
        \item $\forall x_1, \dots x_n$ then there is axactly one set with elements $x_1,\dots x_n$

        \item $x\cup (y\cup z) = (x\cup y)\cup z$
    \end{itemize}
}

\defin{Power set Axiom}{Let $x\subseteq y$ be the abbreviation for $\forall z(z \in x \to z\in y)$ The \graybf{Powerset Axiom} states, that for every set $x$
there exists a set $z$ consisting of all subsetes $y\subseteq x$ that are themselve sets.
    $$\forall x\exists z \forall y (y\in z \leftrightarrow y\subseteq x)$$
    Notation: $\mathcal{P}(x)\defeq z$.
}
Or in words: ``For every set $x$ there is a set $z$ consisting of all subcollections of $x$ that are themselve sets.''
class relations  


\defin{Classes}{All the unary $\mathcal{L}$-definable relations (w/ parameters) are called classes.}
\bsp{}{
    \begin{itemize}
        \item $\varphi (x) : x = x$ defines a class that is not a set
        \item $\varphi(x) : \exists u (u\in x \land \forall v\in u (v\in x))$
    \end{itemize}
}
\defin{Class functions}{
    Suppose we have a formula $\phi(x_1,\dots x_n, y)$. Then we say $\phi$ defines a class function $R_\phi$ iff 
    \[\forall x_1\dots \forall x_n \forall y \forall y' ((\phi(x,y)\land \phi(\underline{x},y'))\to y = y')\]
    We can then define the domain and image of the class function.
    \[\dom R_\phi : \: \{\underline{x} : \:\exists y \phi( \underline{x},y)\}\]
    \[\imag R_\phi : \: \{y : \:\exists \underline{x} \phi(\underline{x},y)\}\]
    Note that $R_\phi(\underline{x})=y$ iff $\phi(\underline{x},y)$
}
\defin{Axiom of replacement / substitution}{\label{Ax5}
    Let $\varphi(x,y,\underbar{a})$ a $\mathcal{L}$-fla., w/ free variables among $x,y$ and set-parameters $\underbar{a}$.
    Suppose $\varphi$ defines a class function on $\mathcal{U}$, than the followoing is an axiom:
    $$\forall u \exists z \forall y\: (y\in z \leftrightarrow \exists x (x\in u \land \varphi(x,y,\underbar{a})))$$
    i.e. the image of a set under a class function is a set.
}
\defin{Axiom scheme of comprehension}{\label{Ax6}
    Let $\psi(x,\underline{a})$ be an $\mathcal{L}$-formula. Then the followoing is an axiom:
    \[\forall u \exists z \forall v \:(v\in z \leftrightarrow(v\in u\land \forall \psi(v,\underline{a})))\]
    i.e. all elements of a set that satisfy a given $\mathcal{L}$-formula form a set.
}
\note{}{\ref{Ax6} follows from \ref{Ax5}}
\defin{Set existence}{\label{Ax7}
    \[\exists x x = x\]
    i.e. $U\neq \varnothing$. - this is clear when we view it as a universe of a structure.
}
\note{on the existence of the empty set}{
    Let $u$ be any set (there exists one by \ref{Ax7}), $\psi(x) \equiv x\neq x$ then by \ref{Ax6} $\varnothing \defeq \{x\in u : \psi(x)\}$
    is a set.
}
\note{}{We can derive pairing from replacement \ref{Ax5}, extensionality, powerset and set existence.
From set existence: $\varnothing$ is a set
By powerset, replacement(comprehension): $\{\varnothing\}$ is a set.
    \[\mathcal{P}(\{\varnothing\}) = \{\varnothing ,\{\varnothing\}\}\]
    is a set.
    Then by defining a class function $R_\phi (x) = y$, 
    $$\phi (x,y)\equiv (x=\varnothing\land y = a)\lor (x = \{\varnothing\}\land y = b)$$
}
\subsubsection*{Functions}
\note{}{If the domain of a class function is a set then the graph of the class function is a set. %check if really is graph
    $R_\phi$ defined by $\phi(x,y,\underline{a})$ then the domain $u\defeq \dom R_\phi \in \mathcal{U}$
    Image $v\defeq \imag R_\phi \in \mathcal{U}$ by replacement 
    \[\{(x,y) : x\in u \land y \in v \land \phi(x,y,\underline{a})\}\]
    is the graph of $R_\phi$
    The above would be a set if $u\times v$ which can be shown by using comprehenson (exercise).
}
\defin{Function}{A function $f:a\to b$ where $a,b$ are sets is a subset of $a\times b$ that satisfies the followoing
    \begin{itemize}
        \item $\forall x \: (x\in a \to \exists y \in b (x,y)\in f)$
        \item $\forall x \forall y \forall y' (((x,y)\in f \land (x,y')\in f)\to y = y')$
    \end{itemize}
}
\subsubsection*{Families of sets and cartesian products}
Suppose we have a function $a: I \to X$. set $a_i$ to the unique $x\in X$ s.th. $(i,x)\in a$
\defin{}{
    \[\bigcup_{i\in I}a_i = \{z\in \bigcup X : \exists i\in I z\in a_i\}\]
    \[\bigcap_{i\in I}a_i = \{z\in \bigcup X : \forall i\in I z\in a_i\}\]
    \[\prod_{i\in I}a_i = \{f:I\to \bigcup X : \forall i\in I z\in a_i\}\]
}
\note{}{
    If $I=\varnothing$ then $\bigcap_{i\in I}a_i = \bigcup X$
}
\subsubsection*{Ordinals}
Types of well ordered sets
\defin{}{
    Let $R$ class relation, $C$ class
    Then $R$ defines a strict ordering on $C$, if
    \[\forall x \forall y \forall z R(x,y)\to (C(x)\land C(y))\]
    \[\forall x \forall y \forall z \lnot(R(x,y)\land R(y,x))\]
    \[\forall x \forall y \forall z R(x,y)\land R(y,z)\to R(x,z))\]
    The ordering is linear if additionally
    \[\forall x \forall y (C(x)\land C(y))\to (x=y\lor R(x,y)\lor R(y,x))\]
}
\defin{Well ordering}{Let $R$ be a strict ordering on $C$ 
    $x$ is a set such that $\forall y \in x C(y)$
    Then $x$ is called well-ordered by $R$, if
    \[\forall \varnothing \neq y \subseteq x \text{has a smallest element}\]
    i.e.
    \[\forall y \biggl((\varnothing \neq y \land y\subseteq x) \to \exists y' (y'\in y \land \forall z (z\in y \to( R(y',z)\lor y'=z)))\biggr)\]
}
\defin{Initial segment}{
    Let $x$ be a set, well ordered by $R$.
    Then $y\subseteq x$ is called an initial segment of $x$, if
    \[\forall x\forall t s\in x \land t \in x \to  \biggl((t\in y \land s<t)\to s\in y\biggr)\]
}
notation: If $y\in x$, $x$ is well-ordered
Then $\delta_y(x) = \{z\in x : z<y\}$\
$<$  strict, $\implies$ $y\notin \delta_y(x)$


\note{}{If $x$ is well-ordered and $y\subseteq x$ then
    \begin{center}
        $y$ is an initial segment of $x$ iff $y = x $ or $y = \delta_z(x)$ for some $z\in x$
    \end{center}

    Let $y\subseteq x$ an initial segment. Suppose $x\neq y$ that means $x\backslash y \neq \varnothing$
    Let $z$ be the smallest element of $x\backslash y$ (exists by well-ordering of $x$) 
}
\defin{Propper class}{A class $C$ is a proper class if it is not a set.
    i.e. if $C$ is given by $\phi(x,\underline{a})$ then there is not $z\in \mathcal{U}$ such that
    $\forall x x\in z \iif \phi (x,\underline{a})$
}
\bsp{}{$\mathcal{U}$ is a proper class
    If $\mathcal{U}$ was a set then $\{x : x\notin x\}$ would be a set.

    the class of all ordinals is a proper class.
}
\subsubsection*{Well-orderings of classes and ordinals}
\defin{well-ordering (class)}{A class relation $R$ defining a strict ordering on a class $C$
    is a well-ordering, if \\
    For every $x\in C$ the class initial segment $\delta_x(C) = \{y : R(y,x)\}$ is a set which is well ordered by $R$
}
\defin{Tranistivity of sets}{A set $x$ is called transitive if 
\[\forall y (y\in x \to y\subseteq x)\]
}
\note{}{
    It corresponds to Tranistivity of the belonging relation $\in$. $z \in y\in x \to z\in x$
}
\defin{Ordinal}{An ordinal is a transitive set which is well ordered by $\in$}
\note{}{The collection of all ordinals is a class TODO text
    Write down formula 
}
\bsp{}{
    \begin{itemize}
        \item $\varnothing$, $\{\varnothing\}$, $\{\varnothing , \{\varnothing\}\}$ are ordinals
    \end{itemize}
}
\lemma{Characterization of ordinals}{$\alpha$ ordinal
    \begin{itemize}
        \item the initil segments of $\alpha$ are $\alpha$ itself and the elements of $\alpha$
        \item if $\beta \in \alpha$ then $\beta $ is an ordinal.
        \item $\alpha\notin \alpha$
    \end{itemize}
}{Problem set}
\lemma{}{
    Let $\alpha,\beta\in \theta$ TODO then either $\alpha = \beta$, or $\alpha\in \beta$ or $\beta\in \alpha$.
}{
    Let $\gamma \defeq \alpha\cap \beta$
    Claim: $\gamma$ is initial segment of both $\alpha $ and $\beta$
    \begin{claimproof}
        $x\in y\in \gamma$ then $x\in y \in \alpha$ and $x\in y \in \beta$. but $\alpha, \beta$ are ordinals, so
        $x\in \alpha$ and $x \in \beta$ and $x \in \gamma$
    \end{claimproof}
    Then by previous lemma, either 
    \begin{itemize}
        \item $\gamma = \alpha$ and $\gamma = \beta$: done 
        \item $\gamma = \alpha$ and $\gamma \in \beta$: $\alpha\in \beta$
        \item $\gamma \in \alpha$ and $\gamma = \beta$: $\beta\in \alpha$.
        \item $\gamma \in \alpha$ and $\gamma \in \beta$ we have $\gamma \in \alpha\cap \beta = \gamma$ which is impossible
    \end{itemize}
}
\prop{}{$\theta_n$ is well-ordered by $\in$}{
    We need to show that if $\alpha\in On$ then $\delta_{\alpha}(On)$ is a set that is well ordered by $\in$.

    $\delta_{\alpha}(On) = \{\beta \in On : \beta\in \alpha\} = \alpha$
    Because $\alpha$ is a well-ordered set, and by the lemma TODO $On$ is linearly ordered by $\in$
}
\lemma{}{
    $On$ is a proper class
}{
    Suppose $On$ would be a set $z$.\\
    $On$ is well ordered by $\in$
    $On$ is transitive: $y\in x \in On$ then $y\in On$ by lemma
    so $On$ would be an ordinal and $On\in On$ which is not possible by Lemma TODO.
}
\note{}{$\alpha,\beta\in On$ then $\alpha\subseteq \beta$ iff $\alpha\in \beta$ or $\alpha = \beta$
    $\to$: If $\alpha\neq \beta$. Then by linear ordering of $\in$ 
}