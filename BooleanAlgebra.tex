\chapter{Boolean Algebra}
From \cite{krivine1998théorie}?
Our language in this chapter will be $\mathcal{L} = \{0,1,+,\cdot,\overline{\phantom{x}}\}$, where $+,\cdot$ are binary operations and $\overline{\phantom{x}}$ is a unary operation. We will sometimes write $x+y\cdot z$ instead of $x + (y\cdot z)$ for more clarity.
The axioms for boolean algebras are
\begin{equation*}
    \begin{matrix*}[l]
        \forall x,y,z \: \bigl(x+(y+z) = (x+y)+z \land x\cdot (y\cdot z) = (x \cdot y ) \cdot z\bigr) & \text{(Associativity $+,\cdot$)}\\[3pt]
        \forall x,y \: \bigl(x+y=y+x \land x\cdot y = y\cdot x\bigr) & \text{(Commutativity of $+,\cdot$) }\\[3pt]
        \forall x \: \bigl(x+x = x \land x\cdot x = x\bigr)& \text{(Idempotence) }\\[3pt]
        \forall x,y,z \: \bigl(x\cdot (y+z) = x\cdot y + x\cdot z \land x+(y\cdot z) = (x+y)\cdot (x+z)\bigr) & \text{(Distributivity) }\\[3pt]
        \forall x,y \: \bigl(x\cdot (x+ y) = x  = x+ (x\cdot y) \bigr)& \text{(Absorbtion)}\\[3pt]
        \forall x,y \: \bigl(\overline{x+y} = \overline{x}\cdot \overline{y}\land \overline{x\cdot y} = \overline{x}+ \overline{y}\bigr)& \text{(De Morgan's Laws) }\\[3pt]
        \forall x \: \bigl( x+0 = x \land x\cdot 0 = 0 \bigr)& \text{(Laws of $0$)}\\[3pt]
        \forall x \: \bigl( x+1 = 1\land x\cdot 1 = x \bigr)& \text{(Laws of $1$)}\\[3pt]
        \forall x \: \bigl(x + \overline{x} = 1 \land x\cdot \overline{x} = 0 \land \overline{\overline{x}} = x\bigr)& \text{(Laws of $\overline{\phantom{x}}$)}\\
    \end{matrix*}
\end{equation*}
$BA$ denotes the $\mathcal{L}$-theory of boolean algebras, i.e. the deductive closure of the axioms above.\\%TODO is this needed? 2025-02-07
Every model of $BA$ is called a boolean algebra.

\note{}{Every boolean algebra $\mathcal{B}$ can be partially ordered by 
$$x\leq y \quad\iif\quad x+y = y$$ 
It is easy to see that $\leq$ is reflexive, antisymmetric and transitive. In this ordering the smallest set is $0$ and the largest one is $1$.
In this notion the supremum and infimum of two elements are equal to
\[\sup\{x,y\} = x+y, \quad \inf\{x,y\} = x\cdot y\quad\text{(see Exercises)}\]
That makes every boolean algebra to a complemented, distributive \imp{lattice}, see appendix.
}
\bsp{}{$X\neq \varnothing$ and $S\subseteq \mathcal{P}(X)$ such that 
    \begin{itemize}
        \item $\varnothing\in S$
        \item $X\in S$ 
        \item $S$ is closed under (finite) intersections and unions and complements.
    \end{itemize}
    Then $(S; \varnothing,X,\cup,\cap,\overline{\phantom{x}})$ is called a boolean algebra of sets and $\leq$ corresponds to $\subseteq$.
}
\note{}{Conversly, it is true that every boolean algebra is isomorphic to a boolean algebra of sets. In fact it can be embedded in a boolean algebra of all subsets of a set. The next section yields the choice of the set.}
\section{Stone Representation Theorem}
\defin{(Ultra-)Filter}{Suppose $\mathcal{B}\models BA$. A non-empty subset $F\subseteq B$ is called a filter on $\mathcal{B}$, if
\begin{itemize}[leftmargin=1.5cm]
    \item[(F1)] $0\notin F$
    \item[(F2)] $\forall a \forall b\: (a\in F \land b\in F) \to a\cdot b\in F$
    \item[(F3)] $\forall a \forall b\: (a\in F \land a \leq b) \to b\in F$
\end{itemize}
An \imp{Ultrafilter}of $\mathcal{B}$ is a filter $\mathcal{F}$ that satisfies
\begin{itemize}[leftmargin=1.5cm]
    \item[(UF)]$\forall a\in B a\in \mathcal{F} \lor \overline{a}\in \mathcal{F}$.
\end{itemize}
}
The congruence with filters defined in functional analisis seems unambiguous, 
however the one key difference
here is that a filter on $\mathcal{B}$ is a subset of $B$. If $\mathcal{B}$ is a boolean algebra of 
the set $X$ then a filter in our definition would coincide with a filter on the maximal element $X$.

Recall: Let $X\neq \varnothing$, $\tau \subseteq \mathcal{P}(X)$ then $(X,\tau)$ is called a topological space, if 
    \begin{enumerate}[leftmargin=1.5cm]
        \item[(T1)] $\varnothing, X\in \tau$
        \item[(T2)] $\forall I \forall (\sigma_i)_i\in \tau^I: \bigcup_{i\in I}(\sigma_i)\in \tau$
        \item[(T3)] $\forall n\in \NN \forall (\sigma_i)_i\in \tau^\{1,\dots n\}: \bigcup_{1\leq i\leq n}(\sigma_i)\in \tau$
    \end{enumerate}
    And $\tau'\subseteq \mathcal{P}(X)$ is called a base for the topology $\tau$ on $X$,
    if every open set in $\tau$ is the union of sets in $\tau'$.
\defin{Stone space}{
    A stone space is a non-empty topological space $(X,\tau)$ which 
    \begin{enumerate}
        \item has a basis $\sigma \subseteq \tau$ (for the topology) of clopen sets
        \[\begin{matrix*}[l]
            \forall V \in \sigma \: X\setminus V \in \tau & \text{(clopen)}\\[3px]
            \forall U \in \tau \forall x\in U \exists V\in \sigma \: x\in V\subseteq U & \text{(basis)}
        \end{matrix*}\]
        \item is compact (every open cover contains a finite subcover)
        \item and hausdorff (every two distinct points can be seperated by open sets) 
        \[\forall x, y\in X\: x\neq y \to \exists U_x,U_y\in \tau \: x\in U_x \land y\in U_y\land U_x\cap U_y= \varnothing\]
    \end{enumerate}

    Let $\mathcal{B}$ be a boolean algebra. We define 
    $S(\mathcal{B}) \defeq\{\text{the set of all ultrafilters on $\mathcal{B}$}\}$
    and call it the stonespace of $\mathcal{B}$.\outernote{$S(\mathcal{B})$}
}
The clarification that $S(\mathcal{B})$ is indeed a stonespace in the sense of the above definition will be given by the Stone-Representation theorem.

\note{}{
    Let $\mathcal{B}\models BA$ then 
    \begin{enumerate}
        \item for a filter $F$ on $\mathcal{B}$ it holds $1\in F$ (by $F\neq \varnothing$ and (F3))
        \item Any subset $F'\subseteq B$ satisfying (F3) satisfies (F1) iff $F'\neq B$.
        \item $\langle a\rangle \defeq \{x\in S(\mathcal{B}): a\in x\}$ where $a$ runs through $B$ forms a basis for a topology on the stonespace of $\mathcal{B}$
    \end{enumerate}
}
Back on $\langle a\rangle = \{x\in S(\mathcal{B}): a\in x\}$. It holds
\begin{itemize}
    \item $\varnothing = \langle 0\rangle $
    \item $S(\mathcal{B}) = \langle 1\rangle $
    \item $\langle a\rangle \cap \langle b\rangle  = \langle a\cdot b\rangle $
\end{itemize}
Every filter on $B$ can be extended to an ultrafilter on $\mathcal{B}$ (Zorn's Lemma).

In fact, suppose $\mathcal{F}\subseteq B$ has (FIP). i.e. for any $n\in\NN$ $\forall f_1,\dots f_n\in \mathcal{F}$ it is $f_1\cdot \dots \cdot f_n \neq 0$. Then $\mathcal{F}$ extends to an ultrafilter on $\mathcal{B}$.
(see Exercises)

\thm{Stone Representation Theorem}{
    \begin{enumerate}[label=(\roman*)]
        \item If $\mathcal{B}\models BA$, then $S(\mathcal{B})$ is a Stone-space
        \item If $\mathcal{S}$ is a Stone space then the clopen subsets of $\mathcal{S}$ form a boolean algebra denoted by $B(\mathcal{S})$.
        \item Every boolean algebra $\mathcal{B}$ is isomorphic to the boolean algebra $B(S(\mathcal{B}))$ with $a\mapsto \langle a\rangle$. Hence $\mathcal{B}$ is isomorphic to a subalgebra of the boolean algebra $\mathcal{P}(S(\mathcal{B}))$ of sets
        \item Every stonespace $\mathcal{S}$ is homeomorphic to the stonespace $S(B(\mathcal{S}))$
        $$x\mapsto \{a\in S(\mathcal{B}) : x\in a\}$$
    \end{enumerate}
}{
    \begin{enumerate}[label=(\roman*)]
        \item We have the base for a topology $\langle a\rangle= \{x\in S(\mathcal{B}): a\in x\}$. 
        $\langle a\rangle$ is clopen : It is clearly open.
        $$\langle a\rangle^c = \overline{\langle a\rangle} = \{x\in S(\mathcal{B}): a\notin x\}=\{x\in S(\mathcal{B}): \overline{a}\in x\}=  \langle \overline{a}\rangle$$
        hausdorff: 
        Let $x,y\in S(\mathcal{B})$ such that $x\neq y$.
        then $\exists a\in B a\in x\land \overline{a}\in y$ 

        then $x\in \langle a\rangle, y\in \langle \overline{a}\rangle$.

        compact:
        we use the fact: If $X$ is a topological space then 

        Every open cover of $X$ contains a finite subcover iff 
        any family of closed sets which has (FIP), has non-empty intersection.

        Supposed $(F_i)_{i\in I}$ a family of closed subsets of $S(\mathcal{B})$ such that $I\neq  \varnothing$ and $\bigcap_{i\in I}F_i = \varnothing$. we want to show that there is a finite intersection which is empty, i.e. $\exists i_1, \dots i_k\in I \bigcap_{1\leq m\leq k}F_{i_m}$
        such that $(F_i)_{i\in I}$ can not have (FIP).

        WMA that $F_i = \langle a_i \rangle$ for some $a_i\in B$.
        Assume $\bigcap_{i\in I}\langle a_i\rangle = \varnothing$.
       
        If $a_i\cdot \dots \cdot_{i_k} \neq 0$ For all $\{i_1,\dots i_k\}\subseteq I$ Then $\{a_i : i\in I\}\subseteq B$ has (FIP), so it extends to an ultrafilter on $\mathcal{B}$ (using Zorns lemma).
        \begin{multicols}{2}
            $X$ set, a filter / ultrafilter on $X$ is some $\mathcal{F}\subseteq \mathcal{P}(X)$ s.th.
            If $\mathcal{F}$ has (FIP) then $\mathcal{F}$ extends to UF
        
            $\mathcal{B}$ boolean algebra, then a filter / ultrafilter of $\mathcal{B}$ is a subset $\mathcal{F}\subseteq B$ s.th.
            If $\mathcal{F}\subseteq B$ has (FIP), then $\mathcal{F}$ extends to UF $\mathcal{U}$ of $\mathcal{B}$
            i.e. $\forall i\in I a_i\in \mathcal{U}$ so $\mathcal{U}\in \bigcap_{i\in I}F_i $ but we assumed $\bigcap_{i\in I}\langle a_i\rangle = \varnothing$
        \end{multicols}
        So there exists some $i_1\dots i_k$ such that $\bigcap_{1\leq j\leq k}\langle a_{i_j}\rangle =\varnothing$ and 
    \end{enumerate}
}
\defin{Atomic, Atomless}{An atom is an element of a boolean algebra such that $a\neq 0$ and there is no element in the boolean algebra that is strictly inbetween $0$ and $a$.
\[\forall y (0\leq y\leq a\to (y=0\lor y=a))\]
A boolean algebra $\mathcal{B}$ is called atomic, if 
\[\forall a (a\neq 0 \to \exists y (y\leq a \land y \text{ is atomic}))\]
A boolean algebra is atomless if it contains no atoms
}
\note{}{There exists boolean algebras that are neither atomic nor atomless.}
\note{}{Axioms for atomic boolean algebras: add
    \[\forall a (a\neq 0 \to \exists y (y\leq x \land y\neq 0 \land \forall z(0\leq z\leq y\to (z=0\lor z=y))))\]

    Axioms for atomless: add
    \[\forall y y\neq 0 \to \exists z (0<z<y)\]
}


\section{Lindenbaum-Tarski Algebras}
Let $\mathcal{L}$ be a first order Language, $\mathcal{L}_0$ the set of all $\mathcal{L}$-sentences and $\sim$ the logical equivalence relation.
On the quotient set $\mathcal{L}_0/_\sim$ we can define $\land,\lor,\lnot$ py passing to representatives. This is well defined and does not depend on the choice of representatives.

\defin{Lindenbaum-Tarski algebra}{With the above notation
    $$B_L = (\mathcal{L}_0/_\sim;\bot/_\sim,\top/_\sim,\lor,\land,\lnot)$$ forms then a boolean algebra.
(it is called Lindenbaum-Tarski algebra for $\mathcal{L}$)

Note that $\bot$ is logically equivalent to $\exists x x\neq x$
and $\top$ is logically equivalent to $\forall x x=x$

The construction can be extended to equivalence modulo some $\mathcal{L}$-theory $T$ (or $T\subseteq \mathcal{L}_0$)
Let $n\in \NN$ and $\underline{x} = (x_1,\dots x_n)$
For $\varphi,\psi\in \mathcal{L}_{\underline{x}}$, where $\mathcal{L}_{\underline{x}}$ are the $\mathcal{L}$-formulas with free variables among $\{x_1,\dots x_n\}$

Define $\varphi\leq_T\psi$ iff $T\models \forall \underline{x} (\varphi\to \psi)$

We can define $T$-equivalence: $\varphi \sim_T \psi$ iff $\varphi\leq_T \psi$ and $\psi\leq_T \varphi$

$$\mathcal{B}_n = ({\mathcal{L}_{\underline{x}}}/_{\sim_T}; \bot/_{\sim_T},\top/_{\sim_T},\land,\lor,\lnot)$$
Is then again a boolean algebra, whose isomorphism type depends only on $T$ and it is called the $n$-th Lindenbaum-Tarski-algebra of $T$.
In the case we take the $0$-th L-T algebra of $\varnothing$ $B_L= B_0(\varnothing^{\models0})$

}

\defin{Recap}{
    \begin{itemize}
        \item The deductive closure of a set of sentences $\Sigma$ is $\{\varphi : \sigma \models \varphi\}$
        \item A contradiction is any sentence of the form $\varphi \land \lnot \varphi$
        \item A set of sentences is consistent, if its deductive closure does not contain a contradiction.
        \item A $\mathcal{L}$-theory is a set of sentences that is consistent and deductively closed.
    \end{itemize}
}
The question we know ask ourselves is: what is the stone space of a Lindenbaum-Tarski algebra?
\note{}{
    \begin{itemize}
        \item $\mathcal{L}$-theories are indeed exactly the filters of $\mathcal{B}_L$
        \item complete $\mathcal{L}$-theories are exactly the ultrafilters of $\mathcal{B}_L$
    \end{itemize}
}
Let $S_L$ equal the set of all complete $\mathcal{L}$-theories then our compactness theorem

\[\Gamma \models \varphi \implies \exists \Gamma'\subseteq \Gamma \text{ finite } \Gamma\models \varphi\]
is equivalent to 
\thm{Compactness Theorem *}{
    $S_L$ with stone topology is compact.
}{}
Two things we would like to show:
\begin{itemize}
    \item Compactness theorem $\implies$ Compactness theorem *\\
    By showing that $S_L=S(B_\mathcal{L})$
    \begin{claimproof}
        \begin{itemize}
            \item[$\subseteq$] Let $T$ be complete $\mathcal{L}$-Theory. by consistency and abuse of notation $0\notin T$. and $T$ is closed under conjunction.
            So for all $\varphi,\psi\in T$ we have $\varphi\land\psi\in T$.
            $\varphi\in T$ and $\varphi\leq \psi$ then $\models \varphi\to \psi$ so $\varphi\models \psi$ and $\psi\in T$ bc. $T$ is deductively closed.
            \item[$\supseteq$] Let $x\in S(\mathcal{B}_\mathcal{L})$ 
            completeness: By maximality of $x$, $\forall \varphi$ either $\varphi/_\sim \in x$ or $\lnot \varphi/_\sim\in x$.

            deductively cloesdness: $x\models \gamma$ then by compactness Theorem $\exists x'\subseteq x x'\models \gamma$ and $x'\in x$, so by if $x'\in x$ and $x'\leq \gamma$, then $\gamma\in x$ hence deductive closure

            consistency: $0\notin x$ and $x$ is deductively closed.
        \end{itemize}
    \end{claimproof}
    \item Compactness theorem * $\implies$ compactness theorem
    $\Gamma = \{\gamma_i : i\in I\}$ set of $\mathcal{L}$-sentences.
    We want: $\Gamma\models \varphi$ then $\exists \Gamma'\subseteq \Gamma \text{ finite } \Gamma'\models \varphi$

    \begin{claimproof}
        Suppose, by contradiction that it is not the case.

        $\forall I'\stackrel{\text{ fin }}{\subseteq} I \{\gamma_i : i\in I'\}\cup \{\lnot \varphi\}$ is consistent.

        \[\implies \forall I'\stackrel{\text{ fin }}{\subseteq} I \bigcap_{i\in I'} \langle\varphi_i\rangle\cap \langle\lnot \varphi\rangle \neq \varnothing\]

        \[\{\langle\varphi_i\rangle : i\in I\}\cup \{ \langle\lnot \varphi\rangle \}\]
        is a collection of closed sets with (FIP).
        By using compactness of $S_L$ with stone topology,
        \[\bigcap_{i\in I} \langle\varphi_i\rangle\cap \langle\lnot \varphi\rangle \neq \varnothing\]
        hence $\Gamma \not\models \varphi$.
    \end{claimproof}

\end{itemize}

























