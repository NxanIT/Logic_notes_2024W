\chapter{Model Theory}
The sections \ref{mt:sec:LST} to \ref{MT:sec:NSA} are sourced from \cite[chapter~2]{EndertonHerbertB2001AMIt} and and the theory of o-minimality (from \ref{MT:sec:o-min} onwards) can be found in \cite{van1998tame}.
\section{LST-Theorem}\label{mt:sec:LST}
LST stands for Löwenheim-Skolem-Tarski and is the combination of the ``upward Löwenheim-Skolem theorem'' with the ``downward Löwenheim-Skolem theorem''.
\thm{LST-Theorem}{Let $\Gamma$ be a set of $\mathcal{L}$-formulas. $|\mathcal{L}| = \lambda$ and lets assume 
    $\Gamma$ is satisfiable in some infinite structure.\\
    Then for every cardinal $\kappa\geq \lambda$, $\Gamma$ is satisfiable in a structure of cardinality $\kappa$.
}{ %2024-12-13: TODO check proof
    add $\kappa$ many new constants to the language $\mathcal{L}$.

    $\mathcal{L}' = \mathcal{L}\cup \{c_\alpha : \alpha < \kappa\}$

    $\Sigma = \{c_\alpha \neq c_\beta : \alpha\leq \beta, \ \alpha,\beta\leq \kappa\}$

    Then $\Gamma\cup \Sigma$ is finitly satisfiable in $\mathcal{L}'$.
    This is because $\Gamma$ is satisfiable in some infinite structure. By compactness $\Gamma\cup \Sigma$ is satisfiable.
    We have $\mathcal{A}\models \Gamma \cup \Sigma$ then $|\mathcal{A}|\geq \kappa$.

    By the proof of completeness theorem, $\Gamma\cup \Sigma$ has a model of size $\leq \kappa$.
    Hence it is exactly of size $\kappa$. Take the reduct of $\mathcal{A}$ to the language $\mathcal{L}$.%TODO: define reduct
}
\bsp{}{The language of ZFC $\mathcal{L} = \{\in\}$ is countable, so Löwenheim-Skolem guaranties that ZFC has a countable model.
But ZFC knows that there are uncountable sets (see Cantors Theorem \ref{5:Thm:Cantor}).
This is called skolems paradox.
explanation: some bijections are missing
}
\bsp{}{\label{ComplNotImplyCath}\begin{enumerate}
    \item $\overline{\RR}$ real field. $\Thm(\overline{\RR})$ has a countable model. $\RR_\text{alg}$
    \item $\mathcal{N} = (\NN,0,S,+,\cdot)$\\
    Claim: there exists a countable structure $\mathcal{M}$ such that $\mathcal{N}\equiv \mathcal{M}$ but $\mathcal{N}\ncong \mathcal{M}$
    One way is to add new constant c to language
    $\Sigma = \{0<c, S0<c,\dots\}$ is fin satisfiable. So $\Sigma\cup Th (\mathcal{N})$ is fin satisfiable by compactness it is satisfiable

    Take the reduct to original language. $\mathcal{M}$. and $\mathcal{M}$ not isomorphic to $\mathcal{N}$, bc
    A bijection of $M\to \NN$ would have to map $c$ somewhere but for every $S^k0<c$ for every $k$ wont be preserved by any map.
\end{enumerate}
}
\section{Theories and completeness}
\defin{Theory}{A theory $T$ is a set of sentences that is closed under logical implication.
\[T\models \sigma \implies \sigma \in T\]
}
\note{}{If $\mathcal{L}$ is a language. Then 
\begin{itemize}
    \item there is a smallest $\mathcal{L}$-theory. The set of all valid $\mathcal{L}$-sentences.
    \item and also a largest $\mathcal{L}$-theory. The set of all $\mathcal{L}$-sentences.
\end{itemize}
}


\defin{Theory of structures}{Let $\mathcal{K}$ some class of $\mathcal{L}$- structures. Then 
    $$\Th(\mathcal{K}) = \{\sigma : \sigma \text{ $\mathcal{L}$-sentence and for every $K \in \mathcal{K}$ } \sigma\in \Th(K)\}$$
}
\note{}{$\Th(\mathcal{K})$ is a theory. 
\[\text{if }\Th(\mathcal{K})\models \sigma \text{ then }  \sigma\in \Th(\mathcal{K}) \]
}

\bsp{}{

\begin{itemize}
    \item $\mathcal{L} = \{0,1,+,\cdot,-\}$ $\mathcal{F}$ the class of fields then $\Th(\mathcal{F})$ is the set of sentences truein every field.
\end{itemize}
}
%2024-12-13: TODO edit below
Recall that $\Mod(\Sigma)$ is the class of all models of $\Sigma$.
$\Th(\Mod \Sigma)$ might not be the set $\Sigma$ but it is the set of all 
sentences true in all models of $\Sigma$.
Which is the set of all sentences that are logically implied by $\Sigma$ 

Or in other words: The set of all consequences of $\Sigma$
\defin{$C_n$}{$ C_n(\Sigma)\defeq \Th(\Mod \Sigma)$
}
\note{}{$\Sigma$ is a theory iff $C_n(\Sigma) = \Sigma$}

\defin{}{We say that a theory $T$ is complete, if for every sentence 
$\sigma$ either $\sigma\in T$ or $\lnot \sigma\in T$.}

\bsp{}{
    $\mathcal{A}$ a $\mathcal{L}$-structure, then $\Th(\mathcal{A})$ is complete.
}

\note{}{$\Th(\mathcal{K})$ is complete, iff any $K_1,K_2\in \mathcal{K}$ are elementarily equivalent.

A theory $T$ is complete iff any to models are elementarily equivalent.
}

\bsp{}{
    \begin{itemize}
        \item The theory of fields is not complete.
        \item The theory of algebraically closed fields of characteristic $0$ is complete 
        (That is non-trivial)
    \end{itemize}
}
\defin{axiomatizability}{
    \begin{itemize}
        \item A theory $T$ is finitely axiomatizable if there is a sentence 
        $\sigma$ such that $C_n(\sigma) = T$.
        \item A theory $T$ is axiomatizable, if there is a decidable set 
        $\Sigma$ such that $C_n(\Sigma) = T$.
    \end{itemize}
}
\bsp{}{
    \begin{itemize}
        \item The theory of fields (common theory of all fields) is finitely axiomatizable.
        \item Theo theory of fields of characteristic $0$ is axiomatizable. $\Psi\cup \{1+1\neq 0, 1+1+1\neq 0,\dots\}$ 
        It is however not finitely axiomatizable. If 
        $\Psi_0\subseteq \Psi\cup \{1+1\neq 0, 1+1+1\neq 0,\dots\}$ 
        finite, then $\Psi_0$ has a model of characteristic $p$ for some sufficiently large $p$.
    \end{itemize}
}
\thm{}{If $C_n(\Sigma)$ is finitlely axiomatizable then there exists a finite subset           
    $\Sigma_0\subseteq \Sigma$ such that $C_n(\Sigma_0) = C_n(\Sigma)$}{
    Suppose $C_n(\Sigma)$ is finitely axiomatizable. So $C_n(\sigma) = C_n(\Sigma)$. 
    Then there is a finite subset $\Sigma_0\subseteq \Sigma$ such that $\Sigma_0 \models \sigma$.
    And we get $C_n(\Sigma_0) = C_n(\Sigma)$
}
\defin{}{A theory $T$ is $\aleph_0$-categorical, if any two infinite countable models of 
    $T$ are isomorphic.
    Futhermore for some infinite cardinal $\kappa$ a theory $T$ is called $\kappa$-categorical, if every two models of cardinality $\kappa$ are isomorphic.
}
\thm{Los-Vaught test}{For a theory $T$ in a countable language with only infinite models it holds\\
    If $T$ is $\kappa$-categorical for some infinite cardinality $\kappa$ then $T$ is complete.
}{
    Let $T$ be $\kappa$-categorical.
    Want: If $\mathcal{A},\mathcal{B}\models T$ then $\mathcal{A}\equiv \mathcal{B}$.
    Note: both $\mathcal{A}$ and $\mathcal{B}$ are infinite.
    By LST there exists structures $\mathcal{A}'$ and $\mathcal{B}'$ with $\mathcal{A}\equiv \mathcal{A}'$ and $\mathcal{B} \equiv \mathcal{B}'$ and $|\mathcal{A}'|, |\mathcal{B}'| = \kappa$.
    By $\kappa$-categorical we have $\mathcal{A}' \cong \mathcal{B}'$ so $\mathcal{A}\equiv \mathcal{B}$
}
\note{}{completness does not imply categorical.
\begin{itemize}
    \item The theory of natural numbers is not $\aleph_0$-categorical. See Example \ref{ComplNotImplyCath}
    \item RCF not $\kappa$-categorical for all infinite cardinalities $\kappa$
    Not $\aleph_0$ categorical real clo of $\mathbb{Q}(\pi)$, real closure of $\mathbb{Q}$
    not uncountable categorical $\overline{\mathbb{R}}$, $\overline{\mathbb{R}(\varepsilon)}, 0<\varepsilon <\frac{1}{n}$ for every $n\in \NN$.
\end{itemize}
}
\section{Theory of algebraic closed fields}
\thm{}{The theory of algebraic closed fields of characteristic $p$ ACF$_p$, where $p$ is either $0$ or prime is complete.\cite[Theorem 26J, p.158]{EndertonHerbertB2001AMIt}}{Note that we have a
\begin{itemize}
    \item  countable language
    \item with no finite models
\end{itemize}
Let $\mathcal{K}_1,\mathcal{K}_2\models \text{ACF}_p$ such that $|K_1| = |K_2|=\kappa$ uncountable.
$F_1$ prime field of $\mathcal{K}_1$, $F_2$ prime field of $\mathcal{K}_2$.

Note $F_1, F_2$ are determined by $p$ if $p=0$ then $F_1 = F_2 = \mathbb{Q}$ and if $p$ prime then $F_1 = F_2 = \mathbb{F}_p$

Define $F \defeq F_1 = F_2$.
$B_1$ trancendence base of $\mathcal{K}_1$ over $F$ 
$B_2$ trancendence base of $\mathcal{K}_2$ over $F$ 
\begin{itemize}
    \item $B$ is trancendence base of $K$ over $F$ if $B$ is a $\subseteq$-maximal subset of $K$ which is algebraically closed then 
    \item $B\subseteq K$ is algebraically TODO
\end{itemize}
$F(B_1)$, $F(B_2)$ subfields of $\mathcal{K}_1,\mathcal{K}_2$
\begin{itemize}
    \item alg cl $F(B_1) = \mathcal{K}_1$
    \item alg cl $F(B_2) = \mathcal{K}_2$
\end{itemize}
Fact: Let $F$ subfield of $K$. if $F$ is countable and $K$ uncountable, then any transe basis $B$ of $K$ oer $F$ is of cardinality $|K|$, hence uncountable.

Steinitz: Two ACF are isomorphic iff they have the same characteristic and there trancendence spaces have the same cardinality.
}
\subsubsection{Lefschetz Principle}%2024-12-13: TODO source needed
\prop{Lefschetz Principle}{Let $\mathcal{C} = (\mathbb{C},0,1,+,\cdot,-)$ For a sentence in the language of $\mathcal{C}$
    Then the following are equivalent:
    \begin{itemize}
        \item $\mathcal{C}\models \sigma$
        \item $\mathcal{A}\models \sigma $ for every $\mathcal{A}\models $ ACF$_0$
        \item ACF$_0$ $\models \sigma$
        \item for all sufficiently large primes $p$ ACF$_p$ $\models \sigma$
        \item For infinitely many primes $p$ ACF$_p$ $models \sigma$
    \end{itemize}}{ 
    Sketch:
    \begin{itemize}
        \item[(a), (b), (c)] are equivalent by completeness of ACF$_0$
        \item[(c)$\implies$ (d)] ACF$_0$ $\models \sigma$ 
        so there is $T_0\subseteq \text{ACF}_0$ such that $T_0\models \sigma$ therefore there exists a sufficiently large prime $p$ such that $\text{ACF}_p\models \sigma$. 
        \item[(d) $\implies$ (e)]TODO
        \item[(e) $\implies$ (c)] If $\text{ACF}_0\vDash \sigma$ than $\text{ACF}_0\models \sigma$
    \end{itemize}
}
Example of the Lefschetz Principle:
\prop{Ax–Grothendieck}{\footnote{Alexander Grothendieck}
    Let $f:\mathbb{C}^n \to \mathbb{C}^n$ be a polynomial map. If $f$ is injective, then $f$ is surjective.
}{
    Our language is $\mathcal{L} = \{0,1,+,-,\cdot\}$. Note that there is an $\mathcal{L}$-sentence $\Phi_d$ such that a Field $F$ 
    
    $F\models \Phi_d$ iff for every polynomial map $f:F^n\to F^n$ whose TODO coord. function is of degree at most $d$, if $f$ is injective then $f$ is surjective.

    By Lefschetz principle it is enough to show for sufficiently large 
    primes $p$, $\text{ACF}_p\models \Phi_d$ for all $d\in \NN$.
    Since $\text{ACF}_p$ is complete, it is enough to show that every injective polynomial map $f:K^n\to K^n$ is surjective, where $K = TODO$
    Let $f:K^n\to K^n$ be a polynomial map.

    Then there is a finite subfield $K_0$ of $K$ such that all coefficients of $f$ come from $K_0$.
    Let $y\in K^n$. Then there is a finite subfield $K_1$ of $K$ such that $y\in K_1$ and $K_0\subseteq K_1 \subseteq K$. Since $f:K^n_1\to K^n_1$ is injective and $K_1$ finite, $f|_{K_1}$ is surjective onto $K_1$. So there is $x\in K_1^n$ such that $f(x)=y$.
}
\note{}{Later, a purely geometric proof was found by Borel.}

Another use of \L oś-Vaught

\prop{}{\[(\mathbb{Q},<_\mathbb{Q}) \equiv (\RR, <_\RR)\]}{%2024-12-13: TODO from here on \prop might actually be theorems, check
    $\mathcal{L} = \{<\}$ and note that both $(\mathbb{Q},<_\mathbb{Q}), (\RR, <_\RR)$ are DLO without endpoints, i.e. they satisfy the following axioms
    \begin{enumerate}
        \item $\forall x \forall y (x<y\lor x=y\lor y<x)$
        \item $\forall x \forall y (x<y\to \lnot(y<x))$
        \item $\forall x \forall y \forall z ((x<y\land y<z)\to x<z)$
        \item $\forall x \forall y (x<y\to \exists z(x<z\land z<y))$
        \item $\forall x \exists y \exists z (y<x\land x<z)$
    \end{enumerate}
    TODO
}

\section{Nonstandard Analysis}\label{MT:sec:NSA}
%TODO historical background
\begin{enumerate}
    \item Language $\mathcal{L}$: $=$, $\forall$ ranging over $\RR$,
    \begin{itemize}
        \item $P_R$ TODO
    \end{itemize}
    \item standard structure for $\mathcal{L}$: $\mathcal{R}$ with universe $\RR$, $c_r^\mathcal{R} = r$, $P_R^\mathcal{R} = R$, $f^\mathcal{R}_F = F$.
    \item Nonstandard structure for $\mathcal{L}$: $\mathcal{R}^*$, which is constructed using the compactness theorem 
    \[\Gamma \defeq \Th(\mathcal{R}) \cup \{c_r P_< v_1 : r\in \RR\}\]
    Compactness theorem $\implies$ there exists a $\mathcal{L}$-structure $\mathcal{R}^*$ with $\mathcal{R}^*\models \Gamma [(v_1 | a)]$ for some $a\in R^*$. We have $\mathcal{R}\equiv \mathcal{R}^*$.
    Moreover, $h: \RR \to R^*$ defined by $r\mapsto c_r^*$ is an isomorphism of $\mathcal{R}$ into $\mathcal{R}^*$
    \begin{itemize}
        \item $h$ is injective:
        \item TODO
    \end{itemize}
\end{enumerate}
\note{}{WMA $\mathcal{R}$ substructure of $\mathcal{R}^*$ (se PS)}
Notation: We will write $\!^*B$ instead of $P_B^{\mathcal{R}^*}$.
\bsp{}{what is $\!^*\RR$?
We have that $\mathcal{R}\models \forall x P_\RR$, hence $\mathcal{R}^*\models \forall x \!^*\RR$, so $\!^*\RR = R^* = \text{universe of } \mathcal{R}^*$.
\note{}{
    Let $F$ be an $n$-ary operator on $\RR$. Then $F$ is the restriction of $\!^*F$ to $\RR$. $\!^*c_r = r$.
}
Idea: If we want to show that $\!^*R$ or $\!^*F$ has certain property, then we show
\begin{itemize}
    \item $R$ or $F$ have that property.
    \item property can be expressed in $\mathcal{L}$
\end{itemize}
}
TODO

%26.11.2024
$\mathcal{R}^*\supseteq \mathcal{R}$ such that $\mathcal{R}^*\equiv \mathcal{R}$.
$\mathcal{F} = \{x\in \mathcal{R}^* : \exists r\in \RR \: ^*|x|^* \leq r\}$
$\mathcal{I} = \{x\in \mathcal{R}^* : \forall r\in \RR \: ^*|x|^* < r\}$
\prop{}{
    \begin{enumerate}
        \item $\mathcal{F}$ is a subring of $\mathcal{R}^*$
        \item $\mathcal{I}$ is an ideal in $\mathcal{R}*$
    \end{enumerate}
}{
    \begin{enumerate}
        \item Let $x,y\in \mathcal{F}$ then there exists $a,b\in \RR^{>0}$ such that $^*|x|^* \leq a$ and $^*|y|^* \leq b$. then \[
        \begin{aligned}
            ^*|x \pm^* y|^* \leq ^*|x|^* + ^*|y|^* \leq a+b\in \RR^{>0}\\
            ^*|x \cdot^* y|^* = ^*|x|^*\ ^*\cdot ^*|y|^* \leq a\cdot b\in \RR^{>0}
        \end{aligned}
        \]
        \item $x,y\in \mathcal{I}$ then $\forall a\in \RR^{>0}$ we have $|x|<\frac{a}{2}$
        Then $$|x\pm y| \leq \frac{a}{2} + \frac{a}{2} = a$$
        Let $z$ be finite then $|z|<b\in \RR^{>0}$
        Let $a\in \RR^{>0}$ then $|x|< \frac{a}{b}$ so 
        \[|xz| < \frac{a}{b}b = a \] 
    \end{enumerate}
}
\defin{infinitely close}{$x,y$ are called to be infinitely close ($x\simeq y$),\outernote{$\simeq$} if $y-x\in \mathcal{I}$}
\prop{}{
    \begin{enumerate}
        \item $\simeq$ is an equivalence relation
        \item $\simeq$ is congruent with $^*+,^*\cdot,^*-$
    \end{enumerate}
}{}
\lemma{}{Suppose $\lnot x\simeq y$ and at least one of $x,y$ is finite then there exists $q\in \RR$ 
such that $q$ is betweeen $x$ and $y$}{
    $y-x\notin \mathcal{I}$, wlog. $x<y$ then there exists $b\in \RR$ such that $0<b<y-x$ 
    and by the archimedian property there is $m\in \NN^{>0}$ such that $x<mb$.
    Let $m$ be the smallest such. i.e. $(m-1)b\leq x<mb$.
    And $mb<y$.
}
\prop{}{
    For every $x\in \mathcal{F}$ therre is exactly one $r\in \RR$ such that $x\sim r$
}{
    Let $S \defeq \{r\in \RR : r<x\}$. $S$ is bounded in $\RR$ because $|x|<r_0$ for 
    some $r_0\in \RR^{>0}$.
    Then $r\defeq \sup S$. Claim: $r\simeq x$. Lets assume by contradiction that this is not the case. 
    By the previous lemma, there is $q\in \RR$ such that 
    $r<q<x$ or $x<q<r$. but neither of this things can happen.
    \begin{itemize}
        \item $r<q<x$ is contradiction to $r$ not being an upper bound.
        \item $x<q<r$ is contradiction to $r$ is not the least upper bound.
    \end{itemize}
}
A concequence of that:
\coroll{for each $x\in \mathcal{F}$ there is a unique way of writing of $x$ in the form $r+i$ where $r\in \RR$ and $i\in \mathcal{I}$}
\note{}{If $x= r+i$ then we also write $\st(x) = r$.

}
\prop{}{
    \begin{itemize}
        \item $\st: \mathcal{F}\twoheadrightarrow \RR$ 
        \item $\st (x) = 0$ iff $x\in \mathcal{I}$
        \item $\st ( x\ ^* + y) = \st(x) + \st(y)$
        \item $\st ( x\ ^* \cdot y) = \st(x) \cdot \st(y)$
    \end{itemize}
}{}
\note{}{this says that $\st$ is a homomorphism of $\mathcal{F}$ onto field $\overline{\RR}$ with 
$\ker(\st) = \mathcal{I}$ and $\mathcal{F}/_\mathcal{I}\cong \overline{\RR}$

}
\defin{Convergence (non-standard definition)}{
    $F$ converges at $a$ to $b$ if whenever $x\simeq a$ and $x\neq a$ 
then $^*F(x)\simeq b$.}
\note{}{This definition is equivalent to $\varepsilon-\delta$ definition of convergence in Analysis.
\begin{itemize}
    \item Suppose $F$ converges to $b$ at $a$ in $\varepsilon-\delta$-sense
    \[\mathcal{R}\models \forall \varepsilon>0 \exists \delta>0 \forall z 
    ( |z-a|<\delta \implies |F(z)-b| < \varepsilon)\]
    \[\mathcal{R}^*\models \forall \varepsilon>0 \exists \delta>0 \forall z 
    ( |z-a|<\delta \implies |F(z)-b| < \varepsilon)\]
    Let $\varepsilon>0$ and $\delta>0$ corresponding to $\varepsilon$.
    Let $x\simeq a$ then $|x-a|<r$ for all positive $r\in \RR^{>0}$ so in particular $|x-a|<\delta$, 
    therefore $|F(x)-b|<\varepsilon$ but $\varepsilon$ was arbitrarily, so $st(F(x))=b$.
    \item Suppose $F$ convergences to $b$ at $a$ in the non-standard-sense.
    Then $\forall \varepsilon \in \RR^{>0}$ 
    $$\mathcal{R}^*\models \exists \delta >0 \forall x (|x-a|<\delta \to |F(x)-b|<\varepsilon)$$
    Because $\delta\in \mathcal{I}$ works.
    But then 
    $$\mathcal{R}\models \exists \delta >0 \forall x (|x-a|<\delta \to |F(x)-b|<\varepsilon)$$
\end{itemize}
}
\note{}{If $F$ converges to $b$ at $a$ then $b$ is unique such that for every $i\in \mathcal{I}$ 
the standard part $\st (F(a+i)) = b$. And we use the general notation $\lim_{x\to a}{F(x)}=b$}

\coroll{$F$ continuous at $a$ then $x\simeq a\implies \: ^*F(x)\simeq ^*F(a)$}
Derivatives
From Analysis: If $F:\RR \to \RR$ then $F'(a) = \lim_{h\to 0}\frac{F(a+h)-F(a)}{h}$ 
\defin{}{We will say that $F'(a) = b$ iff $\forall dx\in \mathcal{I}$, $dx\neq 0$ then 
\[\st\bigl(\frac{F(a+dx)-F(a)}{dx}\bigr) = b\]
}
$dF \defeq ^*F(a+dx) - F(a)$ then
$F'(a) = b$ iff $\forall dx\in \mathcal{I}, dx\neq 0$  we have $\st(\frac{dF}{dx}) = b$
$\frac{dF}{dx}$ is an actual division.

\bsp{}{
    $F(x) = x^2$
    \[\frac{dF}{dx} = \frac{(a+dx)^2 - a^2}{dx} = \frac{2dxa + (dx)^2}{dx} = 2a + dx\]
    and $\st\frac{dF}{dx} = 2a$
}
\prop{}{(standard) If $F'(a)$ exists, then $F$ is continuous at $a$.}{
    Assume $F'(a)$ (in the standard sense) exist, then $F'(a)$ is a finite number and 
    $F'(a) \simeq \frac{F(a+dx)-F(a)}{dx}$.
    Therefore $F(a+dx)-F(a)$ has to be infinitesimal ($\in \mathcal{I}$). 
    Which means $F(a+dx) \simeq F(a)$.
}
\prop{Chain rule}{Suppose $G'(a)$ and $F'(G(a))$ exist then $(F\circ G)'(a) = F'(G(a))\cdot G'(a)$
}{
    Note: $^*(F\circ G) = ^*F \circ ^* G$ because 
    $\mathcal{R} \models \forall x F_{f\circ g}(x) = (F_{f}\circ F_{g})(x)$
    \[dG \defeq ^*G(a+dx) - ^*G(a)\]
    \[\begin{aligned}
        dF \defeq& ^*(F\circ G)(a+dx) - ^*(F\circ G)(a)\\
        =& ^*F(^*G(a+dx)) - ^*F(^*G(a))\\
        =& ^*F(^*G(a) + dG)) - ^*F(^*G(a))\\
    \end{aligned}
    \]
    We know that $G(a)$ exists so $G$ is continuous at $a$ and therefore $dG\simeq 0$
    \begin{itemize}
        \item case $dG\neq 0$ then $\frac{dF}{dG} \simeq F'(G(a))$.
        We can re-write $$\frac{dF}{dx} = \frac{dF}{dG}\frac{dG}{dx} = F'(G(a)) \cdot G'(a)$$
        \item case $dG = 0$ then $dF = 0$ and $G'(a) = \frac{dG}{dx} = 0$ and therefore ($dx\neq 0$)
        \[\frac{dF}{dx} =0 = F'(G(a))\frac{dG}{dx}\]
    \end{itemize}
}
\section{o-minimality}\label{MT:sec:o-min}
\bsp{}{
    $\overline{\RR} = (\RR,+,-,\cdot, 0,1,\leq)$ $\overline{\RR} = (\RR,\leq)$ TODO or at least 
    in a very similar language, 
    then by quantifier elimination (QE, Tarski) all the definable sets of $\RR$ are finite unions 
    of points and intervals.
}
\defin{o-minimality}{Let $\mathcal{L} = \{\leq,\dots\}$, $\mathcal{M}$ is an $\mathcal{L}$-structure such that 
$\mathcal{M}\models \text{DLO}$ and the only definable subsets of $M$ are finite union of points and 
intervalls. Then $\mathcal{M}$ or equivalent $\Th(\mathcal{M})$ is called o-minimal.}
o-minimal is not a first order property so to say that a theory is o-minimal is non trivial.
\note{}{Cell decomposition means 
    Suppose $X$ is definable in an o-minimal structure $\mathcal{M}$, $X\subseteq M^n$ then 
    $X$ is a finite union of cells (in dimension 1 these are points or intervalls)
    in $M^2$ it is either the graph of a continuous function or everything inbetween two graphs of continuous 
    functions. (its an inductive definition)
}
\note{}{Have Dedekind complete for definable subets of $M$: 
For $X\subseteq M$ definable then $\inf X, \sup X$ exist in $M_{\pm \infty}$. }

\note{}{If $M$ contains infinitely small elements, for example $M = \!^*\RR$ then $(0,1)\subseteq M$ is not connected.
$O_1 = \{x: \forall n\in \NN^* 0<x<\frac{1}{n}\}$ is open and so is its complement in $(0,1)$. We have 

Note that $O_1$ is however not definable in $M$. If $O_1$ would be definable it would be a finite union 
of points and intervals. It is convex, and not a point. But it is also not an intervall, because then it 
would have by Dedekind completness that $\sup O_1$ exists in $M_{\pm \infty}$, a contradiction. TODO: 
}
\defin{definably connectedness}{$X\subseteq M^m$ is said to be definably connected, if $X$ is definable and $X$ is not the 
disjoint union of two definable, non-empty open sets.}
\lemma{}{
    \begin{enumerate}
        \item The definably connected subsets of $M$ are the intervalls (including singletons) and $\varnothing$.
        \item The image of a definable connected subset $X\subseteq M^n$ under a definable continuous map $f:X\to M^n$ is definably connected. ($f$ is called to be definable, if its graph $\Gamma f \subseteq M^{mn}$ is).
        \item (IVP) If $f:[a,b]\to M$ definable and continuous, then $f$ assumes all values between $f(a)$ and $f(b)$.
    \end{enumerate}
}{Exercise}
\section{o-minimal ordered groups and rings}
\defin{ordered group}{A ordered group is a group with a linear order such that 
\[\forall x \forall y \forall z x<y \to (zx<zy \land xz < yz)\]
}
\bsp{}{
    \begin{itemize}
        \item $(\RR,<,+)$ 
        \item $(\RR^{>0},<,\cdot)$
        \item non-example: $(\RR^*, \cdot,<)$
    \end{itemize}
}
Recall:
\begin{itemize}
    \item $(G,\cdot)$ is divisible, if $\forall n \forall g \exists x g = x^n$, equivalent to $\forall n G^n = G$.
    \item $(G,\cdot)$ is torsion-free, if no element has finite order except for $1$.
\end{itemize}
\prop{}{$(M,<,\cdot, \dots)$ o-minimal such that $(M,<,\cdot)$ ordered group, then
    $(M,<,\cdot)$ abelian, divisible and torsion-free.
}{}
\lemma{}{If $G$ is a definable subgroup of $M$ then $G$ is convex.
}{
    Suppose $G$ is not convex, then there exists $1<a<g$ for some $g\in G$ and $a\in M\backslash G$. 
    Then 
    \[1<a<g<ag<g^2<ag^2<g^3<\dots \]
    but elements alternate being in $G$ and outside of $G$ so $G$ is not definable (finite union).
}
\lemma{}{The only definable subsets of $M$ that are subgroups are $\{1\}$ and $M$.
}{
    Suppose $G\neq \{1\}$ wts. $G = M$.
    From the previous lemma we know that $G$ is convex.
    The idea is $s\defeq \sup G$ then $1<s$ and $(1,s)\subseteq G$. If $s=+\infty$ then $G = M$.
    Suppose $s\neq +\infty$ then 
    Take $1<g<s$ then $g^{-1}s\in (1,s)$
    So $s = g g^{-1}s\in G$ 
    and $s<sg$ thats a contradiction with $s = \sup G$. 
}
\begin{proof}
    of Proposition.
    \begin{itemize}
        \item $(M,\cdot)$ abelian:
        For any $a\in M$ we can look at $C_a = \{x\in M : xa = ax\}$ it is a definable subgroup and contains $a$ it therefore is non-trivial and we have $C_a = M$ for every $a\in M$, so abelian.
        \item For any $n\in \NN^{>0}$ look at $\{x^n: x\in M\}$ non trivial definable subgroup of $M$, hence $=M$.
        \item Every ordered group is torsion-free.
    \end{itemize}
\end{proof}
\defin{Ordered ring}{A ring (assumed to always be associative, with $1$)
 equipped with a linear order such that
\begin{enumerate}
    \item $0<1$
    \item $<$ translation invariant
    \item $<$ invariant under multiplication by positive elements
\end{enumerate}
}

\note{}{
    \begin{itemize}
        \item The additive group $(R,<,+)$ of an ordered ring is a ordered group. 
        \item Ordered rings have no zero-divisors $\forall x \forall y xy = 0 \to (x=0 \lor y = 0)$
        \item $x^2\geq 0$ 
        \item $k\mapsto k \cdot 1 : \ZZ \to \text{ring}$ is a strictly increasing embedding with resprect to the usual ordering on $\ZZ$ that means our characteristic is $0$.
    \end{itemize}
}
\note{}{
    \begin{itemize}
        \item A division ring is a field without commutativity of multiplication, so 
        \[\forall x x\neq 0 \to \exists y xy = 1\]
        \item Suppose ordered ring is also a division ring. Then such $y$ are unique and $yx=1$. Further, $x>0 \to y>0$.
        
        Also the additive group is divisible, the underlying set is DLO w/o endpoints and 
        $(x,y)\to x \cdot y$
        $x\to x^{-1}$ are continuous with resprect to itervall topology.
    \end{itemize}
}
\defin{ordered field}{An ordered field is an ordered division ring with commutative multiplication.}
\defin{real closed field}{ordered field $R$ \outernote{RCF} such that if $f(X)\in R[X]$ and $a<b$ are such that 
$f(a)<0<f(b)$ then there exists a $c\in (a,b)$ such that $f(c)=0$}
\bsp{}{
    \begin{itemize}
        \item $(\RR,+,\cdot,<)$ is RCF
        \item $(\mathbb{Q},+,\cdot,<)$ is not a RCF
    \end{itemize}
}
\prop{}{$(M,<,+,\cdot,\dots)$ o-minimal such that $(M,<,+,\cdot)$ is ordered ring then $(M,<,+,\cdot)$ is RCF.}{
    \begin{itemize}
        \item wts. $(M;<,+,\cdot)$ is ordered division ring.
            For all $a\in M$ $aM$ is additive subgroup of $(M,+)$ hence $aM = M$ if $a\neq 0$.

        \item wts. commutativity of $\cdot$. $\text{Pos}(M) \defeq \{a\in M: a>0\}$ is a subgroup of the multiplicative group of $M$. Let $a\in M$ then bc $M = aM$ we have $b\in M$ such that $1=a \cdot b$ and by note $0<a$ then $0<a^{-1}$.
        so multiplication is commutative on $M$
        \item IVP property for polynomials:
        The ring operations are continuous, see note and use lemma about IVP (c).
    \end{itemize}
}
\subsubsection{Cell decomposition}
Base step: 
\prop{Monotonicity Theorem}{
    Suppose $f:(a,b)\to M$ definable, then there are $a<c_1< \dots < c_k<b$ such that for $(a,c_1)$, $(c_i,c_{i+1})$, $(c_k,b)$ subsets of $(a,b)$ we have: $f$ is either constant or strictly monotinic and continuous.
}{}
\lemma{}{$\exists$ subinterval on which $f$ is const or injective.}{}
\lemma{}{If $f$ injective, then strictly monotone on a subinterval.}{}
\lemma{}{If $f$ strictly monotone, then $f$ continuous on a subinterval.}{}

\begin{proof}
    Proof of Monotonicity theorem:
    Consider 
    $$X \defeq \biggl\{x\in (a,b) : \begin{aligned}
        &\text{on some subinterval containing $x$,}\\
        &\text{$f$ is either constant or strictly monotone and continuous}
    \end{aligned} \biggr\}$$
    remark: $X$ is a definable set.
    Look at $(a,b)-X$ is finite. If not, it would contain subinterval use lemma to get contradiction.
    WMA:\addAbbrev{WMA}{We may assume} $X = (a,b)$ in particular we may assume $f$ continuous.
    By subdividin $(a,b)$ further WMA that we are in one of the following cases
    \begin{enumerate}
        \item[Case 1:] $\forall x \in (a,b)$ $f$ constant on some neighborhood of $x$
        \item[Case 2:] $\forall x \in (a,b)$ $f$ is strictly monotone increasing on some neighborhood 
        of $x$
        \item[Case 3:] $\forall x \in (a,b)$ $f$ is strictly monotone decreasing on some neighborhood 
        of $x$
    \end{enumerate}
    \begin{enumerate}
        \item[Case 1:] $x_0 \in (a,b)$ then 
        $s \defeq \{ x: x_0<x<b \land f \text{ cont. on $[x_0,x)$}\}$
        wts $s=b$ suppose $s<b$ then $f$ constant on neighborhood of $s$ contradiction with 
        definition of $s$ so $f$ continuous on $[x_0,b)$. $f$ constant on $(a,x_0]$ similar.
        \item[Case 2:] $x_0\in (a,b)$ then $s\defeq \{x: x_0<x<b \land f \text{strictly incr. on } 
        [x_0,x) \}$. wts: $s=b$ assume $s<b$ then $f$ is strictly increasing on some neighborhood of 
        $s$ so $f$ strictly increasing on $[x_0,s+\delta)$ for some $\delta>0$, a contradiction to 
        definition of $s$. 
        \item[Case 3:] similar to Case 2. 
    \end{enumerate}
\end{proof}
%03.12.2024
Proof of Lemma 1:
\begin{proof}
    Statement: ``There exists a subinterval on which $f$ is constant or injective.''

    \begin{itemize}
        \item If $y\in R$ so that $f^{-1}$ is infinite (it has to be a finite union of points and intervals), 
        then $f^{-1}(y)$ contains an interval and $f(x)=y$ on that interval.
        \item Suppose $f^{-1}(y)$ is finite for every $y\in R$.\\
        $f(I)$ is infinite and is definable because $f$ is definable, so it contains an interval $J$.
        We can define an inverse to $f$ on $J$ $g:J\to I$, $g(y)$ is the first $x\in I$ such that $f(x)=y<$ (this is definable). $g$ is necessarily injective. $g(J)$ infinite, so contains a subinterval on which $f$ is injective.
    \end{itemize}
\end{proof}

Proof of Lemma 2:
\begin{proof}
    Statement: ``If $f$ injective, then strictly monotone on a subinterval.''

    Suppose $f$ is injective. $f:I=(a,b)\to R$
    pick $x\in (a,b)$ then $(a,x) = \{y\in (a,x): f(y)<f(x)\} \uplus \{y\in (a,x): f(x)<f(y)\}$ is definable disjoint union of definable sets, so one of the subsets has to contain an interval $(c,x)$ with $a \leq c$, similarly for $(x,d)$.
    So for all $x\in I$ we have $x$ satisfies one of the following:
    \begin{itemize}
        \item $\Phi_{++}(x)$ iff $\exists c_1,c_2 (c_1<x<c_2 \land \forall c \in (c_1,x) f(c)>f(x) \land \forall c\in (x,c_2) f(c)>f(x)$
        \item $\Phi_{--}(x)$
        \item $\Phi_{+-}(x)$
        \item $\Phi_{-+}(x)$
    \end{itemize}
    The set of all $x$ that satisfy each $\Phi$ is definable, it therefore is a finite union of points and intervalls.
    After passing to subinterval $(a,b)$ of $I$ WMA that each $x\in I$ 
    satisfies the same $\Phi_{\pm\pm}$.
    \begin{itemize}
        \item $\Phi_{-+}(x)$, on the left everybody is smaller, on the right everybody is bigger.
        \[\forall x\in I s(x)\defeq \sup\{s\in(x,b): f(x)<f(s)\}\]
        If $s(x)<b$ then $\Phi_{-+}(s(x))$, and therefore there is an element $s'>s(x)$ such that $f(x)
        \leq f(s(x))<f(s')$ so $s(x)\geq s'$ which is a contradiction to definition to $s(x)$, therefore 
        $s(x) = b$ for every $x\in (a,b)$. Then $f$ has to be strictly increasing on $(a,b)$.
        \item $\Phi_{+-}(x)$ similar  (monotinic decreasing)
        \item $\Phi_{++}(x)$ $\forall x \in I$.
        $$B \defeq \{x\in I: \forall y\in I (x<y \to f(y)>f(x))\}$$
        $B$ is definable, if $B$ is infinite, it has to contain a subinterval on which $f$ is strictly 
        increasing. 

        WMA $B$ is finite.
        We restrict ourselves to subinterval and may assume $B=\varnothing$. 
        So by injectivity: 
        $$\circledast \quad\forall x \in I \exists y\in I x<y\land f(x)>f(y)$$
        Let $c\in I$. Claim: for every large enough $y\in I$ we have $f(y)<f(c)$.
        \begin{claimproof}
            By contradiction. suppose we can not find a neighborhood of $b$ such that for all elements in this neighborhood $f(y)<f(c)$
            otherwise $f(y)>f(c)$ for all large enough $y$.
            Let $d<b$ be minimal such that 
            \[\forall y\in (d,b) f(y)>f(c)\]
            \begin{itemize}
                \item case $f(d)>f(c)$: $\Phi_{++}(d)$, contradiction with minimality of $d$.
                \item case $f(d)<f(c)$: By $\circledast$ there has to be an $e$ with $d<e<b$ and 
                $f(e)<f(d)$. So $f(e)<f(c)$ which is a contradiction to $\Phi_{++}(d)$
            \end{itemize}
        \end{claimproof} 
        Define $y(c)$ to be the least element of $[c,b)$ for which 
        $$\forall y y(c)<y<b \: f(c)>f(y)$$ 
        $c$ satisfies $\Phi_{++}$, therefore $c<y(c)$ and $f(y(c))<f(c)$ if $y(c)<y<b$.
        The minimality of $y(c)$ implies that $y(c)$ satisfies $\Psi_{+-}$, where 
        \[\Psi_{+-}(v) \iif \exists v_1,v_2 \in I \bigl(v_1<v<v_2 \land \forall z_1,z_2 (v_1<z_1<v\land v<z_2<v_2)\to f(z_1)>f(z_2)\bigr)\]
        But $c$ was arbitrarily so $\forall x\in I \exists v\in I (x<v\land \Psi_{+-}(v))$
        On subinterval $\Psi_{+-}$ 
        we have a contradiction with $\Phi_{++}$, similarly on subinterval for $\Psi_{-+}$.
        \item $\Phi_{--}$ similar to above
    \end{itemize}
\end{proof}
Proof of Lemma 3:
\begin{proof}
    Statement: ``If $f$ strictly monotone, then $f$ continuous on a subinterval.''

    WMA: $f:(a,b)\to R$ strictly monotone increasing.
    $f(I)$ infinite and definable, so $f(I)$ contains an interval $J$.
    Let $r,s\in J$ $r<s$ and $d,e\in I$ with $d<e$ and $f(d)=r$ and $f(e) = s$.
    restrict $f$ to $(d,e)$ and we get an increasing bijection $(d,e)\to (r,s)$.
    Our topology is the order topology, so $f$ is continuous on $(d,e)$
\end{proof}
So we have proved the monotonicity Theorem.
\note{}{If $f:(a,b)\to R$ is definable, then $\lim_{x\to c^-}f(x)$ exists in $R_{\infty}$ 
    for $c\in (a,b]$. 
    And further $\lim_{x\to c^+}f(x)$ exists in $R_\infty$ for $c\in [a,b)$ 
    If furthermore $f:[a,b]\to R$ is continuous and definable, then $f$ assumes a 
    minimum and maximum on $[a,b]$ 
}
On of the important tools in o-minimality theory is the cell cecomposition theorem:
\defin{Cell}{
    Let $(i_1,\dots i_n)$ a sequence in $\{0,1\}$. An $(i_1,\dots i_n)$-cell is defined inductively:
    \begin{itemize}
        \item $(0)$-cell: $\{r\}\subseteq R$,
        \item $(1)$-cell: $(a,b)\subseteq R$, $a<b$, $a,b\in R$.
        \item $(i_1,\dots i_k, 0)$-cell: $\Gamma f\subseteq R^{k+1}$, where $f$ is definable and continuous function $f:X\to R$, where $X$ is a $(i_1,\dots, i_k)$-cell
        \item $(i_1,\dots i_k, 1)$-cell: is a the set 
        $$(f,g) = \{(\underline{x},x_{k+1})\in R^{k+1} : f(\underline{x}) < x_{k+1}<g(\underline{x})\}$$ 
        $f:X\to R, g: X\to R$, $f<g$ $f,g$ are definable and continuous on $X$, which is a $(i_1,\dots i_k)$-cell. $f$ may be constantly $-\infty$ and $g$ may be constantly $\infty$.
    \end{itemize}
}
\note{Cells have nice topological properties}{
    \begin{itemize}
        \item every $(1,\dots 1)$-cell are precisely the cells that are open in their ambient space. continuity of the functions is important.
        \item The union of finitlely many non-open cells has empty interior.
        \item Each cells is locally closed i.e. open in its closure.
        \item Each cell is homeomorphic to an open cell under a coordinate projection Example $(1,0,0,1)$-cell or $(1,0)$-cell with $(x_1,x_2)\mapsto x_2$
        \item If $A\subseteq R^{n+1}$, then $\pi A\subseteq R^n$ cell $\pi(x_1,\dots x_{n+1})\mapsto (x_1,\dots x_n)$
        \item Every cell is definalby connected. You can proof this by induction on the cell. $\{r\}$ and open intervals are definably connected. If the projection of a cell is definably connected, then the fibre above it is either an open interval or a single point. It is even definable path connected. If there would exist an open disjoint cover there exists an open disjoint cover of the fibre, which is not possible.
    \end{itemize}
}
\defin{decomposition}{A decomposition of $R^m$ is a finite partition of $R^m$ into cells defined inductively:
\begin{itemize}
    \item decomposition of $R^1 = R$: 
    \[\bigl\{(-\infty,a_1), \{a_1\},(a_1,a_2)\dots (a_k,\infty)\bigr\}\]
    \item A decomposition of $R^{n+1}$ is a finite partition of $R^{n+1}$ into cells $C$ such that the collection of $\pi C$ is a decomposition of $R^n$.
\end{itemize}
}
\thm{Cell decomposition}{
    \begin{itemize}
        \item[$(\RomanNum{1} _m)$] Let  $A_1,\dots A_k\subseteq R^m$ definable sets. Then there is a decomposition of $R^m$ partitioning each $A_i$.
        \item[$(\RomanNum{2} _m)$] Given a definable function $f: A\to R$, $A\subseteq R^m$ there is a decomposition $\mathcal{D}$ of $R^m$ partitioning $A$ such that for every $B\in \mathcal{D}$ $f|_B:B\to R$ is continuous. 
    \end{itemize}
}{
    By induction on $m$.
    Base step: \begin{itemize}
        \item $(\RomanNum{1} _1)$ o-minimality
        \item $(\RomanNum{2} _1)$ monotonicity theorem.
    \end{itemize}
    Proof idea: Suppose we have $$
    \begin{cases}
        (\RomanNum{1} _1)\dots (\RomanNum{1} _m)\\
        (\RomanNum{2} _1)\dots (\RomanNum{2} _m)
    \end{cases}\bigr\} \implies (\RomanNum{1} _{m+1}), (\RomanNum{2} _{m+1})
    $$ 
}
\defin{}{A definably connected component of a non-empty definable Subset $X\subseteq R^m$ is a definably-connected subset of $X$ which is maximal wrt being definably connected}
\bsp{}{$X\subseteq R^m$ definable. Then it is definably connected iff $X$ definably path connected 
i.e. 
$$\forall x , y\in X\exists f:[0,1]\to X \text{ definable and continuous with } f(0)=x \land f(1)=y$$
}

\prop{}{Suppose $X\subseteq R^m$ is definable and non-empty, then $X$ has only finitlely many 
    definably connected components. The components are both open and closed in $X$ and they form a 
    finite partition of $X$.}
{
    Let $\{C_1, \dots C_k\}$ be a partition of $X$ into cells. $I\subseteq \{1,\dots k\}$ then
    $C_I \defeq \bigcup_{i\in I}C_i$. 
    Let $C'$ be the maximal among the $C_I$ that is definable connected.
    Claim: For $Y\subseteq X$ definable connected such that $Y\cap C'\neq \varnothing$ then $Y\subseteq C'$.
    \begin{claimproof}
        $C_Y \defeq \bigcup \{C_i : C_i\cap Y \neq \varnothing\}$
        Then $Y\subseteq C_Y$. and $C_Y$ is definably (finite union) connected union of definably connected set $Y$ and finitly meny cells that have non-empty intersection with $Y$.
        Then $C_Y\cap C'$ contains $Y\cap C'\neq \varnothing$. So if we take $C_Y \cup C'$ has to be again definably connected. By maximality $C_Y \cup C'= C'$ and $Y\subseteq C_Y \subseteq C'$.
    \end{claimproof}
    Hence 
    \begin{itemize}
        \item $C'$ definably connected component of $X$
        \item The sets $C'$ form a finite partition of $X$
        \item $C'$ are the only definable connected components of $X$
    \end{itemize}
    The closure in $X$ of a definably connected subset of $X$ is definable connected. (see topology)
    So the $C'$ are closed in $X$. 
    They are also open because the complement in $X$ is a finite union of closed subsets.
}
\note{}{The above Proposition is not true if we drop the requirement ``definable''}

\defin{Definable families}{
    Let $S\subseteq R^{m+n}$ definable. For $a\in R^m$ we put
    $$S_a = \{\underline{x}\in R^n : (a,\underline{x})\in S\}\subseteq R^n$$
    $S$ describes the family of sets $(S_a)_{a\in R^m}$.
    And the sets $S_a$ are called the fibers of $S$. 
}
\bsp{}{$\mathcal{R}=(\RR,<,+,\cdot)$ 
\[ax^2 + bxy + c y^2 + dx + ey + f = 0\]
defines $S\subseteq R^6\times R^2$.
The fibers are: 
\begin{itemize}
    \item points, circles, ellipse, hyperbola, parabola
    \item and the limiting cases: $\varnothing$, 2 lines intersecting each other, 2 parallel lines, one line, $RR^2$
\end{itemize}
}
\note{}{In o-minimal structures there are only finitly many homomorphism types in a definably family. (If there are infinitely many fibres, then only finitlely many are not homeomorphic to each other).}
\prop{}{
    \begin{enumerate}
        \item $C$ cell in $R^{m+n}$, $a\in \pi_m^{m+n}C$ (where 
        $\pi^{m+n}_m (x_1,\dots x_{m+n}) = (x_1,\dots x_m)$) Then $C_a$ is a cell in $R^n$
     \item $\mathcal{D}$ decomposition of $R^{m+n}$, and $a\in R^m$ then 
     $$\mathcal{D}_a = \{C_a : C\in \mathcal{D} \land a \in \pi^{m+n}_m(C)\}$$ is a decomposition  of $R^m$.
    \end{enumerate}
}{
    \begin{enumerate}
        \item induction on $n$. If $n=1$, $a\in \pi^{m+1}_m C$ 
        Then $C_a$ is one of the below
        \begin{itemize}
            \item If $C$ is a $(i_1,\dots i_m,0)$-cell then $C = \Gamma f$, $f:\pi^{m+1}_m C \to R$ definalby continuous. 
            $a\in \pi^{m+1}_mC$ then $C_a = \{f(a)\}\subseteq R$
            \item If $C$ is a $(i_1,\dots i_m,1)$-cell then $C = (f,g)$, $C_a = (f(a),g(a))$
        \end{itemize}
        Suppose the statement holds for some $n$ then let
        $C\subseteq R^{m+n+1}$ be a cell. Consider the two projections
        $\pi^{m+n+1}_{m+n}, \pi^{m+n}_m$ and
        $$\pi^{m+n}_m\circ \pi^{m+n+1}_{m+n} : R^{m+n+1}\to R^m$$ 
        Two options: 
        Either $C = \Gamma f$, then 
        $$C_a = \Gamma f_a \text{ where } f_a: ( \pi^{m+n+1}_{m+n} C)_a\to R$$ and
        $f_a(x) = f(a,x)$\\
        Or $C = (f,g)_D$ i.e. $f,g:D\to R$, $D\subseteq R^{m+n}$ cell, $D = \pi^{m+n+1}_{m+n}$
        Then $C_a = (f_a, g_a)_E$, $E = D_a$. in both cases, $C_a$ is a cell.
        \item Exercise.
    \end{enumerate}
}

\coroll{Let $S\subseteq R^m \times R^n$ a definable family then there exists $M_S\in \NN$ such that 
    for all $a\in R^m$ $S_a\subseteq R^m$ has a partition into $M_S$ many cells.
}
\begin{proof}
    $S\subseteq R^m\times R^n$ $\mathcal{D}$ decomposition of $R^m\times R^n$ that partions $S$. Then $S$ is a finite union of cells from $\mathcal{D}$, each fiber $S_a$ is a finite union of $C_a, C\in \mathcal{D}$ but $C_a$ is a cell by Proposition. A bound: $|\mathcal{D}|$.
\end{proof}

\note{}{There is a uniform bound on $\#$ of definable connected somponents of sets in definable family.}


\thm{}{$\mathcal{R}= (R;<,\dots)$ o-minimal $\mathcal{L}$-structure, $R'=(R';<,\dots)$ $\mathcal{L}$-structure. 
If $R\equiv R'$ then $R'$ is o-minimal.
}{
    $S\subseteq R$ definable, $S = \{r\in R: \mathcal{R}\to \varphi(x) [r]\}$ might use parameters from 
    $R$. If $\varphi$ is a $\mathcal{L}$-fla. over $\varnothing$.
    Then 
    $$\begin{aligned}
        \mathcal{R}\models \exists x_1,x_2,x_3 &(x_1 \neq x_2\land x_1\neq x_3 \land x_2\neq x_3 \\
        &\land \forall c ((x_1<c<x_2 \to \varphi(c))\land (c=x_3 \to \varphi(c))\\
        &\qquad\land \lnot (x_1<c<x_2\lor c = x_3)\to \lnot \varphi(c)))
    \end{aligned}$$ 
    But if $\varphi$ uses parameters, we TODO
    For all $S\subseteq R^{m+1}$ need formula $\forall \underline{a}\in R^m$ ``$S_a$ is finite union of points and intervals''
    Idea: Subset of $R$ definableby formula w/ param is just a fiber in a definable family that is parameter-free definable.
    $S_a\subseteq R$ definable. By the note, there is some number $M_S$ that only depends on the TODO
    Such that for each $\underline{a}\in R^m$ $S_a$ is a finite union of at most $M_S$ cells.

    Then \[\begin{aligned}
        \mathcal{R}\models \forall \underline{z}\exists x_1 \dots \exists x_{M_S+1}&\bigl(\forall y (y<x_1 \to \varphi_{\underline{z}}(y))\lor \forall y (y<x_1 \to \lnot \varphi_{\underline{z}}(y))\bigr)\\
        &\land \bigl(\forall y (x_1<y<x_2\to \varphi_{\underline{z}}(y))\lor\forall y (x_1<y<x_2\to \lnot\varphi_{\underline{z}}(y))\bigr)\\
        &\dots%TODO
    \end{aligned}\]
    By elementarily equivalence:
    $\mathcal{R}'\models \dots$.
}
%\note{}{0-min is strong o-minimality.} TODO

